\begin{tikz*}[%
	every node/.style={align=center,rounded rectangle,draw,minimum width=8.5em,minimum height=3.25em}
]
	\node(orig) [rectangle] {Исходная \\ программа};
	\node(trans) [below=of orig] {\alert<2>{Трансляция} \\ \alert<2>{исходного кода}};
	\node(improve) [right=of trans] {\alert<4>{Модификация} \\ \alert<4>{структуры}};
	\node(rev-eng) [above=of improve] {\alert<3>{Обратная} \\ \alert<3>{инженерия}};
	\node(module) [right=of improve] {\alert<5>{Модуляризация}};
	\node(doc) [rectangle,above=of module] {Документация};
	\node(restruct) [rectangle,below=of improve] {Структур. \\ программа};
	\node(reeng) [rectangle,below=of module] {Модиф. \\ программа};
	\node(data-reeng) [right=of module] {\alert<6>{Реинженерия} \\ \alert<6>{данных}};
	\node(orig-data) [rectangle,above=of data-reeng] {Исходные \\ данные};
	\node(data) [rectangle,below=of data-reeng] {Модиф. \\ данные};

	\draw[->] (orig) -- (trans);
	\draw[->] (trans) -- (rev-eng);
	\draw[->] (trans) -- (improve);
	\draw[->] (rev-eng) -- (doc);
	\draw[->] (improve) -- (restruct);
	\draw[->] (doc) -- (module);
	\draw[->] (restruct) -- (module);
	\draw[->] (module) -- (reeng);
	\draw[->] (orig-data) -- (data-reeng);
	\draw[->] (reeng) -- (data-reeng);
	\draw[->] (data-reeng) -- (data);

	\node<2> [note={(trans.south)},below=1em of trans.south,anchor=north west] {%
		автоматизированный перевод программы \\ на новую версию ЯП или другой ЯП};
	\node<3> [note={(rev-eng.south)},below=1em of rev-eng.south,anchor=north] {%
		автоматизированное создание документации \\ на основе структуры программы};
	\node<4> [note={(improve.south)},below=1em of improve.south,anchor=north] {%
		упрощение структуры программы \\ (частичная автоматизация)};
	\node<5> [note={(module.south)},below=1em of module.south,anchor=north] {%
		группирование связанных частей программы \\ и удаление избыточности (вручную)};
	\node<6> [note={(data-reeng.south)},below=1em of data-reeng.south,anchor=north east] {%
		обработка данных для согласования \\ с изменениями в системе \\ (напр., изменение схемы БД)};
\end{tikz*}
