\documentclass{a4beamer}
%% Lectures - common definitions

\usextensions{tikz}
\usetikzlibrary{shapes.multipart,shapes.callouts,shapes.geometric}
\input{fix-callouts.inc} % Fixes absolute positioning of rectangle callouts

\newif\ifbigpages \bigpagesfalse
\ifdim\paperwidth >20cm
	\bigpagestrue
\fi

\tikzset{%
	note/.style={rectangle callout,draw=none,callout pointer width=1em,%
		align=flush left,font=\footnotesize,inner sep=0.5em,%
		fill=blue!15,fill opacity=0.95,text opacity=1.0,callout absolute pointer=#1},
	node distance=2em and 2.75em
}
\ifbigpages
	% Scale all arrow tips by the factor of 2.5
	\let\old@pgf@arrow@call=\pgf@arrow@call
	\def\pgf@arrow@call#1{%
		\@tempdima=\pgflinewidth%
		\pgfsetlinewidth{2.5\pgflinewidth}%
		\old@pgf@arrow@call{#1}%
		\pgfsetlinewidth{\@tempdima}%
	}
	\def\pgfarrowsleftextend#1{\pgfmathsetlength{\pgf@xa}{1.5*#1}}
	\def\pgfarrowsrightextend#1{\pgfmathsetlength{\pgf@xb}{1.5*#1}}
\fi

%% Load listings package
\usepackage{listings}

%% Are we inside a comment?
\newif\iflstcomment \lstcommentfalse

\lstset{%
	tabsize=4,
	showstringspaces=false,
	basicstyle=\linespread{1.25}\ttfamily\small,
	keywordstyle=\bfseries,
	commentstyle=\lstcommentstyle,
	numbers=left,
	numberstyle=\footnotesize\color{gray},
	xleftmargin=2.5em,
	extendedchars=true,
	escapechar=\$,
	escapebegin=\iflstcomment\begingroup\lstcommentstyle\fi,
	escapeend=\iflstcomment\endgroup\fi
}

\def\lstcommentstyle{\color{gray}}

\lst@AddToHook{AfterBeginComment}{\global\lstcommenttrue}
\let\orig@lst@EndComment=\lst@EndComment
\def\lst@EndComment{\global\lstcommentfalse\orig@lst@EndComment}
\lst@AddToHookAtTop{EOL}{%
	\lst@ifLmode\global\lstcommentfalse\fi% XXX Sloppy way to determine comment end
}

%% Python with docstrings treated as comments
\lstdefinelanguage[doc]{python}[]{python}{%
	deletestring=[s]{"""}{"""},%
	morecomment=[s]{"""}{"""}%
}%

%% JavaScript language
\lstdefinelanguage{javascript}%
	{morekeywords={break,case,catch,%
		const,constructor,continue,default,do,else,false,%
		finally,for,function,if,in,instanceof,%
		new,null,prototype,%
		return,switch,this,throw,%
		true,try,typeof,var,while},%
	sensitive,%
	morecomment=[l]//,%
	morecomment=[s]{/*}{*/},%
	morestring=[b]",%
	morestring=[b]',%
}[keywords,comments,strings]%

%% C# language (4.0?)
\lstdefinelanguage{csharp}%
	{morekeywords={abstract,as,%
		base,bool,byte,case,catch,char,%
		checked,class,const,continue,%
		decimal,default,delegate,do,double,%
		else,enum,event,explicit,extern,%
		false,finally,fixed,float,for,foreach,%
		goto,if,implicit,in,int,interface,%
		internal,is,lock,long,%
		namespace,new,null,object,operator,out,%
		override,params,private,protected,public,%
		readonly,ref,return,sbyte,sealed,%
		short,sizeof,stackalloc,static,string,%
		struct,switch,this,throw,true,try,%
		typeof,uint,ulong,unchecked,unsafe,ushort,%
		using,virtual,void,volatile,while%
	},%
	sensitive,%
	morecomment=[l]//,%
	morecomment=[s]{/*}{*/},%
	morestring=[b]",%
	morestring=[b]',%
}[keywords,comments,strings]%

%% Translation for fact environment
\deftranslation[to=russian]{Fact}{Наблюдение}

%% Inline code snippets
\def\code#1{\texttt{#1}}
\def\codekw#1{\code{\textbf{#1}}}

\def\quoteauthor#1{\par\footnotesize\upshape\hfill—~#1}

%% English term
\def\engterm#1{(англ. \textit{#1})}
%% Term with explanation below (to be used in diagrams)
\def\termwithexpl#1#2{#1\strut{}\\\small\color{gray}(\textit{#2})\strut{}}
%% External link
\def\extlink#1#2{\href{#1}{\color[rgb]{0.7,0.7,1.0}\dashbar{#2}}}
%% Internal link
\def\inlink#1#2{\hyperlink{#1}{\color[rgb]{0.7,0.7,1.0}\dashbar{#2}}}
%% Explanation for a list item
\def\itemexpl#1{\begingroup\small\vspace{0.75ex}#1\par\endgroup}



\begingroup
	\catcode`\^=12
	\gdef\texthat{^}
\endgroup


\lecturetitle{Программная инженерия. Лекция №22 — Управление конфигурацией ПО (часть 2)}
\title[Конфигурация 2]{Управление конфигурацией ПО (часть 2)}
\author{Алексей Островский}
\institute{\small{Физико-технический учебно-научный центр НАН Украины}\vspace{2ex}}
\date{30 апреля 2015 г.}

\begin{document}
	\frame{\titlepage}

	\section{Построение}

	\frame{
		\frametitle{Построение системы}

		\begin{Definition}
			\textbf{Построение системы} \engterm{system building} — процесс создания полной исполняемой версии 
			программной системы путем компиляции и~связывания \engterm{linking} компонентов программы, 
			внешних библиотек, файлов конфигурации и~т.\,п.
		\end{Definition}

		\vspace{2.5ex}
		\begin{columns}[t]
			\begin{column}{0.45\textwidth}
				\textbf{Входные данные}

				\vspace{0.5ex}
				исходный код; \\
				файлы конфигурации; \\
				файлы данных (напр., локализация); \\
				внешние библиотеки; \\
				компиляторы и другие инструменты; \\
				тесты.
			\end{column}
			\begin{column}{0.45\textwidth}
				\textbf{Выходные данные}

				\vspace{0.5ex}
				исполняемые файлы; \\
				документация; \\
				результаты тестирования; \\
				упаковка исполняемых файлов (напр., JAR-архивы при~разработке в~Java); \\
				развертывание~ПО на~целевой системе.
			\end{column}
		\end{columns}
	}

	\subsection{Требования к инструментам построения}

	\frame{
		\frametitle{Инструменты построения}

		\textbf{Требования к инструментам построения:}
		\begin{itemize}
			\item
			Комплексное использование компиляторов (напр., \code{gcc}) и компоновщиков (напр., \code{ld}) 
			для~создания исполняемого кода, готового к~развертыванию.

			\vspace{0.5ex}
			\item
			Минимизация количества повторных действий: отслеживание изменившихся исходных файлов, 
			чтобы компилировать / компоновать только изменившиеся модули.

			\vspace{0.5ex}
			\item
			Использование промежуточных артефактов (напр., объектных файлов) 
			для~повышения модульности и~ускорения построения.

			\vspace{0.5ex}
			\item
			Отчеты об ошибках: остановка построения при~ошибке компиляции или~компоновки; 
			возможность быстро локализовать ошибку.
		\end{itemize}
	}

	\frame{
		\frametitle{Инструменты построения}

		\textbf{Требования к инструментам построения} (продолжение):
		\begin{itemize}
			\item
			Конфигурация построения: возможность задания параметров, 
			влияющих на~особенности построения системы (напр., дополнительные опции для~компиляторов / компоновщиков; 
			архитектура целевой системы).

			\vspace{0.5ex}
			\item
			Тестирование: автоматическая прогонка модульных / интеграционных тестов 
			после~компоновки для~определения корректности поведения системы.

			\vspace{0.5ex}
			\item
			Интеграция с системой управления версиями: запрос актуальных версий компонентов 
			из~хранилища перед~началом построения.

			\vspace{0.5ex}
			\item
			Дополнительные режимы построения для создания документации и~других вспомогательных ресурсов.
		\end{itemize}
	}

	\subsection{Классификация инструментов построения}

	\frame{
		\frametitle{Классификация инструментов построения}

		\begin{itemize}
			\item
			\textbf{Самостоятельные} — инструменты, вызываемые вручную (напр., из~командной строки).

			\vspace{1ex}
			\textbf{Достоинства:} большая гибкость; четкая спецификация действий при~построении.

			\vspace{0.5ex}
			\textbf{Недостатки:} большой объем конфигурационных файлов.

			\vspace{1.5ex}
			\item
			\textbf{Интегрированные} — утилиты, встроенные в интегрированные среды разработки (IDE).

			\vspace{0.5ex}
			{\small\textbf{NB.} Интегрированные инструменты построения часто основаны на~самостоятельных 
			(напр., построение в~Visual~Studio работает на~основе MSBuild).}

			\vspace{1ex}
			\textbf{Достоинства:} простота использования; минимальная потребность в~конфигурации.

			\vspace{0.5ex}
			\textbf{Недостатки:} недостаточная гибкость; потребность в~дополнительном~ПО для~построения системы.
		\end{itemize}
	}

	\frame{
		\frametitle{Классификация инструментов построения}

		\begin{center}
			\begin{tabular}{|p{0.2\textwidth}|p{0.25\textwidth}|p{0.4\textwidth}|}
			\hline
				\centering\textbf{Утилита} & \centering\textbf{Язык проекта} & \centering\textbf{Язык конфигурации} \cr
			\hline
				make & любой & Makefile \cr
			\hline
				Apache Ant & Java & XML \cr
			\hline
				NAnt & языки .NET & XML \cr
			\hline
				MSBuild & языки .NET, другие & XML \cr
			\hline
				Apache Maven & Java & \raggedright XML / модель проекта \engterm{Project Object Model} \cr
			\hline
				Gradle & Java, Groovy, другие & DSL на основе Groovy \cr
			\hline
			\end{tabular}
		\end{center}

		\vspace{1ex}
		\textbf{Примечание.} Большинство инструментов построения в разной степени поддерживают плагины, 
		расширяющие список допустимых действий.
	}

	\frame{
		\frametitle{Классификация инструментов построения}

		\begin{center}
			\begin{tikz*}[%
	every node/.style={align=center}
]
	\node(arrow-st) [coordinate] {};
	\node(arrow-tip) [coordinate,right=30em of arrow-st] {};

	\node(make-c) [coordinate,right=7.5em of arrow-st] {};
	\node(make) [above=0.5em of make-c,anchor=south] {make, \\ nmake};
	\node(ant-c) [coordinate,right=15em of arrow-st] {};
	\node(ant) [above=0.5em of ant-c,anchor=south] {Ant, \\ NAnt, \\ MSBuild};
	\node(maven-c) [coordinate,right=22.5em of arrow-st] {};
	\node(maven) [above=0.5em of maven-c,anchor=south] {Maven, \\ Gradle};

	\node(speed) [below=0.5em of arrow-st] {Явная конфигурация};
	\node(complete) [below=0.5em of arrow-tip] {Соглашение};

	\draw[->] (arrow-st) -- (arrow-tip);
	\draw (make-c) ++(0,0.5em) -- ++(0,-1em);
	\draw (ant-c) ++(0,0.5em) -- ++(0,-1em);
	\draw (maven-c) ++(0,0.5em) -- ++(0,-1em);
\end{tikz*}

		\end{center}

		\begin{itemize}
			\item
			\textbf{make} — отсутствие каких-либо априорных сведений о~структуре программной системы; 
			возможно построение произвольной системы с~использованием любых инструментов.

			\item
			\textbf{Apache Ant} — минимальные сведения о структуре программной системы (проект Java); 
			построение с~применением утилит разработки~Java (javac, jar, javadoc, …).

			\item
			\textbf{Apache Maven} — неявное соглашение о~структуре программной системы 
			(проект Java с~размещением исходного кода в~фиксированных директориях), 
			заданные наперед цели построения. 
			Конфигурационные файлы описывают \emph{отступления} от~соглашения.
		\end{itemize}
	}

	\subsection{Понятия построения}

	\frame{
		\frametitle{Цели построения}

		\begin{Definition}
			\textbf{Цель построения} \engterm{build target} — режим работы инструмента построения, 
			который специфицируется при~вызове инструмента (например, как~параметр командной строки) 
			и~означает проведение действий для~создания определенных промежуточных или~конечных программных артефактов.
		\end{Definition}

		\vspace{1ex}
		\textbf{Примеры целей построения:}
		\begin{itemize}
			\item
			компиляция всех исходных файлов;
			\item
			\emph{(зависит от предыдущего)} компоновка полученных объектных файлов в единый исполняемый файл;
			\item
			создание документации;
			\item
			удаление промежуточных артефактов построения (напр., объектных файлов).
		\end{itemize}
	}

	\frame{
		\frametitle{Зависимости}

		\begin{fact}
			Режим построения может зависеть от успешного выполнения другого режима построения.
		\end{fact}

		\vspace{1ex}
		\textbf{Пример:} компоновка объектных файлов зависит от компиляции исходного кода.

		\vspace{1ex}
		\textbf{Организация режимов построения:}
		\begin{itemize}
			\item
			Организация в виде ациклического ориентированного графа: 
			декларации зависимости (ребра) между режимами (вершины); режимы выполняются согласно 
			\extlink{http://en.wikipedia.org/wiki/Topological_sorting}{топологической сортировке}.

			\vspace{0.5ex}
			\textbf{Примеры утилит:} make, Apache Ant.

			\item
			Линейная организация (жизненный цикл): режимы построения выполняются в~фиксированном порядке.

			\vspace{0.5ex}
			\textbf{Примеры утилит:} Apache Maven.
		\end{itemize}
	}

	\subsection{Make}

	\frame{
		\frametitle{Make}

		\begin{Definition}
			\textbf{Make} — утилита для автоматического построения произвольных проектов из исходных файлов, 
			использующая сценарии построения (Makefile).
		\end{Definition}

		\vspace{1ex}
		\textbf{Поддерживаемые ОС:} *NIX; Windows (через Cygwin; nmake — аналог из MS Visual Studio).

		\vspace{1ex}
		\textbf{Достоинства:}
		\begin{itemize}
			\item
			построение произвольных программных проектов;
			\item
			отсутствие необходимости в плагинах для задействования внешних инструментов;
			\item
			возможность динамического добавления правил во время построения.
		\end{itemize}

		\vspace{0.5ex}
		\textbf{Недостатки:}
		\begin{itemize}
			\item
			большой объем файлов конфигурации, затраты на их создание и поддержку 
			(решается использованием генераторов);
			\item
			проблемы переносимости между ОС.
		\end{itemize}
	}

	\frame{
		\frametitle{Составляющие Makefile}

		\begin{itemize}
			\item
			\textbf{Макросы} — $\sim$~переменные в языках программирования, подставляются в~правила и~действия; 
			позволяют конфигурировать построение (напр., используемые компиляторы).

			\vspace{0.5ex}
			\textbf{Синтаксис:}
			\lstinputlisting[language=make,escapechar=!]{code-mk-syntax-1}

			\vspace{1ex}
			\item
			\textbf{Включения} других конфигурационных файлов Makefile.

			\vspace{0.5ex}
			\textbf{Синтаксис:}
			\lstinputlisting[language=make,escapechar=!,morekeywords={include}]{code-mk-syntax-2}
		\end{itemize}
	}

	\frame{
		\frametitle{Составляющие Makefile}

		\begin{itemize}
			\item
			\textbf{Правила} — \emph{зависимости} (другие режимы построения и~файлы), 
			которые влияют на~\emph{цель} (режим построения или~выходной файл / файлы).

			\vspace{1ex}
			\item
			\textbf{Действия} — произвольные команды, выполняемые в~оболочке \code{sh} для~создания выходных файлов.

			\vspace{0.5ex}
			\textbf{Синтаксис:}
			\lstinputlisting[language=make,escapechar=!]{code-mk-syntax-3}
		\end{itemize}
	}

	\frame{
		\frametitle{Пример Makefile для построения программы на C}

		\lstinputlisting[language=make,escapechar=!]{code-mk-c}
	}

	\frame{
		\frametitle{Дополнительные возможности make}

		\begin{itemize}
			\item
			\textbf{Правила на основе суффиксов} \engterm{suffix rules} — шаблон правил для~построения 
			определенного типа файлов (напр., объектных файлов \code{*.o} на~основе исходных файлов \code{*.c}).

			\vspace{0.5ex}
			\item
			\textbf{Шаблонные правила} \engterm{pattern rules} — обобщение правил на~основе суффиксов
			с~шаблонным видом выходных файлов.

			\vspace{0.5ex}
			\item
			Использование \textbf{подстановочных знаков} (* и ?) в~списках зависимостей.

			\vspace{0.5ex}
			\item
			\textbf{Функции} для преобразования макросов.

			\vspace{0.5ex}
			\item
			\textbf{Директивы ветвления} для условного выполнения команд или~объявления макросов.

			\vspace{0.5ex}
			\item
			\textbf{Интерпретация кода}, напр., для~динамического добавления правил.
		\end{itemize}
	}

	\frame{
		\frametitle{Построение с помощью make}

		\begin{center}
			\code{\only<3>{autoconf \&\& }\only<2->{./configure \&\& }make \&\& [sudo] make install}
		\end{center}

		\begin{itemize}
			\only<3>{%
				\item
				\textbf{autoconf:} Создает сценарий конфигурации configure на~основе списка проверок и~т.\,п.; 
				генерирует сценарии Makefile с~помощью шаблонов.
			}
			\only<2->{%
				\item
				\textbf{configure:} пределяет конфигурацию построения и~создает файлы, включаемые в~основной Makefile.
			}
			\item
			\textbf{make:} компилирует и компонует составляющие приложения;
			\item
			\textbf{make install:} устанавливает полученные выполняемые файлы / библиотеки в~конечный пункт назначения.
		\end{itemize}

		\only<1>{%
			\vspace{1ex}
			\textbf{Проблема:} Определение конфигурации построения (используемых компиляторов, архитектуры целевой системы, …) 
			средствами make невозможно или~чрезвычайно сложно.
		}
		\only<2>{%
			\textbf{Проблема:} Создание сценария конфигурации вручную занимает много усилий.
		}
	}

	\subsection{Ant}

	\frame{
		\frametitle{Apache Ant}

		\begin{Definition}
			\textbf{Apache Ant} — утилита для автоматического построения проектов, 
			написанных на~языке программирования Java.
		\end{Definition}

		\vspace{1ex}
		\textbf{Поддерживаемые ОС:} произвольные (требуется Java Runtime).

		\vspace{1ex}
		\textbf{Достоинства:}
		\begin{itemize}
			\item портируемость (базовые задания не зависят от операционной системы);
			\item широкие возможности по фильтрации файлов для передачи в команды;
			\item интеграция с системой тестирования JUnit.
		\end{itemize}

		\vspace{0.5ex}
		\textbf{Недостатки:} \\
		ограниченность встроенных команд (для расширения требуются внешние плагины).
	}

	\frame{
		\frametitle{Формат файлов конфигурации Ant}

		\begin{itemize}
			\item
			\textbf{Переменные} (могут подставляться в аргументы команд).

			\vspace{0.5ex}
			\textbf{Синтаксис:} \code{<property name="имя" value="значение" />}

			\vspace{1ex}
			\item
			\textbf{Цели построения.}

			\vspace{0.5ex}
			\textbf{Синтаксис:} \\
			\code{%
				<target name="имя" description="описание" \\
				\hspace{4em} depends="зависимости (цели построения)"> \\
				\hspace{2em} команды \\
				</target>}

			\vspace{1ex}
			\item
			\textbf{Команды.}

			\vspace{0.5ex}
			\textbf{Синтаксис:} XML-тег с атрибутами и / или внутренними элементами, зависящими от~типа~команды.
		\end{itemize}
	}

	\frame{
		\frametitle{Пример файла конфигурации Ant}

		\lstinputlisting[language=Ant,escapechar=\#,firstline=2]{code-build.xml}
	}

	\section[CI]{Непрерывная интеграция}

	\frame{
		\frametitle{Непрерывная интеграция}

		\begin{Definition}
			\textbf{Непрерывная интеграция} \engterm{continuous integration, CI} — метод разработки~ПО, 
			основанный на~частой (несколько раз в~день) фиксации всех~изменений, вносимых в~систему разработчиками.
		\end{Definition}

		\vspace{1ex}
		\textbf{Автор:} Гради Буч (Grady Booch).

		\vspace{1ex}
		\textbf{Связанные технологии:}
		\begin{itemize}
			\item экстремальное программирование \engterm{extreme programming, XP}; 
			\item разработка через тестирование \engterm{test-driven development, TDD}.
		\end{itemize}

		\vspace{1ex}
		\textbf{Цели:}
		\begin{itemize}
			\item
			предотвращение проблем интеграции между компонентами программной системы;
			\item
			проверка изменений с помощью систем автоматического тестирования.
		\end{itemize}
	}

	\subsection{Принципы}

	\frame{
		\frametitle{Принципы непрерывной интеграции}

		\begin{itemize}
			\item
			\textbf{Система управления версиями}, охватывающая все артефакты, необходимые для~построения проекта;

			\vspace{0.5ex}
			\item
			\textbf{инструменты построения} для автоматического создания промежуточных выпусков;

			\vspace{0.5ex}
			\item
			\textbf{автоматическое тестирование} системы при построении;

			\vspace{0.5ex}
			\item
			\textbf{регулярное фиксирование изменений} (не реже раза в день) с~построением проекта 
			после~фиксирования;

			\vspace{0.5ex}
			\item
			\textbf{выделенный сервер} для построения / тестирования (копия целевой системы);

			\vspace{0.5ex}
			\item
			\textbf{прозрачность и доступность результатов построения} для~разработчиков и~заказчиков.
		\end{itemize}
	}

	\frame{
		\frametitle{Разработка согласно CI}

		\begin{figure}
			\begin{tikz*}[%
	every node/.style={align=center},
	edge/.style={font=\footnotesize\itshape},
	label/.style={font=\bfseries},
	line/.style={minimum height=1em,draw=blue,fill=blue!50,minimum width=#1},
	point/.style={circle,fill=black,minimum size=0.333em,inner sep=0pt}
]
	\node(label-dev) [label] {Среда разработки};
	\node(dev) [right=of label-dev,line=7.5em] {};
	\node [edge,above=0pt of dev] {построение, разработка, \\ тестирование};
	\node(rbound) [coordinate,right=15em of dev.east] {};
	\draw (label-dev.east) -- (dev.west);
	\draw (dev.east) -- (rbound);

	\node(label-build) [label,below=5em of label-dev.south east,anchor=north east] {Сервер построения};
	\node(build) [right=12.5em of label-build,line=7.5em] {};
	\node [edge,below=0pt of build] {построение, \\ тестирование};
	\draw (label-build.east) -- (build.west);
	\draw (build.east) -- (build.east -| rbound);

	\node(label-repo) [label,below=5em of label-build.south east,anchor=north east] {Хранилище версий}; 
	\draw (label-repo.east) -- (label-repo.east -| rbound);

	\node(checkin-src) [point,right=1em of label-repo.east] {};
	\node(checkin-dst) [point,right=1em of label-dev.east] {};
	\draw[->] (checkin-src) -- node[pos=0.75,right,edge]{check-in} (checkin-dst);
	\node(build-src) [point,right=1em of dev.east] {};
	\node(build-dst) at (build-src |- build.west) [point] {};
	\draw[->] (build-src) -- node[right,edge]{отправка} (build-dst);
	\node(commit-src) [point,right=1em of build.east] {};
	\node(commit-dst) at (commit-src |- label-repo.east) [point] {};
	\draw[->] (commit-src) -- node[right,edge]{commit, merge} (commit-dst);
\end{tikz*}

			\caption{При разработке согласно CI после внесения изменений в~программную систему 
				и~успешного построения на~компьютере разработчика приложение строится на~тестовом сервере.}
		\end{figure}
	}

	\frame{
		\frametitle{Преимущества и недостатки CI}

		\textbf{Преимущества:}
		\begin{itemize}
			\item
			быстрая локализация дефектов интеграции;
			\item
			более равномерное распределение внесения изменений в систему;
			\item
			постоянная доступность актуальной рабочей версии программной системы для~тестирования, 
			демонстрирования или~выпуска;
			\item
			возможность быстрого отката дефектного кода без~существенной потери функциональности.
		\end{itemize}

		\vspace{1ex}
		\textbf{Недостатки:}
		\begin{itemize}
			\item
			необходимость устройства выделенного сервера тестирования с~параметрами, 
			сходными с~целевой системой;
			\item
			для больших систем построение может занимать много времени.
		\end{itemize}
	}

	\section{Выпуски}

	\frame{
		\frametitle{Выпуски ПО}

		\begin{Definition}
			\textbf{Выпуск ПО} \engterm{software release} — версия программной системы, 
			предназначенная для~использования вне~отдела разработки.
		\end{Definition}

		\vspace{1ex}
		\textbf{Аспекты управления выпусками:}
		\begin{itemize}
			\item
			идентифицируемость — определение версий исходных файлов, инструментов, среды, задействованных в~построении системы;
			\item
			повторяемость — возможность повторного построения системы ($\Rightarrow$ сохранение версий исходных файлов и~т.\,п.);
			\item
			согласованность — автоматизация создания выпусков и~их~составляющих (напр., программ установки~ПО).
		\end{itemize}
	}

	\subsection{Составляющие}

	\frame{
		\frametitle{Состав выпуска}

		\textbf{Обязательные элементы:}
		\begin{itemize}
			\item
			выполняемый код (двоичные выполняемые файлы, библиотеки, сценарии, …)
		\end{itemize}
		или
		\begin{itemize}
			\item
			исходные файлы для построения программы в~среде пользователя.
		\end{itemize}

		\vspace{1ex}
		\textbf{Необязательные элементы:}
		\begin{itemize}
			\item
			программа установки;
			\item
			файлы конфигурации;
			\item
			файлы данных (напр., локализация сообщений, используемых в программе);
			\item
			электронная и печатная документация.
		\end{itemize}
	}

	\frame{
		\frametitle{Жизненный цикл выпусков ПО}

		\textbf{Технические (внутренние) выпуски:}
		\begin{itemize}
			\item
			рабочий выпуск \engterm{development release, pre-alpha, nightly build} — для~использования разработчиками: 
			создания архитектуры, проектирования и~кодирования компонентов системы и~т.\,п.;
			\item
			альфа-выпуск \engterm{alpha release} — для тестирования по~методу белого ящика;
			\item
			бета-выпуск \engterm{beta release} — для тестирования по~методу черного ящика;
			\item
			кандидат \engterm{release candidate, RC} — бета-выпуск, в котором устранено большинство дефектов, 
			потенциально готовый для публичного использования.
		\end{itemize}

		\vspace{1ex}
		\textbf{Публичные выпуски:}
		\begin{itemize}
			\item
			RTM (release to manufacturing) — выпуск, предназначенный для распространения и~развертывания 
			на~системах потребителей.
		\end{itemize}
	}

	\subsection{Планирование}

	\frame{
		\frametitle{Планирование выпусков}

		\textbf{Факторы, влияющие на расписание выпусков:}
		\begin{itemize}
			\item
			обнаруженные дефекты, требующие исправление в~виде отдельного выпуска или~патча;
			\item
			изменение среды выполнения (напр., новая версия операционной системы 
			или~используемых внешних библиотек);
			\item
			законы эволюции ПО:
			\begin{itemize}
				\item необходимость добавления новой функциональности (в~т.\,ч. для~успешной конкуренции);
				\item исправление ошибок, связанных с новыми функциями;
			\end{itemize}
			\item
			расписание, составленное отделом маркетинга;
			\item
			требования заказчика.
		\end{itemize}
	}

	\subsection{Нумерация версий}

	\frame{
		\frametitle{Нумерация версий ПО}

		\textbf{Цели:}
		\begin{itemize}
			\item
			идентификация выпусков;
			\item
			определение совместимости API различных выпусков;
			\item
			определение совместимости между программами и зависимостей в репозиториях приложений;
			\item
			маркетинг.
		\end{itemize}

		\vspace{1ex}
		\textbf{Схемы нумерации:}
		\begin{itemize}
			\item
			целые числа (1, 2, 3, …);
			\item
			десятичные дроби (1.0, 1.01, 1.1, 1.2, …);
			\item
			последовательность чисел (1.0.0, 1.0.1, 1.0.2, 1.1.0, …);
			\item
			даты;
			\item
			произвольные строки.
		\end{itemize}
	}

	\frame{
		\frametitle{Семантическая нумерация}

		\textbf{Формат версии:} \quad\code{major.minor[.patch[-suffix]]}

		\vspace{1ex}
		\begin{itemize}
			\item
			\textbf{major} — главная версия, целое неотрицательное число, увеличивающееся при~внесении значительных изменений 
			(напр., несовместимых модификаций API);

			\vspace{0.5ex}
			\textbf{major = 0} означает бета-выпуск;

			\vspace{1ex}
			\item
			\textbf{minor} — второстепенная версия, целое неотрицательное число, 
			увеличивающееся при~внесении обратно совместимых изменений;

			\vspace{1ex}
			\item
			\textbf{patch} — версия построения, целое неотрицательное число, увеличивающееся при~каждом построении системы 
			(напр., при~исправлении ошибок). По~умолчанию равна~нулю.

			\vspace{1ex}
			\item
			\textbf{suffix} — алфавитно-цифровая последовательность для~обозначения внутренних выпусков, напр., \code{alpha}, \code{rc.1}.
		\end{itemize}
	}

	\frame{
		\frametitle{Пример нумерации версий ПО}

		\begin{figure}
			\begin{tikz*}[%
	every node/.style={align=center},
	label/.style={font=\bfseries},
	point/.style={circle,fill=black,minimum size=0.333em,inner sep=0pt},
	ver/.style={rectangle,draw,minimum height=2em}
]
	\node(label-beta) [label] {Бета-выпуски};
	\node(label-pub) [label,below=5em of label-beta.south east,anchor=north east] {Публичные \\ выпуски};
	\node(label-tech) [label,below=5em of label-pub.south east,anchor=north east] {Технические \\ выпуски};

	\node(01) [ver,right=1em of label-beta] {0.1.0};
	\node(02) [ver,right=1.5em of 01] {0.2.0};
	\node(09) [ver,right=3em of 02] {0.9.0};
	\node(010) [ver,right=1.5em of 09] {0.10.0};
	\node(011) [ver,right=1.5em of 010] {0.11.0};
	\node(beta-src) [point,right=1em of 011] {};

	\draw[->] (01) -- (02);
	\draw[->,dashed] (02) -- (09);
	\draw[->] (09) -- (010);
	\draw[->] (010) -- (011);
	\draw[->] (011) -- (beta-src);

	\node(beta-dst) [point,right=1em of label-pub] {};
	\draw[->] (beta-src) |- ($ (beta-dst.center) + (0,3.5em) $) -- (beta-dst);

	\node(tech-src) [point,right=1em of beta-dst] {};
	\node(tech-dst) at (tech-src |- label-tech) [point] {};
	\node(pub-dst) [point,right=1em of tech-src] {};

	\node(100) [ver,right=1em of pub-dst] {1.0.0};
	\node(101) [ver,right=1.5em of 100] {1.0.1};
	\node(102) [ver,right=1.5em of 101] {1.0.2};
	\node(110) [ver,right=1.5em of 102] {1.1.0};
	\node(111) [ver,right=1.5em of 110] {1.1.1};
	\node(200) [ver,right=3em of 111] {2.0.0};

	\draw[->] (beta-dst) -- (tech-src);
	\draw[->] (tech-src) -- (tech-dst);
	\draw[->] (pub-dst) -- (100);
	\draw[->] (100) -- (101);
	\draw[->] (101) -- (102);
	\draw[->] (102) -- (110);
	\draw[->] (110) -- (111);
	\draw[->,dashed] (111) -- (200);

	\node(100-alpha) [ver,right=1em of tech-dst] {1.0.0-alpha};
	\node(100-beta) [ver,right=1.5em of 100-alpha] {1.0.0-beta\strut{}};
	\node(100-rc1) [ver,right=1.5em of 100-beta] {1.0.0-rc.1\strut{}};
	\node(100-rc2) [ver,right=1.5em of 100-rc1] {1.0.0-rc.2\strut{}};
	\node(pub-src) [point,right=1em of 100-rc2] {};
	\draw[->] (tech-dst) -- (100-alpha);
	\draw[->] (100-alpha) -- (100-beta);
	\draw[->] (100-beta) -- (100-rc1);
	\draw[->] (100-rc1) -- (100-rc2);
	\draw[->] (100-rc2) -- (pub-src);
	\draw[->] (pub-src) |- ($ (pub-dst.center) + (0,-3.5em) $) -- (pub-dst);
\end{tikz*}

			\caption{%
				\textbf{Примечание.} Технические выпуски создаются для~каждого или~почти~каждого 
				публичного выпуска~ПО; они~не~показаны на~схеме из-за~нехватки места.}
		\end{figure}
	}

	\section{Заключение}

	\subsection{Выводы}
	
	\frame{
		\frametitle{Выводы}

		\begin{enumerate}
			\item
			Управление версиями ПО связанно с~двумя другими аспектами управления конфигурацией — 
			построением программной системы и~управлением выпусками.

			\vspace{0.5ex}
			\item
			Инструменты автоматического построения~ПО различаются областью применения. 
			Наиболее универсальная утилита построения — make — может выполнять построение произвольных проектов. 
			Для~Java-проектов одним из~популярных средств построения является Apache Ant.

			\vspace{0.5ex}
			\item
			Автоматическое построение и управление версиями используются в~непрерывной интеграции (\emph{continuous integration}) 
			для~устранения ошибок интеграции при~коллективной разработке.

			\vspace{0.5ex}
			\item
			Выпуски ПО — это версии программной системы, предназначенные для~тестирования (технические выпуски) 
			или~использования потребителями (публичные выпуски). 
			Для~идентификации выпусков используются различные способы нумерации версий, 
			напр., семантическая нумерация (\emph{semantic versioning}).
		\end{enumerate}
	}
	
	\subsection{Материалы}
	
	\frame{
		\frametitle{Материалы}
		
		\begin{thebibliography}{9}
			\bibitem[1]{1}
			Sommerville, Ian
			\newblock Software Engineering.
			\newblock {\footnotesize Pearson, 2011. — 790 p.}

			\bibitem[2]{2}
			GNU
			\newblock GNU Make Manual.
			\newblock {\footnotesize\url{https://www.gnu.org/software/make/manual/make.html}}

			\bibitem[3]{3}
			Fowler, Martin
			\newblock Continuous Integration.
			\newblock {\footnotesize\url{http://martinfowler.com/articles/continuousIntegration.html}}

			\bibitem[4]{4}
			Preston-Werner, Tom
			\newblock Semantic versioning.
			\newblock {\footnotesize\url{http://semver.org/}}
		\end{thebibliography}
	}
	
	\frame{
		\frametitle{}
		
		\begin{center}
			\Huge Спасибо за внимание!
		\end{center}
	}

\end{document}

