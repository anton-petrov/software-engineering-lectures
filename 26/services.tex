\documentclass{a4beamer}
%% Lectures - common definitions

\usextensions{tikz}
\usetikzlibrary{shapes.multipart,shapes.callouts,shapes.geometric}
\input{fix-callouts.inc} % Fixes absolute positioning of rectangle callouts

\newif\ifbigpages \bigpagesfalse
\ifdim\paperwidth >20cm
	\bigpagestrue
\fi

\tikzset{%
	note/.style={rectangle callout,draw=none,callout pointer width=1em,%
		align=flush left,font=\footnotesize,inner sep=0.5em,%
		fill=blue!15,fill opacity=0.95,text opacity=1.0,callout absolute pointer=#1},
	node distance=2em and 2.75em
}
\ifbigpages
	% Scale all arrow tips by the factor of 2.5
	\let\old@pgf@arrow@call=\pgf@arrow@call
	\def\pgf@arrow@call#1{%
		\@tempdima=\pgflinewidth%
		\pgfsetlinewidth{2.5\pgflinewidth}%
		\old@pgf@arrow@call{#1}%
		\pgfsetlinewidth{\@tempdima}%
	}
	\def\pgfarrowsleftextend#1{\pgfmathsetlength{\pgf@xa}{1.5*#1}}
	\def\pgfarrowsrightextend#1{\pgfmathsetlength{\pgf@xb}{1.5*#1}}
\fi

%% Load listings package
\usepackage{listings}

%% Are we inside a comment?
\newif\iflstcomment \lstcommentfalse

\lstset{%
	tabsize=4,
	showstringspaces=false,
	basicstyle=\linespread{1.25}\ttfamily\small,
	keywordstyle=\bfseries,
	commentstyle=\lstcommentstyle,
	numbers=left,
	numberstyle=\footnotesize\color{gray},
	xleftmargin=2.5em,
	extendedchars=true,
	escapechar=\$,
	escapebegin=\iflstcomment\begingroup\lstcommentstyle\fi,
	escapeend=\iflstcomment\endgroup\fi
}

\def\lstcommentstyle{\color{gray}}

\lst@AddToHook{AfterBeginComment}{\global\lstcommenttrue}
\let\orig@lst@EndComment=\lst@EndComment
\def\lst@EndComment{\global\lstcommentfalse\orig@lst@EndComment}
\lst@AddToHookAtTop{EOL}{%
	\lst@ifLmode\global\lstcommentfalse\fi% XXX Sloppy way to determine comment end
}

%% Python with docstrings treated as comments
\lstdefinelanguage[doc]{python}[]{python}{%
	deletestring=[s]{"""}{"""},%
	morecomment=[s]{"""}{"""}%
}%

%% JavaScript language
\lstdefinelanguage{javascript}%
	{morekeywords={break,case,catch,%
		const,constructor,continue,default,do,else,false,%
		finally,for,function,if,in,instanceof,%
		new,null,prototype,%
		return,switch,this,throw,%
		true,try,typeof,var,while},%
	sensitive,%
	morecomment=[l]//,%
	morecomment=[s]{/*}{*/},%
	morestring=[b]",%
	morestring=[b]',%
}[keywords,comments,strings]%

%% C# language (4.0?)
\lstdefinelanguage{csharp}%
	{morekeywords={abstract,as,%
		base,bool,byte,case,catch,char,%
		checked,class,const,continue,%
		decimal,default,delegate,do,double,%
		else,enum,event,explicit,extern,%
		false,finally,fixed,float,for,foreach,%
		goto,if,implicit,in,int,interface,%
		internal,is,lock,long,%
		namespace,new,null,object,operator,out,%
		override,params,private,protected,public,%
		readonly,ref,return,sbyte,sealed,%
		short,sizeof,stackalloc,static,string,%
		struct,switch,this,throw,true,try,%
		typeof,uint,ulong,unchecked,unsafe,ushort,%
		using,virtual,void,volatile,while%
	},%
	sensitive,%
	morecomment=[l]//,%
	morecomment=[s]{/*}{*/},%
	morestring=[b]",%
	morestring=[b]',%
}[keywords,comments,strings]%

%% Translation for fact environment
\deftranslation[to=russian]{Fact}{Наблюдение}

%% Inline code snippets
\def\code#1{\texttt{#1}}
\def\codekw#1{\code{\textbf{#1}}}

\def\quoteauthor#1{\par\footnotesize\upshape\hfill—~#1}

%% English term
\def\engterm#1{(англ. \textit{#1})}
%% Term with explanation below (to be used in diagrams)
\def\termwithexpl#1#2{#1\strut{}\\\small\color{gray}(\textit{#2})\strut{}}
%% External link
\def\extlink#1#2{\href{#1}{\color[rgb]{0.7,0.7,1.0}\dashbar{#2}}}
%% Internal link
\def\inlink#1#2{\hyperlink{#1}{\color[rgb]{0.7,0.7,1.0}\dashbar{#2}}}
%% Explanation for a list item
\def\itemexpl#1{\begingroup\small\vspace{0.75ex}#1\par\endgroup}




\lecturetitle{Программная инженерия. Лекция №26 — Веб-сервисы}
\title[Веб-сервисы]{Сервисная архитектура приложений. Веб-сервисы}
\author{Алексей Островский}
\institute{\small{Физико-технический учебно-научный центр НАН Украины}\vspace{2ex}}
\date{21 мая 2015 г.}

\begin{document}
	\frame{\titlepage}

	\section{Введение}

	\subsection{SOA}

	\frame{
		\frametitle{Сервис-ориентированная архитектура}

		\begin{Definition}
			\textbf{Сервис-ориентированная архитектура} \engterm{service-oriented architecture, SOA} — 
			парадигма программирования, в которой для обеспечения модульности применяются 
			распределенные слабо связанные компоненты (\emph{сервисы}), взаимодействующие с~помощью стандартизованных протоколов.
		\end{Definition}

		\vspace{1ex}
		\textbf{Характеристики сервисов:}
		\begin{itemize}
			\item
			модульность — сервис представляет логически связанные функции 
			в~определенной предметной области с~заданными входами и~выходами;

			\item
			автономность — отсутствие наблюдаемых для пользователей зависимостей;

			\item
			сокрытие реализации — рассматривается как «черный ящик».
		\end{itemize}
	}

	\frame{
		\frametitle{Преимущества и недостатки SOA}

		\textbf{Достоинства:}
		\begin{itemize}
			\item
			открытость, стандартизация протоколов доступа;
			\item
			поддержка параллелизма, масштабируемость (напр., за счет прозрачных для~клиента балансировщиков нагрузки);
			\item
			отказоустойчивость.
		\end{itemize}

		\vspace{1ex}
		\textbf{Недостатки:}
		\begin{itemize}
			\item
			зависимость от состояния сетевых соединений;
			\item
			дополнительные вычислительные ресурсы, ПО и~затраты для~поддержки масштабирования;
			\item
			проблемы обеспечения безопасности данных, качества обслуживания и~т.\,п.
		\end{itemize}
	}

	\subsection{Веб-сервисы}

	\frame{
		\frametitle{Веб-сервисы}

		\begin{Definition}
			\textbf{Веб-сервис} \engterm{web service} — программная система с~возможностью взаимодействия 
			с~другими программами через сеть, обладающая заданным интерфейсом и~протоколом сообщений 
			для~обмена данными.
		\end{Definition}

		\vspace{1ex}
		\textbf{Характеристики веб-сервиса:}
		\begin{itemize}
			\item
			\textbf{интерфейс} веб-сервиса ($\simeq$ интерфейс компонента): 
			определяемые операции, типы входных и~выходных данных;

			\item
			\textbf{формат спецификации интерфейса:} 
			на~основе формального представления (языка спецификации) или~неформального описания;

			\item
			используемый \textbf{протокол передачи данных} (HTTP, UDP, …);

			\item
			\textbf{формат представления данных:} на основе XML, JSON, простого текста, …
		\end{itemize}
	}

	\frame{
		\frametitle{Разработка веб-сервисов}

		\textbf{Цели разработки:}
		\begin{itemize}
			\item минимизация количества обращений к сервису;
			\item скрытие состояния сервиса (хранение состояния — задача клиента; 
			состояние может передаваться в сообщениях).
		\end{itemize}

		\vspace{1ex}
		\textbf{Этапы разработки:}
		\begin{enumerate}
			\item определение функциональности; 
			\item описание операций и сообщений; 
			\item имплементация; 
			\item тестирование; 
			\item развертывание.
		\end{enumerate}
	}

	\frame{
		\frametitle{Классификация веб-сервисов}

		\textbf{Типы веб-сервисов:}
		\begin{itemize}
			\item
			\textbf{Утилитарные} — реализующие функциональность общего назначения, 
			которая может использоваться в~различных предметных областях другими сервисами.

			\vspace{0.5ex}
			\textbf{Пример:} конвертер валюты.

			\item
			\textbf{Бизнес-сервисы} — реализующие функциональность, специфичную для~предметной области.

			\vspace{0.5ex}
			\textbf{Пример:} вычисление кредитного рейтинга.

			\item
			\textbf{Координационные} — комплексные бизнес-процессы, 
			зачастую реализуемые с~помощью более~простых веб-сервисов.

			\vspace{0.5ex}
			\textbf{Пример:} управление магазином (прием заказов, инвентаризация, оплата, …).
		\end{itemize}

		\vspace{0.5ex}
		\textbf{Ориентация веб-сервисов:}
		\begin{itemize}
			\item
			сущности — поведение, аналогичное объектам в~ООП (напр., манипуляции с~БД);
			\item
			задания — выполнение действий без привязки к сущностям предметной области.
		\end{itemize}
	}

	\section{SOAP-сервисы}

	\frame{
		\frametitle{SOAP-сервисы}

		\begin{Definition}
			\textbf{Веб-сервис} в узком смысле, \textbf{SOAP-сервис} — веб-сервис, в~котором спецификация интерфейса 
			и~передача данных определены стандартами W3C.
		\end{Definition}

		\begin{center}
			\begin{tikz*}[%
	every node/.style={align=center},
	label/.style={font=\small},
	item/.style={ellipse,draw,minimum height=3em}
]
	\node(consumer) [item] {Потребитель};
	\node(registry) [item,above right=5em and 6em of consumer] {Реестр};
	\node(provider) [item,right=12em of consumer] {Поставщик};
	\node(service) [item,rectangle,below=4em of provider] {Сервис};

	\draw[<->] (consumer) -- node[label,above left] {поиск (UDDI)} (registry);
	\draw[<->] (consumer) -- node[label,below] {связывание (SOAP)} (provider);
	\draw[<->] (registry) -- node[label,above right] {публикация} (provider);
	\draw[->] (service) -- node[label,left] {спецификация (WSDL)} (provider);
\end{tikz*}


			\vspace{1ex}
			\figureexpl{Схема взаимодействия с веб-сервисом}
		\end{center}
	}

	\frame{
		\frametitle{Стандарты SOAP-сервисов}

		\textbf{Основные стандарты:}
		\begin{itemize}
			\item
			\textbf{SOAP} — протокол передачи данных для~вызова операций, определенных интерфейсом сервиса;

			\item
			\textbf{WSDL} — стандарт для~определения интерфейса сервиса;

			\item
			\textbf{UDDI} (universal description, discovery, and integration) — стандарт для~обнаружения 
			активных сервисов в~сети (расположение WSDL-описания интерфейса и~т.\,п.);

			\item
			\textbf{WS-BPEL} — стандарт для высокоуровневого описания программ, использующих веб-сервисы.
		\end{itemize}

		\vspace{1ex}
		\textbf{Вспомогательные стандарты:} 
		\begin{itemize}
			\item защита данных (WS-Security); 
			\item транзакции в распределенных сервисах (WS-Transactions);
			\item контроль передачи сообщений (WS-Reliable Messaging), …
		\end{itemize}
	}

	\subsection{WSDL}

	\frame{
		\frametitle{WSDL}

		\begin{Definition}
			\textbf{WSDL} (web service description language) — язык спецификации интерфейса веб-сервисов, 
			использующий XML.
		\end{Definition}

		\vspace{1ex}
		\textbf{Содержимое спецификации:}
		\begin{itemize}
			\item
			Операции, предоставляемые сервисом ($\simeq$ методы в ООП), соответствующие входные и~возвращаемые данные;
			\item
			формат сообщений для взаимодействия с сервисом;
			\item
			(необязательно) типы данных, используемые в сообщениях;
			\item
			определение конкретных протоколов доступа к операциям (с помощью SOAP или~других методов).
		\end{itemize}
	}

	\frame{
		\frametitle{Понятия WSDL 2.0}

		\begin{itemize}
			\item
			\textbf{Интерфейс} — набор операций для веб-сервиса.

			\vspace{0.5ex}
			\item
			\textbf{Операция} — определение способа обращения к~сервису; 
			$\sim$~вызов функции или~метода в~ЯП.

			\vspace{0.5ex}
			\item
			\textbf{Типы данных} — определения используемой структуры входных / выходных сообщений 
			для~операций с~помощью XML Schema.

			\vspace{0.5ex}
			\item
			\textbf{Привязка} \engterm{binding} — спецификация способа доступа к~определенному интерфейсу, 
			в~частности, протокол связи.

			\vspace{0.5ex}
			\item
			\textbf{Конечная точка} \engterm{endpoint} — адрес доступа к~веб-сервису 
			(чаще всего — простой HTTP-адрес), соответствующий некоторой привязке.

			\vspace{0.5ex}
			\item
			\textbf{Сервис} — набор конечных точек, обладающих общим интерфейсом.
		\end{itemize}
	}

	\frame{
		\frametitle{Структура файлов WSDL 2.0}

		\lstinputlisting[language=xml]{code-wsdl-generic.xml}
	}

	\frame{
		\frametitle{Пример: типы данных в WSDL 2.0}

		\lstinputlisting[language=java]{code-wsdl-interface.java}

		\lstinputlisting[language=xml]{code-wsdl-types.xml}
	}

	\frame{
		\frametitle{Пример: описание сервиса в WSDL 2.0}

		\lstinputlisting[language=xml]{code-wsdl-service.xml}
	}

	\subsection{SOAP}

	\frame{
		\frametitle{SOAP}

		\begin{Definition}
			\textbf{SOAP} (simple object access protocol) — протокол для обмена структурированными данными 
			с~веб-сервисами через~сеть (напр., поверх HTTP-соединения).
		\end{Definition}

		\vspace{1ex}
		\textbf{Сообщение сервису:}
		\begin{itemize}
			\item
			заголовок сообщения — нефункциональные характеристики запроса (приоритетность, время обработки, …);
			\item
			тело сообщения — список операций веб-сервиса и соответствующих параметров.
		\end{itemize}

		\vspace{1ex}
		\textbf{Ответное сообщение:}
		\begin{itemize}
			\item
			тело сообщения — список с результатами выполнения операций;
			\item
			отказы — информация об отказах при проведении операций.
		\end{itemize}
	}

	\frame{
		\frametitle{Пример SOAP-сообщения}

		\textbf{Запрос:}
		\vspace{-0.5ex}
		\lstinputlisting[language=xml]{code-soap-request.xml}

		\textbf{Ответ:}
		\vspace{-0.5ex}
		\lstinputlisting[language=xml]{code-soap-response.xml}
	}

	\subsection{BPEL}

	\frame{
		\frametitle{BPEL}

		\begin{Definition}
			\textbf{WS-BPEL} (web service business process execution language) — язык на~основе XML 
			для~описания бизнес-процессов, координирующих веб-сервисы.
		\end{Definition}

		\vspace{1ex}
		\textbf{Базовые инструкции:}
		\begin{itemize}
			\item
			ветвление (\codekw{if} — \codekw{elseif} — \codekw{else});
			\item
			цикл (\codekw{while});
			\item
			последовательность действий (\codekw{sequence});
			\item
			параллельные действия (\codekw{flow}).
		\end{itemize}

		\vspace{1ex}
		\textbf{Возможности:}
		\begin{itemize}
			\item
			обмен данными с веб-сервисами, извлечение сведений из~ответов с~помощью XPath;
			\item
			синхронизация параллельных действий;
			\item
			обработка событий и исключительных ситуаций.
		\end{itemize}
	}

	\frame{
		\frametitle{Достоинства и недостатки BPEL}

		\textbf{Достоинства:}
		\begin{itemize}
			\item
			высокий уровень абстракции;
			\item
			не зависит от парадигмы программирования (ООП, структурное программирование, …);
			\item
			ориентация на специфичную для~веб-сервисов функциональность 
			(параллельные запросы, разбор данных, …);
		\end{itemize}

		\vspace{1ex}
		\textbf{Недостатки:}
		\begin{itemize}
			\item
			чрезмерная абстрактность побуждает к~созданию дополнений для~BPEL, 
			несовместимых между~собой (что~противоречит сути стандарта);
			\item
			отсутствие встроенной поддержки новых технологий (WSDL 2.0, REST-сервисов, …);
			\item
			централизованная модель управления.
		\end{itemize}
	}

	\subsection{Разработка}

	\frame{
		\frametitle{Разработка с SOAP-сервисами}

		\textbf{Способы разработки сервисов:}
		\begin{itemize}
			\item
			top-down — вначале разрабатывается WSDL-описание сервиса, затем на~его~основе~— реализация на~ЯП;

			\vspace{0.5ex}
			\item
			bottom-up — WSDL-описание генерируется на~основе готовых интерфейсов и~классов.
		\end{itemize}

		\vspace{1ex}
		\textbf{Вспомогательные инструменты:} 
		обработка поступающих запросов и~их~трансляция в~вызовы методов имплементации 
		(напр., при~помощи Apache~Axis в Java~EE).

		\vspace{1ex}
		\textbf{Разработка клиента:}
		\begin{itemize}
			\item
			автоматическая генерация интерфейса и~клиентского стаба 
			на~основе WSDL-описания сервиса;

			\vspace{0.5ex}
			\item
			клиентский стаб позволяет обращаться к~сервису как~к~локальному объекту.
		\end{itemize}
	}

	\frame{
		\frametitle{Пример: SOAP-сервис в JavaEE}

		\lstinputlisting[language=java]{code-jax-ws.java}
	}

	\section{REST}

	\frame{
		\frametitle{REST}

		\begin{Definition}
			\textbf{Передача репрезентативного состояния} \engterm{representational state transfer, REST} — 
			архитектура распределенных приложений, предназначенная для~создания масштабируемых веб-сервисов, 
			которая определяется как набор ограничений.
		\end{Definition}

		\begin{problem}
			Соответствуют ли операции сервиса методам одного объекта?
		\end{problem}

		Если да, то:
		\begin{itemize}
			\item
			как обрабатывать одновременные запросы?
			\item
			как масштабировать сервис?
		\end{itemize}

		\vspace{1ex}
		\textbf{Решение:} отсутствие состояния сервиса; 
		каждая операция выполняется независимо от~других (но~может модифицировать данные, 
		с~которыми работает сервис).
	}

	\subsection{Ограничения}

	\frame{
		\frametitle{REST}

		\textbf{Ограничения REST-архитектуры:}
		\begin{itemize}
			\item
			модель «клиент — сервер» для разделения ответственности;
			\item
			отсутствие хранимого состояния при~взаимодействии клиента и~сервера;
			\item
			кэшируемость запросов к веб-сервисам (согласно~спецификации HTTP-протокола);
			\item
			прозрачная многослойная архитектура (напр., для~подключения балансировщиков нагрузки);
			\item
			унифицированный интерфейс сервисов:
			\item
			доступ к ресурсам с помощью различных URI-адресов;
			\item
			режим обработки возвращенных данных определяется в ответе (напр., как~спецификатор MIME).
		\end{itemize}
	}

	\frame{
		\frametitle{Стандарты в REST}

		\textbf{NB.} REST не определяет стандартов для взаимодействия, определения интерфейса и~т.\,п. 
		Веб-сервис на~основе~SOAP теоретически может удовлетворять ограничениям REST-архитектуры.

		\vspace{1ex}
		\textbf{Часто используемые стандарты:}
		\begin{center}
			\vspace{-0.5ex}
			\begin{tabular}{p{0.325\textwidth}p{0.55\textwidth}}
				\raggedright\textbf{Протокол передачи данных:} & HTTP \cr
				\hline
				\raggedright\textbf{Идентификация операции и~параметров:} & 
					\raggedright с помощью URI (\code{http://example.com/api/\alert{add/2,3}}) и~/~или~параметров~HTTP; \cr
				\hline
				\textbf{Возвращаемые данные:} & 
					\raggedright XML, JSON, plain text (может определяться в~запросе при~помощи параметра HTTP Accept); \cr
				\hline
				\raggedright\textbf{Спецификация интерфейса:} & 
					\raggedright неформальная, с~помощью документации на~API; 
					определение допустимых операций для~ресурсов согласно~\extlink{http://en.wikipedia.org/wiki/HATEOAS}{HATEOAS}. \cr
			\end{tabular}
		\end{center}
	}

	\subsection{Сравнение с SOAP-сервисами}

	\frame{
		\frametitle{Сравнение SOAP- и REST-сервисов}

		\textbf{Преимущества SOAP-сервисов:}
		\begin{itemize}
			\item
			стандартизация всех аспектов сервисов;
			\item
			наличие вспомогательных технологий (безопасность информации, транзакции, …).
		\end{itemize}

		\vspace{1ex}
		\textbf{Преимущества REST-сервисов:}
		\begin{itemize}
			\item
			отсутствие дополнительной нагрузки, связанной с~использованием 
			«тяжелых» протоколов (SOAP, WSDL);
			\item
			более высокая скорость разработки за~счет использования неявных соглашений 
			\engterm{convention over~configuration};
			\item
			легкость доступа и создания клиентов.
		\end{itemize}
	}

	\subsection{Пример реализации}

	\frame{
		\frametitle{HTTP-доступ к REST-сервисам}

		\begin{center}
			\begin{tabular}{|p{0.1\textwidth}|p{0.375\textwidth}|p{0.375\textwidth}|}
				\hline
				\centering\textbf{Метод}
					& \centering\textbf{Коллекция ресурсов} (напр., {\small\code{http://example.com/api/\alert{books}}})
					& \centering\textbf{Отдельный ресурс} (напр., {\small\code{http://example.com/api/\alert{books/1000}}}) \cr
				\hline
				GET & \raggedright получение списка ресурсов (возможно, с~доп.~информацией)
					& \raggedright получение представления запрашиваемого ресурса \cr
				\hline
				PUT & \raggedright замена коллекции целиком
					& \raggedright замена или (при~отсутствии) создание нового ресурса с~заданным URI \cr
				\hline
				POST & \raggedright создание нового ресурса в~коллекции
					& \raggedright не используется \cr
				\hline
				DELETE & \raggedright удаление коллекции целиком
					& \raggedright удаление запрашиваемого ресурса \cr
				\hline
			\end{tabular}
		\end{center}
	}

	\frame{
		\frametitle{Пример интерфейса REST-сервиса}

		\textbf{Сервис:} целочисленные последовательности (напр., числа Фибоначчи).

		\textbf{Базовый URL:} http://example.com/api/

		\vspace{1ex}
		\begin{itemize}
			\item	
			\textbf{GET http://example.com/api/}

			Возвращает список зарегистрированных целочисленных последовательностей в~формате~JSON.

			\vspace{1ex}
			\textbf{Запрос:} \code{GET http://example.com/api/}

			\textbf{Ответ:} \code{HTTP 200 OK; Content-Type: application/json}
			\lstinputlisting[language={}]{code-list.json}
		\end{itemize}
	}

	\frame{
		\frametitle{Пример интерфейса REST-сервиса (продолжение)}

		\begin{itemize}
			\item
			\textbf{GET http://example.com/api/<id>}

			Возвращает сведения о~последовательности~\code{<id>} в~формате~JSON. 

			\vspace{1ex}
			\textbf{Ошибки:}
			\begin{itemize}
				\item
				Если \code{<id>} не~является зарегистрированной последовательностью, 
				возвращается ошибка HTTP~404 с~сообщением в~формате text/plain.
			\end{itemize}

			\vspace{1ex}
			\textbf{Запрос:} \code{GET http://example.com/api/fib}

			\textbf{Ответ:} \code{HTTP 200 OK; Content-Type: application/json}
			\lstinputlisting[language={}]{code-seq.json}

			\textbf{Запрос:} \code{GET http://example.com/api/non-existent}

			\textbf{Ответ:} \code{HTTP 404 Not Found; Content-Type: text/plain} \\
			\code{Unknown integer sequence identifier: 'non-existent'}
		\end{itemize}
	}

	\frame{
		\frametitle{Пример интерфейса REST-сервиса (продолжение)}

		\begin{itemize}
			\item
			\textbf{GET http://example.com/api/<id>/<index>}

			Возвращает член последовательности~\code{<id>} в~формате~text/plain.

			\vspace{1ex}
			\textbf{Ошибки:}
			\begin{itemize}
				\item
				Если~\code{<id>} не~является зарегистрированной последовательностью, 
				возвращается ошибка HTTP~404 с~сообщением в~формате~text/plain. 

				\item
				Если~\code{<index>} не~является числом или~не~выполняются ограничения на~индекс, 
				возвращается ошибка HTTP~400 с~телом, содержащем описание ошибки в~формате~text/plain.
			\end{itemize}

			\vspace{1ex}
			\textbf{Запрос:} \code{GET http://example.com/api/fib/100}

			\textbf{Ответ:} \code{HTTP 200 OK; Content-Type: text/plain} \\
			\code{354224848179261915075}

			\vspace{1ex}
			\textbf{Запрос:} \code{GET http://example.com/api/fib/1000000000}

			\textbf{Ответ:} \code{HTTP 400 Bad Request; Content-Type: text/plain} \\
			\code{Index too large: 1000000000}
		\end{itemize}
	}

	\subsection{Разработка}

	\frame{
		\frametitle{Разработка с REST-сервисами}

		\textbf{Способы разработки сервисов:}
		\begin{itemize}
			\item
			как часть веб-приложений с использованием архитектуры MVC;

			\vspace{0.5ex}
			\item
			как составляющая модулей для сервера приложений 
			(напр., \extlink{http://docs.oracle.com/javaee/7/tutorial/jaxrs.htm}{JAX-RS} в~рамках Java~EE).
		\end{itemize}

		\vspace{1ex}
		\textbf{Способы разработки клиентов:}
		\begin{itemize}
			\item
			с помощью специализированных~API (напр., 
			\extlink{http://docs.oracle.com/javaee/7/tutorial/jaxrs-client.htm}{JAX-RS Client API} для~Java~EE);

			\vspace{0.5ex}
			\item
			при помощи API общего назначения для отправки и~обработки HTTP-запросов наподобие~\extlink{http://curl.haxx.se/}{libcurl} 
			+~средства для~сериализации / десериализации данных.
		\end{itemize}
	}

	\frame{
		\frametitle{Пример: REST-сервис в JavaEE}

		\lstinputlisting[language=java]{code-jax-rs.java}
	}

	\section{Заключение}

	\subsection{Выводы}
	
	\frame{
		\frametitle{Выводы}

		\begin{enumerate}
			\item
			Веб-сервисы — один из способов реализации компонентов в~распределенных приложениях. 
			Существуют два~типа~веб-сервисов: SOAP-сервисы и~REST-сервисы.

			\vspace{0.5ex}
			\item
			SOAP-сервисы используют стандарты доступа и~описания интерфейса, предложенные W3C, — SOAP и WSDL, соответственно. 
			SOAP-сервисы громоздки, зато~обладают дополнительными возможностями, отсутствующими в~REST (напр., адресация). 
			Для~создания композиций сервисов есть~язык~BPEL.

			\vspace{0.5ex}
			\item
			REST-сервисы используют для~передачи данных средства протокола HTTP (напр., кэширование и~определение типа содержимого).
			В~отличие от~SOAP-сервисов интерфейс REST-сервисов определяется неформально в~документации.
		\end{enumerate}
	}
	
	\subsection{Материалы}
	
	\frame{
		\frametitle{Материалы}
		
		\begin{thebibliography}{9}
			\bibitem[1]{1}
			Sommerville, Ian
			\newblock Software Engineering.
			\newblock {\footnotesize Pearson, 2011. — 790 p.}

			\bibitem[2]{2}
			Лавріщева К.\,М. 
			\newblock Програмна інженерія (підручник). 
			\newblock {\footnotesize К., 2008. — 319 с.}
		\end{thebibliography}
	}
	
	\frame{
		\frametitle{}
		
		\begin{center}
			\Huge Спасибо за внимание!
		\end{center}
	}

\end{document}

