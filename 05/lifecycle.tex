\documentclass{a4beamer}
%% Lectures - common definitions

\usextensions{tikz}
\usetikzlibrary{shapes.multipart,shapes.callouts,shapes.geometric}
\input{fix-callouts.inc} % Fixes absolute positioning of rectangle callouts

\newif\ifbigpages \bigpagesfalse
\ifdim\paperwidth >20cm
	\bigpagestrue
\fi

\tikzset{%
	note/.style={rectangle callout,draw=none,callout pointer width=1em,%
		align=flush left,font=\footnotesize,inner sep=0.5em,%
		fill=blue!15,fill opacity=0.95,text opacity=1.0,callout absolute pointer=#1},
	node distance=2em and 2.75em
}
\ifbigpages
	% Scale all arrow tips by the factor of 2.5
	\let\old@pgf@arrow@call=\pgf@arrow@call
	\def\pgf@arrow@call#1{%
		\@tempdima=\pgflinewidth%
		\pgfsetlinewidth{2.5\pgflinewidth}%
		\old@pgf@arrow@call{#1}%
		\pgfsetlinewidth{\@tempdima}%
	}
	\def\pgfarrowsleftextend#1{\pgfmathsetlength{\pgf@xa}{1.5*#1}}
	\def\pgfarrowsrightextend#1{\pgfmathsetlength{\pgf@xb}{1.5*#1}}
\fi

%% Load listings package
\usepackage{listings}

%% Are we inside a comment?
\newif\iflstcomment \lstcommentfalse

\lstset{%
	tabsize=4,
	showstringspaces=false,
	basicstyle=\linespread{1.25}\ttfamily\small,
	keywordstyle=\bfseries,
	commentstyle=\lstcommentstyle,
	numbers=left,
	numberstyle=\footnotesize\color{gray},
	xleftmargin=2.5em,
	extendedchars=true,
	escapechar=\$,
	escapebegin=\iflstcomment\begingroup\lstcommentstyle\fi,
	escapeend=\iflstcomment\endgroup\fi
}

\def\lstcommentstyle{\color{gray}}

\lst@AddToHook{AfterBeginComment}{\global\lstcommenttrue}
\let\orig@lst@EndComment=\lst@EndComment
\def\lst@EndComment{\global\lstcommentfalse\orig@lst@EndComment}
\lst@AddToHookAtTop{EOL}{%
	\lst@ifLmode\global\lstcommentfalse\fi% XXX Sloppy way to determine comment end
}

%% Python with docstrings treated as comments
\lstdefinelanguage[doc]{python}[]{python}{%
	deletestring=[s]{"""}{"""},%
	morecomment=[s]{"""}{"""}%
}%

%% JavaScript language
\lstdefinelanguage{javascript}%
	{morekeywords={break,case,catch,%
		const,constructor,continue,default,do,else,false,%
		finally,for,function,if,in,instanceof,%
		new,null,prototype,%
		return,switch,this,throw,%
		true,try,typeof,var,while},%
	sensitive,%
	morecomment=[l]//,%
	morecomment=[s]{/*}{*/},%
	morestring=[b]",%
	morestring=[b]',%
}[keywords,comments,strings]%

%% C# language (4.0?)
\lstdefinelanguage{csharp}%
	{morekeywords={abstract,as,%
		base,bool,byte,case,catch,char,%
		checked,class,const,continue,%
		decimal,default,delegate,do,double,%
		else,enum,event,explicit,extern,%
		false,finally,fixed,float,for,foreach,%
		goto,if,implicit,in,int,interface,%
		internal,is,lock,long,%
		namespace,new,null,object,operator,out,%
		override,params,private,protected,public,%
		readonly,ref,return,sbyte,sealed,%
		short,sizeof,stackalloc,static,string,%
		struct,switch,this,throw,true,try,%
		typeof,uint,ulong,unchecked,unsafe,ushort,%
		using,virtual,void,volatile,while%
	},%
	sensitive,%
	morecomment=[l]//,%
	morecomment=[s]{/*}{*/},%
	morestring=[b]",%
	morestring=[b]',%
}[keywords,comments,strings]%

%% Translation for fact environment
\deftranslation[to=russian]{Fact}{Наблюдение}

%% Inline code snippets
\def\code#1{\texttt{#1}}
\def\codekw#1{\code{\textbf{#1}}}

\def\quoteauthor#1{\par\footnotesize\upshape\hfill—~#1}

%% English term
\def\engterm#1{(англ. \textit{#1})}
%% Term with explanation below (to be used in diagrams)
\def\termwithexpl#1#2{#1\strut{}\\\small\color{gray}(\textit{#2})\strut{}}
%% External link
\def\extlink#1#2{\href{#1}{\color[rgb]{0.7,0.7,1.0}\dashbar{#2}}}
%% Internal link
\def\inlink#1#2{\hyperlink{#1}{\color[rgb]{0.7,0.7,1.0}\dashbar{#2}}}
%% Explanation for a list item
\def\itemexpl#1{\begingroup\small\vspace{0.75ex}#1\par\endgroup}



\lecturetitle{Программная инженерия. Лекция №5 — Стандарт и модели жизненного цикла.}

\title[Стандарт и модели жизненного цикла]{Стандарт и модели жизненного цикла}
\author{Алексей Островский}
\institute{\small{Физико-технический учебно-научный центр НАН Украины}\vspace{2ex}}
\date{17 октября 2014 г.}

\begin{document}
	\frame{\titlepage}
	
	\section{Жизненный цикл}
	
	\frame{
		\frametitle{Жизненный цикл}
		
		\begin{tikz*}[%
	every node/.style={rectangle,align=center,minimum height=3em},
	every label/.style={font=\small\itshape,minimum height=0pt},
	imp/.style={font=\small\bfseries\color{red}}
]
	\node(design) {Проектирование};
	\node(constr) [right=of design] {Конструирование};
	\node(testing) [below=of design] {Тестирование};
	\node(maint) [below=of constr] {Сопровождение};
	
	\node<2->(quest) at ($ (design)!0.5!(maint) $) {\LARGE\bfseries\color{red} ???};
	
	\draw[dotted] (design) to (constr);
	\draw[dotted] (design) to (testing);
	\draw[dotted] (design) to (maint);
	\draw[dotted] (constr) to (testing);
	\draw[dotted] (constr) to (maint);
	\draw[dotted] (testing) to (maint);
	
	\node(devel) [fit=(design.north west) (constr.north east) (maint.south east),draw,inner xsep=1em,label=below:Разработка] {};
	\node(contract) [left=of devel,label={[imp]below:\only<3->{???}}]{Договор \\ с заказчиком};
	\node(product) [right=of devel,label={[imp]below:\only<4->{???}}] {Готовый \\ продукт};
	
	\draw[->] (contract) to (devel);
	\draw[->] (devel) to (product);
\end{tikz*}

		\\[0.5ex]
		\figureexpl{\textbf{Жизненный цикл} — схема упорядочивания работ, касающихся проектирования и разработки программного продукта.}
		
		\vspace{1ex}
		\visible<2->{\textbf{Проблемы:}}
		\begin{enumerate}
			\item<2-> Как соотносятся между собой различные процессы разработки ПО?
			\item<3-> Каким образом организовано взаимодействие с заказчиком и конечными пользователями?
			\item<4-> Что считается конечным продуктом разработки?
		\end{enumerate}
	}
	
	\subsection{Стандарт ISO 12207}
	
	\frame{
		\frametitle{Стандарт ISO 12207}

		\textbf{Содержание стандарта:}
		\begin{itemize}
			\item 23 процесса разработки;
			\item 95 родов деятельности по разработке \engterm{activity};
			\item 325 заданий \engterm{task};
			\item 224 результатов выполнения процессов \engterm{outcome}.
		\end{itemize}

		\vspace{1ex}
		\textbf{NB.} Стандарт определяет \emph{составляющие} процессов разработки ПО, но~не~\emph{последовательность} их выполнения.
	}
	
	\subsection{Структура процессов ЖЦ}
	
	\frame{
		\frametitle{Структура процессов ЖЦ}
		
		\begin{tikz*}[%
	every node/.style={rectangle,draw,align=center,minimum height=3em},
	group/.style={},%{inner xsep=2em},
	hilight/.style={font=\only<#1>{\color{red}}},
]
	\node(task1) [hilight=2] {Задание$_{1}$};
	\node(task2) [right=0.5em of task1,hilight=2] {Задание$_{2}$};
	\node(task-dots) [right=0.5em of task2,draw=none] {$\cdots$};
	\node(act-label) [below=0.5em of task1,draw=none,minimum height=0pt,hilight=3] {Действие$_{1}$};
	\node(act1) [group,fit=(task1) (task-dots) (act-label)] {};
	
	\node(taskk) [right=2em of task-dots,hilight=2] {Задание$_{k}$};
	\node(taskk1) [right=0.5em of taskk,hilight=2] {Задание$_{k+1}$};
	\node(task-dots2) [right=0.5em of taskk1,draw=none] {$\cdots$};
	\node(act-label2) [below=0.5em of taskk,draw=none,minimum height=0pt,hilight=3] {Действие$_{2}$};
	\node(act2) [group,hilight=2,fit=(taskk) (task-dots2) (act-label2)] {};
	
	\node(act-dots) [right=1em of act2,draw=none] {$\cdots$};
	\node(process-label) [below=4em of $(act1.west)!0.5!(act-dots.east)$,minimum height=0pt,draw=none,hilight=4] {Процесс ЖЦ};
	\node(process) [group,fit=(act1) (act-dots) (process-label)] {};
	
	\node(input) [below left=of process-label] {Входные данные};
	\node(output) [below right=of process-label,hilight=5] {Итог процесса};
	\draw[->] (input.north) to (input.north |- process.south);
	\draw[<-] (output.north) to (output.north |- process.south);
\end{tikz*}

		
		\vspace{1ex}
		\begin{overlayarea}{\textwidth}{0.27\textheight}
			\only<2>{
				\begin{Definition}
					\textbf{Задание} \engterm{task} — требование, рекомендация или допустимое действие 
					для достижения определенного итога процесса.
				\end{Definition}
			}
			\only<3>{
				\begin{Definition}
					\textbf{Действие} \engterm{activity} — набор связанных заданий в пределах процесса.
				\end{Definition}
			}
			\only<4>{
				\begin{Definition}
					\textbf{Процесс} — набор взаимосвязанных действий, преобразующих поданную на вход информацию.
				\end{Definition}
			}
			\only<5>{
				\begin{Definition}
					\textbf{Итог процесса} \engterm{process outcome} — наблюдаемый результат достижения цели выполнения процесса 
					(программный артефакт, изменение состояния системы, выполнение требования и~т.\,п.).
				\end{Definition}
			}
		\end{overlayarea}
	}
	
	\subsection{Классификация процессов}
	
	\frame{
		\frametitle{Классификация процессов}
		
		\begin{overlayarea}{\textwidth}{0.7\textheight}
			\begingroup
	\makeatletter%
	\let\@orig@itemize\itemize
	\def\itemize{%
		\@orig@itemize%
		\let\@orig@item\item%
		\def\item{\vspace{-0.5ex}\@orig@item}%
	}
	\makeatother
	\begin{tikz*}[%
		every node/.style={rectangle,align=center,inner sep=0.5em}
	]
		\node(all) {\bfseries Процессы ЖЦ};
		\node(org) [below=of all] {
			Организационные \\ процессы
		};
		\node(main) [left=of org.north west,anchor=north east] {
			Основные \\ процессы
		};
		\node(aux) [right=of org.north east,anchor=north west] {
			Процессы \\ поддержки
		};
		\node(manage) [below left=2em and -2em of org] {
			Управление
		}; 
		\node(improve) [below right=2em and -2em of org] {
			Модификация \\ процессов
		};
		
		\draw (all) to (main.north);
		\draw (all) to (aux.north);
		\draw (all) to (org.north);
		\draw (org) to (manage.north);
		\draw (org) to (improve.north);
		
		\node<2> [note={(main.south)},below=4em of main.south,anchor=north west,text width=13em] {
			\begin{itemize}
				\item приобретение;
				\item поставка;
				\item разработка;
				\item эксплуатация;
				\item сопровождение.
			\end{itemize}
		};
		\node<3> [note={(aux.south)},below=4em of aux.south,anchor=north east,text width=13em] {
			\begin{itemize}
				\item документирование;
				\item конфигурация; 
				\item QA; 
				\item верификация; 
				\item валидация; 
				\item аудит; 
				\item оценка.
			\end{itemize}
		};
		\node<4> [note={(manage.south)},below=1em of manage.south,anchor=north west,text width=13em] {
			\begin{itemize}
				\item организацией; 
				\item проектом; 
				\item качеством;
				\item риском;
				\item знаниями;
				\item орг. обеспечение;
				\item измерение.
			\end{itemize}
		};
		\node<5> [note={(improve.south)},below=1em of improve.south,anchor=north east,text width=13em] {
			\begin{itemize}
				\item внедрение; 
				\item оценка;
				\item совершенствование.
			\end{itemize}
		};
	\end{tikz*}
\endgroup

		\end{overlayarea}
	}
	
	\frame{
		\frametitle{Основные процессы ЖЦ}

		\begin{enumerate}
			\item<1- | alertitem@1>
			\textbf{Приобретение} \engterm{acquisition} — начальный процесс ЖЦ, определяющий действия заказчика.

			\only<1>{
				\itemexpl{\textbf{Составляющие:} инициация и подготовка запроса на разработку; оформление и актуализация контракта; приемка ПО.}
			}

			\item<1- | alertitem@2>
			\textbf{Поставка} \engterm{supply} — совместные действия заказчика и разработчика по~составлению общего плана управления проектом 
			(\emph{project management plan}).

			\item<1- | alertitem@3>
			\textbf{Разработка} \engterm{development} — действия разработчика по созданию ПО.

			\only<3>{
				\itemexpl{\textbf{Составляющие:} анализ требований, создание дизайна системы и компонентов; 
				кодирование; модульное, интеграционное и системное тестирование ПО.}
			}

			\item<1- | alertitem@4>
			\textbf{Эксплуатация} \engterm{operation} — действия обслуживающей организации, обеспечивающей эксплуатацию системы конечными пользователями.

			\only<4>{
				\itemexpl{\textbf{Составляющие:} функциональное тестирование, проверка правильности эксплуатации; руководства по использованию.}
			}

			\item<1- | alertitem@5>
			\textbf{Сопровождение} \engterm{maintenance} — действия организации, сопровождающей продукт 
			(управление модификациями, поддержка функциональности, инсталляция и~т.\,п.). 

			\only<5>{
				\itemexpl{\textbf{Составляющие:} анализ вопросов сопровождения и модификации; разработка планов модификации; 
				миграция; вывод из эксплуатации.}
			}
		\end{enumerate}
	}
	
	\section[Модели ЖЦ]{Модели жизненного цикла}
	
	\frame{
		\frametitle{Модели жизненного цикла}

		\begin{Definition}
			\textbf{Модель жизненного цикла} — это схема выполнения работ и задач в рамках процессов, 
			обеспечивающая разработку, эксплуатацию и сопровождение программного продукта.
		\end{Definition}

		\vspace{1ex}
		\textbf{Составляющие} модели:
		\begin{itemize}
			\item разработка требований или технического задания;
			\item разработка эскизного или технического проекта;
			\item программирование и проектирование рабочего проекта;
			\item пробная эксплуатация;
			\item сопровождение и улучшение;
			\item снятие с эксплуатации.
		\end{itemize}
	}

	\frame{	
		\frametitle{Цели моделей ЖЦ}

		\begin{itemize}
			\item Планирование и распределение работ между разработчиками;
			\item управление проектом разработки;
			\item обеспечение взаимодействия между разработчиками и заказчиком;
			\item контроль работ, оценка промежуточных артефактов ЖЦ на соответствие требованиям; 
			\item оценка конечного продукта и затрат на его получение;
			\item согласование промежуточных результатов с заказчиком;
			\item проверка правильности конечного продукта (тестирование), оценка его соответствия требованиям;
			\item усовершенствование процессов ЖЦ по результатам разработки.
		\end{itemize}
	}
	
	\subsection{Классификация моделей ЖЦ}
	
	\frame{
		\frametitle{Классификация моделей ЖЦ}
		
		\begin{tikz*}[%
	every node/.style={rectangle,align=center,minimum height=3em}
]
	\node(all) {\bfseries Модели ЖЦ};
	\node(waterfall) [above left=of all] {Каскадная};
	\node(spiral) [above right=of all] {Спиральная};
	\node(incr) [below left=of all] {Инкрементная};
	\node(evo) [below right=of all] {Эволюционная};
	
	\node(agile) [below=of $(incr.south)!0.5!(evo.south)$] {Гибкая \\ методология}; 
	\node(rad) [below right=of spiral] {RAD};
	
	\draw (all) to (waterfall);
	\draw (all) to (incr);
	\draw (all) to (spiral);
	\draw (all) to (evo);
	\draw (incr) to (agile);
	\draw (evo) to (rad);
	\draw[dashed] (spiral) to (rad);
	\draw[dashed] (evo) to (agile);
\end{tikz*}

	}
	
	\subsection{Каскадная модель}
	
	\frame{
		\frametitle{Каскадная модель}

		\begin{tikz*}[%
	every node/.style={rectangle,align=center,minimum height=3.25em,minimum width=8.5em}
]
	\node(req) [draw] {Инженерия \\ требований};
	\node(req-doc) [right=5em of req] {Спецификация \\ требований};
	\node(design) [below=of req,draw] {Проектирование};
	\node(des-doc) at (req-doc.south |- design.east) {архитектурная / проектная \\ документация};
	\node(impl) [below=of design,draw] {Имплементация};
	\node(impl-doc) at (req-doc.south |- impl.east) {Техническая \\ документация};
	\node(test) [below=of impl,draw] {Тестирование};
	\node(maint) [below=of test,draw] {Сопровождение};
	
	\draw[->] (req) to (design);
	\draw[->,dashed] (req) to (req-doc);
	\draw[->] (design) to (impl);
	\draw[->,dashed] (design) to (des-doc);
	\draw[->] (impl) to (test);
	\draw[->,dashed] (impl) to (impl-doc);
	\draw[->] (test) to (maint);

	\draw[decorate,decoration={brace,amplitude=0.4em,raise=5pt}] (req.west |- test.west) 
		-- node[left,minimum width=12em] {Пользовательская \\ документация; \\ маркетинговая \\ документация} (req.west);
\end{tikz*}

		\\[1ex]
		\figureexpl{\textbf{Каскадная модель} \engterm{waterfall model} — применение традиционного инженерного подхода 
		к разработке ПО.}
	}
	
	\frame{
		\frametitle{Каскадная модель (продолжение)}

		\textbf{Основная идея:} значительное внимание уделяется инженерии требований и~проектированию, 
		чтобы застраховаться от возможных затратных ошибок.
		
		\vspace{1ex}
		\textbf{Недостатки модели:}
		\begin{itemize}
			\item жесткое ограничение последовательности действий по~разработке;
			\item игнорирование меняющихся нужд пользователей, факторов операционной среды, что~приводит к~изменению требований во~время разработки;
			\item большой период между внесением и~обнаружением ошибки.
		\end{itemize}
		
		\vspace{1ex}
		\textbf{Целесообразность применения:} комплексные системы, для которых долгий цикл сопровождения более важен, 
		чем затраты на разработку.
	}
	
	\subsection{Инкрементная модель}
	
	\frame{
		\frametitle{Инкрементная модель}
		
		{\small\begin{tikz*}[%
	every node/.style={rectangle,draw,align=center,minimum height=2.5em}
]
	\node(analysis) {Анализ \\ требований};
	\node(design1) [above right=of analysis] {Проектирование};
	\node(impl1) [right=of design1] {Реализация};
	\node(test1) [right=of impl1] {Тестирование};
	\node(release1) [right=of test1,ellipse] {Выпуск 1};
	\draw[->] (analysis) |- (design1);
	\draw[->] (design1) to (impl1);
	\draw[->] (impl1) to (test1);
	\draw[->] (test1) to (release1);
	
	\node(design2) [right=of analysis] {Проектирование};
	\node(impl2) [right=of design2] {Реализация};
	\node(test2) [right=of impl2] {Тестирование};
	\node(release2) [right=of test2,ellipse] {Выпуск 2};
	\draw[->] (analysis) to (design2);
	\draw[->] (design2) to (impl2);
	\draw[->] (impl2) to (test2);
	\draw[->] (test2) to (release2);
	
	\draw[->,dashed] (release1) to (release2);
	
	\node(design3) [below right=of analysis] {Проектирование};
	\node(impl3) [right=of design3,label=below:$\vdots$] {Реализация};
	\node(test3) [right=of impl3] {Тестирование};
	\node(release3) [right=of test3,ellipse] {Выпуск 3};
	\draw[->] (analysis) |- (design3);
	\draw[->] (design3) to (impl3);
	\draw[->] (impl3) to (test3);
	\draw[->] (test3) to (release3);
	
	\draw[->,dashed] (release2) to (release3);
\end{tikz*}
}
		\\[0.5ex]
		\figureexpl{\textbf{Инкрементная модель} — разработка продукта итерациями, каждая из которых 
		завершается выпуском работоспособной и~осмысленной версии.}
	}
	
	\frame{
		\frametitle{Инкрементная модель (продолжение)}
		
		\textbf{Основная идея:} последовательное наращивание функциональных возможностей программного продукта 
		с применением на каждой итерации всех процессов каскадной модели.
		
		\vspace{1ex}
		\textbf{Недостатки модели:} требования фиксированы на протяжении всего процесса разработки.
		
		\vspace{1ex}
		\textbf{Целесообразность применения:}
		\begin{itemize}
			\item необходима быстрая реализация возможностей системы;
			\item существует декомпозиция системы на составляющие части, реализуемые как~самостоятельные промежуточные или готовые продукты;
			\item возможно увеличение финансирования на разработку отдельных частей продукта.
		\end{itemize}
	}
	
	\subsection{Эволюционная модель}
	
	\frame{
		\frametitle{Эволюционная модель}
		
		\begin{tikz*}[%
	node distance=2.0em and 3.0em,
	every node/.style={rectangle,draw,align=center,minimum height=2.5em}
]
	\node(init) {Начальные \\ требования};
	\node(proto) [below=of init] {Разработка \\ прототипа};
	\node(req) [right=of proto] {Конкретизация \\ требований};
	\node(diff) [below=of $(proto.south)!0.5!(req.south)$] {Внесение \\ изменений};
	\node(release) [below=of diff] {Выпуск};
	\node(customer) [above=of req,ellipse] {Заказчик};
	
	\draw[->] (init) to (proto);
	\draw[->] (proto) to (req);
	\draw[->] (req) to (diff);
	\draw[->] (diff) to (proto);
	\draw[->] (diff) to (release);
	\draw[<->,dashed] (init) to (customer);
	\draw[<->,dashed] (req) to (customer);
\end{tikz*}

		\\[1ex]
		\figureexpl{\textbf{Эволюционная модель} — разработка ПО с использованием функциональных прототипов, 
			которые \emph{эволюционируют} в~элементы конечного продукта.}
	}
	
	\frame{
		\frametitle{Эволюционная модель (продолжение)}
		
		\textbf{Основные идеи:}
		\begin{itemize}
			\item создание множества прототипов (т.\,е.~неполных версий) продукта для~определения и~уточнения требований пользователей;
			\item
			интенсивное использование средств автоматизации: визуальных сред разработки пользовательского интерфейса, 
			СУБД, ЯП высокого уровня абстракции, генераторов кода и~т.\,п.
		\end{itemize}
		
		\vspace{1ex}
		\textbf{Недостатки модели:} 
		\begin{itemize}
			\item дополнительные затраты на разработку прототипов; 
			\item недостаточный анализ системы ($\Rightarrow$ неоптимальная архитектура); 
			\item риск интерпретации заказчиком прототипов как~финального продукта.
		\end{itemize}
		
		\vspace{1ex}
		\textbf{Целесообразность применения:} проекты, для~которых важен пользовательский интерфейс.
	}

	\subsection{Спиральная модель}
	
	\frame{
		\frametitle{Спиральная модель}
		
		\begingroup
	\footnotesize%
	\begin{tikz*}[%
		x=2em,y=2em,
		xscale=3.25,yscale=1.9,
		every node/.style={rectangle,align=center,minimum height=2.5em},
		objective/.style={draw,font=\small\bfseries}
	]
		\draw[->] (-3.5,0) to (4.25,0);
		\draw[->] (0,-4) to (0,4) node[right]{Стоимость системы};

		\draw[thick,draw=blue,domain=0.01:21.991,samples=150,smooth,variable=\t,xscale=0.19,yscale=0.175] plot({\t*sin(\t r)}, {\t*cos(\t r)});
		
		\node at (0.1, -0.2) [pin=84:Концепция \\ требований] {};
		\node at (-0.4, -0.2) {План ЖЦ};
		\node at (-0.4, 0.3) {План \\ требований};
		\node at (0.8, 0.2) {Прототип 1};
		\node at (2.0, 0.2) {Прототип 2};
		\node at (3.2, 0.4) {Функциональный \\ прототип};
		\node at (0.95, -0.3) {Требования};
		\node at (0.5, -1.0) {Верификация \\ и валидация};
		\node at (-0.5, -1.0) {План \\ разработки};
		\node at (2.0, -0.5) {Черновое \\ проектирование};
		\node at (0.5, -2.2) {Верификация \\ и валидация};
		\node at (-0.5, -2.2) {План \\ тестирования};
		\node at (3.3, -0.5) {Проектирование};
		\node at (3.0, -1.4) {Кодирование};
		\node at (2.55, -2.1) {Интеграция};
		\node at (1.7, -2.8) {Тестирование};
		\node at (0.6, -3.3) {Имплементация};
		
		\node at (-0.3, -3.85) {\bfseries Выпуск};
		
		\node at (-2.75, 3.25) [objective] {1. Определение \\ целей};
		\node at (3.5, 3.25) [objective] {2. Минимизация \\ риска};
		\node at (3.5, -3.25) [objective] {3. Разработка \\ и тестирование};
		\node at (-2.75, -3.25) [objective] {4. Планирование \\ след. итерации};
	\end{tikz*}
\endgroup

	}
	
	\frame{
		\frametitle{Спиральная модель (продолжение)}
		
		\textbf{Основные идеи:}
		\begin{itemize}
			\item контроль риска — раннее тестирование наиболее сложных частей ПО; 
			выбор распределения между работами (анализ требований, проектирование, создание прототипов, тестирование); 
			выбор уровня детализации.
			
			\item выбор модели процесса разработки (каскадная модель, прототипирование) на~каждой итерации. 
		\end{itemize}
		
		\vspace{1ex}
		\textbf{Недостатки модели:} проблема выбора момента начала новой итерации.
		
		\vspace{1ex}
		\textbf{Целесообразность применения:} комплексные дорогостоящие проекты, для которых критически важна минимизация риска.
	}
	
	\section{Гибкая методология}
	
	\subsection{Гибкая методология}
	
	\frame{
		\frametitle{Гибкая методология разработки ПО}
		
		\begin{tikz*}[%
	xscale=2.0,
	every node/.style={rectangle,draw,align=center,minimum height=2.5em,font=\small}
]
	\node(customer) [ellipse] {Заказчик};
	\node(init) at (90:6em) {Планирование \\ итерации};
	\node(req) at (45:6em) {Анализ \\ требований};
	\node(design) at (0:6em) {Проектирование};
	\node(code) at (-45:6em) {Кодирование};
	\node(test) at (-90:6em) {Тестирование};
	\node(doc) at (-135:6em) {Документирование};
	\node(release) at (180:6em) [ellipse] {Выпуск};
	\node(prior) at (135:6em) {Переоценка \\ приоритетов};
	
	\draw[->] (init) to (req);
	\draw[->] (req) to (design);
	\draw[->] (design) to (code);
	\draw[->] (code) to (test);
	\draw[->] (test) to (doc);
	\draw[->] (doc) to (release);
	\draw[->] (release) to (prior);
	\draw[->] (prior) to (init);
	
	\draw[<->,dashed] (customer) to (init);
	\draw[<->,dashed] (customer) to (req);
	\draw[<->,dashed] (customer) to (test);
	\draw[<->,dashed] (customer) to (release);
\end{tikz*}

		
		\vspace{0.5ex}
		\begin{quote}
			\hfill\begin{tabular}{rcl}
				Личности и взаимодействие & > & процессы и инструменты \\
				Работающее ПО & > & исчерпывающая документация \\
				Сотрудничество с заказчиком & > & согласование условий контракта \\
				Реагирование на изменения & > & следование плану \\
			\end{tabular}\hfill\strut{}
			
			\hfill\figureexpl{— Agile Manifesto, 2001}
		\end{quote}
	}
	
	\frame{
		\frametitle{Принципы гибкой методологии}
	
		\begin{enumerate}
			\item
			\textbf{Личности и взаимодействие} — важность самоорганизации и взаимодействий между разработчиками 
			(напр., парное программирование); многофункциональность каждого исполнителя (проектирование, кодирование, тестирование, …).
			
			\item
			\textbf{Работающий продукт} — работающее ПО лучше отражает процесс разработки для~заказчика лучше, чем документы.
			
			\item
			\textbf{Сотрудничество с заказчиком} — доработка и конкретизация требований в процессе разработки; 
			постоянное присутствие представителя заказчика при разработке ПО.
			
			\item
			\textbf{Реагирование на изменения} — фокус на быстрое внедрение изменений и~непрерывную разработку \engterm{continuous development}.
		\end{enumerate}
	}
	
	\subsection{Методы гибкой разработки}
	
	\frame{
		\frametitle{Методы гибкой разработки}
		
		\begin{itemize}
			\item
			\textbf{Непрерывная интеграция} \engterm{continuous integration} — частая (несколько раз в~день) автоматизированная сборка программного продукта, 
			чтобы выявить интеграционные проблемы.
			
			\item
			\textbf{Проблемно-ориентированное проектирование} \engterm{domain-driven design} — создание концептуальных моделей предметной области 
			с целью упростить ее~понимание разработчиками.
			
			\item
			\textbf{Парное программирование} \engterm{pair programming} — 1-й~разработчик пишет код, 2-й~проверяет его на~правильность.
			
			\item
			\textbf{Разработка через тестирование} \engterm{test-driven development} — написание набора тестов, проверяющих функциональность элемента~ПО, 
			с~последующим кодированием для~прохождения этого набора.
		\end{itemize}
	}
	
	\frame{
		\frametitle{Методы гибкой разработки (продолжение)}
		
		\begin{itemize}
			\item
			\textbf{Автоматизированное модульное тестирование} \engterm{unit testing} — немедленная проверка всех изменений, вносимых в код.
			
			\item
			\textbf{Шаблоны проектирования} \engterm{design patterns} — типовые конструктивные элементы программной системы, 
			задающие взаимодействие нескольких компонентов, а также роли и сферы ответственности исполнителей.
			
			\item
			\textbf{Рефакторинг кода} \engterm{code refactoring} — преобразование кода без изменения функциональности программной системы 
			с целью создания общей архитектуры системы.
		\end{itemize}
	}

	\subsection{Характеристики гибкой методологии}
	
	\frame{
		\frametitle{Характеристики гибкой методологии}
		
		\textbf{Недостатки модели:}
		\begin{itemize}
			\item
			неоптимальная архитектура системы вследствие отсутствия требований, фиксированных на протяжении всего процесса разработки;
			
			\item
			риск снижения качества продукта из-за множества изменений, вносимых без~достаточного тестирования 
			и проверки на соответствие общей архитектуре системы.
		\end{itemize}
		
		\vspace{1ex}
		\textbf{Целесообразность применения:} программные проекты с необходимостью частых выпусков; 
		ПО с быстро меняющимися требованиями (напр., веб-сервисы).
	}
	
	\section{Заключение}

	\subsection{Выводы}
	
	\frame{
		\frametitle{Выводы}
		
		\begin{enumerate}
			\item
			Процессы, согласно которым разрабатывается ПО, описаны в стандарте ISO 12207.
			В~то~же~время, этот стандарт не содержит последовательность выполнения процессов.
			
			\item
			Различные подходы к расписанию процессов жизненного цикла ПО представлены в~моделях ЖЦ — каскадной, итеративной, эволюционной и спиральной; 
			основными различиями моделей являются их подход к инженерии требований и~(а)цикличность процессов ЖЦ.
			
			\item
			Более современный подход к разработке — гибкая методология программирования (\emph{agile development}), в которой основной фокус делается 
			не на планировании, а~на~взаимодействии внутри коллектива разработчиков и с заказчиком ПО.
		\end{enumerate}
	}
	
	\subsection{Материалы}
	
	\frame{
		\frametitle{Материалы}
		
		\begin{thebibliography}{9}
			\bibitem[1]{1}
			Лавріщева К.\,М. 
			\newblock Програмна інженерія (підручник). 
			\newblock {\footnotesize К., 2008. — 319 с.}

			\bibitem[2]{2}
			US Department of Health \& Human Services
			\newblock Selecting a Development Approach.
			\newblock {\footnotesize\href{http://www.cms.gov/Research-Statistics-Data-and-Systems/CMS-Information-Technology/XLC/Downloads/SelectingDevelopmentApproach.pdf}{\nolinkurl{http://www.cms.gov/Research-Statistics-Data-and-Systems/.../XLC/Downloads/SelectingDevelopmentApproach.pdf}}}
			\newblock {\footnotesize(неплохой обзор различных методологий разработки)}
			
			\bibitem[3]{3}
			Fowler, Martin 
			\newblock The New Methodology.
			\newblock {\footnotesize\url{http://martinfowler.com/articles/newMethodology.html}}
			\newblock {\footnotesize(описание гибкой методологии разработки ПО)}
		\end{thebibliography}
	}

	
	\frame{
		\frametitle{}
		
		\begin{center}
			\Huge Спасибо за внимание!
		\end{center}
	}
\end{document}
