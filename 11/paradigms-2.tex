\documentclass{a4beamer}
%% Lectures - common definitions

\usextensions{tikz}
\usetikzlibrary{shapes.multipart,shapes.callouts,shapes.geometric}
\input{fix-callouts.inc} % Fixes absolute positioning of rectangle callouts

\newif\ifbigpages \bigpagesfalse
\ifdim\paperwidth >20cm
	\bigpagestrue
\fi

\tikzset{%
	note/.style={rectangle callout,draw=none,callout pointer width=1em,%
		align=flush left,font=\footnotesize,inner sep=0.5em,%
		fill=blue!15,fill opacity=0.95,text opacity=1.0,callout absolute pointer=#1},
	node distance=2em and 2.75em
}
\ifbigpages
	% Scale all arrow tips by the factor of 2.5
	\let\old@pgf@arrow@call=\pgf@arrow@call
	\def\pgf@arrow@call#1{%
		\@tempdima=\pgflinewidth%
		\pgfsetlinewidth{2.5\pgflinewidth}%
		\old@pgf@arrow@call{#1}%
		\pgfsetlinewidth{\@tempdima}%
	}
	\def\pgfarrowsleftextend#1{\pgfmathsetlength{\pgf@xa}{1.5*#1}}
	\def\pgfarrowsrightextend#1{\pgfmathsetlength{\pgf@xb}{1.5*#1}}
\fi

%% Load listings package
\usepackage{listings}

%% Are we inside a comment?
\newif\iflstcomment \lstcommentfalse

\lstset{%
	tabsize=4,
	showstringspaces=false,
	basicstyle=\linespread{1.25}\ttfamily\small,
	keywordstyle=\bfseries,
	commentstyle=\lstcommentstyle,
	numbers=left,
	numberstyle=\footnotesize\color{gray},
	xleftmargin=2.5em,
	extendedchars=true,
	escapechar=\$,
	escapebegin=\iflstcomment\begingroup\lstcommentstyle\fi,
	escapeend=\iflstcomment\endgroup\fi
}

\def\lstcommentstyle{\color{gray}}

\lst@AddToHook{AfterBeginComment}{\global\lstcommenttrue}
\let\orig@lst@EndComment=\lst@EndComment
\def\lst@EndComment{\global\lstcommentfalse\orig@lst@EndComment}
\lst@AddToHookAtTop{EOL}{%
	\lst@ifLmode\global\lstcommentfalse\fi% XXX Sloppy way to determine comment end
}

%% Python with docstrings treated as comments
\lstdefinelanguage[doc]{python}[]{python}{%
	deletestring=[s]{"""}{"""},%
	morecomment=[s]{"""}{"""}%
}%

%% JavaScript language
\lstdefinelanguage{javascript}%
	{morekeywords={break,case,catch,%
		const,constructor,continue,default,do,else,false,%
		finally,for,function,if,in,instanceof,%
		new,null,prototype,%
		return,switch,this,throw,%
		true,try,typeof,var,while},%
	sensitive,%
	morecomment=[l]//,%
	morecomment=[s]{/*}{*/},%
	morestring=[b]",%
	morestring=[b]',%
}[keywords,comments,strings]%

%% C# language (4.0?)
\lstdefinelanguage{csharp}%
	{morekeywords={abstract,as,%
		base,bool,byte,case,catch,char,%
		checked,class,const,continue,%
		decimal,default,delegate,do,double,%
		else,enum,event,explicit,extern,%
		false,finally,fixed,float,for,foreach,%
		goto,if,implicit,in,int,interface,%
		internal,is,lock,long,%
		namespace,new,null,object,operator,out,%
		override,params,private,protected,public,%
		readonly,ref,return,sbyte,sealed,%
		short,sizeof,stackalloc,static,string,%
		struct,switch,this,throw,true,try,%
		typeof,uint,ulong,unchecked,unsafe,ushort,%
		using,virtual,void,volatile,while%
	},%
	sensitive,%
	morecomment=[l]//,%
	morecomment=[s]{/*}{*/},%
	morestring=[b]",%
	morestring=[b]',%
}[keywords,comments,strings]%

%% Translation for fact environment
\deftranslation[to=russian]{Fact}{Наблюдение}

%% Inline code snippets
\def\code#1{\texttt{#1}}
\def\codekw#1{\code{\textbf{#1}}}

\def\quoteauthor#1{\par\footnotesize\upshape\hfill—~#1}

%% English term
\def\engterm#1{(англ. \textit{#1})}
%% Term with explanation below (to be used in diagrams)
\def\termwithexpl#1#2{#1\strut{}\\\small\color{gray}(\textit{#2})\strut{}}
%% External link
\def\extlink#1#2{\href{#1}{\color[rgb]{0.7,0.7,1.0}\dashbar{#2}}}
%% Internal link
\def\inlink#1#2{\hyperlink{#1}{\color[rgb]{0.7,0.7,1.0}\dashbar{#2}}}
%% Explanation for a list item
\def\itemexpl#1{\begingroup\small\vspace{0.75ex}#1\par\endgroup}



\usetikzlibrary{decorations.pathreplacing,shapes.misc}


\lecturetitle{Программная инженерия. Лекция №11 — Парадигмы программирования (часть 2).}
\title[Парадигмы программирования 2]{Парадигмы программирования (часть 2)}
\author{Алексей Островский}
\institute{\small{Физико-технический учебно-научный центр НАН Украины}\vspace{2ex}}
\date{5 декабря 2014 г.}

\begin{document}
	\frame{\titlepage}

	\frame{
		\frametitle{Парадигмы программирования}

		\begin{tikz*}[%
	every node/.style={rectangle,minimum height=2.25em,minimum width=7.5em}
]
	\node(paradigm) {\large\textbf{Программирование}};
	\node(declarative) [above=of paradigm] {Декларативное};
	\node(functional) [left=of declarative] {Функциональное};
	\node(logic) [right=of declarative] {Логическое};
	\node(imperative) [below=of paradigm] {Императивное};
	\node(struct) [below left=of imperative] {Структурное};
	\node(modular) [below=of imperative] {Модульное};
	\node(oo) [below right=of imperative] {ООП};

	\node(component) [below=of oo] {\textbf{Компонентное}};
	\node(service) [below=of component] {\textbf{Сервисное}};
	\node(aspect) [left=of service] {\textbf{Аспектное}};
	\node(agent) [right=of service] {\textbf{Агентное}};

	\draw[->] (paradigm) to (declarative);
	\draw[->] (paradigm) to (imperative);
	\draw[->] (declarative) to (functional);
	\draw[->] (declarative) to (logic);
	\draw[->] (imperative) to (struct);
	\draw[->] (imperative) to (modular);
	\draw[->] (imperative) to (oo);
	\draw[->] (oo) to (component);
	\draw[->] (component) to (service);
	\draw[->] (component) to (aspect);
	\draw[->] (component) to (agent);
\end{tikz*}

	}

	\section[Компоненты]{Компонентно-ориентированное программирование}

	\subsection{Определение}

	\frame{
		\frametitle{Причины появления компонентов}

		\begin{overlayarea}{\textwidth}{0.4\textheight}
			\begin{tikz*}[%
	every node/.style={rectangle,draw,minimum height=2.5em},
	label/.style={draw=none,minimum height=3em,font=\itshape}
]
	\node(object-1) {Объект 1};
	\node(object-2) [below=4em of object-1] {Объект 2};
	\node(object-3) [right=6em of object-1] {Объект 3};
	\node(object-4) [below=4em of object-3] {Объект 4};
	\node(input) [circle,fill,minimum height=1.25em,left=4em of object-1] {};
	\node(output-i) [circle,fill,minimum height=1.25em,right=4em of object-4] {};
	\node(output) [circle,draw,minimum height=1.5em] at (output-i.center) {};

	\node(brace-start) at ($ (object-4.south east) + (0.5em, 0) $) [coordinate]{};
	\node(brace-end) at ($ (object-2.south west) + (-0.5em, 0) $) [coordinate]{};
	\draw<2>[decorate,decoration={brace,amplitude=6pt}] (brace-start) -- node[label,below]{Компонент} (brace-end);

	\draw[<->] (object-1) to (object-2);
	\draw[<->] (object-1) to (object-3);
	\draw[<->] (object-1) to (object-4);
	\draw[<->] (object-3) to (object-4);
	\draw[->] (input) to (object-1);
	\draw[->] (input) to (object-2);
	\draw[->] (object-3) to (output);
	\draw[->] (object-4) to (output);
\end{tikz*}

		\end{overlayarea}

		\uncover<2->{%
			\vspace{1ex}
			\begin{itemize}
				\item
				\textbf{повторное использование} — объекты/классы тесно связаны между собой, 
				что~затрудняет использование в другой среде;

				\item
				\textbf{модульность} — связанность объектов препятствует выделению автономных модулей;

				\item
				\textbf{быстрая разработка} — затраты на конфигурацию объектов для новой системы чересчур высоки.
			\end{itemize}
		}
	}

	\frame{
		\frametitle{Компонентно-ориентированное программирование}

		\begin{Definition}
			\textbf{Компонентно-ориентированное программирование} \engterm{component-based software engineering} — 
			подход к разработке ПО, основанный на использовании слабо связанных \engterm{loosely coupled} 
			независимых программных компонентов в единой программной системе.
		\end{Definition}

		\vspace{1ex}
		КОП — развитие ООП с акцентом на повторное использование кода (software reuse).

		\vspace{1ex}
		\begin{Definition}
			\textbf{Компонент} — самостоятельный программный продукт, который соответствует определенной компонентной модели 
			и может сочетаться с другими компонентами без~модификации кода согласно заданному стандарту.
		\end{Definition}
	}

	\frame{
		\frametitle{Особенности парадигмы}

		\begin{itemize}
			\item
			\textbf{Разделение интерфейса и реализации} — интерфейс полностью определяет функциональность компонента; 
			его~реализация может прозрачно измениться, в~т.\,ч.~во~время~исполнения программы.

			\item
			\textbf{Использование стандартов} — спецификация определений интерфейсов и~формата взаимодействия компонентов. 
			Компонент может~быть написан на~любом~ЯП при~соблюдении стандартов.

			\item
			\textbf{Промежуточный слой} \engterm{middleware}, реализующий взаимодействие между~компонентами.

			\item
			\textbf{Процесс разработки}, оптимизированный для~работы с~компонентами повторного использования.
		\end{itemize}
	}

	\subsection{Характеристики компонентов}

	\frame{
		\frametitle{Характеристики компонентов}

		\begin{itemize}
			\item
			\textbf{Стандартизация} — соответствие компонентной модели, в которой находится компонент. 
			Модель определяет способ задания интерфейса, метаданных, документации компонента; 
			инструменты для композиции и развертывания компонентов.

			\item
			\textbf{Независимость} — все зависимости от других компонентов должны быть явно заданы в~интерфейсе.

			\item
			\textbf{Использование в составе системы} — все внешние взаимодействия должны производиться 
			через задекларированные интерфейсы; компонент должен предоставить информацию о своих методах и свойствах.

			\item
			\textbf{Способность к развертыванию} — компонент должен предоставлять имплементацию, 
			которая позволяет развернуть его в определенной среде с~помощью стандартных средств.

			\item
			\textbf{Документация} — интерфейс компонента должен быть документирован для~упрощения его~повторного использования.
		\end{itemize}
	}

	\frame{
		\frametitle{Отличия компонентов от объектов}

		\begin{itemize}
			\item
			Объект выражает одно понятие предметной области; компонент может соответствовать нескольким связанным понятиям.

			\item
			Объекты тесно связаны между собой; компоненты связаны слабо, 
			все зависимости должны быть задекларированы в интерфейсе компонента.

			\item
			Для объектов интерфейс и имплементация тесно связаны между собой (напр., определяются одним ЯП); 
			для компонентов интерфейс и имплементация разграничены.

			\item
			Компоненты не имеют наблюдаемого состояния, все экземпляры компонента неразличимы (выполняется не всегда).
		\end{itemize}
	}

	\frame{
		\frametitle{Интерфейсы компонентов}

		\begin{figure}
			\begin{tikz*}[%
	description/.style={rectangle,fill=blue!15,align=center,font=\small}
]
	\node(component) [rectangle,draw,minimum width=10em,minimum height=10em] {\textbf{Компонент}};

	\node(req-2) [left=4em of component.west,coordinate] {};
	\node(req-1) [above=2em of req-2,coordinate] {};
	\node(req-3) [below=2em of req-2,coordinate] {};
	\node(req) [description,left=5em of component.west,anchor=east] {Функциональность, \\ которую должны \\ предоставить \\ внешние компоненты};

	\node(provides-2) [above right=1em and 4em of component.east,coordinate] {};
	\node(provides-3) [below right=1em and 4em of component.east,coordinate] {};
	\node(provides-1) [above=2em of provides-2,coordinate] {};
	\node(provides-4) [below=2em of provides-3,coordinate] {};
	\node(provides) [description,right=5em of component.east,anchor=west] {Функциональность, \\ предоставляемая \\ компонентом};
	
	\draw[)-] (req-1) -- (req-1 -| component.west);
	\draw[)-] (req-2) -- (req-2 -| component.west);
	\draw[)-] (req-3) -- (req-3 -| component.west);

	\draw[o-] (provides-1) -- (provides-1 -| component.east);
	\draw[o-] (provides-2) -- (provides-2 -| component.east);
	\draw[o-] (provides-3) -- (provides-3 -| component.east);
	\draw[o-] (provides-4) -- (provides-4 -| component.east);
\end{tikz*}

			\caption{UML-диаграмма компонента}
		\end{figure}

		\vspace{2ex}
		\textbf{Сервис} — развитие компонента, в котором отсутствуют зависимости от внешних компонентов.
	}

	\subsection{Компонентная модель}

	\frame{
		\frametitle{Компонентная модель}

		\begin{Definition}
			\textbf{Компонентная модель} — совокупность стандартов, касающихся реализации, документирования и развертывания компонентов.
		\end{Definition}

		\vspace{1ex}
		\textbf{Аспекты компонентной модели:}
		\begin{itemize}
			\item способ задания интерфейса (операции над компонентом, их параметры, генерируемые исключения);
			\item использование компонентов (способ идентификации, конфигурирование, …);
			\item развертывание (организация бинарного кода, необходимого для запуска компонента, 
			напр., используемых сторонних библиотек).
		\end{itemize}

		\vspace{1ex}
		\textbf{Примеры компонентных моделей:}
		\begin{itemize}
			\item Java EE / Enterprise Java Beans;
			\item Microsoft COM, Microsoft .NET;
			\item CORBA.
		\end{itemize}
	}

	\subsection{Разработка с компонентами}

	\frame{
		\frametitle{Разработка с компонентами}

		\textbf{Типы процессов разработки} с использованием компонентов:
		\begin{itemize}
			\item
			Разработка \emph{для} повторного использования \engterm{development for reuse} — разработка программных компонентов, 
			которые могут использоваться в~различных контекстах с~минимальными затратами.

			\vspace{0.75ex}
			\textbf{Составляющие:} обобщение функциональности компонента, минимизация зависимостей, 
			стандартизация обработки исключительных ситуаций, сохранение в~репозиторий, …

			\vspace{2ex}
			\item
			Разработка \emph{с} повторным использованием \engterm{development with reuse} — разработка программных систем, 
			использующих готовые компоненты и сервисы.

			\vspace{0.75ex}
			\textbf{Составляющие:} выработка требований, подбор, тестирование и объединение компонентов в единую систему.
		\end{itemize}
	}

	\section[Аспекты]{Аспектно-ориентированное программирование}

	\frame{
		\frametitle{Аспектно-ориентированное программирование}

		\begin{Definition}
			\textbf{Аспектно-ориентированное программирование} — парадигма программирования, 
			основанная на обеспечении модульности при помощи выделения \emph{сквозной функциональности} в отдельные модули.
		\end{Definition}

		\vspace{2ex}
		\textbf{Принцип разделения ответственности} \engterm{separation of concerns}:

		Каждый элемент компьютерной программы (в ООП — объекты, методы объектов) должен решать строго одну задачу; 
		функции различных элементов не должны перекрываться.

		\vspace{2ex}
		\textbf{Аспектно-ориентированные среды:} AspectJ (на основе Java), AspectC++.
	}

	\subsection{Сквозная функциональность}

	\frame{
		\frametitle{Сквозная функциональность}

		{\small%(рис.: Основная функциональность: Добавление пользователей, Работа с учетной записью, Управление пользователями. Сквозная функциональность: Безопасность данных, %Отказоустойчивость)
\begin{tikz*}[%
	every node/.style={align=center},
	concern/.style={minimum width=1em,minimum height=15em,draw=blue,fill=blue!15},
	cross-concern/.style={minimum width=30em,minimum height=1em,draw,fill=white},
	label/.style={font=\bfseries}
]
	\node(main-1-label) {Добавление \\ пользователей};
	\node(main-2-label) [right=of main-1-label] {Работа с \\ учетной записью};
	\node(main-3-label) [right=of main-2-label] {Управление \\ пользователями};
	\node(main-label) [label,above=1em of main-2-label] {Основная функциональность};

	\node(main-1) [concern,below=of main-1-label]{};
	\node(main-2) [concern,below=of main-2-label]{};
	\node(main-3) [concern,below=of main-3-label]{};

	\node(cross-1) [cross-concern,below=4em of main-2.north] {};
	\node(cross-2) [cross-concern,above=4em of main-2.south] {};
	\node(cross-2-label) [right=1em of cross-2] {Отказоустойчивость};
	\node(cross-1-label) at (cross-2-label.center |- cross-1.center) {Безопасность \\ данных};
	\node(cross) [label,below=of cross-2-label] {Сквозная \\ функциональность};
\end{tikz*}
}
	}

	\frame{
		\frametitle{Сквозная функциональность}

		\begin{Definition}		
			\textbf{Сквозная функциональность} \engterm{cross-cutting concern} — это функциональность программы, 
			которую невозможно полностью выделить в отдельные сущности (классы, методы). 
		\end{Definition}

		\vspace{1ex}
		\textbf{Следствия сквозной функциональности:} 
		\begin{itemize}
			\item дублирование кода; 
			\item сильные зависимости между элементами системы.
		\end{itemize}

		\vspace{1ex}
		\textbf{Примеры сквозной функциональности:} 
		\begin{itemize}
			\item ведение лога;
			\item аутентификация; 
			\item синхронизация; 
			\item кэширование; 
			\item обработка транзакций.
		\end{itemize}
	}

	\subsection{Элементы АОП}

	\frame{
		\frametitle{Терминология АОП}

		\begin{itemize}
			\item
			\textbf{Аспект} — модуль, реализующий сквозную функциональность; 
			содержит определение \emph{срезов} и~связанных~с~ними \emph{советов}.

			\vspace{0.5ex}
			\item
			\textbf{Совет} \engterm{advice} — код, реализующий сквозную функциональность. 
			Совет может выполняться до, после или~вместо основного кода.

			\vspace{0.5ex}
			\item
			\textbf{Точка соединения} \engterm{join point} — точка в~основной программе, 
			в~которой применяется \emph{совет}.

			\vspace{0.5ex}
			\item
			\textbf{Срез} \engterm{pointcut} — набор точек соединения,
			определяющий область применения совета.

			\vspace{0.5ex}
			\item
			\textbf{Внедрение} \engterm{weaving} — изменение основного кода 
			для~добавления функциональности аспекта.
		\end{itemize}
	}

	\frame{
		\frametitle{Модель точек соединения}

		\textbf{Модель точек соединения} \engterm{join point model} — определение возможных мест для~внедрения советов.

		\vspace{2ex}
		\textbf{Точки соединения в AspectJ:}
		\begin{itemize}
		\item вызов методов и конструкторов;
		\item инициализация классов или объектов;
		\item доступ или изменение полей объекта;
		\item обработка исключительных ситуаций.
		\end{itemize}

		\vspace{1ex}
		\textbf{Контекст выполнения:} класс (объект \code{\textbf{this}} / класс, где находится код), пакет.
	}

	\subsection{Примеры}

	\frame{
		\frametitle{Примеры точек соединения}

		\textbf{Точки соединения в AspectJ:}
		\begin{itemize}
			\item
			\code{call(\textbf{void} show())}

			Вызов метода \code{show()} в произвольном классе.

			\vspace{1ex}
			\item
			\code{call(* Point.set*())}

			Вызов метода с названием, начинающимся на \code{set}, в классе \code{Point}.

			\vspace{1ex}
			\item
			\code{within(com.example.*) \&\& call(*.\textbf{new}(\textbf{int}))}

			Вызов конструктора, принимающего один аргумент типа \code{\textbf{int}}, в классах из пакета \code{com.example}.

			\vspace{1ex}
			\item
			\code{handler(ArrayOutOfBoundsException)}

			Обработка исключения \code{ArrayOutOfBoundsException}.

			\vspace{1ex}
			\item
			\code{call(\textbf{protected} !\textbf{static} * *(..))}

			Вызов любого защищенного нестатического метода.
		\end{itemize}
	}

	\frame{
		\frametitle{Пример}

		\textbf{Основной код:}
		\lstinputlisting[language=java]{code-aspect.java}
	}

	\frame{
		\frametitle{Пример}

		\textbf{Код аспекта:}
		\lstinputlisting[language={[aspectj]java}]{code-aspect.aj}
	}

	\subsection{Внедрение аспектов}

	\frame{
		\frametitle{Внедрение аспектов}

		\begin{figure}
			{\small\begin{tikz*}[%
	every node/.style={draw,align=center,minimum height=2.5em},
	aspect/.style={rectangle,draw,minimum width=12.5em},
	class/.style={rectangle split,rectangle split parts=2,align=left}
]
	\node(aspect-auth) [aspect] {Аспект аутентификации};
	\node(aspect-log) [aspect,below=of aspect-auth] {Аспект ведения лога};
	\node(class-main) [class,text width=12.5em,below=of aspect-log] {%
		\hfill\textbf{Reader}\hfill\strut{}
		\nodepart{two}
		…\strut{} \\
		\code{updateInfo(…);} \\
		…
	};

	\node(weaver) [rounded rectangle,right=4em of aspect-log] {\textbf{Компоновщик}};

	\node(class-patched) [class,text width=12.5em,right=4em of weaver] {%
		\hfill\textbf{Reader}\hfill\strut{}
		\nodepart{two}
		…\strut{} \\
		\textit{аутентификация} \\
		\code{updateInfo(…);} \\
		\textit{запись в лог} \\
		…
	};

	\draw[->] (aspect-auth.east) -| (weaver.north);
	\draw[->] (aspect-log) -- (weaver);
	\draw[->] (class-main.east) -| (weaver.south);
	\draw[->] (weaver) -- (class-patched);
\end{tikz*}
}\vspace{-1ex}
			\caption{Внедрение аспектов в AspectJ}
		\end{figure}

		\textbf{Внедрение кода аспектов:}
		\begin{itemize}
			\item на этапе компиляции (как фрагментов исходного кода);
			\item на этапе линковки (как скомпилированных фрагментов) — \textbf{основной способ};
			\item на этапе выполнения (за счет рефлексии).
		\end{itemize}
	}

	\section[Сервисы]{Сервис-ориентированная архитектура}

	\frame{
		\frametitle{Сервис-ориентированная архитектура}

		\begin{Definition}
			\textbf{Сервис-ориентированная архитектура} \engterm{service-oriented architecture} — 
			парадигма программирования, в которой для обеспечения модульности применяются 
			распределенные слабо связанные компоненты (\emph{сервисы}), взаимодействующие с~помощью стандартизованных протоколов.
		\end{Definition}

		\vspace{2ex}
		\textbf{Характеристики сервисов:}
		\begin{itemize}
			\item
			модульность — сервис представляет логически связанные функции 
			в~определенной предметной области с~заданными входами и~выходами;

			\item
			автономность — отсутствие наблюдаемых для пользователей зависимостей;

			\item
			сокрытие реализации — рассматривается как «черный ящик».
		\end{itemize}
	}

	\subsection{Веб-сервисы}

	\frame{
		\frametitle{Веб-сервисы}

		\begin{Definition}
			\textbf{Веб-сервис} — сервис, идентифицирующийся по~адресу URL и~взаимодействующий 
			по~Интернету с~помощью высокоуровневых протоколов на~основе HTTP, TCP/IP.
		\end{Definition}

		\vspace{1ex}
		\begin{figure}
			\begin{tikz*}[%
	every node/.style={align=center},
	label/.style={font=\small},
	item/.style={ellipse,draw,minimum height=3em}
]
	\node(consumer) [item] {Потребитель};
	\node(registry) [item,above right=5em and 6em of consumer] {Реестр};
	\node(provider) [item,right=12em of consumer] {Поставщик};
	\node(service) [item,rectangle,below=4em of provider] {Сервис};

	\draw[<->] (consumer) -- node[label,above left] {поиск (UDDI)} (registry);
	\draw[<->] (consumer) -- node[label,below] {связывание (SOAP)} (provider);
	\draw[<->] (registry) -- node[label,above right] {публикация} (provider);
	\draw[->] (service) -- node[label,left] {спецификация (WSDL)} (provider);
\end{tikz*}
\vspace{-1ex}
			\caption{Схема взаимодействия с веб-сервисом}
		\end{figure}
	}

	\frame{
		\frametitle{Спецификация веб-сервисов}

		\textbf{Содержимое спецификации:}
		\begin{itemize}
			\item
			Операции, предоставляемые сервисом ($\simeq$ методы в ООП), соответствующие входные и~возвращаемые данные;
			\item
			формат сообщений для взаимодействия с сервисом;
			\item
			(необязательно) типы данных, используемые в сообщениях;
			\item
			определение конкретных протоколов доступа к операциям (с помощью SOAP или~других методов).
		\end{itemize}

		\vspace{1ex}
		\textbf{Язык описания:} WSDL (web service definition language) — на основе XML.
	}

	\frame{
		\frametitle{Использование веб-сервисов}

		\textbf{Протокол:} SOAP (simple object access protocol) — на основе XML.

		\vspace{1ex}
		\textbf{Сообщение сервису:}
		\begin{itemize}
			\item
			заголовок сообщения — нефункциональные характеристики запроса (приоритетность, время обработки, …);
			\item
			тело сообщения — список операций веб-сервиса и соответствующих параметров.
		\end{itemize}

		\vspace{1ex}
		\textbf{Ответное сообщение:}
		\begin{itemize}
			\item
			тело сообщения — список с результатами выполнения операций;
			\item
			отказы — информация об отказах при проведении операций.
		\end{itemize}
	}

	\subsection{Разработка веб-сервисов}

	\frame{
		\frametitle{Разработка веб-сервисов}

		\textbf{Цели разработки:}
		\begin{itemize}
			\item минимизация количества обращений к сервису;
			\item скрытие состояния сервиса (хранение состояния — задача клиента; 
			состояние может передаваться в сообщениях).
		\end{itemize}

		\vspace{1ex}
		\textbf{Этапы разработки:}
		\begin{enumerate}
			\item определение функциональности; 
			\item описание операций и сообщений; 
			\item имплементация; 
			\item тестирование; 
			\item развертывание.
		\end{enumerate}
	}

	\frame{
		\frametitle{Разработка веб-сервисов}

		\textbf{Средства автоматизации:}
		\begin{itemize}
			\item
			Инструменты для создания WSDL-описания на основе классов.

			\item
			Серверы приложений — веб-серверы, обеспечивающие автоматическое развертывание 
			и~выполнение кода веб-сервиса, а также разбор / формирование SOAP-сообщений 
			и~обмен данными со~средой, где выполняется сервис.

			\item
			Инструменты для автоматической генерации кода на~основе WSDL для~доступа к~сервису как к~локальному объекту.

			\item
			Средства для композиции сервисов (Business Process Execution Language, BPEL).
		\end{itemize}
	}

	\subsection{Альтернативы веб-сервисам}

	\frame{
		\frametitle{Альтернативы веб-сервисам}

		\textbf{Недостатки веб-сервисов на основе SOAP:}
		\begin{itemize}
			\item
			«тяжеловесность» используемых протоколов SOAP и WSDL;
			\item
			необходимость в комплексных средствах поддержки (сервере приложений и~т.\,п.);
			\item
			отсутствие привязок типов данных к языкам программирования.
		\end{itemize}

		\vspace{1ex}
		\textbf{Альтернативные реализации СОА:}
		\begin{itemize}
			\item REST;
			\item CORBA;
			\item DCOM;
			\item Java RMI;
			\item Apache Thrift.
		\end{itemize}
	}

	\section{Заключение}

	\subsection{Выводы}
	
	\frame{
		\frametitle{Выводы}

		\begin{enumerate}
			\item
			ООП в чистом виде обладает некоторыми недостатками, затрудняющими повторное использование кода. 
			Для улучшения повторного использования используются парадигмы на основе ООП, 
			в частности компонентно-ориентированное, аспектное и сервисное программирование.

			\vspace{0.5ex}
			\item
			Компонент — самостоятельный программный продукт, реализующий логически замкнутый набор функций системы. 
			В отличие от объектов и классов, интерфейс и~реализация компонентов всегда разграничены, 
			что упрощает многоязыковую и~мультиплатформенную разработку.

			\vspace{0.5ex}
			\item
			Аспект — модуль, объединяющий сквозную функциональность (напр., ведение лога, обработка исключений). 
			Использование аспектов позволяет избежать дублирования кода и избыточных связей между элементами системы.

			\vspace{0.5ex}
			\item
			Сервис — развитие идеи компонента для распределенных приложений. Для~коммуникации с~сервисами используются сетевые протоколы.
		\end{enumerate}
	}
	
	\subsection{Материалы}
	
	\frame{
		\frametitle{Материалы}
		
		\begin{thebibliography}{9}
			\bibitem[1]{1}
			Sommerville, Ian
			\newblock Software Engineering.
			\newblock {\footnotesize Pearson, 2011. — 790 p.}

			\bibitem[2]{2}
			Лавріщева К.\,М. 
			\newblock Програмна інженерія (підручник). 
			\newblock {\footnotesize К., 2008. — 319 с.}

			\bibitem[3]{3}
			Heineman, G.\,T.; Councill, W.\,T.
			\newblock Component-Based Software Engineering: Putting the Pieces Together.
			\newblock {\footnotesize (компоненты и сервисы)}

			\bibitem[4]{4}
			Jacobson, I.; Ng P.-W.
			\newblock Aspect-Oriented Software Development with Use Cases.
			\newblock {\footnotesize (аспекты)}
		\end{thebibliography}
	}
	
	\frame{
		\frametitle{}
		
		\begin{center}
			\Huge Спасибо за внимание!
		\end{center}
	}
\end{document}
