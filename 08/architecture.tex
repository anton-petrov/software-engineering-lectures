\documentclass{a4beamer}
%% Lectures - common definitions

\usextensions{tikz}
\usetikzlibrary{shapes.multipart,shapes.callouts,shapes.geometric}
\input{fix-callouts.inc} % Fixes absolute positioning of rectangle callouts

\newif\ifbigpages \bigpagesfalse
\ifdim\paperwidth >20cm
	\bigpagestrue
\fi

\tikzset{%
	note/.style={rectangle callout,draw=none,callout pointer width=1em,%
		align=flush left,font=\footnotesize,inner sep=0.5em,%
		fill=blue!15,fill opacity=0.95,text opacity=1.0,callout absolute pointer=#1},
	node distance=2em and 2.75em
}
\ifbigpages
	% Scale all arrow tips by the factor of 2.5
	\let\old@pgf@arrow@call=\pgf@arrow@call
	\def\pgf@arrow@call#1{%
		\@tempdima=\pgflinewidth%
		\pgfsetlinewidth{2.5\pgflinewidth}%
		\old@pgf@arrow@call{#1}%
		\pgfsetlinewidth{\@tempdima}%
	}
	\def\pgfarrowsleftextend#1{\pgfmathsetlength{\pgf@xa}{1.5*#1}}
	\def\pgfarrowsrightextend#1{\pgfmathsetlength{\pgf@xb}{1.5*#1}}
\fi

%% Load listings package
\usepackage{listings}

%% Are we inside a comment?
\newif\iflstcomment \lstcommentfalse

\lstset{%
	tabsize=4,
	showstringspaces=false,
	basicstyle=\linespread{1.25}\ttfamily\small,
	keywordstyle=\bfseries,
	commentstyle=\lstcommentstyle,
	numbers=left,
	numberstyle=\footnotesize\color{gray},
	xleftmargin=2.5em,
	extendedchars=true,
	escapechar=\$,
	escapebegin=\iflstcomment\begingroup\lstcommentstyle\fi,
	escapeend=\iflstcomment\endgroup\fi
}

\def\lstcommentstyle{\color{gray}}

\lst@AddToHook{AfterBeginComment}{\global\lstcommenttrue}
\let\orig@lst@EndComment=\lst@EndComment
\def\lst@EndComment{\global\lstcommentfalse\orig@lst@EndComment}
\lst@AddToHookAtTop{EOL}{%
	\lst@ifLmode\global\lstcommentfalse\fi% XXX Sloppy way to determine comment end
}

%% Python with docstrings treated as comments
\lstdefinelanguage[doc]{python}[]{python}{%
	deletestring=[s]{"""}{"""},%
	morecomment=[s]{"""}{"""}%
}%

%% JavaScript language
\lstdefinelanguage{javascript}%
	{morekeywords={break,case,catch,%
		const,constructor,continue,default,do,else,false,%
		finally,for,function,if,in,instanceof,%
		new,null,prototype,%
		return,switch,this,throw,%
		true,try,typeof,var,while},%
	sensitive,%
	morecomment=[l]//,%
	morecomment=[s]{/*}{*/},%
	morestring=[b]",%
	morestring=[b]',%
}[keywords,comments,strings]%

%% C# language (4.0?)
\lstdefinelanguage{csharp}%
	{morekeywords={abstract,as,%
		base,bool,byte,case,catch,char,%
		checked,class,const,continue,%
		decimal,default,delegate,do,double,%
		else,enum,event,explicit,extern,%
		false,finally,fixed,float,for,foreach,%
		goto,if,implicit,in,int,interface,%
		internal,is,lock,long,%
		namespace,new,null,object,operator,out,%
		override,params,private,protected,public,%
		readonly,ref,return,sbyte,sealed,%
		short,sizeof,stackalloc,static,string,%
		struct,switch,this,throw,true,try,%
		typeof,uint,ulong,unchecked,unsafe,ushort,%
		using,virtual,void,volatile,while%
	},%
	sensitive,%
	morecomment=[l]//,%
	morecomment=[s]{/*}{*/},%
	morestring=[b]",%
	morestring=[b]',%
}[keywords,comments,strings]%

%% Translation for fact environment
\deftranslation[to=russian]{Fact}{Наблюдение}

%% Inline code snippets
\def\code#1{\texttt{#1}}
\def\codekw#1{\code{\textbf{#1}}}

\def\quoteauthor#1{\par\footnotesize\upshape\hfill—~#1}

%% English term
\def\engterm#1{(англ. \textit{#1})}
%% Term with explanation below (to be used in diagrams)
\def\termwithexpl#1#2{#1\strut{}\\\small\color{gray}(\textit{#2})\strut{}}
%% External link
\def\extlink#1#2{\href{#1}{\color[rgb]{0.7,0.7,1.0}\dashbar{#2}}}
%% Internal link
\def\inlink#1#2{\hyperlink{#1}{\color[rgb]{0.7,0.7,1.0}\dashbar{#2}}}
%% Explanation for a list item
\def\itemexpl#1{\begingroup\small\vspace{0.75ex}#1\par\endgroup}



\usetikzlibrary{shapes.misc}


\lecturetitle{Программная инженерия. Лекция №8 — Архитектура программных систем.}
\title[Моделирование ПС]{Архитектура программных систем}
\author{Алексей Островский}
\institute{\small{Физико-технический учебно-научный центр НАН Украины}\vspace{2ex}}
\date{14 ноября 2014 г.}

\begin{document}
	\frame{\titlepage}
	
	\section{Введение}
	
	\frame{
		\frametitle{Архитектура ПО}

		\begin{Definition}
			\textbf{Архитектура программного проекта} — высокоуровневое представление структуры системы 
			и~спецификация ее компонентов и логики их взаимодействия.
		\end{Definition}

		\vspace{1ex}
		\textbf{Преимущества} использования архитектуры ПО:
		\begin{itemize}
			\item основа для анализа системы на ранних этапах ее разработки;
			\item основа для повторного использования компонентов и решений;
			\item упрощение принятия решений касательно разработки, развертывания и~поддержки~ПО;
			\item упрощение диалога с заказчиком;
			\item уменьшение рисков и снижение затрат на~производство~ПО.
		\end{itemize}
	}
	
	\subsection{Виды архитектуры}
	
	\frame{
		\frametitle{Виды архитектуры ПО}
		
		\begin{itemize}
			\item
			\textbf{Архитектура отдельных программ.}
			
			\vspace{0.5ex}
			\textbf{Цель:} определить разбиение программы на составляющие.
			
			\begin{center}
				\begin{tabular}{rcl}
					Составляющие & $=$ & функциональные требования \cr
					Архитектура & $=$ & нефункциональные требования \cr
				\end{tabular}
			\end{center}

			\vspace{1.5ex}
			\item
			\textbf{Архитектура сложных систем.}
			
			\vspace{0.5ex}
			\textbf{Цель:} определить структуру взаимодействия системы с~другими системами, программами и~компонентами; 
			определить местоположение системы в~распределенной среде.

			\vspace{0.5ex}
			\itemexpl{(будет рассмотрена в весеннем семестре)}
		\end{itemize}
	}

	\subsection{Архитектурные решения}
	
	\frame{
		\frametitle{Архитектурные решения}

		\textbf{Вопросы,} касающиеся архитектуры ПО:
		\begin{itemize}
			\item распределение программы по нескольким потокам выполнения;
			\item определение структуры системы (выделение компонентов и субкомпонентов);
			\item контроль над выполнением компонентов;
			\item выполнение нефункциональных требований к программе;
			\item использование готовых шаблонов архитектуры и проектирования; 
			\item документирование архитектуры. 
		\end{itemize}
	}
	
	\subsection{Требования и архитектура}
	
	\frame{
		\frametitle{Требования и архитектура}
		
		\begin{center}\linespread{1.0}
			\begin{tabular}{|p{0.25\textwidth}|p{0.65\textwidth}|}
				\hline
				\centering \textbf{Требование} & \centering \textbf{Влияние на архитектуру} \cr
				\hline
				Производительность & \raggedright преимущественно большие компоненты, отказ от~распределенности, параллельное исполнение кода. \cr
				\hline
				Безопасность & \raggedright использование многослойной архитектуры; локализация механизмов авторизации в небольшом числе компонентов. \cr
				\hline
				\raggedright Доступность (\textit{availability}) & \raggedright наличие избыточных компонентов; локализация сбоев. \cr
				\hline
				\raggedright Удобство сопровождения (\textit{maintainability}) & \raggedright использование малых компонентов; минимизация зависимостей; 
					отсутствие совместного использования данных. \cr
				\hline
			\end{tabular}
		\end{center}
	}
	
	\section[Представления]{Представления архитектуры}
	
	\frame{
		\frametitle{Представления архитектуры}
		
		\begin{center}
			\begin{tikz*}[%
	every node/.style={rectangle,draw,align=center,minimum height=3em}
]
	\node(logical) {\alert<2>{Логическое} \\ \alert<2>{представление}};
	\node(process) [below=8em of logical] {\alert<4>{Процессное} \\ \alert<4>{представление}};
	\node(devel) [right=14em of logical] {\alert<3>{Представление} \\ \alert<3>{разработки}};
	\node(phys) [below=8em of devel] {\alert<5>{Физическое} \\ \alert<5>{представление}};
	\node(use-case) [below right=4em and 7em of logical,anchor=center,rounded rectangle] {\alert<6>{Сценарии}};
	\draw[->] (logical) to (process);
	\draw[->] (logical) to (devel);
	\draw[->] (process) to (phys);
	\draw[->] (devel) to (phys);
	\draw[->,dotted] (logical) to (use-case);
	\draw[->,dotted] (process) to (use-case);
	\draw[->,dotted] (devel) to (use-case);
	\draw[->,dotted] (phys) to (use-case);
	
	\node [draw=none,below=0em of phys.south] {};
	
	\node<2> [note={(logical.south east)},below right=1em and 0.5em of logical.south,anchor=north west] {%
		\textbf{Логическое представление} \engterm{logical view} — \\ ключевые сущности системы как объекты \\ и классы объектов.
		\\[1.5ex]
		\textbf{Цель:} реализация функциональных требований \\ в объектах.
	};
	\node<3> [note={(devel.south west)},below left=1em and 0.5em of devel.south,anchor=north east] {%
		\textbf{Представление разработки} \engterm{development view} — \\ 
		распределение компонентов системы \\ для имплементации различными разработчиками \\ и командами.
		\\[1.5ex]
		\textbf{Цель:} организация разработки.
	};
	\node<4> [note={(process.north east)},above right=1em and 0.5em of process.north,anchor=south west] {%
		\textbf{Процессное представление} \engterm{process view} — \\
		взаимодействие процессов во время выполнения \\ системы.
		\\[1.5ex]
		\textbf{Цель:} анализ нефункциональных требований \\ (производительности, доступности, …).
	};
	\node<5> [note={(phys.north west)},above left=1em and 0.5em of phys.north,anchor=south east] {%
		\textbf{Физическое представление} \engterm{physical view} — \\
		организация компонентов в распределенной среде.
		\\[1.5ex]
		\textbf{Цель:} планирование развертывания системы.
	};
	\node<6> [note={(use-case.south)},below=1em of use-case.south] {%
		\textbf{Сценарии} \engterm{use case view} — взаимодействие \\
		между компонентами системы.
		\\[1.5ex]
		\textbf{Цель:} выделение компонентов, проверка \\ и уточнение архитектуры.
	};
\end{tikz*}

			
			\vspace{1ex}
			\figureexpl{Модель представлений \extlink{http://en.wikipedia.org/wiki/4\%2B1_architectural_view_model}{4+1} [Kruchten, 1995]}
		\end{center}
	}
	
	\subsection{Создание представлений}
	
	\frame{
		\frametitle{Создание представлений}
		
		\textbf{Место представлений} в проектировании ПО:
		\begin{itemize}
			\item
			Упрощение обсуждения решений, касающихся проектирования системы.
			
			\vspace{0.5ex}
			\textbf{Потребители:} заинтересованные стороны (напр., заказчики), менеджеры. 
			
			\vspace{0.5ex}
			\textbf{Средства:} неформальные представления (напр., блочные диаграммы).
			
			\vspace{1ex}
			\item
			Документирование \emph{уже разработанной} архитектуры для улучшения ее понимания и~развития приложения.

			\vspace{0.5ex}			
			\textbf{Потребители:} разработчики, отдел сопровождения.

			\vspace{0.5ex}			
			\textbf{Средства:} формальные представления (UML, специализированные языки).
		\end{itemize}
		
		\vspace{1ex}
		\textbf{Agile development:} детализованные представления \emph{не нужны}.
	}
	
	\subsection{Инструменты}
	
	\frame{
		\frametitle{Инструменты создания представлений}
		
		\textbf{UML:}
		\begin{center}\linespread{1.0}
			\begin{tabular}{|l|l|}
				\hline
				\centering \textbf{Представление} & \centering \textbf{Диаграмма UML} \cr
				\hline
				логическое & классов, последовательности, коммуникации \cr\hline
				процессное & деятельности \cr\hline
				разработка & компонентов, пакетов \cr\hline
				физическое & развертывания \cr\hline
				сценарии & вариантов использования \cr
				\hline
			\end{tabular}
		\end{center}
		
		\vspace{1ex}
		\textbf{Другие средства:} языки описания архитектуры \engterm{architecture description language, ADL} — AADL, C2, Darwin, Wright.
	}
	
	\section[Шаблоны]{Архитектурные шаблоны}
	
	\frame{
		\frametitle{Архитектурные шаблоны}
		
		\begin{Definition}
			\textbf{Архитектурный шаблон} \engterm{architecture pattern} — абстрактное описание применяющегося~на практике подхода 
			к организации программной системы.
		\end{Definition}
		
		\vspace{1ex}
		\textbf{Составляющие шаблона:} описание, границы применения, сильные и слабые стороны, примеры.
		
		\vspace{1ex}
		\textbf{Примеры шаблонов:} 
		\begin{itemize}
			\item \inlink{sec:mvc}{Model — View — Controller (MVC)}; 
			\item \inlink{sec:layered}{многослойная архитектура}; 
			\item \inlink{sec:server}{клиент-серверная архитектура}; 
			\item \inlink{sec:pipe}{конвейерная архитектура}.
		\end{itemize}
	}
	
	\subsection{MVC}\label{sec:mvc}
	
	\frame{
		\frametitle{MVC}
		
		\begin{center}
			\begin{tikz*}[%
	block/.style={rectangle split,draw,rectangle split parts=2,align=center},
	every two node part/.style={font=\small,align=left},
	label/.style={font=\footnotesize}
]
	\node(ctrl) [block] {%
		\textbf{Контроллер}
		\nodepart{two}
		Обновление моделей \\ и выбор представления \\
		в зависимости от действий \\ пользователя.
	};
	\node(view) [block,right=8.5em of ctrl] {%
		\textbf{Представление}
		\nodepart{two}
		Создание наглядного \\
		отображения моделей; \\
		запрос изменения моделей; \\
		захват событий.
	};
	\node(model) [block,below right=8em and 4.25em of ctrl,anchor=center] {%
		\textbf{Модель}
		\nodepart{two}
		Сохранение состояния; \\
		уведомление отображений \\
		об изменениях состояния.
	};

	\node(_tmp) at ($ (ctrl.east) + (0,0.5em) $) [coordinate] {};
	\draw[->] (_tmp) -- node[label,above]{выбор представления} (_tmp -| view.west);
	\node(_tmp) at ($ (ctrl.east) + (0,-0.5em) $) [coordinate] {};
	\draw[<-] (_tmp) -- node[label,below]{польз. события} (_tmp -| view.west);
	\draw[->] (ctrl.south) |- node[label,pos=0.3,left,align=right]{изменение \\ состояния} (model.west);
	\draw[->] (model.east) -| node[label,pos=0.7,right,align=left]{уведомление \\ об изменениях} (view.south);
	\node(_tmp) at ($ (model.north east) + (-1em, 0) $) [coordinate] {};
	\draw[<-] (_tmp) -- node[label,left,align=right] {запрос состояния} (_tmp |- view.south);
\end{tikz*}

			
			\vspace{1ex}
			\figureexpl{Общая модель архитектуры MVC}
		\end{center}
	}
	
	\frame{
		\frametitle{MVC — описание}
	
		\begin{center}\begin{tabular}{rp{0.75\textwidth}}
			\textbf{Описание:} & \raggedright Отделяет представление данных и взаимодействие с~пользователем от~хранимых данных. \cr
			\textbf{Применение:} & \raggedright если необходимо несколько способов взаимодействия и/или~отображения данных. \cr
			\textbf{Примеры:} & \raggedright веб-приложения; графические приложения (напр., 
				библиотеки Qt или~Android Development Kit). \cr
			\textbf{Преимущества:} & \raggedright независимость данных от их представления; изменение данных в~одном представлении 
				обновляет другие представления. \cr
			\textbf{Недостатки:} & \raggedright избыточное усложнение системы, если модель данных и~взаимодействия ограничены.
		\end{tabular}\end{center}
	}
	
	\frame{
		\frametitle{MVC — пример}
		
		\begin{center}
			\begin{tikz*}[%
	every node/.style={align=center},
	block/.style={rectangle split,draw,rectangle split parts=2,align=center},
	every two node part/.style={font=\small,align=left},
	label/.style={font=\footnotesize}
]
	\node(ctrl) [block] {%
		\textbf{Контроллер}
		\nodepart{two}
		Обработка HTTP-запросов; \\
		системная логика; \\
		проверка данных форм.
	};
	\node(view) [block,right=8.5em of ctrl] {%
		\textbf{Представление}
		\nodepart{two}
		Создание веб-страниц \\
		на основе шаблонов; \\
		отображение форм.
	};
	\node(model) [block,below right=8em and 4.25em of ctrl,anchor=center] {%
		\textbf{Модель}
		\nodepart{two}
		Взаимодействие с СУБД; \\
		логика предметной области.
	};

	\draw[->] (ctrl) -- node[label,below]{выбор представления} (view);
	\draw[->] (ctrl.south) |- node[label,pos=0.3,left,align=right]{запросы на изменение/ \\ создание/удаление} (model.west);
	\draw[->] (model.east) -| node[label,pos=0.7,right,align=left]{уведомление \\ об изменениях} (view.south);
	\node(_tmp) at ($ (model.north east) + (-1em, 0) $) [coordinate] {};
	\draw[<-] (_tmp) -- node[label,left,align=right] {запрос состояния} (_tmp |- view.south);

	\node(browser) [ellipse,draw,above right=3em and 4em of ctrl,anchor=center] {Веб-браузер};
	\draw[->] (view) to (browser);
	\draw[->] (browser) -- node[label,right] {польз. события} (ctrl);
\end{tikz*}

			
			\vspace{1ex}
			\figureexpl{Архитектура MVC для веб-приложений}
		\end{center}
	}
	
	\subsection{Многослойная архитектура}\label{sec:layered}
	
	\frame{
		\frametitle{Многослойная архитектура}
		
		\begin{center}
			\begin{tikz*}[%
	every node/.style={rectangle,draw,align=center,minimum width=22em,minimum height=3em}
]
	\node(ui) {Пользовательский интерфейс};
	\node(hi) [below=of ui] { Управление интерфейсом; \\ аутентификация и авторизация};
	\node(low) [below=of hi] {Логика предметной области и приложения; \\ общесистемные утилиты};
	\node(sys) [below=of low] {Поддержка системы (ОС, СУБД, …)};

	\node(_tmp) [coordinate] at ($ (ui.south) + (-2em, 0) $) {};
	\draw[->] (ui.south -| _tmp) to (hi.north -| _tmp);
	\draw[->] (hi.south -| _tmp) to (low.north -| _tmp);
	\draw[->] (low.south -| _tmp) to (sys.north -| _tmp);
	
	\node(_tmp) [coordinate] at ($ (ui.south) + (2em, 0) $) {};
	\draw[<-] (ui.south -| _tmp) to (hi.north -| _tmp);
	\draw[<-] (hi.south -| _tmp) to (low.north -| _tmp);
	\draw[<-] (low.south -| _tmp) to (sys.north -| _tmp);
\end{tikz*}

		
			\vspace{1ex}
			\figureexpl{Общая модель четырехслойной архитектуры ПО}
		\end{center}
	}
	
	\frame{
		\frametitle{Многослойная архитектура — описание}
		
		\begin{center}\begin{tabular}{rp{0.75\textwidth}}
			\textbf{Описание:} & \raggedright Выделяет в системе несколько слоев, представляющих различные уровни детализации. 
				Компоненты из каждого слоя используются в~следующем. \cr
			\textbf{Применение:} & \raggedright разработка на основе существующей системы; 
				разработка несколькими командами; высокие требования к защищенности. \cr
			\textbf{Примеры:} & операционные системы. \cr
			\textbf{Преимущества:} & \raggedright модульность системы (возможность замены отдельных слоев); 
				дублирование функциональности на разных уровнях для повышения отказоустойчивости. \cr
			\textbf{Недостатки:} & \raggedright сложность размежевания уровней; снижение производительности.
		\end{tabular}\end{center}
	}
	
	\subsection{Клиент-серверная архитектура}\label{sec:server}
	
	\frame{
		\frametitle{Клиент-серверная архитектура}
		
		\begin{center}
			\begin{tikz*}[%
	every node/.style={rectangle,draw,align=center,minimum height=3em},
	server/.style={rectangle split,rectangle split parts=2,rectangle split part align=center},
	every label/.style={draw=none,minimum height=0pt,font=\bfseries}
]
	\node(server1) [server,label=above:Сервер 1] {
		Реализация\strut{}
		\nodepart{two}
		Интерфейс\strut{}
	};
	\node(server2) [server,label=above:Сервер 2,right=of server1] {
		Реализация\strut{}
		\nodepart{two}
		Интерфейс\strut{}
	};
	\node(server-dots) [draw=none,right=of server2] {$\cdots$};
	\node(server3) [server,label=above:Сервер $n$,right=of server-dots] {
		Реализация\strut{}
		\nodepart{two}
		Интерфейс\strut{}
	};
	
	\node(middleware) [below=of $ (server1.south west)!0.5!(server3.south east) $,minimum width=30em,minimum height=3.5em] {};
	\node [above=0pt of middleware.south,draw=none,minimum height=0pt] {Промежуточный слой \engterm{middleware}};
	
	\node(_tmp) [coordinate,below=of middleware.south] {};
	\node(client2) at (server2.south |- _tmp) [anchor=north] {\textbf{Клиент 2}};
	\node(client-dots) at (server-dots.south |- _tmp) [draw=none,anchor=north] {$\cdots$};
	\node(client1) at (server1.south |- _tmp) [anchor=north] {\textbf{Клиент 1}};
	\node(client3) at (server3.south |- _tmp) [anchor=north] {\textbf{Клиент $m$}};
	
	\draw[<->] (server1.south) to (middleware.north -| server1.south);
	\draw[<->] (server2.south) to (middleware.north -| server2.south);
	\draw[<->] (server3.south) to (middleware.north -| server3.south);
	\draw[<->] (client1.north) to (middleware.south -| client1.north);
	\draw[<->] (client2.north) to (middleware.south -| client2.north);
	\draw[<->] (client3.north) to (middleware.south -| client3.north);
	
	\draw[<->,dotted,bend right] (server1.south |- middleware.north) to node(server-link)[coordinate]{} (server2.south |- middleware.north);
	\node<2> [note={(server-link)},below=1em of server-link,anchor=north west] {%
		Сервера могут обмениваться \\ данными между собой \\ прозрачно для клиентов.
	};
\end{tikz*}

		
			\vspace{1ex}
			\figureexpl{Общая модель клиент-серверной архитектуры}
		\end{center}
	}
	
	\frame{
		\frametitle{Клиент-серверная архитектура — описание}

		\begin{center}\begin{tabular}{rp{0.75\textwidth}}
			\textbf{Описание:} & \raggedright Функциональность системы организована в виде сервисов, 
				соответствующих различным серверам. \cr
			\textbf{Применение:} & \raggedright большой объем и распределенность данных; 
				необходимость удаленного доступа к данным; балансирование нагрузки. \cr
			\textbf{Примеры:} & \raggedright веб-приложения. \cr
			\textbf{Преимущества:} & \raggedright высокая доступность; снижение требований к клиентам. \cr
			\textbf{Недостатки:} & \raggedright уязвимость к DoS-атакам; потенциальное снижение производительности; 
				сложность управления.
		\end{tabular}\end{center}
	}
	
	\frame{
		\frametitle{Клиент-серверная архитектура — пример}
		
		\begin{center}
			\begin{tikz*}[%
	every node/.style={rectangle,draw,align=center,minimum height=3em},
	every label/.style={draw=none,minimum height=0pt,font=\bfseries}
]
	\node(server1){
		Сервер \\ статики
	};
	\node(server2) [right=of server1] {
		Веб-сервер
	};
	\node(server3) [right=of server2] {
		Сервер БД
	};
	
	\node(middleware) [below=of $ (server1.south west)!0.5!(server3.south east) $,minimum width=30em,minimum height=4.5em] {};
	\node(intranet) [dashed,below=of $ (server1.south west)!0.5!(server3.south east) $,minimum width=20em,minimum height=2em] 
		{Сеть хостинга};
	\node [above=0pt of middleware.south,draw=none,minimum height=0pt] {Интернет};
	
	\node(client2) [below=of middleware] {\textbf{Клиент 2:} \\ web-crawler};
	\node(client1) [left=of client2] {\textbf{Клиент 1:} \\ пользователь};
	\node(client3) [right=of client2] {\textbf{Клиент 3:} \\ веб-админ};
	
	\draw[<->] (server1.south) to (middleware.north -| server1.south);
	\draw[<->] (server2.south) to (middleware.north -| server2.south);
	\draw[<->] (server3.south) to (middleware.north -| server3.south);
	\draw[<->] (client1.north) to (middleware.south -| client1.north);
	\draw[<->] (client2.north) to (middleware.south -| client2.north);
	\draw[<->] (client3.north) to (middleware.south -| client3.north);
\end{tikz*}

			
			\vspace{1ex}
			\figureexpl{Клиент-серверная архитектура для веб-сервера}
		\end{center}
	}
	
	\subsection{Конвейерная архитектура}\label{sec:pipe}
	
	\frame{
		\frametitle{Конвейерная архитектура}
		
		\begin{center}
			\begin{tikz*}[%
	every node/.style={rectangle,draw,align=center,minimum height=3em},
	every label/.style={draw=none,minimum height=0pt,font=\bfseries}
]
	\node(input) [font=\bfseries] {Входные \\ данные};
	\node(filter1) [right=of input]{Фильтр 1};
	\node(filter2) [right=of filter1]{Фильтр 2};
	\node(filter3) [below=of $ (filter1.south east)!0.5!(filter2.south west) $,anchor=north]{Фильтр 3};
	\node(filter4) [above=of filter2]{Фильтр 4};
	\node(filter5) [right=of filter2]{Фильтр 5};
	\node(output1) [right=of filter5,font=\bfseries]{Выход 1};
	\node(output2) [right=of filter4,font=\bfseries]{Выход 2};
	
	\draw[->] (input) to (filter1);
	\draw[->] (filter1) to (filter2);
	\draw[->] (filter1.south) |- (filter3.west);
	\draw[->] (filter3.east) -| (filter2.south);
	\draw[<->] (filter2) to (filter4);
	\draw[->] (filter2) to (filter5);
	\draw[->] (filter5) to (output1);
	\draw[->] (filter4) to (output2);
\end{tikz*}

		
			\vspace{1ex}
			\figureexpl{Абстрактная модель конвейерной архитектуры \engterm{pipe and filter architecture}}
		\end{center}
	}
	
	\frame{
		\frametitle{Конвейерная архитектура — описание}
	
		\begin{center}\begin{tabular}{rp{0.75\textwidth}}
			\textbf{Описание:} & \raggedright Организация обработки данных в виде конвейера, 
				в котором каждый~компонент (фильтр) выполняет операции одного типа. \cr
			\textbf{Применение:} & \raggedright приложения, ориентированные на данные (обработка транзакций, интеллектуальный анализ); 
				большое количество однотипных данных. \cr
			\textbf{Примеры:} & \raggedright pipe в *NIX; обработка звука и видео; системы Map/Reduce. \cr
			\textbf{Преимущества:} & \raggedright хорошая масштабируемость; простота и понятность; 
				возможности повторного использования компонентов. \cr
			\textbf{Недостатки:} & \raggedright необходимость преобразования входных/выходных данных к~стандартному виду.
		\end{tabular}\end{center}
	}
	
	\frame{
		\frametitle{Конвейерная архитектура — пример}
	
		\textbf{Команды:} \code{\% ls -l | grep error | tee -a error-files.txt | less}
			
		\textbf{Эффект:} Выводит информацию о файлах, имя которых содержит \textit{error} с разбиением текста на~страницы, 
		и добавляет эту информацию в конец файла \textit{error-files.txt}.
		
		\vspace{1ex}
		\begin{center}	
			\begin{tikz*}[%
	node distance=2em and 3em,
	every node/.style={rectangle,draw,align=center,minimum height=2.5em,minimum width=4em},
	data/.style={rounded rectangle,minimum width=7.5em},
	label/.style={draw=none,minimum height=0pt,,minimum width=0pt,font=\footnotesize}
]
	\node(kb) [data]{Клавиатура};
	\node(ls) [right=of kb]{ls};
	\node(grep) [right=of ls]{grep};
	\node(tee) [right=of grep]{tee};
	\node(less) [right=of tee]{less};
	\node(console) [data,below=of grep]{Консоль};
	\node(file) [data,above=of tee]{error-files.txt};
	
	\draw[->] (kb) -- node[label,above]{in} (ls);
	\draw[->] (ls) -- node[label,above]{in/out} (grep);
	\draw[->] (grep) -- node[label,above]{in/out} (tee);
	\draw[->] (tee) -- node[label,above]{in/out} (less);
	\draw[->] (less) |- node[label,below,pos=0.7]{out} (console);
	\draw[->] (tee) -- node[label,left]{out,+a} (file);
	
	\draw[->] (ls) -- node[label,left]{err} (console);
	\draw[->] (grep) -- node[label,left]{err} (console);
	\draw[->] (tee) -- node[label,left]{err} (console);
	\draw[->] (less) -- node[label,below]{err} (console);
\end{tikz*}

			
			\vspace{1ex}
			\figureexpl{Пример конвейера для цепочки команд Linux с указанием направления стандартных потоков}
		\end{center}
	}
	
	\section{Заключение}

	\subsection{Выводы}
	
	\frame{
		\frametitle{Выводы}
		
		\begin{enumerate}
			\item
			Архитектура программной системы определяет ее организацию и разбиение на~компоненты. 
			На архитектуру влияют нефункциональные требования к системе (напр., производительность и надежность). 
			
			\vspace{0.5ex}
			\item
			Помимо составляющих системы, архитектура может определять распределение работ между разработчиками, 
			взаимодействие компонентов и их организацию в~распределенной системе. 
			Для представления архитектуры могут использоваться формальные (UML, ADL) и неформальные (диаграммы) методы.

			\vspace{0.5ex}			
			\item
			Опыт разработки архитектуры различных систем собран в архитектурные шаблоны. 
			Часто используемые шаблоны: MVC, многослойная архитектура, клиент-сервер и~конвейер.
		\end{enumerate}
	}
	
	\subsection{Материалы}
	
	\frame{
		\frametitle{Материалы}
		
		\begin{thebibliography}{9}
			\bibitem[2]{2}
			Sommerville, Ian
			\newblock Software Engineering.
			\newblock {\footnotesize Pearson, 2011. — 790 p.}
		
			\bibitem[1]{1}
			Лавріщева К.\,М. 
			\newblock Програмна інженерія (підручник). 
			\newblock {\footnotesize К., 2008. — 319 с.}
		\end{thebibliography}
	}
	
	\frame{
		\frametitle{}
		
		\begin{center}
			\Huge Спасибо за внимание!
		\end{center}
	}
\end{document}
