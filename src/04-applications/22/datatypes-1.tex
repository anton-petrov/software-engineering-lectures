\documentclass{a4beamer}
%% Lectures - common definitions

\usextensions{tikz}
\usetikzlibrary{shapes.multipart,shapes.callouts,shapes.geometric}
\input{fix-callouts.inc} % Fixes absolute positioning of rectangle callouts

\newif\ifbigpages \bigpagesfalse
\ifdim\paperwidth >20cm
	\bigpagestrue
\fi

\tikzset{%
	note/.style={rectangle callout,draw=none,callout pointer width=1em,%
		align=flush left,font=\footnotesize,inner sep=0.5em,%
		fill=blue!15,fill opacity=0.95,text opacity=1.0,callout absolute pointer=#1},
	node distance=2em and 2.75em
}
\ifbigpages
	% Scale all arrow tips by the factor of 2.5
	\let\old@pgf@arrow@call=\pgf@arrow@call
	\def\pgf@arrow@call#1{%
		\@tempdima=\pgflinewidth%
		\pgfsetlinewidth{2.5\pgflinewidth}%
		\old@pgf@arrow@call{#1}%
		\pgfsetlinewidth{\@tempdima}%
	}
	\def\pgfarrowsleftextend#1{\pgfmathsetlength{\pgf@xa}{1.5*#1}}
	\def\pgfarrowsrightextend#1{\pgfmathsetlength{\pgf@xb}{1.5*#1}}
\fi

%% Load listings package
\usepackage{listings}

%% Are we inside a comment?
\newif\iflstcomment \lstcommentfalse

\lstset{%
	tabsize=4,
	showstringspaces=false,
	basicstyle=\linespread{1.25}\ttfamily\small,
	keywordstyle=\bfseries,
	commentstyle=\lstcommentstyle,
	numbers=left,
	numberstyle=\footnotesize\color{gray},
	xleftmargin=2.5em,
	extendedchars=true,
	escapechar=\$,
	escapebegin=\iflstcomment\begingroup\lstcommentstyle\fi,
	escapeend=\iflstcomment\endgroup\fi
}

\def\lstcommentstyle{\color{gray}}

\lst@AddToHook{AfterBeginComment}{\global\lstcommenttrue}
\let\orig@lst@EndComment=\lst@EndComment
\def\lst@EndComment{\global\lstcommentfalse\orig@lst@EndComment}
\lst@AddToHookAtTop{EOL}{%
	\lst@ifLmode\global\lstcommentfalse\fi% XXX Sloppy way to determine comment end
}

%% Python with docstrings treated as comments
\lstdefinelanguage[doc]{python}[]{python}{%
	deletestring=[s]{"""}{"""},%
	morecomment=[s]{"""}{"""}%
}%

%% JavaScript language
\lstdefinelanguage{javascript}%
	{morekeywords={break,case,catch,%
		const,constructor,continue,default,do,else,false,%
		finally,for,function,if,in,instanceof,%
		new,null,prototype,%
		return,switch,this,throw,%
		true,try,typeof,var,while},%
	sensitive,%
	morecomment=[l]//,%
	morecomment=[s]{/*}{*/},%
	morestring=[b]",%
	morestring=[b]',%
}[keywords,comments,strings]%

%% C# language (4.0?)
\lstdefinelanguage{csharp}%
	{morekeywords={abstract,as,%
		base,bool,byte,case,catch,char,%
		checked,class,const,continue,%
		decimal,default,delegate,do,double,%
		else,enum,event,explicit,extern,%
		false,finally,fixed,float,for,foreach,%
		goto,if,implicit,in,int,interface,%
		internal,is,lock,long,%
		namespace,new,null,object,operator,out,%
		override,params,private,protected,public,%
		readonly,ref,return,sbyte,sealed,%
		short,sizeof,stackalloc,static,string,%
		struct,switch,this,throw,true,try,%
		typeof,uint,ulong,unchecked,unsafe,ushort,%
		using,virtual,void,volatile,while%
	},%
	sensitive,%
	morecomment=[l]//,%
	morecomment=[s]{/*}{*/},%
	morestring=[b]",%
	morestring=[b]',%
}[keywords,comments,strings]%

%% Translation for fact environment
\deftranslation[to=russian]{Fact}{Наблюдение}

%% Inline code snippets
\def\code#1{\texttt{#1}}
\def\codekw#1{\code{\textbf{#1}}}

\def\quoteauthor#1{\par\footnotesize\upshape\hfill—~#1}

%% English term
\def\engterm#1{(англ. \textit{#1})}
%% Term with explanation below (to be used in diagrams)
\def\termwithexpl#1#2{#1\strut{}\\\small\color{gray}(\textit{#2})\strut{}}
%% External link
\def\extlink#1#2{\href{#1}{\color[rgb]{0.7,0.7,1.0}\dashbar{#2}}}
%% Internal link
\def\inlink#1#2{\hyperlink{#1}{\color[rgb]{0.7,0.7,1.0}\dashbar{#2}}}
%% Explanation for a list item
\def\itemexpl#1{\begingroup\small\vspace{0.75ex}#1\par\endgroup}




\lecturetitle{Программная инженерия. Лекция №18 — Интерфейсы и типы данных (часть 1)}
\title[Типы данных 1]{Интерфейсы и типы данных (часть 1)}
\author{Алексей Островский}
\institute{\small{Физико-технический учебно-научный центр НАН Украины}\vspace{2ex}}
\date{9 апреля 2015 г.}

\begin{document}
	\frame{\titlepage}

	\section{Вступление}

	\frame{
		\frametitle{Типы данных в программировании}

		Интерпретация строки в C++ как разных типов данных:
		\begin{center}
			\begin{tabular}{|p{0.2\textwidth}|p{0.35\textwidth}|}
				\hline
				\textbf{Тип} & \textbf{Представление} \cr
				\hline
				std::string & "Datatype" \cr
				\hline
				signed char[8] & 68 97 116 97 116 121 112 101 \cr
				\hline
				char[8] & D a t a t y p e \cr
				\hline
				short[4] & 24900 24948 31092 25968 \cr
				\hline
				int[2] & 1635017028 1701869940 \cr
				\hline
				long & 7309475735980499268 \cr
				\hline
				float[2] & 2.81751e+20 7.09755e+22 \cr
				\hline
				double & 4.27256e+180 \cr
				\hline
			\end{tabular}
		\end{center}

		\vspace{2ex}
		Каков смысл заданной последовательности байт? 
		Какие операции к ней применимы?
	}

	\subsection{Система типов}

	\frame{
		\frametitle{Система типов}

		\begin{Definition}
			\textbf{Тип данных} — множество представимых в машинной памяти величин, 
			которые характеризуются своими свойствами и производимыми над ними операциями.
		\end{Definition}

		\vspace{1ex}
		\begin{Definition}
			\textbf{Система типов языка программирования} — совокупность правил, определяющих свойство типа 
			для конструкций языка (переменных, выражений, функций, модулей, …).
		\end{Definition}

		\vspace{1ex}
		\textbf{Цели системы типов:}
		\begin{itemize}
			\item
			\textbf{основная:} определение интерфейсов для взаимодействия с частями программы и~обеспечение 
			их~корректного использования с~целью устранения ошибок;

			\item обеспечение функциональности языка (напр., динамическая диспетчеризация в~ООП);
			\item повышение доступности программы для понимания.
		\end{itemize}
	}

	\subsection{Типы данных и интерфейсы}

	\frame{
		\frametitle{Типы данных и интерфейсы}

		\textbf{Роль ТД в определении интерфейсов:}
		\begin{itemize}
			\item
			\textbf{Функциональные ЯП:} \\ 
			спецификация допустимых аргументов и возвращаемых результатов для определяемых функций.

			\vspace{1ex}
			\item
			\textbf{Процедурные ЯП:} \\ 
			то же + спецификация глобального состояния программы.

			\vspace{1ex}
			\item
			\textbf{Объектно-ориентированные ЯП:} \\
			спецификация классов / объектов в виде набора допустимых методов для работы с~инкапсулированными данными.
		\end{itemize}
	}

	\frame{
		\frametitle{Интерфейсы классов}

		\textbf{Области видимости интерфейса:}
		\begin{itemize}
			\item
			публичные \engterm{public} операции — доступны всем клиентам;
			\item
			защищенные \engterm{protected} операции — доступны классу и подклассам;
			\item
			частные \engterm{private} операции — доступны только классу;
			\item
			внутренние \engterm{package-wide} операции 
			— доступны классам, находящимся в~одном модуле с~данным (методы без спецификатора видимости в~Java, \textbf{internal} в~C\#).
		\end{itemize}

		\vspace{1ex}
		\textbf{Способ определения:}
		\begin{itemize}
			\item методы;
			\item свойства;
			\item точки расширения (напр., события).
		\end{itemize}
	}

	\section{ISO 11404}

	\frame{
		\frametitle{Стандарт ISO/IEC 11404}

		\textbf{Стандарт ISO/IEC 11404} «Типы данных общего назначения» \engterm{General purpose datatypes} 
		— содержит описание типов данных, независимое от языка программирования.

		\vspace{1.5ex}
		\textbf{Содержание стандарта:}
		\begin{itemize}
			\item
			спецификация для формального описания типов данных, независимого от~реализации (ЯП~и~среды выполнения);
			\item
			определение базовых типов данных (содержимое + допустимые операции);
			\item
			способы генерации новых типов данных из базовых.
		\end{itemize}
	}

	\subsection{Классификация ТД}

	\frame{
		\frametitle{Классификация типов данных}

		\textbf{Типы данных по способу определения:}
		\begin{itemize}
			\item
			примитивные — задаваемые через аксиомы, т.\,е. независимые от других определений типов данных;
			\item
			генерируемые \engterm{generated} — определяемые через другие типы данных.
		\end{itemize}

		\vspace{1ex}
		\textbf{По структуре:}
		\begin{itemize}
			\item
			атомарные — ТД, значения которых семантически неделимы (напр., целые числа);

			\vspace{1ex}
			\item
			агрегационные — ТД со сложной внутренней структурой (напр., массивы). Подвиды:
			\begin{itemize}
				\item однородные / неоднородные — компоненты структуры имеют один / разные типы;
				\item фиксированный / переменный размер;
				\item уникальность и упорядоченность компонент;
				\item метод доступа (произвольный / последовательный).
			\end{itemize}
		\end{itemize}
	}

	\frame{
		\frametitle{Общие свойства типов}

		\textbf{Операция сравнения} ($=$):
		\begin{itemize}
			\item
			универсальная: $\forall a, b \in T \quad (a = b) \lor (a \neq b)$;
			\item
			непротиворечивая: $\nexists a, b \in T \quad (a = b) \land (a \neq b)$;
			\item
			тривиальная: $\forall a \in T \quad a = a$;
			\item
			коммутативная: $\forall a, b \in T \quad (a = b) \Leftrightarrow (b = a)$;
			\item
			транзитивная: $\forall a, b, c \in T \quad (a = b) \land (b = c) \Rightarrow (a = c)$.
		\end{itemize}

		{\small%
			\textbf{Примечание.} Во многих ЯП сущесвуют две операции сравнения — по~содержимому и~по~ссылке 
			(\code{.equals()} и~\code{==}~в~Java, \code{==} и~\codekw{is}~в~Python).
		}

		\vspace{1.5ex}
		\textbf{Необязательная операция упорядочивания ($\leq$):}
		\begin{itemize}
			\item
			универсальная: $\forall a, b \in T \quad (a \leq b) \lor (a \leq b)$;
			\item
			связанная с равенством: $\forall a, b \in T \quad (a \leq b) \land (a \leq b) \Rightarrow (a = b)$;
			\item
			транзитивная: $\forall a, b, c \in T \quad (a \leq b) \land (b \leq c) \Rightarrow (a \leq c)$.
		\end{itemize}
	}

	\frame{
		\frametitle{Общие свойства типов}

		\textbf{(Не)ограниченность:}
		\begin{itemize}
			\item
			Тип данных $T$ называется ограниченным сверху, если в нем задана операция упорядочивания 
			и~существует значение $U \in T$: $\forall a \in T \quad a \leq U$.
			\item
			Ограниченность снизу: $\exists L \in T: \forall a \in T \quad L \leq a$.
		\end{itemize}

		\vspace{1ex}
		\textbf{Аппроксимация:}
		\begin{itemize}
			\item
			точные типы;
			\item
			типы, в которых вычислительная модель налагает ограничения на точность операций 
			(напр., числа с плавающей запятой согласно стандарту IEEE 754).
		\end{itemize}

		\vspace{1ex}
		\textbf{Мощность:}
		\begin{itemize}
			\item
			конечное множество значений;
			\item
			счетное множество значений;
			\item
			несчетное множество значений (требуют аппроксимации, т.\,к. представление в~ЭВМ 
			несчетного множества значений невозможно).
		\end{itemize}
	}

	\subsection{Примитивные типы данных}

	\frame{
		\frametitle{Примитивные типы данных}

		\begin{itemize}
			\item
			\textbf{Булев тип.} \\
			\codekw{bool} в~C\#; \codekw{boolean} в~Java; \codekw{bool} в~Python.

			\vspace{0.5ex}\item
			\textbf{Состояние} \engterm{state} — конечное множество различимых неупорядоченных значений.

			\vspace{0.5ex}\item
			\textbf{Перечисление} — конечное множество упорядоченных значений. \\
			\codekw{enum} в~C / C++, C\#, Java.

			\vspace{0.5ex}\item
			\textbf{Символ.} \\ \codekw{char} в~C / C++, C\#, Java.

			\vspace{0.5ex}\item
			\textbf{Ординальный тип} \engterm{ordinal}: \code{"first"}, \code{"second"}, \code{"third"}, …

			\vspace{0.5ex}\item
			\textbf{Дата и время.} \\ \code{DATETIME} в~SQL.

			\vspace{0.5ex}\item
			\textbf{Целые числа.} \\
			$\sim$~BigInteger в~Java, \codekw{long} в~Python.
		\end{itemize}
	}

	\frame{
		\frametitle{Примитивные типы данных (продолжение)}

		\begin{itemize}
			\item
			\textbf{Рациональные числа.} \\
			\code{fractions.Fraction} в~Python.

			\vspace{0.5ex}\item
			\textbf{Масштабированный тип} \engterm{scaled} — рациональные числа с фиксированным знаменателем,
			напр., валютный тип (знаменатель равен 100). \\
			$\sim$~BigDecimal в~Java, \code{decimals.Decimal} в~Python.

			\vspace{0.5ex}\item
			\textbf{Действительные числа.} \\
			\codekw{float} и \codekw{double} в~С++, C\#, Java, 
			\codekw{float} в~Python.

			\vspace{0.5ex}\item
			\textbf{Комплексные числа.} \\
			\codekw{complex} в~Python.

			\vspace{0.5ex}\item
			\textbf{Пустой тип} \engterm{void} — объект, необходимый с точки зрения синтаксиса или~семантики, 
			не~содержащий информации. \\
			\codekw{void} в~C++, C\#, Java.
		\end{itemize}
	}

	\subsection{Генерируемые ТД}

	\frame{
		\frametitle{Подтипы}

		\textbf{Подтипы типов данных:}
		\begin{itemize}
			\item диапазон значений;
			\vspace{0.5ex}\item перечисление значений из базового типа;
			\vspace{0.5ex}\item исключение значений из базового типа;
			\vspace{0.5ex}\item ограничение размера базового типа-коллекции;
			\vspace{0.5ex}\item расширение базового типа.
		\end{itemize}

		\vspace{1ex}
		\textbf{Примеры:} целочисленные типы в ЯП, напр., \codekw{short}, \codekw{int}
		и~\codekw{long} в~C++, C\#, Java; \codekw{int}~в~Python.
	}

	\frame{
		\frametitle{Генерируемые типы данных}

		\textbf{Генерируемые типы данных:}
		\begin{itemize}
			\item
			\textbf{Выбор} \engterm{choice} — значение типа принадлежит одному из конечного набора базовых типов, 
			в~зависимости от~состояния. \\
			\codekw{union} в С / С++.

			\vspace{0.5ex}\item
			\textbf{Указатель} на другой тип. \\ 
			Указатели и ссылки в С / С++.

			\vspace{0.5ex}\item
			\textbf{Процедурный тип}, задаваемый типами и направлением аргументов (входные или~выходные) и~типом возвращаемого значения. \\
			Функции в функциональных ЯП; в Python, JavaScript (без выделения подтипов).
		\end{itemize}
	}

	\subsection{Агрегационные типы данных}

	\frame{
		\frametitle{Агрегационные типы данных}

		\begin{itemize}
			\item
			\textbf{Запись} \engterm{record} — компоненты (значения различных типов) 
			определяются с~помощью именованных полей.

			\vspace{0.5ex}\item
			\textbf{Класс} \engterm{class} — компоненты (значения различных типов, а~также методы) 
			определяются с помощью именованных полей; определены наследование и~перегрузка компонентов.

			\vspace{0.5ex}\item
			\textbf{Множество} \engterm{set} — имплементация математического понятия подмножества 
			пространства базового типа.

			\vspace{0.5ex}\item
			\textbf{Мешок} \engterm{bag} — неупорядоченный набор с~повторениями значений базового типа.
		\end{itemize}
	}

	\frame{
		\frametitle{Агрегационные типы данных (продолжение)}

		\begin{itemize}
			\item
			\textbf{Последовательность} \engterm{sequence} — упорядоченный набор значений базового типа с~повторениями 
			(упорядочивание задано извне).
			\begin{itemize}
				\item
				\textbf{Строка} \engterm{string} — последовательность символов.
			\end{itemize}

			\vspace{0.5ex}\item
			\textbf{Массив} \engterm{array} — отображение между конечным пространством индексов 
			(декартово произведение одного или более конечных типов) и~значениями типа элементов.

			\vspace{0.5ex}\item
			\textbf{Таблица} \engterm{table} — конечное множество ассоциаций, заданное на декартовом произведении базовых типов 
			($\sim$ реляционная таблица SQL).
		\end{itemize}
	}

	\subsection{Примеры}

	\frame{
		\frametitle{Примеры определений ТД}

		\begin{itemize}
			\item
			Целочисленный тип со знаком, занимающий 4 байта памяти:

			\code{\textbf{type} Int = \textbf{new integer range} (-2147483648 .. 2147483647)}

			\vspace{0.5ex}
			\item
			Строка символов Unicode:

			\code{\textbf{type} String = \textbf{new sequence of character} (\{ iso standard 10646 \})}

			\vspace{0.5ex}
			\item
			Дерево:

			\code{%
			\textbf{type} Tree = \textbf{record} (\\
			\quad label: String, \\
			\quad branches: \textbf{set of} (Tree) \\
			)}

			\vspace{0.5ex}
			\item
			Интерфейс процедуры для вычисления чисел Фибоначчи:

			\code{\textbf{type} Fib = \textbf{procedure} ( \textbf{in} n: Int \textbf{selecting} (0 .. *) ) \textbf{returns} ( \textbf{integer} )}
		\end{itemize}
	}

	\section{Иерархия ТД}

	\subsection{Система ТД с изолированными типами}

	\frame{
		\frametitle{Иерархия типов данных (Java)}

		\begin{figure}
			\begin{tikz*}[%
	every node/.style={rectangle,align=center,minimum width=7.5em,minimum height=3em}
]
	\node(Object) {\textbf{Базовый класс} \\ (Object)};
	\node(wrapper) [above=of Object] {Обертки \\ (Boolean, Byte, …, \\ Float, Double)};
	\node(primitive) [right=of wrapper] {Примитивные ТД: \\ 
		\codekw{boolean}, \codekw{byte}, \codekw{char}, \codekw{short}, \\
		\codekw{int}, \codekw{long}, \codekw{float}, \codekw{double}};

	\node(array) [left=of Object] {Массивы};
	\node(String) [left=of wrapper] {Строки \\ (String)};
	\node(Throwable) [right=of Object] {Прерывания \\ (Throwable)};
	\node(Error) [right=of Throwable] {Ошибки \\ (Error)};
	\node(Exception) [below=of Throwable] {Исключения \\ (Exception)};

	\node(Class) [below left=of Object] {Метаклассы \\ (Class<?>)};

	\node(intf) [below=of Object] {Интерфейсы};
	\node(annot) [below left=2em and -1em of intf] {Аннотации};
	\node(func) [below right=2em and -1em of intf] {Функциональные \\ интерфейсы \\ (в Java 8)};

	\draw[->] (Object) -- (array);
	\draw[->] (Object) -- (String);
	\draw[->] (Object) -- (Throwable);
	\draw[->] (Object) -- (Class);
	\draw[->] (Throwable) -- (Error);
	\draw[->] (Throwable) -- (Exception);
	\draw[->] (Object) -- (wrapper);
	\draw[<->,dashed] (wrapper) -- (primitive);
	\draw[->,dashed] (Object) -- (intf);
	\draw[->] (intf) -- (annot);
	\draw[->] (intf) -- (func);
\end{tikz*}

			\caption{В C / C++, Java, PHP примитивные типы объектов изолированы (не являются потомками других типов).}
		\end{figure}
	}

	\frame{
		\frametitle{Боксинг типов данных}

		\begin{Definition}
			\textbf{Боксинг} \engterm{boxing} — преобразование примитивного~ТД 
			к~соответствующему \inlink{p:ref-types}{ссылочному типу данных} и~обратно.
		\end{Definition}

		\vspace{1ex}
		\textbf{Пример:}
		\lstinputlisting[language=java]{code-boxing.java}
	}

	\subsection{Унифицировання система типов}

	\frame{
		\frametitle{Иерархия типов данных (Python)}

		\begin{figure}
			\begin{tikz*}[%
	every node/.style={rectangle,align=center,minimum width=7.5em,minimum height=3em}
]
	\node(object) {\textbf{Базовый класс} \\ (object)};
	\node(number) [above=of object] {Числовые ТД \\ (\codekw{int}, \codekw{long}, \codekw{float}, \codekw{complex})};
	\node(set) [left=of object] {Множества \\ (\codekw{set}, \codekw{frozenset})};
	\node(dict) [below left=of object] {Ассоциации \\ (\codekw{dict})};
	\node(seq) [below=of dict] {Последовательности \\ (\codekw{tuple}, \codekw{list}, \codekw{xrange})};
	\node(str) [right=of seq] {Строки \\ (\codekw{str}, \codekw{unicode})};
	\node(bool) [above left=of object] {Булев тип \\ (\codekw{bool})};
	\node(module) [above right=of object] {Модули \\ (\code{module})};
	\node(function) [right=of object] {Функции \\ (\code{function})};
	\node(type) [below=of object] {Метакласс \\ (\codekw{type})};
	\node(Exception) [below right=of object] {Исключения \\ (Exception)};

	\draw[->] (object) -- (number);
	\draw[->] (object) -- (seq);
	\draw[->] (object) -- (set);
	\draw[->] (object) -- (dict);
	\draw[->] (object) -- (bool);
	\draw[->] (object) -- (module);
	\draw[->] (object) -- (function);
	\draw[->] (object) -- (type);
	\draw[->] (seq) -- (str);
	\draw[->] (object) -- (Exception);
\end{tikz*}

			\caption{Python — ЯП с унифицированной системой типов: все ТД наследуются от базового.}
		\end{figure}
	}

	\frame{
		\frametitle{Иерархия типов данных (Python)}

		\lstinputlisting[language=python,firstline=4]{code-inheritance.py}
	}

	\subsection{Интерфейс типов данных в ООП}

	\frame{
		\frametitle{Интерфейс базового типа данных}

		\textbf{Методы класса Object (Java):}
		\lstinputlisting[language=java]{code-Object.java}
	}

	\frame{
		\frametitle{Интерфейс типов данных в ООП}

		\textbf{Спецификация интерфейса объектов:}
		\begin{itemize}
			\item
			Методы.

			\vspace{0.5ex}
			\item
			Свойства (синтаксический сахар для методов).

			\vspace{0.5ex}
			\item
			События (свойства процедурного типа).

			\vspace{0.5ex}
			\item
			Перегрузка операций помимо сравнения (напр., арифметических):
			\begin{itemize}
				\item \codekw{operator} в C++; 
				\item методы \code{\_\_lt\_\_}, \code{\_\_le\_\_} и т.\,д. в~Python.
			\end{itemize}

			\vspace{0.5ex}
			\item
			Приведение к другим ТД помимо строкового:
			\begin{itemize}
				\item \code{\textbf{operator} <Type>()} в С++; 
				\item методы \code{\_\_int\_\_}, \code{\_\_float\_\_} и т.\,д. в~Python.
			\end{itemize}

			\vspace{0.5ex}
			\item
			Манипуляции с содержимым объекта, напр., методы для изменения логики доступа к~полям:
			\begin{itemize}
				\item методы \code{\_\_getattr\_\_}, \code{\_\_setattr\_\_}, \code{\_\_delattr\_\_} в~Python; 
				\item методы \code{\_\_get} и \code{\_\_set} в PHP.
			\end{itemize}
		\end{itemize}
	}

	\section{Заключение}

	\subsection{Выводы}
	
	\frame{
		\frametitle{Выводы}

		\begin{enumerate}
			\item
			Система типов данных — основной инструмент спецификации интерфейсов в~языках программирования. 
			Использование типов данных позволяет повысить понимание программы и~служит средством верификации программы.

			\vspace{0.5ex}
			\item
			Стандарт ISO/IEC 11404 определяет классификацию типов данных и~базовые типы, независимые от~языка программирования. 
			В~этом стандарте также даны способы построения новых типов данных на основе базовых.

			\vspace{0.5ex}
			\item
			ЯП имплементируют систему типов данных в различном объеме и с различными средствами для~ее~расширения. 
			В объектно-ориентированных ЯП система типов данных представляет собой иерархию с~одним или~несколькими корневыми классами.
		\end{enumerate}
	}
	
	\subsection{Материалы}
	
	\frame{
		\frametitle{Материалы}
		
		\begin{thebibliography}{9}
			\bibitem[1]{1}
			Лавріщева К.\,М. 
			\newblock Програмна інженерія (підручник). 
			\newblock {\footnotesize К., 2008. — 319 с.}

			\bibitem[2]{2}
			ISO / IEC
			\newblock General Purpose Datatypes.
			\newblock {\footnotesize \url{http://standards.iso.org/ittf/PubliclyAvailableStandards/c039479_ISO_IEC_11404_2007(E).zip}}

			\bibitem[3]{3}
			Документация Python
			\newblock Data Model.
			\newblock {\footnotesize \url{https://docs.python.org/2/reference/datamodel.html}}
		\end{thebibliography}
	}
	
	\frame{
		\frametitle{}
		
		\begin{center}
			\Huge Спасибо за внимание!
		\end{center}
	}

	\subsection{Приложения}

	\frame{
		\frametitle{Приложение. Ссылочные типы данных}
		\label{p:ref-types}

		\begin{Definition}
			\textbf{Ссылочный тип} \engterm{reference type} — тип данных, ссылающийся на область памяти 
			($\sim$указатель с автоматическим разыменованием).
		\end{Definition}

		\vspace{0.5ex}
		\begin{center}\small
			\begin{tabular}{|p{0.4\textwidth}|p{0.4\textwidth}|}
				\hline
				\centering\textbf{Типы, содержащие значение} \engterm{value type} & \centering\textbf{Ссылочные типы} \cr
				\hline
				\raggedright присвоение — копирование данных & \raggedright присвоение — копирование ссылки \cr
				\hline
				\raggedright сравнение по значению & \raggedright сравнение по ссылке или значению \cr
				\hline
				\raggedright размещаются в стеке & \raggedright размещаются в куче \cr
				\hline
			\end{tabular}

			\vspace{3ex}
			\begin{tabular}{|p{0.2\textwidth}|p{0.35\textwidth}|p{0.25\textwidth}|}
				\hline
				\centering\textbf{ЯП} & \centering\textbf{Типы, содержащие значение} 
					& \centering\textbf{Ссылочные типы} \cr
				\hline
				Java & примитивные типы & все остальные \cr
				\hline
				C\# & примитивные, структуры (\codekw{struct}) & классы (\codekw{class}) \cr
				\hline
				Python & \raggedright числовые, булев (размещаются в~куче) & все остальные \cr
				\hline
				Функциональные & все & $-$ \cr
				\hline
			\end{tabular}
		\end{center}
	}

	\frame{
		\frametitle{Приложение. Агрегационные типы данных в ЯП}

		\begin{center}\small
			\begin{tabular}{|p{0.2\textwidth}|p{0.3\textwidth}|p{0.3\textwidth}|}
				\hline
				\centering\textbf{Тип} & \centering\textbf{Java} & \centering\textbf{Python} \cr
				\hline
				Массив & \raggedright \code{T[]} (одномерный), \code{T[]…[]} (многомерный)
					& \raggedright Есть в модулях \code{array} и~\code{numpy} \cr
				\hline
				Строка & \code{String} & \code{str}, \code{unicode} \cr
				\hline
				Последовательность & \code{List<T>}
					& \raggedright \code{list} (изменяемая), \code{tuple} (неизменяемая) \cr
				\hline
				Множество & \code{Set<T>} & \raggedright \code{set} (изменяемое), \code{frozenset} (неизменяемое) \cr
				\hline
				\raggedright Ассоциативная таблица & \code{Map<K, V>} & \code{dict} \cr
				\hline
			\end{tabular}
		\end{center}

		{\small%
			\textbf{Обозначения:} \code{T} — тип элементов; \code{K} — тип ключей, \code{V} — тип значений.
		}

		\vspace{1ex}
		\begin{itemize}
			\item
			\textbf{Статическая типизация:} тип элементов агрегационного типа указывается с~помощью шаблонов 
			(\codekw{template}~в~C++; generics в~Java, C\#).
			\item
			\textbf{Динамическая типизация:} агрегационные типы могут содержать произвольные элементы.
		\end{itemize}
	}

	\frame{
		\frametitle{Приложение. Функции как тип данных}

		\begin{Definition}
			Наличие \textbf{функций первого класса} в ЯП — работа с функциями как с другими типами данных, в~частности:
			\begin{itemize}
				\item
				поддержка передачи функций как параметров и~возврата~их как~результатов функций;
				\item
				присваивание функций переменным;
				\item
				(опционально) наличие анонимных функций, вложенных функций и~замыканий.
			\end{itemize}
		\end{Definition}

		\textbf{ЯП с функциями первого класса:} Python, JavaScript, функциональные ЯП (Scheme, Haskell, ML, Scala).

		\vspace{1ex}
		\textbf{Частичная поддержка функций как типа данных:}
		\begin{itemize}
			\item C\# (делегаты и анонимные функции); 
			\item Java 8 (функциональные интерфейсы и анонимные классы для замыканий); 
			\item C / C++ (указатели на функции).
		\end{itemize}
	}

	\frame{
		\frametitle{Приложение. Работа с функциями в Python}

		\lstinputlisting[language=python,firstline=4]{code-func.py}
	}
\end{document}

