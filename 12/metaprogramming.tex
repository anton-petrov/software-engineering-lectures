\documentclass{a4beamer}
%% Lectures - common definitions

\usextensions{tikz}
\usetikzlibrary{shapes.multipart,shapes.callouts,shapes.geometric}
\input{fix-callouts.inc} % Fixes absolute positioning of rectangle callouts

\newif\ifbigpages \bigpagesfalse
\ifdim\paperwidth >20cm
	\bigpagestrue
\fi

\tikzset{%
	note/.style={rectangle callout,draw=none,callout pointer width=1em,%
		align=flush left,font=\footnotesize,inner sep=0.5em,%
		fill=blue!15,fill opacity=0.95,text opacity=1.0,callout absolute pointer=#1},
	node distance=2em and 2.75em
}
\ifbigpages
	% Scale all arrow tips by the factor of 2.5
	\let\old@pgf@arrow@call=\pgf@arrow@call
	\def\pgf@arrow@call#1{%
		\@tempdima=\pgflinewidth%
		\pgfsetlinewidth{2.5\pgflinewidth}%
		\old@pgf@arrow@call{#1}%
		\pgfsetlinewidth{\@tempdima}%
	}
	\def\pgfarrowsleftextend#1{\pgfmathsetlength{\pgf@xa}{1.5*#1}}
	\def\pgfarrowsrightextend#1{\pgfmathsetlength{\pgf@xb}{1.5*#1}}
\fi

%% Load listings package
\usepackage{listings}

%% Are we inside a comment?
\newif\iflstcomment \lstcommentfalse

\lstset{%
	tabsize=4,
	showstringspaces=false,
	basicstyle=\linespread{1.25}\ttfamily\small,
	keywordstyle=\bfseries,
	commentstyle=\lstcommentstyle,
	numbers=left,
	numberstyle=\footnotesize\color{gray},
	xleftmargin=2.5em,
	extendedchars=true,
	escapechar=\$,
	escapebegin=\iflstcomment\begingroup\lstcommentstyle\fi,
	escapeend=\iflstcomment\endgroup\fi
}

\def\lstcommentstyle{\color{gray}}

\lst@AddToHook{AfterBeginComment}{\global\lstcommenttrue}
\let\orig@lst@EndComment=\lst@EndComment
\def\lst@EndComment{\global\lstcommentfalse\orig@lst@EndComment}
\lst@AddToHookAtTop{EOL}{%
	\lst@ifLmode\global\lstcommentfalse\fi% XXX Sloppy way to determine comment end
}

%% Python with docstrings treated as comments
\lstdefinelanguage[doc]{python}[]{python}{%
	deletestring=[s]{"""}{"""},%
	morecomment=[s]{"""}{"""}%
}%

%% JavaScript language
\lstdefinelanguage{javascript}%
	{morekeywords={break,case,catch,%
		const,constructor,continue,default,do,else,false,%
		finally,for,function,if,in,instanceof,%
		new,null,prototype,%
		return,switch,this,throw,%
		true,try,typeof,var,while},%
	sensitive,%
	morecomment=[l]//,%
	morecomment=[s]{/*}{*/},%
	morestring=[b]",%
	morestring=[b]',%
}[keywords,comments,strings]%

%% C# language (4.0?)
\lstdefinelanguage{csharp}%
	{morekeywords={abstract,as,%
		base,bool,byte,case,catch,char,%
		checked,class,const,continue,%
		decimal,default,delegate,do,double,%
		else,enum,event,explicit,extern,%
		false,finally,fixed,float,for,foreach,%
		goto,if,implicit,in,int,interface,%
		internal,is,lock,long,%
		namespace,new,null,object,operator,out,%
		override,params,private,protected,public,%
		readonly,ref,return,sbyte,sealed,%
		short,sizeof,stackalloc,static,string,%
		struct,switch,this,throw,true,try,%
		typeof,uint,ulong,unchecked,unsafe,ushort,%
		using,virtual,void,volatile,while%
	},%
	sensitive,%
	morecomment=[l]//,%
	morecomment=[s]{/*}{*/},%
	morestring=[b]",%
	morestring=[b]',%
}[keywords,comments,strings]%

%% Translation for fact environment
\deftranslation[to=russian]{Fact}{Наблюдение}

%% Inline code snippets
\def\code#1{\texttt{#1}}
\def\codekw#1{\code{\textbf{#1}}}

\def\quoteauthor#1{\par\footnotesize\upshape\hfill—~#1}

%% English term
\def\engterm#1{(англ. \textit{#1})}
%% Term with explanation below (to be used in diagrams)
\def\termwithexpl#1#2{#1\strut{}\\\small\color{gray}(\textit{#2})\strut{}}
%% External link
\def\extlink#1#2{\href{#1}{\color[rgb]{0.7,0.7,1.0}\dashbar{#2}}}
%% Internal link
\def\inlink#1#2{\hyperlink{#1}{\color[rgb]{0.7,0.7,1.0}\dashbar{#2}}}
%% Explanation for a list item
\def\itemexpl#1{\begingroup\small\vspace{0.75ex}#1\par\endgroup}




\lecturetitle{Программная инженерия. Лекция №12 — ЯП. Метапрограммирование.}
\title[Метапрограммирование]{Языки программирования. Метапрограммирование}
\author{Алексей Островский}
\institute{\small{Физико-технический учебно-научный центр НАН Украины}\vspace{2ex}}
\date{12 декабря 2014 г.}

\begin{document}
	\frame{\titlepage}

	\section[ЯП]{Языки программирования}

	\frame{
		\frametitle{Языки программирования}

		\begin{Definition}
			\textbf{Язык программирования} — формальный язык для записи инструкций, выполняемых исполнителем (чаще всего — ЭВМ).
		\end{Definition}

		\vspace{1ex}
		\textbf{Составляющие языка:}
		\begin{itemize}
			\item
			Синтаксис — набор правил, определяющих последовательности символов, 
			составляющих допустимую программу или ее фрагмент.

			\item
			Семантика — правила, определяющие значение различных конструкций языка.

			\item
			Система выполнения и стандартная библиотека — определяют функциональность, 
			доступную для всех имплементаций языка без подключения внешних модулей.
		\end{itemize}
	}

	\frame{
		\frametitle{Классификация языков программирования}

		\textbf{Способы классификации ЯП:}
		\begin{itemize}
			\item
			по способу представления инструкций — текстовые, графические, смешанные;
			\item
			по конкретизации инструкций — императивные (инструкции определяют последовательность действий), 
			декларативные (инструкции определяют конечный результат, но не способ его достижения);
			\item
			по поддерживаемым парадигмам программирования;
			\item
			по выразительной силе — полные и неполные по Тьюрингу;
			\item
			по области применения — общего назначения и предметно-ориентированные;
			\item
			по семантическим характеристикам, напр., особенностям системы типов данных.
		\end{itemize}
	}

	\subsection{Синтаксис ЯП}

	\frame{
		\frametitle{Синтаксис ЯП}

		\begin{center}
			\textbf{Синтаксис = форма ЯП.}
		\end{center}

		\vspace{1ex}
		\textbf{Описание синтаксиса:}
		\extlink{http://ru.wikipedia.org/wiki/\%D0\%A4\%D0\%BE\%D1\%80\%D0\%BC\%D0\%B0_\%D0\%91\%D1\%8D\%D0\%BA\%D1\%83\%D1\%81\%D0\%B0_\%E2\%80\%94_\%D0\%9D\%D0\%B0\%D1\%83\%D1\%80\%D0\%B0}{формы Бэкуса — Наура} (БНФ).

		\vspace{1ex}
		\textbf{Роль в обработке программы:}
		\begin{itemize}
			\item
			лексический анализ — группирование символов программы в лексемы (напр., числа, строки, ключевые слова);
			\item
			синтаксический анализ — построение дерева разбора на основе лексем.
		\end{itemize}

		\vspace{1ex}
		\textbf{Синтаксис БНФ:}
		\begin{center}
			\code{<выражение> ::= <альтернатива 1> | <альтернатива 2> | … |  <альтернатива n>}
		\end{center}

		\figureexpl{
			В БНФ может использоваться синтаксис регулярных выражений (\code{+} = повтор 1 или~более раз, 
			\code{[]} — выбор из~нескольких вариантов и~т.\,п.).
		}
	}

	\frame{
		\frametitle{Пример БНФ}

		\textbf{БНФ для списков:}
		\lstinputlisting[language={}]{code-bnf.yacc}

		\vspace{1ex}
		\textbf{Допустимые списки:}
		\begin{itemize}
			\item \code{[]}
			\item \code{[1, [2, 3], foo, Bar]}
			\item \code{[-123456789, [[[]], [+987654321, baZZ]]]}
		\end{itemize}
	}

	\subsection{Семантика ЯП}

	\frame{
		\frametitle{Семантика ЯП}

		\begin{center}
			\textbf{Семантика = содержание или смысл ЯП.}
		\end{center}

		\vspace{1ex}
		\textbf{Описание семантики:} математические и логические модели.

		\vspace{1ex}
		\textbf{Составляющие семантики:}
		\begin{itemize}
			\item
			статическая семантика — ограничение на инструкции, не формализуемые в БНФ 
			(напр., декларирование переменных, \extlink{http://en.wikipedia.org/wiki/Definite_assignment_analysis}{анализ инициализации}, 
			ограничения, связанные с системой типов);

			\item
			динамическая семантика — определение поведения отдельных конструкций языка, 
			преобразование дерева разбора в машинные инструкции;

			\item
			формальная семантика — анализ кода программы для получения ее~характеристик, напр., 
			для доказательства корректности (логика Хоара) или~оптимизации.
		\end{itemize}
	}

	\frame{
		\frametitle{Различия между синтаксисом и семантикой}

		\textbf{Синтаксические и семанические ошибки в Java:}
		\lstinputlisting[language=java]{code-semantics.java}
	}

	\subsection{Система типов}

	\frame{
		\frametitle{Система типов}

		\begin{Definition}
			\textbf{Система типов языка программирования} — совокупность правил, определяющих свойство типа 
			для конструкций языка (переменных, выражений, функций, модулей, …).
		\end{Definition}

		\vspace{1ex}
		\textbf{Цели системы типов:}
		\begin{itemize}
			\item
			\textbf{основная:} определение интерфейсов для взаимодействия с частями программы и~обеспечение 
			их~корректного использования с~целью устранения ошибок;

			\item обеспечение функциональности языка (напр., динамическая диспетчеризация в~ООП);
			\item рефлексия;
			\item оптимизация;
			\item повышение доступности программы для понимания.
		\end{itemize}
	}

	\frame{
		\frametitle{Классификация типов}

		\textbf{Категории типов в ЯП:}
		\begin{itemize}
			\item примитивные типы (булев тип, целые и вещ. числа);
			\item ссылки и указатели;
			\item составные типы (напр., массивы и записи);
			\item объекты;
			\item функции;
			\item типы типов (напр., \code{Class<?>} в Java).
		\end{itemize}

		\vspace{1ex}
		\textbf{Унификация типов в ООП:}
		\begin{itemize}
			\item языки, в которых все типы наследуются от базового типа (C\#, Python);
			\item языки с изолированными примитивными типами (C++, Java, JavaScript).
		\end{itemize}
	}

	\frame{
		\frametitle{Проверка типов в различных ЯП}

		\textbf{Java:}
		\lstinputlisting[language=java]{code-str-int.java}

		\vspace{0.5ex}
		\textbf{Python:}
		\lstinputlisting[language=python]{code-str-int.py}

		\vspace{0.5ex}
		\textbf{PHP:}
		\lstinputlisting[language=php,escapechar=\#]{code-str-int.php}

		\vspace{0.5ex}
		\textbf{C:}
		\lstinputlisting[language=c]{code-str-int.c}
	}

	\frame{
		\frametitle{Статическая и динамическая типизация}

		\textbf{Статическая типизация} — определение типов всех конструкций языка на этапе компиляции программы.

		\vspace{1ex}
		\textbf{Виды статической типизации:}
		\begin{itemize}
			\item
			явная — типы конструкций декларируются программистом (напр., при объявлении переменных);
			\item
			неявная — тип переменных выводится в процессе компиляции. Примеры:
			\begin{itemize}
				\item \code{\textbf{var} x = 5} в C\#; 
				\item \code{List<> list = \textbf{new} ArrayList<String>()} в Java 7+.
			\end{itemize}
		\end{itemize}

		\vspace{1ex}
		\textbf{ЯП со статической типизацией:} C++, Pascal, Java, C\#.

		\vspace{2.5ex}
		\textbf{Динамическая типизация} — определение типов некоторых конструкций и проверка соответствующих ограничений 
		во время выполнения программы.

		\vspace{1ex}
		\textbf{ЯП с динамической типизацией:} Python, PHP, Perl, JavaScript.
	}

	\frame{
		\frametitle{Утиная типизация}

		\begin{Definition}
			\textbf{Утиная типизация} \engterm{duck typing} — вид динамической типизации, при которой 
			корректность использования объекта определяется набором его методов и свойств, а~не~типом.
		\end{Definition}

		\vspace{1ex}
		\textbf{ЯП с утиной типизацией:}
		\begin{itemize}
			\item языки с ООП на основе прототипов (JavaScript, Lua);
			\item Python; 
			\item Smalltalk.
		\end{itemize}

		\vspace{1ex}
		\textbf{Пример динамической не утиной типизации:} подсказки типов \engterm{data hinting} в~PHP.
	}

	\frame{
		\frametitle{Сильная и слабая типизация}

		\textbf{Типобезопасность} \engterm{type safety} — предотвращение языком программирования ошибок 
		согласования типов (напр., интерпретации целого числа \code{int} как вещественного \code{float}).

		\vspace{1ex}
		\textbf{Безопасность памяти} \engterm{memory safety} — предотвращение ЯП доступа к~оперативной памяти, 
		не выделенной в ходе выполнения программы.

		\vspace{2.5ex}
		\textbf{Сильная типизация} — высокая степень типобезопасности и~безопасности памяти 
		(напр., отсутствие неявных приведений типов, указателей).

		\vspace{0.5ex}
		\textbf{Примеры ЯП с сильной типизацией:} Python, Java, функциональные ЯП.

		\vspace{1.5ex}
		\textbf{Слабая типизация} — возможность обхода безопасности типов и~памяти 
		с~помощью конструкций ЯП (напр., \code{void*} в С/С++).

		\vspace{0.5ex}
		\textbf{Примеры ЯП со слабой типизацией:} C, C++, Visual Basic.
	}

	\frame{
		\frametitle{Полиморфизм}

		\begin{Definition}
			\textbf{Полиморфизм} — использование единого интерфейса для сущностей различных типов.
		\end{Definition}

		\vspace{1ex}
		\textbf{Виды полиморфизма:}
		\begin{itemize}
			\item
			Специальный (ad hoc) полиморфизм — определение различных реализаций для~конечного числа 
			фиксированных наборов входных типов (напр., перегрузка функций / методов).

			\item
			Параметрический полиморфизм — определение обобщенной реализации для~произвольного типа 
			(напр., шаблоны в C++; generics в Java).

			\item
			Полиморфизм подтипов — использование интерфейса класса для любого производного от~него подкласса (применяется в ООП).
		\end{itemize}
	}

	\subsection{Реализация ЯП}

	\frame{
		\frametitle{Реализация ЯП}

		\begin{itemize}
			\item
			\textbf{Компилируемые ЯП} — исходный код преобразуется в машинный \emph{до начала} выполнения.

			\vspace{0.5ex}
			\textbf{Примеры:} C/C++, Pascal.

			\vspace{1ex}\item
			\textbf{Интерпретируемые ЯП} — преобразование кода в машинные инструкции \emph{по ходу} выполнения.

			\vspace{0.5ex}
			\textbf{Примеры:} PHP, JavaScript.

			\vspace{1ex}\item
			Компиляция в независимый от платформы код и его интерпретация.

			\vspace{0.5ex}
			\textbf{Примеры:} Java (байт-код), C\# (CLR — common language runtime).
		\end{itemize}
	}

	\section{Метапрограммирование}

	\frame{
		\frametitle{Метапрограммирование}

		\begin{Definition}
			\textbf{Метапрограммирование} — разработка программ, обращающихся с~программами как~с~данными:
			\begin{itemize}
				\item
				создание программ, порождающих другие программы (в том числе во время компиляции);
				\item
				разработка программ, модифицирующих свой код во время исполнения.
			\end{itemize}
		\end{Definition}

		\vspace{1ex}
		\textbf{Цели метапрограммирования:}
		\begin{itemize}
			\item минимизация затрат на разработку;
			\item оптимизация кода;
			\item автоматизация разработки за счет использования высокоуровневых абстракций.
		\end{itemize}
	}

	\frame{
		\frametitle{Виды метапрограммирования}

		\begin{enumerate}
			\item
			\textbf{Шаблоны} — автоматическая генерация мало различающихся параметризованных фрагментов~кода.

			\vspace{0.5ex}
			\textbf{Примеры параметров:} типы данных, размеры массивов.

			\vspace{0.5ex}
			\textbf{Примеры:} препроцессоры в C и сходных языках; шаблоны в C++.

			\vspace{1ex}
			\item
			\textbf{Интерпретация} произвольного исходного кода с~помощью специальных функций.

			\vspace{0.5ex}
			\textbf{Примеры:} Функция \code{eval} в Python, PHP, JavaScript и~других интерпретируемых~ЯП.

			\vspace{1ex}\item
			Использование \textbf{рефлексии} \engterm{reflection} для самомодификации программы 
			во~время~выполнения.

			\vspace{1ex}\item
			Использование \textbf{предметно-специфичных языков} \engterm{domain-specific languages} 
			с~последующей трансляцией или~интерпретацией при~помощи~ЯП общего~назначения.
		\end{enumerate}
	}

	\subsection{Рефлексия}

	\frame{
		\frametitle{Рефлексия}

		\begin{Definition}
			\textbf{Рефлексия} или \textbf{отражение} \engterm{reflection} — отслеживание и изменение 
			структуры и~поведения программы во время ее выполнения.
		\end{Definition}

		\vspace{1ex}
		\textbf{Возможности рефлексии:}
		\begin{itemize}
			\item
			поиск и извлечение информации о типах во время исполнения;
			\item
			определение дополнительной информации о классах / методах (напр., данные об~аннотациях в Java);
			\item
			создание или переопределение классов / методов во время работы программы.
		\end{itemize}
	}

	\frame{
		\frametitle{Варианты использования рефлексии}

		\begin{itemize}
			\item
			\textbf{Тестирование:} создание mock-объектов — заглушек, реализующих требуемые функции программного окружения.

			\vspace{1ex}\item
			\textbf{Контейнеры объектов} (напр., серверы приложений): управление жизненным циклом объектов.

			\vspace{1ex}\item
			\textbf{Интерпретация предметно-ориентированных языков:} установление связи между объектами ЯП 
			общего назначения и инструкциями DSL.
		\end{itemize}
	}

	\frame{
		\frametitle{Пример рефлексии}

		\textbf{Рефлексия в JavaScript:}
		\lstinputlisting[language=javascript]{code-reflection.js}
	}

	\section{DSL}

	\frame{
		\frametitle{DSL}

		\begin{Definition}
			\textbf{Предметно-ориентированный язык} \engterm{domain-specific language} — компьютерный язык, 
			специализированный для конкретной области применения.
		\end{Definition}

		\vspace{1ex}
		\textbf{Примеры ПОЯ}

		\begin{center}
			\begin{tabular}{|l|p{0.5\textwidth}|}
				\hline
				\centering\textbf{Область} & \centering\textbf{Язык(и)} \cr
				\hline
				Веб-страницы (отображение) & HTML, CSS \cr
				Компьютерная верстка & \TeX / \LaTeX \cr
				Веб-страницы (генерация) & \raggedright Языки шаблонов (\extlink{http://en.wikipedia.org/wiki/Twig_\%28template_engine\%29}{Twig}, 
					\extlink{http://en.wikipedia.org/wiki/Jinja_(template_engine)}{Jinja} / шаблоны Django, 
					\extlink{http://en.wikipedia.org/wiki/JavaServer_Pages}{JSP}) \cr
				Матричное программирование & Matlab, Octave \cr
				Реляционные СУБД & SQL \cr
				\hline
			\end{tabular}
		\end{center}
	}

	\subsection{Причины появления}

	\frame{
		\frametitle{Причины появления DSL}

		\textbf{Задача:} создание представлений (view) в рамках архитектуры MVC.

		\textbf{Наивное решение:} использование ЯП общего назначения (напр., Python).

		\vspace{1ex}
		\textbf{Наивное представление в ЯП Python}
		\lstinputlisting[language=python]{code-view.py}
	}

	\frame{
		\frametitle{Причины появления DSL}

		\textbf{Недостатки использования ЯП общего назначения:} 
		\begin{itemize}
			\item
			громоздкость кода (избыток функций вывода и т.\,п.); 
			\item 
			соблазн внести код, не относящийся к созданию представления; 
			\item 
			разработчик интерфейса сайта может не знать ЯП общего назначения.
		\end{itemize}

		\vspace{1ex}
		\textbf{Решение:} использование языка шаблонов (HTML с дополнительными управляющими конструкциями).

		\vspace{1ex}
		\lstinputlisting[language=html]{code-view.html}
	}

	\frame{
		\frametitle{Место DSL в разработке ПО}

		\textbf{Варианты использования DSL:}
		\begin{enumerate}
			\item DSL используется для конфигурации компонентов, написанных на ЯП общего назначения;
			\item DSL встраивается в ЯП общего назначения (напр., Java Scripting Engine);
			\item DSL транслируется (чаще всего — в код на ЯП высокого уровня).
		\end{enumerate}

		\vspace{1ex}
		\textbf{Инструменты для разработки DSL:}
		\begin{itemize}
			\item
			средства лексического и синтаксического анализа (yacc, lex, bison, ANTLR);
			\item
			инструменты моделирования предметной области (Eclipse Modeling Framework).
		\end{itemize}
	}

	\subsection{Преимущества и недостатки}

	\frame{
		\frametitle{Преимущества DSL}

		\begin{itemize}
			\item
			\textbf{Разделение ответственности} \engterm{separation of concerns}~— 
			независимость семантики языка от~его~реализации;

			\vspace{1ex}
			\item
			\textbf{высокий уровень абстракции} DSL — позволяет сократить объем решения, 
			повысить его~доступность и~упростить отладку;

			\vspace{1ex}
			\item
			упрощение \textbf{портирования} — при переносе в~другую~среду меняется лишь имплементация~DSL;

			\vspace{1ex}
			\item
			легкость в освоении использовании \textbf{профильными специалистами}.
		\end{itemize}
	}

	\frame{
		\frametitle{Недостатки DSL}

		\begin{itemize}
			\item
			\textbf{Затраты} на проектирование, имплементацию и сопровождение языка и~соответствующих инструментов 
			(напр., поддержки среды разработки);

			\vspace{1ex}
			\item
			проблема \textbf{определения границ}~DSL;

			\vspace{1ex}
			\item
			трудности \textbf{интеграции}~DSL в~программную систему;

			\vspace{1ex}
			\item
			проблема \textbf{стандартизации} языков одной предметной области;

			\vspace{1ex}
			\item
			\textbf{низкая производительность} по~сравнению с~языками общего назначения.
		\end{itemize}
	}

	\section{Заключение}

	\subsection{Выводы}
	
	\frame{
		\frametitle{Выводы}

		\begin{enumerate}
			\item
			Языки программирования характеризуются синтаксисом (формой) и семантикой (содержанием). 
			Семантика языка может использоваться не только в процессе компиляции / интерпретации, 
			но и при анализе программ, напр., при~доказательстве их корректности.

			\vspace{0.5ex}
			\item
			Важная часть семантики языка программирования — система типов. 
			ЯП может обладать статической или динамической, строгой или слабой типизацией.

			\vspace{0.5ex}
			\item
			Метапрограммирование (порождающее программирование) — парадигма программирования, 
			в которой код программ рассматривается как данные. 
			Основными инструментами метапрограммирования являются рефлексия и~использование предметно-ориентированных языков.
		\end{enumerate}
	}
	
	\subsection{Материалы}
	
	\frame{
		\frametitle{Материалы}
		
		\begin{thebibliography}{9}
			\bibitem[1]{1}
			Czarnecki, K.; Eisenecker, U.\,W.
			\newblock Generative Programming.

			\bibitem[2]{2}
			Fowler, M.
			\newblock Domain Specific Languages. 
			\newblock {\footnotesize\url{http://martinfowler.com/dsl.html}}

			\bibitem[3]{3}
			Лавріщева К.\,М. 
			\newblock Програмна інженерія (підручник). 
			\newblock {\footnotesize К., 2008. — 319 с.}
		\end{thebibliography}
	}
	
	\frame{
		\frametitle{}
		
		\begin{center}
			\Huge Спасибо за внимание!
		\end{center}
	}
\end{document}
