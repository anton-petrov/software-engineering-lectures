\documentclass{a4beamer}
%% Lectures - common definitions

\usextensions{tikz}
\usetikzlibrary{shapes.multipart,shapes.callouts,shapes.geometric}
\input{fix-callouts.inc} % Fixes absolute positioning of rectangle callouts

\newif\ifbigpages \bigpagesfalse
\ifdim\paperwidth >20cm
	\bigpagestrue
\fi

\tikzset{%
	note/.style={rectangle callout,draw=none,callout pointer width=1em,%
		align=flush left,font=\footnotesize,inner sep=0.5em,%
		fill=blue!15,fill opacity=0.95,text opacity=1.0,callout absolute pointer=#1},
	node distance=2em and 2.75em
}
\ifbigpages
	% Scale all arrow tips by the factor of 2.5
	\let\old@pgf@arrow@call=\pgf@arrow@call
	\def\pgf@arrow@call#1{%
		\@tempdima=\pgflinewidth%
		\pgfsetlinewidth{2.5\pgflinewidth}%
		\old@pgf@arrow@call{#1}%
		\pgfsetlinewidth{\@tempdima}%
	}
	\def\pgfarrowsleftextend#1{\pgfmathsetlength{\pgf@xa}{1.5*#1}}
	\def\pgfarrowsrightextend#1{\pgfmathsetlength{\pgf@xb}{1.5*#1}}
\fi

%% Load listings package
\usepackage{listings}

%% Are we inside a comment?
\newif\iflstcomment \lstcommentfalse

\lstset{%
	tabsize=4,
	showstringspaces=false,
	basicstyle=\linespread{1.25}\ttfamily\small,
	keywordstyle=\bfseries,
	commentstyle=\lstcommentstyle,
	numbers=left,
	numberstyle=\footnotesize\color{gray},
	xleftmargin=2.5em,
	extendedchars=true,
	escapechar=\$,
	escapebegin=\iflstcomment\begingroup\lstcommentstyle\fi,
	escapeend=\iflstcomment\endgroup\fi
}

\def\lstcommentstyle{\color{gray}}

\lst@AddToHook{AfterBeginComment}{\global\lstcommenttrue}
\let\orig@lst@EndComment=\lst@EndComment
\def\lst@EndComment{\global\lstcommentfalse\orig@lst@EndComment}
\lst@AddToHookAtTop{EOL}{%
	\lst@ifLmode\global\lstcommentfalse\fi% XXX Sloppy way to determine comment end
}

%% Python with docstrings treated as comments
\lstdefinelanguage[doc]{python}[]{python}{%
	deletestring=[s]{"""}{"""},%
	morecomment=[s]{"""}{"""}%
}%

%% JavaScript language
\lstdefinelanguage{javascript}%
	{morekeywords={break,case,catch,%
		const,constructor,continue,default,do,else,false,%
		finally,for,function,if,in,instanceof,%
		new,null,prototype,%
		return,switch,this,throw,%
		true,try,typeof,var,while},%
	sensitive,%
	morecomment=[l]//,%
	morecomment=[s]{/*}{*/},%
	morestring=[b]",%
	morestring=[b]',%
}[keywords,comments,strings]%

%% C# language (4.0?)
\lstdefinelanguage{csharp}%
	{morekeywords={abstract,as,%
		base,bool,byte,case,catch,char,%
		checked,class,const,continue,%
		decimal,default,delegate,do,double,%
		else,enum,event,explicit,extern,%
		false,finally,fixed,float,for,foreach,%
		goto,if,implicit,in,int,interface,%
		internal,is,lock,long,%
		namespace,new,null,object,operator,out,%
		override,params,private,protected,public,%
		readonly,ref,return,sbyte,sealed,%
		short,sizeof,stackalloc,static,string,%
		struct,switch,this,throw,true,try,%
		typeof,uint,ulong,unchecked,unsafe,ushort,%
		using,virtual,void,volatile,while%
	},%
	sensitive,%
	morecomment=[l]//,%
	morecomment=[s]{/*}{*/},%
	morestring=[b]",%
	morestring=[b]',%
}[keywords,comments,strings]%

%% Translation for fact environment
\deftranslation[to=russian]{Fact}{Наблюдение}

%% Inline code snippets
\def\code#1{\texttt{#1}}
\def\codekw#1{\code{\textbf{#1}}}

\def\quoteauthor#1{\par\footnotesize\upshape\hfill—~#1}

%% English term
\def\engterm#1{(англ. \textit{#1})}
%% Term with explanation below (to be used in diagrams)
\def\termwithexpl#1#2{#1\strut{}\\\small\color{gray}(\textit{#2})\strut{}}
%% External link
\def\extlink#1#2{\href{#1}{\color[rgb]{0.7,0.7,1.0}\dashbar{#2}}}
%% Internal link
\def\inlink#1#2{\hyperlink{#1}{\color[rgb]{0.7,0.7,1.0}\dashbar{#2}}}
%% Explanation for a list item
\def\itemexpl#1{\begingroup\small\vspace{0.75ex}#1\par\endgroup}




\lecturetitle{Программная инженерия. Лекция №19 — Интерфейсы и типы данных (часть 2)}
\title[Типы данных 2]{Интерфейсы и типы данных (часть 2)}
\author{Алексей Островский}
\institute{\small{Физико-технический учебно-научный центр НАН Украины}\vspace{2ex}}
\date{16 апреля 2015 г.}

\begin{document}
	\frame{\titlepage}

	\section{Типобезопасность}

	\frame{
		\frametitle{Типобезопасность}

		\begin{Definition}
			\textbf{Типобезопасность} \engterm{type safety} — мера, в~которой язык программирования 
			или~стиль написания программы предотвращает ошибки типов.
		\end{Definition}

		\begin{Definition}
			\textbf{Ошибка типа} \engterm{type error} — дефектное или нежелательное поведение программы, 
			вызванное различием между~ожидаемым и~действительным смыслом данных, связанных с~переменной 
			или~функцией / методом.
		\end{Definition}

		\vspace{1ex}
		\textbf{Виды типобезопасности:}
		\begin{itemize}
			\item статическая (определение ошибок во время компиляции);
			\item динамическая (поиск ошибок во время выполнения).
		\end{itemize}

		\vspace{1ex}
		В «хорошем» ЯП результат любого выражения является корректным значением типа, 
		который может~быть определен на~основе выражения во~время компиляции.
	}

	\frame{
		\frametitle{Безопасность памяти}

		\begin{Definition}
			\textbf{Безопасность памяти} \engterm{memory safety} — мера, в которой язык или стиль программирования 
			защищены от ошибок доступа к памяти.
		\end{Definition}

		\vspace{1ex}
		\textbf{Ошибки, связанные с памятью:}
		\begin{itemize}
			\item переполнение буфера, кучи в целом или стека;
			\item использование неинициализированных переменных;
			\item некорректная работа с динамической памятью:
			\item работа с указателем после освобождения памяти;
			\item многократное высвобождение одного указателя;
			\item неправильная обработка нулевых указателей.
		\end{itemize}
	}

	\subsection{Безопасность и системы ТД}

	\frame{
		\frametitle{Сильная и слабая типизация}

		\begin{Definition}
			\textbf{Сильная / слабая типизация} \engterm{strong / weak type system} — 
			степень соблюдения языком программирования безопасности типов и памяти.
		\end{Definition}

		\vspace{1ex}
		\textbf{Сильная типизация:}
		\begin{itemize}
			\item
			отсутствие указателей / ссылок, арифметики указателей;
			\item
			отсутствие различающихся представлений одних и тех же данных 
			(таких как \codekw{union} в~C / C++);
			\item
			минимальное количество неявных приведений типов;
			\item
			отсутствие неочевидных для программиста спецификаций операций 
			(таких как перегрузка операторов в~C++).
		\end{itemize}

		\vspace{1ex}
		\textbf{ЯП с сильной типизацией:} Python, Java, функциональные ЯП.

		\textbf{ЯП со слабой типизацией:} C, C++, Visual Basic.
	}

	\frame{
		\frametitle{Статическая и динамическая типизация}

		\textbf{Статическая типизация} — определение типов всех конструкций языка на этапе компиляции программы
		(слабая форма верификации программы).

		\vspace{1ex}
		\textbf{Виды статической типизации:}
		\begin{itemize}
			\item
			явная — типы конструкций декларируются программистом (напр., при объявлении переменных);
			\item
			неявная — тип переменных выводится в процессе компиляции. Примеры:
			\begin{itemize}
				\item \code{\textbf{var} x = 5} в C\#; 
				\item \code{List<> list = \textbf{new} ArrayList<String>()} в Java 7+.
			\end{itemize}
		\end{itemize}

		\vspace{1ex}
		\textbf{ЯП со статической типизацией:} C++, Pascal, Java, C\#.

		\vspace{2.5ex}
		\textbf{Динамическая типизация} — определение типов некоторых конструкций и проверка соответствующих ограничений 
		во время выполнения программы.

		\vspace{1ex}
		\textbf{ЯП с динамической типизацией:} Python, PHP, Perl, JavaScript.
	}

	\subsection{Совместимость типов}

	\frame{
		\frametitle{Совместимость типов}

		\begin{Problem}
			Проверить соответствие типа всех выражений ожидаемому в конкретной ситуации. 
			Понятие соответствия специфично для конкретного ЯП.
		\end{Problem}

		\vspace{1ex}
		\textbf{Конексты, где необходимо согласование:}
		\begin{itemize}
			\item присвоение \code{<переменная> = <выражение>} (типы переменной и выражения);
			\item вызов функции / метода (соответствие аргументов сигнатуре).
		\end{itemize}

		\vspace{1ex}
		\textbf{Методы определения совместимости:}
		\begin{itemize}
			\item
			\textbf{«Плоская» система типов:} совместимость = эквивалентность; 
			определяется исходя из~деклараций или структуры данных.

			\vspace{1ex}
			\item
			\textbf{Иерархическая система типов:} совместимость определяется отношениями 
			тип — подтип (задекларированными или неявными).
		\end{itemize}
	}

	\frame{
		\frametitle{Номинальная и структурная типизация}

		\textbf{Номинальная типизация} — вид системы типов, при которой совместимость и~эквивалентность типов 
		определяется на~основе явных деклараций (наследования, имплементации интерфейсов и~т.\,п.).

		\vspace{0.5ex}
		\textbf{ЯП с номинальной типизацией:} C\#, Java; C++ (основные типы).

		\vspace{1.5ex}
		\textbf{Структурная типизация} — вид системы типов, при~которой совместимость и~эквивалентность типов 
		определяется на~основе внутренней структуры объектов \emph{во~время компиляции}.

		\vspace{0.5ex}
		\textbf{ЯП со структурной типизацией:} C++ (шаблоны); функциональные ЯП (Haskell, ML).

		\vspace{1.5ex}
		\textbf{Утиная типизация} — вид системы типов, при~которой совместимость и~эквивалентность типов 
		определяется на~основе структуры объектов \emph{во~время выполнения}.

		\vspace{0.5ex}
		\textbf{ЯП с утиной типизацией:} Python, JavaScript.
	}

	\subsection{Утиная типизация}

	\frame{
		\frametitle{Утиная типизация}

		\begin{Definition}
			\textbf{Утиная типизация} \engterm{duck typing} — вид динамической типизации, при которой 
			корректность использования объекта определяется набором его методов и свойств, а~не~типом.
		\end{Definition}

		\vspace{1ex}
		\textbf{ЯП с утиной типизацией:}
		\begin{itemize}
			\item языки с ООП на основе прототипов (JavaScript, Lua);
			\item Python; 
			\item Smalltalk.
		\end{itemize}

		\vspace{1ex}
		\textbf{Пример динамической не утиной типизации:} подсказки типов \engterm{data hinting} в~PHP.
	}
	
	\frame{
		\frametitle{Пример: утиная типизация в Python}

		\lstinputlisting[language={[doc]python},firstline=4]{code-duck.py}
	}

	\section{Приведение ТД}

	\frame{
		\frametitle{Приведение типов в различных ЯП}

		\textbf{Java:}
		\lstinputlisting[language=java]{code-str-int.java}

		\vspace{0.5ex}
		\textbf{Python:}
		\lstinputlisting[language=python]{code-str-int.py}

		\vspace{0.5ex}
		\textbf{PHP:}
		\lstinputlisting[language=php,escapechar=\#]{code-str-int.php}

		\vspace{0.5ex}
		\textbf{C:}
		\lstinputlisting[language=c]{code-str-int.c}
	}

	\subsection{Виды приведения ТД}

	\frame{
		\frametitle{Неявное приведение типов данных}

		\begin{Definition}
			\textbf{Неявное приведение ТД} \engterm{implicit conversion, coercion} — приведение типов данных, 
			осуществляемое автоматически компилятором.
		\end{Definition}

		\vspace{1ex}
		\textbf{Примеры:}
		\begin{itemize}
			\item
			Преобразование между числовыми типами с повышением разрядности или множества представимых чисел 
			(\codekw{byte}~$\rightarrow$ \codekw{short}~$\rightarrow$ \codekw{int}~$\rightarrow$ 
				\codekw{long}~$\rightarrow$ \codekw{float}~$\rightarrow$ \codekw{double}).

			\textbf{Код (Java):}
			\lstinputlisting{code-coerce.java}

			\vspace{0.5ex}
			\item
			Приведение объектов к строковому представлению; зачастую — с помощью средств ООП: 
			\begin{itemize}
				\item метод \code{toString()} в Java; 
				\item метод \code{ToString()} в C\#; 
				\item метод \code{str.format()} и ключевое слово / функция \code{print} в~Python.
			\end{itemize}
		\end{itemize}
	}

	\frame{
		\frametitle{Явное приведение типов данных}

		\begin{Definition}
			\textbf{Явное приведение ТД} \engterm{explicit conversion} — приведение ТД, 
			специфицируемое в~исходном коде программы.
		\end{Definition}

		\vspace{1ex}
		\textbf{Подвиды:}
		\begin{itemize}
			\item
			С динамической проверкой типа во время выполнения:
			\begin{itemize}
				\item приведение типов в Java; 
				\item оператор \code{\textbf{dynamic\_cast}<T>} в C++.
			\end{itemize}

			\vspace{1ex}
			\item
			Без динамической проверки типа:
			\begin{itemize}
				\item \code{\textbf{static\_cast}<T>} в C++; 
				\item оператор \codekw{as} в C\#.
			\end{itemize}
		\end{itemize}
	}

	\frame{
		\frametitle{Пример: приведение ТД в C\#}

		\lstinputlisting[language=csharp]{code-conversion.cs}
	}

	\subsection{Полиморфизм}

	\frame{
		\frametitle{Полиморфизм}

		\begin{Definition}
			\textbf{Полиморфизм} — использование единого интерфейса для сущностей различных типов.
		\end{Definition}

		\vspace{1ex}
		\textbf{Виды полиморфизма:}
		\begin{itemize}
			\item
			Специальный (ad hoc) полиморфизм — определение различных реализаций для~конечного числа 
			фиксированных наборов входных типов (напр., перегрузка функций / методов).

			\item
			Параметрический полиморфизм — определение обобщенной реализации для~произвольного типа 
			(напр., шаблоны в C++; generics в Java и C\#).

			\item
			Полиморфизм подтипов — использование интерфейса класса для любого производного от~него подкласса (применяется в ООП).
		\end{itemize}
	}

	\subsubsection{Специальный полиморфизм}

	\frame{
		\frametitle{Пример: специальный полиморфизм (Java)}

		\lstinputlisting[language=java]{code-poly-special.java}
	}

	\subsubsection{Параметрический полиморфизм}

	\frame{
		\frametitle{Пример: параметрический полиморфизм (C++)}

		\lstinputlisting[language=c++]{code-poly-parametric.cpp}
	}

	\frame{
		\frametitle{Пример: параметрический полиморфизм (Java)}

		\lstinputlisting[language=java]{code-poly-parametric.java}

		Информация о параметризации типа в Java доступна только во~время компиляции; 
		во~время выполнения эти~сведения \extlink{http://en.wikipedia.org/wiki/Type_erasure}{стираются}:

		\lstinputlisting[language=java]{code-poly-parametric-2.java}
	}

	\frame{
		\frametitle{Особенности параметрического полиморфизма}

		\textbf{C++:} шаблоны \engterm{templates}.
		\begin{itemize}
			\item
			Используется для объявления широкого круга конструкций (функции, структуры, классы);
			\item
			аргументы — произвольные переменные (не обязательно типы данных);
			\item
			возможность явного указания частных реализаций;
			\item
			для каждого набора аргументов генерируется свой код.
		\end{itemize}

		\vspace{1ex}
		\textbf{Java и C\#:} шаблонные типы \engterm{generics}.
		\begin{itemize}
			\item
			Используется для объявления классов / интерфейсов и отдельных методов;
			\item
			аргументы — типы данных (в Java — только ссылочные);
			\item
			общий код для всех аргументов;
			\item
			стирание типов во время выполнения (Java).
		\end{itemize}
	}

	\subsubsection{Полиморфизм подтипов}

	\frame{
		\frametitle{Полиморфизм подтипов}

		\textbf{Принцип подстановки Барбары Лисков} \engterm{Liskov substitution principle, LSP}:

		Функции, использующие базовый тип данных (напр., в качестве аргументов), 
		должны уметь использовать произвольные подтипы этого типа, не зная об этом.

		\vspace{1ex}
		{\small(LSP — один из пяти базовых принципов объектно-ориентированного проектирования 
			\extlink{http://en.wikipedia.org/wiki/SOLID_\%28object-oriented_design\%29}{SOLID}.)}

		\vspace{2ex}
		\textbf{NB.} Полиморфизм подтипов означает наследование интерфейсов, наследование в~ООП — наследование имплементации. 
		Один тип может быть подтипом неродственных типов в~ЯП без~множественного наследования.

		\vspace{1ex}
		\textbf{Пример (Java):}
		\begin{itemize} 
			\item 
			ArrayList — подтип интерфейсов List, Iterable, Collection, Cloneable, Serializable, RandomAccess;
			\item 
			ArrayList — подтип классов AbstractList, AbstractCollection и Object.
		\end{itemize}
	}

	\frame{
		\frametitle{Пример: полиморфизм подтипов (Java)}

		\lstinputlisting[language=java]{code-poly-subtypes.java}
	}

	\section{Подтипы в ООП}

	\frame{
		\frametitle{Подтипы и наследование}

		\begin{Definition}
			\textbf{Наследование} ($=$ наследование реализации, \textit{code inheritance}) — 
			перенос для~использования, расширения или~модификации \emph{реализации} методов класса.
		\end{Definition}

		{\small%
			\textbf{Множественное наследование} — копирование реализации из нескольких источников.

			\textbf{ЯП с множественным наследованием:} C++, Python.
		}

		\vspace{1ex}
		\begin{Definition}
			\textbf{Отношение «тип — подтип»} ($=$ наследование интерфейса, \textit{subtyping}) — 
			копирование \emph{описания} методов класса для определения совместимости типов.
		\end{Definition}

		\vspace{1ex}
		\textbf{Промежуточные варианты:} \\
		типажи \engterm{traits}, примеси \engterm{mixins} — $\sim$ интерфейсы с частично реализованными методами.

		\textbf{ЯП с типажами / примесями:} методы в интерфейсах по умолчанию (Java~8); \codekw{trait} в~PHP.
	}

	\subsection{Интерфейсы ООП}

	\frame{
		\frametitle{Интерфейсы ООП}

		\begin{Definition}
			\textbf{Интерфейс в ООП} — способ задания отношения подтипов на~основе контрактов 
			(т.\,е. описания требуемой функциональности).
		\end{Definition}

		\vspace{1ex}
		\textbf{ЯП, поддерживающие интерфейсы:} C\#, D, Delphi / Object Pascal, Java, PHP.

		\vspace{0.5ex}
		\textbf{Эмуляция интерфейсов:} C++ (за счет классов с чистыми виртуальными функциями и~множественного наследования).

		\vspace{1ex}
		\textbf{Примеры интерфейсов (Java):}
		\begin{itemize}
			\item Cloneable — указывает на то, что класс поддерживает клонирование;
			\item Serializable — класс поддерживает сохранение данных;
			\item Comparable<T> — объекты класса сравнимы с объектами класса T;
			\item List<T> — класс представляет собой список элементов класса T.
		\end{itemize}
	}

	\frame{
		\frametitle{Пример: интерфейсы в Java}

		\lstinputlisting[language=java]{code-intf.java}
	}

	\frame{
		\frametitle{Сравнение наследования и подтипов}

		\begin{center}\small
			\begin{tabular}{|p{0.03\textwidth}|p{0.03\textwidth}|p{0.375\textwidth}|p{0.375\textwidth}|}
				\hline
				\multicolumn{2}{|c|}{} & \multicolumn{2}{c|}{Наследование} \cr
				\cline{3-4}
				\multicolumn{2}{|c|}{} & \centering $+$ & \centering $-$ \cr
				\hline
				\multirow{2}*{\rotatebox{90}{Подтип\hspace{2ex}\strut{}}} & \centering $+$ 
					& \raggedright Наследование в Java, C\# 
						\vspace{0.5ex}\par \code{\textbf{class} A \{ \}} 
						\par \code{\textbf{class} B \textbf{extends} A \{ \}}
					& \raggedright Расширение интерфейсов в Java, C\# 
						\vspace{0.5ex}\par \code{\textbf{intefrace} A \{ \}} 
						\par \code{\textbf{interface} B \textbf{extends} A \{ \}} \cr
				\cline{2-4}
				& \centering $-$ 
					& \raggedright Наследование в С++ с~модификаторами \codekw{protected} и~\codekw{private}
						\vspace{0.5ex}\par \code{\textbf{class} A \{ \};} 
						\par \code{\textbf{class} B : \textbf{protected} A \{ \};}
					& \raggedright Независимые типы данных 
						\vspace{0.5ex}\par \code{\textbf{class} A \{ \}} 
						\par \code{\textbf{class} B \{ \}} \cr
				\hline
			\end{tabular}
		\end{center}
	}

	\subsection{Ковариантность и контравариантность}

	\frame{
		\frametitle{Ковариантность и контравариантность}

		\textbf{Ковариантность:} 
		\code{ParametricType<S>} — подтип \code{ParametricType<T>}, если \code{S} — подтип \code{T}.

		\vspace{1ex}
		\textbf{Примеры использования:}
		\begin{itemize}
			\item
			итераторы (C\#):
			\lstinputlisting[language=csharp]{code-covariance.cs}

			\item
			массивы в C\#, Java:
			\lstinputlisting[language=java]{code-covariance.java}
		\end{itemize}
	}

	\frame{
		\frametitle{Ковариантность и контравариантность}

		\textbf{Контравариантность:} \code{ParametricType<T>} — подтип \code{ParametricType<S>}, 
		если \code{S} — подтип \code{T}.

		\vspace{1ex}
		\textbf{Пример использования:} сравнение (C\#):
		\lstinputlisting[language=csharp]{code-contravariance.cs}

		\vspace{1ex}
		\textbf{Инвариантность:} \code{ParametricType<T>} и \code{ParametricType<S>} не связаны.

		\vspace{1ex}
		\textbf{Пример:} изменяемые коллекции (C\#):
		\lstinputlisting[language=csharp]{code-invariance.cs}
	}

	\frame{
		\frametitle{Ковариантность и контравариантность (Java)}

		Декларации ко/контравариантности в Java: во время использования типа, а~не~во~время его~декларации:
		\begin{itemize}
			\item
			\code{? \textbf{extends} T} — ковариантное определение типа данных в шаблоне;
			\item
			\code{? \textbf{super} T} — контравариантное определение типа данных в шаблоне.
		\end{itemize}

		\vspace{0.5ex}
		\begin{overlayarea}{\textwidth}{0.575\textheight}
			\only<1,2>{%
				\textbf{Пример (ковариантное определение):}
				\lstinputlisting[language=java]{code-wildcards-1.java}
			}
			\only<3,4>{%
				\textbf{Пример (контравариантное определение):}
				\lstinputlisting[language=java]{code-wildcards-2.java}
			}
		\end{overlayarea}
	}

	\section{Заключение}

	\subsection{Выводы}
	
	\frame{
		\frametitle{Выводы}

		\begin{enumerate}
			\item
			Основное применение системы типов данных — устранение ошибок, связанных с~некорректной интерпретацией данных 
			(слабая форма верификации программы). Для~решения этой~задачи вводится понятие совместимых типов~данных.

			\item
			Совместимость типов может определяться с~помощью явных или~неявных приведений, 
			а~также с~помощью полиморфизма, в~частности отношений «тип — подтип».

			\item
			Определение подтипов и наследование в~ООП — связанные, но~различные понятия. 
			В~некоторых случаях определение подтипов нетривиально (например, правила ковариантности / контравариантности 
			для~параметрических типов).
		\end{enumerate}
	}
	
	\subsection{Материалы}
	
	\frame{
		\frametitle{Материалы}
		
		\begin{thebibliography}{9}
			\bibitem[1]{1}
			Лавріщева К.\,М. 
			\newblock Програмна інженерія (підручник). 
			\newblock {\footnotesize К., 2008. — 319 с.}

			\bibitem[2]{2}
			Tratt, Laurence
			\newblock Dynamically typed languages.
			\newblock {\footnotesize\url{http://tratt.net/laurie/research/pubs/html/tratt__dynamically_typed_languages/}}

			\bibitem[3]{3}
			Cardelli Luca, Wegner Peter
			\newblock On understanding types, data abstraction, and polymorphism.
			\newblock {\footnotesize\url{http://lucacardelli.name/Papers/OnUnderstanding.A4.pdf}}
		\end{thebibliography}
	}
	
	\frame{
		\frametitle{}
		
		\begin{center}
			\Huge Спасибо за внимание!
		\end{center}
	}

\end{document}

