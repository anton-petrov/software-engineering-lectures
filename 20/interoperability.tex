\documentclass{a4beamer}
%% Lectures - common definitions

\usextensions{tikz}
\usetikzlibrary{shapes.multipart,shapes.callouts,shapes.geometric}
\input{fix-callouts.inc} % Fixes absolute positioning of rectangle callouts

\newif\ifbigpages \bigpagesfalse
\ifdim\paperwidth >20cm
	\bigpagestrue
\fi

\tikzset{%
	note/.style={rectangle callout,draw=none,callout pointer width=1em,%
		align=flush left,font=\footnotesize,inner sep=0.5em,%
		fill=blue!15,fill opacity=0.95,text opacity=1.0,callout absolute pointer=#1},
	node distance=2em and 2.75em
}
\ifbigpages
	% Scale all arrow tips by the factor of 2.5
	\let\old@pgf@arrow@call=\pgf@arrow@call
	\def\pgf@arrow@call#1{%
		\@tempdima=\pgflinewidth%
		\pgfsetlinewidth{2.5\pgflinewidth}%
		\old@pgf@arrow@call{#1}%
		\pgfsetlinewidth{\@tempdima}%
	}
	\def\pgfarrowsleftextend#1{\pgfmathsetlength{\pgf@xa}{1.5*#1}}
	\def\pgfarrowsrightextend#1{\pgfmathsetlength{\pgf@xb}{1.5*#1}}
\fi

%% Load listings package
\usepackage{listings}

%% Are we inside a comment?
\newif\iflstcomment \lstcommentfalse

\lstset{%
	tabsize=4,
	showstringspaces=false,
	basicstyle=\linespread{1.25}\ttfamily\small,
	keywordstyle=\bfseries,
	commentstyle=\lstcommentstyle,
	numbers=left,
	numberstyle=\footnotesize\color{gray},
	xleftmargin=2.5em,
	extendedchars=true,
	escapechar=\$,
	escapebegin=\iflstcomment\begingroup\lstcommentstyle\fi,
	escapeend=\iflstcomment\endgroup\fi
}

\def\lstcommentstyle{\color{gray}}

\lst@AddToHook{AfterBeginComment}{\global\lstcommenttrue}
\let\orig@lst@EndComment=\lst@EndComment
\def\lst@EndComment{\global\lstcommentfalse\orig@lst@EndComment}
\lst@AddToHookAtTop{EOL}{%
	\lst@ifLmode\global\lstcommentfalse\fi% XXX Sloppy way to determine comment end
}

%% Python with docstrings treated as comments
\lstdefinelanguage[doc]{python}[]{python}{%
	deletestring=[s]{"""}{"""},%
	morecomment=[s]{"""}{"""}%
}%

%% JavaScript language
\lstdefinelanguage{javascript}%
	{morekeywords={break,case,catch,%
		const,constructor,continue,default,do,else,false,%
		finally,for,function,if,in,instanceof,%
		new,null,prototype,%
		return,switch,this,throw,%
		true,try,typeof,var,while},%
	sensitive,%
	morecomment=[l]//,%
	morecomment=[s]{/*}{*/},%
	morestring=[b]",%
	morestring=[b]',%
}[keywords,comments,strings]%

%% C# language (4.0?)
\lstdefinelanguage{csharp}%
	{morekeywords={abstract,as,%
		base,bool,byte,case,catch,char,%
		checked,class,const,continue,%
		decimal,default,delegate,do,double,%
		else,enum,event,explicit,extern,%
		false,finally,fixed,float,for,foreach,%
		goto,if,implicit,in,int,interface,%
		internal,is,lock,long,%
		namespace,new,null,object,operator,out,%
		override,params,private,protected,public,%
		readonly,ref,return,sbyte,sealed,%
		short,sizeof,stackalloc,static,string,%
		struct,switch,this,throw,true,try,%
		typeof,uint,ulong,unchecked,unsafe,ushort,%
		using,virtual,void,volatile,while%
	},%
	sensitive,%
	morecomment=[l]//,%
	morecomment=[s]{/*}{*/},%
	morestring=[b]",%
	morestring=[b]',%
}[keywords,comments,strings]%

%% Translation for fact environment
\deftranslation[to=russian]{Fact}{Наблюдение}

%% Inline code snippets
\def\code#1{\texttt{#1}}
\def\codekw#1{\code{\textbf{#1}}}

\def\quoteauthor#1{\par\footnotesize\upshape\hfill—~#1}

%% English term
\def\engterm#1{(англ. \textit{#1})}
%% Term with explanation below (to be used in diagrams)
\def\termwithexpl#1#2{#1\strut{}\\\small\color{gray}(\textit{#2})\strut{}}
%% External link
\def\extlink#1#2{\href{#1}{\color[rgb]{0.7,0.7,1.0}\dashbar{#2}}}
%% Internal link
\def\inlink#1#2{\hyperlink{#1}{\color[rgb]{0.7,0.7,1.0}\dashbar{#2}}}
%% Explanation for a list item
\def\itemexpl#1{\begingroup\small\vspace{0.75ex}#1\par\endgroup}



\usepackage{listings}
\lstset{%
	morekeywords={assert},
	tabsize=4,
	showstringspaces=false,
	basicstyle=\linespread{1.25}\ttfamily\small,
	keywordstyle=\bfseries,
	commentstyle=\color{gray},
	numbers=left,
	numberstyle=\footnotesize\color{gray},
	xleftmargin=2.5em,
	texcl=true,
	extendedchars=true,
	escapechar=\$
}

\def\code#1{\texttt{#1}}
\def\codekw#1{\texttt{\textbf{#1}}}

\deftranslation[to=russian]{Fact}{Наблюдение}

\lecturetitle{Программная инженерия. Лекция №20 — Интероперабельность}
\title[Интероперабельность]{Интероперабельность}
\author{Алексей Островский}
\institute{\small{Физико-технический учебно-научный центр НАН Украины}\vspace{2ex}}
\date{23 апреля 2015 г.}


\begin{document}
	\frame{\titlepage}

	\frame{
		\frametitle{Проблемы взаимодействия сред выполнения}

		\begin{itemize}
			\item
			\textbf{Байт (8 бит):} целое число со знаком (Java) или без знака (C / C++, C\#)?

			\vspace{0.5ex}
			\item
			Формат и объем поддержки \textbf{сложных типов:} структур, массивов (как реализована многомерность?), 
			множеств, перечислений, объединений.

			\vspace{0.5ex}
			\item
			\textbf{Порядок байтов} в сложных объектах: от младшего к~старшему или~наоборот (определяется ABI конкретной среды)?

			\vspace{0.5ex}
			\item
			\textbf{Кодирование строк:} в UTF-16 (1 символ $=$ 2 байта, представление данных в~оперативной памяти в~Java, C\#, …) 
			или~UTF-8 (переменная длина символа, представление при~постоянном хранении данных)?

			\vspace{0.5ex}
			\item
			\textbf{Сжатие данных?}
		\end{itemize}
	}

	\section{Введение}

	\subsection{Определения}

	\frame{
		\frametitle{Интероперабельность}

		\begin{Definition}
			\textbf{Интероперабельность} \engterm{interoperability} — возможность обмениваться данными для~программ, 
			выполняемых в различных операционных средах.
		\end{Definition}

		\vspace{1ex}
		\textbf{Инструменты} достижения интероперабельности:
		\begin{itemize}
			\item
			стандартизация протоколов взаимодействия;
			\item
			поддержка базовых форматов передачи информации (файлов и т.\,п.);
			\item
			использование согласованного представления данных.
		\end{itemize}

		\vspace{1ex}
		\textbf{Причины использования:}
		\begin{itemize}
			\item
			разделение приложения на слабо связанные компоненты в~соответствии 
			с~компонентно-ориентированным программированием, разделение «зон ответственности» разработчиков;
			\item
			максимизация повторного использования кода;
			\item
			повышение отказоустойчивости и производительности.
		\end{itemize}
	}

	\frame{
		\frametitle{Межпроцессное взаимодействие}

		\begin{Definition}
			\textbf{Межпроцессное взаимодействие} \engterm{inter-process communication, IPC} 
			— спецификации для~организации обмена данными между несколькими потоками выполнения 
			и / или процессами.
		\end{Definition}

		\vspace{1ex}
		\textbf{Категории:}
		\begin{itemize}
			\item локальное — взаимодействие в пределах одного компьютера;
			\item распределенное — в рамках нескольких компьютеров (через сеть).
		\end{itemize}

		\vspace{1ex}
		\textbf{Уровень:}
		\begin{itemize}
			\item 
			низкоуровневое — взаимодействие средствами операционной системы 
			(с~использованием сведений об~ABI);
			\item 
			высокоуровневое — с помощью посредника, позволяющего абстрагироваться от~ABI.
		\end{itemize}
	}

	\subsection{Спецификация взаимодействия}

	\frame{
		\frametitle{Спецификация взаимодействия}

		\textbf{Способ коммуникации:} 
		\begin{itemize}
			\item файлы; 
			\item анонимные / именованные каналы; 
			\item сигналы; 
			\item разделяемая память; 
			\item сокеты; 
			\item сообщения.
		\end{itemize}

		\vspace{1ex}
		\textbf{Формат данных:} 
		\begin{itemize}
			\item бинарный (с согласованной спецификацией);
			\item текстовый (на основе XML, JSON, других форматов).
		\end{itemize}
	}

	\subsection{Низкоуровневое взаимодействие}

	\frame{
		\frametitle{Низкоуровневое взаимодействие: POSIX / System V}

		\textbf{Способы обмена данными:}
		\begin{itemize}
			\item
			\textbf{Сигнал} — пересылка системных сообщений между процессами (обычно с~минимальной нагрузкой данными).
			\item
			\textbf{Сокет} — взаимодействие через физический сетевой интерфейс.
			\item
			Системная \textbf{очередь сообщений}.
			\item
			\textbf{Канал взаимодействия} \engterm{pipe} — двусторонний поток данных, 
			связывающий ввод / вывод двух процессов с последовательным доступом.
			\item
			\textbf{Разделяемая память} — участок оперативной памяти, доступный для~чтения / записи для~нескольких процессов.
		\end{itemize}
	}

	\frame{
		\frametitle{Низкоуровневое взаимодействие: POSIX / System V}

		\textbf{Достоинства:}
		\begin{itemize}
			\item
			высокая скорость передачи данных;
			\item
			широкий круг систем с поддержкой технологии;
			\item
			минимальная зависимость от стороннего ПО.
		\end{itemize}

		\vspace{1ex}
		\textbf{Недостатки:}
		\begin{itemize}
			\item
			ограниченность одним компьютером (кроме сокетов);
			\item
			отсутствие спецификации формата данных ($\Rightarrow$~сложно обнаруживаемые и~локализуемые ошибки);
			\item
			сложность восприятия и модификации программ;
			\item
			необходимость написания шаблонного кода \engterm{boilerplate code}.
		\end{itemize}
	}

	\subsection{Промежуточное ПО}

	\frame{
		\frametitle{Middleware}

		\begin{Definition}
			\textbf{Промежуточное программное обеспечение} \engterm{middleware} — ПО, 
			предоставляющее инструменты для обмена и управления данными для~межпроцессного взаимодействия, 
			которые расширяют встроенные средства операционной системы.
		\end{Definition}

		\vspace{1ex}
		\textbf{Достоинства:}
		\begin{itemize}
			\item
			стандартизация формата данных $\Rightarrow$~поддержка взаимодействия 
			между~различными языками программирования и~ОС;
			\item
			упрощение процесса разработки, улучшение качества кода;
			\item
			инструменты для контроля взаимодействия (отладка, профилирование и т.\,п.).
		\end{itemize}
	}

	\frame{
		\frametitle{Роль промежуточного ПО в построении приложений}

		\begin{center}
			\begin{tikz*}[%
	every node/.style={rectangle,draw,align=center,minimum height=2.5em},
	computer/.style={minimum width=10em,minimum height=6em}
]
	\node(c1) [computer] {Компьютер 1};
	\node(comp1) [below=-2em of c1,fill=white]{Компонент \\ (ЯП 1)};
	\node(intf1) [below=of comp1]{Интерфейс};

	\node(c2) [computer,right=12.5em of c1] {Компьютер 2};
	\node(comp2) [below=-2em of c2,fill=white]{Компонент \\ (ЯП 2)};
	\node(intf2) [below=of comp2]{Интерфейс};

	\node(middleware) [minimum width=30em,anchor=north] at ($ (intf1.south)!0.5!(intf2.south) + (0, -2em) $) {\textbf{Промежуточное ПО}};

	\draw[<->] (comp1) -- (intf1);
	\draw[<->] (intf1.south) -- (intf1.south |- middleware.north);
	\draw[<->] (comp2) -- (intf2);
	\draw[<->] (intf2.south) -- (intf2.south |- middleware.north);
\end{tikz*}

		\end{center}

		\figureexpl{Промежуточное ПО связывает компоненты, определяя их~унифицированный интерфейс
				для~всех~поддерживаемых сред выполнения.}
	}

	\frame{
		\frametitle{Типы промежуточного ПО}

		\begin{center}
			Типы промежуточного ПО $\simeq$ парадигмы программирования.
		\end{center}

		\begin{itemize}
			\item
			Процедурное программирование — \textbf{удаленный вызов процедур}
			\engterm{remote procedure call}.

			\itemexpl{Посредник связывает интерфейсы и реализации отдельных процедур.}

			\item
			Объектно-ориентированное программирование — \textbf{посредник доступа к~объектам}
			\engterm{object request broker}.

			\itemexpl{Посредник хранит интерфейсы и реализации объектов, содержащих методы и атрибуты.}

			\item
			Событийно-ориентированное программирование — \textbf{очередь сообщений} \engterm{message queue}.

			\itemexpl{Посредник позволяет клиентам производить и потреблять сообщения определенного формата.}

			\item
			Реляционное программирование — \textbf{доступ к СУБД} \engterm{SQL data access}.

			\itemexpl{Посредник предоставляет доступ к БД с помощью программных интерфейсов.}
		\end{itemize}
	}

	\frame{
		\frametitle{Характеристики промежуточного ПО}

		\textbf{Метод согласования интерфейсов:}
		\begin{itemize}
			\item
			на основе отдельного языка спецификации;
			\item
			с помощью языков программирования (при поддержке ограниченного набора сред выполнения).
		\end{itemize}

		\vspace{1ex}
		\textbf{Роли пользователей:}
		\begin{itemize}
			\item
			архитектура «клиент — сервер»;
			\item
			однородная среда (peer to peer).
		\end{itemize}

		\vspace{1ex}
		\textbf{Метод обмена данными:}
		\begin{itemize}
			\item
			синхронный (с блокированием выполнения);
			\item
			асинхронный (без блокирования, т.\,е. на основе событий).
		\end{itemize}
	}

	\section{RPC}

	\frame{
		\frametitle{RPC}

		\begin{Definition}
			\textbf{Удаленный вызов процедур} \engterm{remote procedure call, RPC} — протокол межпроцессного взаимодействия, 
			предоставляющий возможность выполнять процедуры в других процессах как локальные.
		\end{Definition}

		\begin{Definition}
			\textbf{Удаленный вызов методов} \engterm{remote method invocation, RMI} — аналог RPC 
			в~объектно-ориентированном программировании.
		\end{Definition}

		\vspace{2.5ex}
		\begin{overlayarea}{\textwidth}{0.275\textheight}
			\begin{center}
				\begin{tikz*}[%
	every node/.style={rectangle,draw,align=center,minimum height=2.5em,minimum width=5em}
]
	\node(client) {Клиент};
	\node(stub) [right=of client] {Stub};
	\node(mw) [right=of stub] {\textbf{Промежуточное} \\ \textbf{ПО}};
	\node(skel) [right=of mw] {Skeleton};
	\node(server) [right=of skel] {Сервер};

	\draw[<->] (client) -- (stub);
	\draw[<->] (stub) -- (mw);
	\draw[<->] (mw) -- (skel);
	\draw[<->] (skel) -- (server);

	\node<2> [note={(stub.south)},below=1em of stub.south,anchor=north west] {%
		преобразует параметры вызываемых процедур \\ для передачи промежуточному ПО \engterm{marshalling}};
	\node<3> [note={(skel.south)},below=1em of skel.south,anchor=north east] {%
		преобразует сообщения от промежуточного ПО \\ в локальные вызовы \engterm{unmarshalling}};
\end{tikz*}

			\end{center}
		\end{overlayarea}
	}

	\subsection{Виды RPC}

	\frame{
		\frametitle{Виды RPC}

		\textbf{Процедурный и смешанный подход:}
		\begin{itemize}
			\item JSON-RPC;
			\item XML-RPC, SOAP.
		\end{itemize}

		\vspace{1ex}
		\textbf{Объектно-ориентированный подход:}
		\begin{itemize}
			\item CORBA;
			\item Java RMI;
			\item Microsoft .NET Remoting, Microsoft DCOM;
			\item Apache Thrift;
			\item ZeroC Internet Communication Engine (ICE).
		\end{itemize}
	}

	\subsection{CORBA}

	\frame{
		\frametitle{CORBA}

		\begin{Definition}
			\textbf{CORBA} (common object request broker architecture) — набор спецификаций, 
			разработанных Object Management Group (OMG) для обеспечения обмена данными на~основе 
			объектно-ориентированных интерфейсов.
		\end{Definition}

		\vspace{1ex}
		\textbf{Составляющие CORBA:}
		\begin{itemize}
			\item
			язык описания интерфейсов объектов (OMG interface description language, IDL);
			\item
			отображение типов данных IDL для различных языков программирования;
			\item
			протокол передачи данных: general inter-orb protocol (GIOP) и его реализация через HTTP 
			— Internet inter-orb protocol (IIOP);
			\item
			вспомогательные средства для передачи (stub / skeleton);
			\item
			инструменты, напр., сервис имен.
		\end{itemize}
	}

	\frame{
		\frametitle{Схема работы CORBA}

		\begin{center}
			\begin{tikz*}[%
	every node/.style={rectangle,draw,align=center,minimum height=2.5em,minimum width=6em},
	big/.style={minimum width=10em,minimum height=6em},
	label/.style={minimum height=0pt,minimum width=0pt,font=\footnotesize\itshape,draw=none}
]
	\node(client) [big] {Код клиента};
	\node(intf-cl) [below=-1.5em of client,fill=white] {Интерфейс \\ объекта};
	\node(stub) [below=of intf-cl] {Stub};
	\node(orb-cl) [below=of stub] {Посредник доступа \\ к объектам (ORB)};
	\draw[<->] (intf-cl) -- (stub);
	\draw[<->] (stub) -- (orb-cl);

	\node(server) [big,right=13em of client] {Код сервера};
	\node(intf-srv) [below=-1.5em of server,fill=white] {Имплементация \\ объекта};
	\node(skel) [below=of intf-srv] {Skeleton};
	\node(orb-srv) [below=of skel] {Посредник доступа \\ к объектам (ORB)};
	\draw[<->] (intf-srv) -- (skel);
	\draw[<->] (skel) -- (orb-srv);

	\draw[<->,dashed] (orb-cl) -- node(net) [above,label]{сетевое соединение} (orb-srv);

	\node<2> [note={(orb-cl.north east)},above=3em of net] {%
		код, предоставляемый поставщиком ORB};
	\node<3> [note={(orb-srv.north west)},above=3em of net] {%
		код, предоставляемый поставщиком ORB};
	\node<4> [note={(stub.east)},above=3em of net] {%
		код, генерируемый утилитами \\ от поставщика ORB};
	\node<5> [note={(skel.west)},above=3em of net] {%
		код, генерируемый утилитами \\ от поставщика ORB};
	\node<6> [note={(intf-cl.south east)},above=3em of net] {%
		код, генерируемый утилитами \\ от поставщика ORB};
\end{tikz*}

		\end{center}
	}

	\frame{
		\frametitle{Генерация кода в CORBA}

		В CORBA предусмотрены инструменты, которые автоматически генерируют:
		\begin{itemize}
			\item
			\textbf{интерфейс объекта} на поддерживаемом ЯП на основе IDL-описания;

			\vspace{0.5ex}
			\item
			\textbf{IDL-описание} на основе интерфейса / класса поддерживаемого ЯП;

			\vspace{0.5ex}
			\item
			\textbf{stub} (клиентский вспомогательный код) на основе IDL, который:
			\begin{itemize}
				\item реализует интерфейс объекта;
				\item адресует вызовы методов посреднику и обрабатывает получаемые результаты.
			\end{itemize}

			\vspace{0.5ex}
			\item
			\textbf{skeleton} (серверный вспомогательный код) на основе IDL, который:
			\begin{itemize}
				\item реализует интерфейс объекта;
				\item является основой для построения имплементации;
				\item преобразует принятые от посредника данные.
			\end{itemize}
		\end{itemize}
	}

	\frame{
		\frametitle{Реализации CORBA}

		\textbf{Системы, поддерживающие архитектуру CORBA:}
		\begin{itemize}
			\item
			Java Development Kit, как альтернатива / дополнение для Java RMI;

			\vspace{0.5ex}
			\item
			WildFly (ранее JBoss) (Java);

			\vspace{0.5ex}
			\item
			omniORB (C++, Python);

			\vspace{0.5ex}
			\item
			ORBit (C, Python);

			\vspace{0.5ex}
			\item
			IIOP.NET (C\# и другие языки MS .NET; поддерживаются не все стандарты).
		\end{itemize}

		\vspace{1ex}
		Несмотря на объектную ориентацию CORBA, для реализации сервисов и клиентов 
		можно использовать другие подходы (напр., процедурное программирование в~C).
	}

	\subsection{IDL}

	\frame{
		\frametitle{IDL}

		\begin{Definition}
			\textbf{Язык описания объектов} \engterm{interface description language / interface definition language} — 
			частный случай языка спецификации, служащий для описания интерфейса объекта независимым от языка реализации способом.
		\end{Definition}

		\vspace{1ex}
		\begin{center}
			\begin{tikz*}[%
	every node/.style={rectangle,draw,minimum height=2.5em,minimum width=7.5em},
	label/.style={draw=none,font=\small\itshape,minimum height=0},
	wrap/.style={inner sep=1em,dashed}
]
	\node(idl) {Интерфейс (IDL)};

	\node(i-srv) [below right=2em and 1em of idl] {Интерфейс (ЯП 2)};
	\node(impl) [below=of i-srv] {Реализация};
	\node(srv) [label,below=1em of impl] {Сервер};
	\node [fit=(i-srv.north west) (i-srv.north east) (srv.south east),wrap] {};

	\node(i-cli) [below left=2em and 1em of idl] {Интерфейс (ЯП 1)};
	\node(use) [below=of i-cli] {Вызовы методов};
	\node(cli) [label,below=1em of use] {Клиент};
	\node [fit=(i-cli.north west) (i-cli.north east) (cli.south east),wrap] {};

	\draw[<->] (idl) -- (i-srv);
	\draw[<->] (i-srv) -- (impl);
	\draw[<->] (idl) -- (i-cli);
	\draw[<->] (i-cli) -- (use);
\end{tikz*}


			\vspace{0.5ex}
			\figureexpl{Промежуточное ПО содержит средства для преобразования от конкретных ЯП к IDL и обратно.}
		\end{center}
	}

	\frame{
		\frametitle{OMG IDL}

		\textbf{Типы данных:}
		\begin{itemize}
			\item
			базовые типы данных:
			\begin{itemize}
				\item булев тип; 
				\item целые числа 2, 4, 8 байт со знаком и без; 
				\item байты; 
				\item 32-, 64- и 96/128-битные числа с плавающей запятой; 
				\item символы и строки ASCII и UTF-16);
			\end{itemize}

			\vspace{1ex}
			\item
			конструкции: 
			\begin{itemize}
				\item структуры \engterm{struct}, 
				\item перечисления \engterm{enum}, 
				\item маркированные объединения \engterm{tagged union}; 
				\item массивы (фиксированное число элементов); 
				\item последовательности (ограниченное сверху или произвольное число элементов);
			\end{itemize}
		\end{itemize}
	}

	\frame{
		\frametitle{OMG IDL}

		\textbf{Типы данных (продолжение):}
		\begin{itemize}
			\item исключения ($\sim$ структуры с особой семантикой);

			\vspace{1ex}
			\item
			интерфейсы объектов, содержащие:
			\begin{itemize}
				\item операции ($\sim$ методы), характеризующиеся входными / выходными параметрами, 
				возвращаемым типом и исключениями;
				\item атрибуты ($\sim$ свойства), в т.\,ч. только для чтения;
			\end{itemize}

			\vspace{1ex}
			\item
			any (произвольный объект); 

			\vspace{1ex}
			\item
			типы, передаваемые по значению \engterm{valuetype} — 
			$\sim$ классы в ООП (поля + методы); методы выполняются на стороне клиента.
		\end{itemize}
	}

	\frame{
		\frametitle{OMG IDL}

		\textbf{Характеристики языка:}
		\begin{itemize}
			\item
			декларации типов (в т.\,ч. вложенные), именованные константы;
			\item
			наследование интерфейсов (в т.\,ч. множественное);
			\item
			модули для группировки связанных интерфейсов;
			\item
			наличие препроцессора, возможность условной компиляции;
			\item
			подключение деклараций из других IDL-файлов.
		\end{itemize}

		\vspace{1ex}
		\textbf{Привязки к языкам программирования:}
		\begin{itemize}
			\item C++;
			\item Java;
			\item Python;
			\item Ruby;
			\item C\# и другие языки CLR (неполные).
		\end{itemize}
	}

	\frame{
		\frametitle{Пример: интерфейс числовой последовательности}

		\lstinputlisting[language={[CORBA]IDL}]{code-int-seq.idl}
	}

	\subsection{Модули и ссылки ORB}

	\frame{
		\frametitle{Модули ORB}

		\textbf{Сервис имен} — идентификация объектов по имени ($\simeq$URI), 
		состоящем из $\geq 0$ контекстов (директорий) и названия (и, возможно, типа) объекта.

		\textbf{Применение:} упрощение доступа к объектам.

		\vspace{2ex}
		\textbf{POA} (portable object adapter) — разделение реализации объекта на 2 части:
		\begin{itemize}
			\item
			CORBA-совместимый объект — код, связанный с преобразованием данных;
			\item
			слуга \engterm{servant} — реализация методов и атрибутов объекта.
		\end{itemize}

		\textbf{Применение:} динамический подбор реализации (напр., в зависимости от метода); 
		балансировка нагрузки.

		\vspace{2ex}
		\textbf{Репозиторий интерфейсов} — содержит все зарегистрированные в CORBA определения типов.

		\textbf{Применение:} получение динамических сведений о структуре объектов.
	}

	\frame{
		\frametitle{Объекты CORBA}

		CORBA предоставляет доступ к объектам в виде ссылок. 
		Интерфейс ссылки состоит из~2~частей: 
		\begin{itemize}
			\item интерфейс, определенный в IDL;
			\item базовый интерфейс \code{CORBA::Object}.
		\end{itemize}

		\vspace{1ex}
		\textbf{Методы базового интерфейса:}
		\begin{itemize}
			\item
			динамическое определение интерфейса: \code{get\_interface}, \code{repository\_id}, \code{is\_a};
			\item
			копирование и удаление ссылок на объект (для управления сборкой мусора): \code{duplicate}, \code{release};
			\item
			тестирование доступности объекта: \code{is\_nil}, \code{non\_existent};
			\item
			сравнение объектов: \code{is\_equivalent}, \code{hash};
			\item
			приведение типов: \code{narrow}.
		\end{itemize}
	}

	\frame{
		\frametitle{Пример: работа с удаленным объектом в Python}

		\lstinputlisting[language=python]{code-corba.py}
	}

	\section{Очередь сообщений}

	\frame{
		\frametitle{Очередь сообщений}

		\begin{Definition}
			\textbf{Очередь сообщений} \engterm{message queue, mailbox} — вид межпроцессного взаимодействия, 
			при котором отправитель и получатель данных слабо связаны \engterm{loosely coupled}.
		\end{Definition}

		\vspace{0.5ex}
		\begin{itemize}
			\item
			Роли сторон

			\textbf{RPC:} ассиметричные — клиент (потребитель) и сервер (реализация); \\
			\textbf{MQ:} симметричные — каждый клиент может получать и отправлять сообщения.

			\item
			Связь 

			\textbf{RPC:} сильная — клиент знает интерфейс сервера; \\
			\textbf{MQ:} слабая — получатель знает только формат сообщений.

			\item
			Передача данных

			\textbf{RPC:} синхронная — по запросу клиента; сервер должен быть активен на момент запроса. \\
			\textbf{MQ:} асинхронная — сообщения хранятся в очереди до доставки клиенту.
		\end{itemize}
	}

	\subsection{Характеристики}

	\frame{
		\frametitle{Характеристики очередей сообщений}

		\textbf{Преимущества по сравнению с RPC:}
		\begin{itemize}
			\item
			отсутствие необходимости спецификации интерфейсов компонентов;
			\item
			быстрая замена и модификация компонентов;
			\item
			нет необходимости в одновременном функционировании всех компонентов;
			\item
			возможность построения сложной архитектуры поставщиков и потребителей сообщений.
		\end{itemize}

		\vspace{1ex}
		\textbf{Недостатки:}
		\begin{itemize}
			\item
			асинхронная модель подходит не для всех архитектур приложений (напр., малопригодна для систем реального времени);
			\item
			затраты на организацию и обеспечение отказоустойчивости брокера сообщений.
		\end{itemize}
	}

	\subsection{Способы организации}

	\frame{
		\frametitle{Организация очереди сообщений}

		\begin{center}
			\begin{tikz*}[%
	every node/.style={rectangle,draw,align=center,minimum height=3em,minimum width=7.5em},
	label/.style={draw=none,font=\small\itshape,minimum height=0pt,minimum width=0pt},
	wrap/.style={inner sep=1em,dashed}
]

	\node(queue) [fill=blue!25] {Очередь};
	\node(client1) [below left=4em of queue]{Клиент 1};
	\node(broker) [above=of queue,label] {Брокер сообщений};
	\node(client2) [below right=4em of queue] {Клиент 2};
	\node [fit=(broker.north west) (broker.north east) (queue.south),wrap] {};

	\draw[->] (client1) -- node[label,left] {отправка\ } (queue);
	\draw[->,bend left] (queue) to node[label,above right] {прием} (client2);
	\draw[<-,bend right] (queue) to node[label,below left] {подтверждение} (client2);
\end{tikz*}

		\end{center}

		Организация очереди сообщений в формате \textbf{point-to-point} (PTP):
		\begin{itemize}
			\item один потребитель для каждого сообщения;
			\item асинхронная работа отправителя и приемника;
			\item сохранение сведений о приеме сообщений.
		\end{itemize}
	}

	\frame{
		\frametitle{Организация очереди сообщений}

		\begin{center}
			\begin{tikz*}[%
	every node/.style={rectangle,draw,align=center,minimum height=3em,minimum width=7.5em},
	label/.style={draw=none,font=\small\itshape,minimum height=0pt,minimum width=0pt},
	wrap/.style={inner sep=1em,dashed}
]

	\node(topic) [fill=blue!25] {Тема (topic)};
	\node(client1) [below=4em of queue]{Клиент 1};
	\node(broker) [above=of queue,label] {Брокер сообщений};
	\node(client2) [right=8em of queue] {Клиент 2};
	\node(client3) [left=8em of queue] {Клиент 3};
	\node [fit=(broker.north west) (broker.north east) (queue.south),wrap] {};

	\draw[->] (client1) -- node[label,left] {отправка\ } (queue);
	\draw[->,bend left=15] (topic) to node[label,above] {прием} (client2);
	\draw[<-,bend right=15] (topic) to node[label,below] {подписка} (client2);
	\draw[->,bend left=15] (topic) to node[label,below] {прием} (client3);
	\draw[<-,bend right=15] (topic) to node[label,above] {подписка} (client3);
\end{tikz*}

		\end{center}

		Организация очереди сообщений в формате \textbf{publish / subscribe}:
		\begin{itemize}
			\item
			неограниченное количество потребителей для каждого сообщения;
			\item
			прием сообщений потребителями только в периоды активности (сообщения не~«откладываются»).
		\end{itemize}
	}

	\subsection{Реализации}

	\frame{
		\frametitle{Реализации очереди сообщений}

		\textbf{Java Message Queue} — стандарт обмена сообщениями в Java, часть Java Enterprise Edition (Java~EE).

		\textbf{Реализации:}
		\begin{itemize}
			\item Apache ActiveMQ;
			\item OpenMQ (часть сервера приложений Glassfish);
			\item IBM WebSphere MQ.
		\end{itemize}

		\vspace{2ex}
		\textbf{AMQP} (advanced message queuing protocol) — протокол передачи сообщений на уровне приложения.

		\textbf{Реализации:}
		\begin{itemize}
			\item Apache Qpid; 
			\item RabbitMQ; 
			\item Microsoft Azure Service Bus; 
			\item ZeroMQ.
		\end{itemize}
	}

	\section{Заключение}

	\subsection{Выводы}
	
	\frame{
		\frametitle{Выводы}

		\begin{enumerate}
			\item
			Интероперабельность — технологии взаимодействия между компонентами программной системы, 
			написанными и~выполняемыми в~различных средах. 
			Близкое к~интероперабельности понятие — межпроцессное взаимодействие (IPC).

			\item
			Низкоуровневые средства для межпроцессного взаимодействия встроены во~все~операционные системы. 
			Высокоуровневые методы IPC используют промежуточное~ПО (\emph{middleware}).

			\item
			Есть две основные категории промежуточного ПО: на~основе процедур / объектов и~на~основе сообщений.

			\item
			Одним из наиболее полных стандартов промежуточного~ПО является CORBA, 
			технология, позволяющая работать с~удаленными объектами как~с~локальными.
		\end{enumerate}
	}
	
	\subsection{Материалы}
	
	\frame{
		\frametitle{Материалы}
		
		\begin{thebibliography}{9}
			\bibitem[1]{1}
			Лавріщева К.\,М. 
			\newblock Програмна інженерія (підручник). 
			\newblock {\footnotesize К., 2008. — 319 с.}

			\bibitem[2]{2}
			Object Management Group
			\newblock CORBA specifications.
			\newblock {\footnotesize\url{http://www.omg.org/spec/}}

			\bibitem[3]{3}
			Oracle
			\newblock The Java EE Tutorial. Chapter 45 "Java Message Service Concepts".
			\newblock {\footnotesize\url{http://docs.oracle.com/javaee/7/tutorial/jms-concepts001.htm}}
		\end{thebibliography}
	}
	
	\frame{
		\frametitle{}
		
		\begin{center}
			\Huge Спасибо за внимание!
		\end{center}
	}

\end{document}

