\documentclass{a4beamer}
%% Lectures - common definitions

\usextensions{tikz}
\usetikzlibrary{shapes.multipart,shapes.callouts,shapes.geometric}
\input{fix-callouts.inc} % Fixes absolute positioning of rectangle callouts

\newif\ifbigpages \bigpagesfalse
\ifdim\paperwidth >20cm
	\bigpagestrue
\fi

\tikzset{%
	note/.style={rectangle callout,draw=none,callout pointer width=1em,%
		align=flush left,font=\footnotesize,inner sep=0.5em,%
		fill=blue!15,fill opacity=0.95,text opacity=1.0,callout absolute pointer=#1},
	node distance=2em and 2.75em
}
\ifbigpages
	% Scale all arrow tips by the factor of 2.5
	\let\old@pgf@arrow@call=\pgf@arrow@call
	\def\pgf@arrow@call#1{%
		\@tempdima=\pgflinewidth%
		\pgfsetlinewidth{2.5\pgflinewidth}%
		\old@pgf@arrow@call{#1}%
		\pgfsetlinewidth{\@tempdima}%
	}
	\def\pgfarrowsleftextend#1{\pgfmathsetlength{\pgf@xa}{1.5*#1}}
	\def\pgfarrowsrightextend#1{\pgfmathsetlength{\pgf@xb}{1.5*#1}}
\fi

%% Load listings package
\usepackage{listings}

%% Are we inside a comment?
\newif\iflstcomment \lstcommentfalse

\lstset{%
	tabsize=4,
	showstringspaces=false,
	basicstyle=\linespread{1.25}\ttfamily\small,
	keywordstyle=\bfseries,
	commentstyle=\lstcommentstyle,
	numbers=left,
	numberstyle=\footnotesize\color{gray},
	xleftmargin=2.5em,
	extendedchars=true,
	escapechar=\$,
	escapebegin=\iflstcomment\begingroup\lstcommentstyle\fi,
	escapeend=\iflstcomment\endgroup\fi
}

\def\lstcommentstyle{\color{gray}}

\lst@AddToHook{AfterBeginComment}{\global\lstcommenttrue}
\let\orig@lst@EndComment=\lst@EndComment
\def\lst@EndComment{\global\lstcommentfalse\orig@lst@EndComment}
\lst@AddToHookAtTop{EOL}{%
	\lst@ifLmode\global\lstcommentfalse\fi% XXX Sloppy way to determine comment end
}

%% Python with docstrings treated as comments
\lstdefinelanguage[doc]{python}[]{python}{%
	deletestring=[s]{"""}{"""},%
	morecomment=[s]{"""}{"""}%
}%

%% JavaScript language
\lstdefinelanguage{javascript}%
	{morekeywords={break,case,catch,%
		const,constructor,continue,default,do,else,false,%
		finally,for,function,if,in,instanceof,%
		new,null,prototype,%
		return,switch,this,throw,%
		true,try,typeof,var,while},%
	sensitive,%
	morecomment=[l]//,%
	morecomment=[s]{/*}{*/},%
	morestring=[b]",%
	morestring=[b]',%
}[keywords,comments,strings]%

%% C# language (4.0?)
\lstdefinelanguage{csharp}%
	{morekeywords={abstract,as,%
		base,bool,byte,case,catch,char,%
		checked,class,const,continue,%
		decimal,default,delegate,do,double,%
		else,enum,event,explicit,extern,%
		false,finally,fixed,float,for,foreach,%
		goto,if,implicit,in,int,interface,%
		internal,is,lock,long,%
		namespace,new,null,object,operator,out,%
		override,params,private,protected,public,%
		readonly,ref,return,sbyte,sealed,%
		short,sizeof,stackalloc,static,string,%
		struct,switch,this,throw,true,try,%
		typeof,uint,ulong,unchecked,unsafe,ushort,%
		using,virtual,void,volatile,while%
	},%
	sensitive,%
	morecomment=[l]//,%
	morecomment=[s]{/*}{*/},%
	morestring=[b]",%
	morestring=[b]',%
}[keywords,comments,strings]%

%% Translation for fact environment
\deftranslation[to=russian]{Fact}{Наблюдение}

%% Inline code snippets
\def\code#1{\texttt{#1}}
\def\codekw#1{\code{\textbf{#1}}}

\def\quoteauthor#1{\par\footnotesize\upshape\hfill—~#1}

%% English term
\def\engterm#1{(англ. \textit{#1})}
%% Term with explanation below (to be used in diagrams)
\def\termwithexpl#1#2{#1\strut{}\\\small\color{gray}(\textit{#2})\strut{}}
%% External link
\def\extlink#1#2{\href{#1}{\color[rgb]{0.7,0.7,1.0}\dashbar{#2}}}
%% Internal link
\def\inlink#1#2{\hyperlink{#1}{\color[rgb]{0.7,0.7,1.0}\dashbar{#2}}}
%% Explanation for a list item
\def\itemexpl#1{\begingroup\small\vspace{0.75ex}#1\par\endgroup}



\usetikzlibrary{shapes.misc}


\lecturetitle{Программная инженерия. Лекция №25 — Хранение данных}
\title[Хранение данных]{Хранение данных. Сериализация и объектно-реляционные отображения}
\author{Алексей Островский}
\institute{\small{Физико-технический учебно-научный центр НАН Украины}\vspace{2ex}}
\date{14 мая 2015 г.}

\begin{document}
	\frame{\titlepage}

	\section{Введение}

	\subsection{Задача хранения данных}

	\frame{
		\frametitle{Data persistence}

		\begin{Definition}
			\textbf{Постоянное хранение данных} \engterm{data persistence} — сохранение информации 
			после~прекращения существования процесса, породившего эту~информацию.
		\end{Definition}

		\vspace{1ex}
		\begin{figure}
			\begin{tikz*}[%
	every node/.style={rectangle,draw,align=center},
	label/.style={draw=none,font=\footnotesize\itshape}
]
	\node(ram) [minimum width=10em,minimum height=10em] {};
	\node(ram-lbl) [draw=none,below=0em of ram.north,anchor=north] {Оперативная \\ память};
	\node(process) [minimum width=7.5em,minimum height=6em,fill=white,
		below left=1em and 1em of ram.east,anchor=center] {Процесс};
	\node(db) [minimum width=10em,minimum height=10em,right=15em of ram] {Постоянная \\ память};

	\node(_tmp) [coordinate,above=1em of process.east] {};
	\draw[->] (_tmp) -- node[label,above] {сохранение} (_tmp -| db.west);
	\node(_tmp) [coordinate,below=1em of process.east] {};
	\draw[<-] (_tmp) -- node[label,below] {загрузка} (_tmp -| db.west);
\end{tikz*}

			\caption{Data persistence подразумевает двусторонний процесс: 
				загрузку данных из~постоянной памяти в~оперативную и~наоборот}
		\end{figure}
	}

	\frame{
		\frametitle{Задача хранения данных}

		\begin{Problem}
			Добавить к~функциональности объекта возможность его~хранения в~постоянном хранилище данных 
			(напр., в~двоичном файле или~в~реляционной базе данных) 
			при~условии минимальной модификации его~внутренней структуры.
		\end{Problem}

		\vspace{1ex}
		\textbf{Проблемы:}
		\begin{itemize}
			\item
			хранение отношений с другими объектами, в~т.\,ч.~рекурсивных;
			\item
			работа с избыточностью (напр., с~множественными ссылками на~один~объект);
			\item
			прозрачное преобразование информации (напр., для~минимизации объема хранимых данных);
			\item
			независимость от среды выполнения.
		\end{itemize}
	}

	\subsection{Представления данных}

	\frame{
		\frametitle{Представления данных}

		\textbf{Оперативная память:}
		\begin{itemize}
			\item
			формат определяется ABI среды выполнения; 
			\item
			оптимизация скорости произвольного доступа к~данным (напр., выравнивание полей объектов); 
			\item 
			связь с другими объектами через указатели.
		\end{itemize}

		\vspace{0.5ex}
		\textbf{Потоки данных} (сериализация): 
		\begin{itemize}
			\item
			формат определяется ABI сервиса сериализации или стандартом (JSON, XML, …); 
			\item
			оптимизация объема данных (напр., преобразование строк UTF-16 $\rightarrow$ UTF-8); 
			\item
			связь с другими объектами через специальные средства.
		\end{itemize}

		\vspace{0.5ex}
		\textbf{Реляционная БД} (объектно-реляционные отображения):
		\begin{itemize}
			\item
			формат определяется схемой БД; 
			\item
			оптимизация отношений между данными; 
			\item
			связь с другими объектами через внешние ключи.
		\end{itemize}
	}

	\frame{
		\frametitle{Примеры хранения данных}

		\textbf{Двоичные потоки данных:}
		\begin{itemize}
			\item
			сохранение / загрузка объектов в виде локальных файлов;
			\item
			передача данных в сети, напр., в CORBA.
		\end{itemize}

		\vspace{0.5ex}
		\textbf{Текстовые потоки данных:}
		\begin{itemize}
			\item
			передача данных в сети, напр., при работе с веб-сервисами;
			\item
			сохранение объектов с произвольной структурой (напр., в БД).
		\end{itemize}

		\vspace{0.5ex}
		\textbf{Базы данных:}
		\begin{itemize}
			\item
			хранение данных о моделях в~архитектуре Model — View — Controller (MVC);
			\item
			хранение сущностей предметной области \engterm{entity} в~Enterprise-приложениях.
		\end{itemize}
	}

	\section{Сериализация}

	\frame{
		\frametitle{Сериализация}

		\begin{Definition}
			\textbf{Сериализация} \engterm{serialization}, \textbf{маршалинг} \engterm{marshalling} — 
			перевод структур данных или~состояния объекта в~формат, доступный для~хранения или~передачи, 
			с~возможностью восстановления из~сериализованных данных семантически эквивалентного объекта.
		\end{Definition}

		\begin{Definition}
			\textbf{Десериализация} \engterm{deserialization, unmarshalling} — процесс, обратный сериалиазации: 
			восстановление состояния объекта или~структуры данных из~сериализованного представления.
		\end{Definition}

		\begin{figure}
			\begin{tikz*}[%
	every node/.style={rectangle,draw,align=center},
	label/.style={draw=none,font=\footnotesize\itshape}
]
	\node(object) [minimum width=7.5em,minimum height=6.5em] {Объект};
	\node(service) [minimum width=7.5em,minimum height=6.5em,right=7.5em of object,dashed] {Сервис \\ сериализации};
	\node(stream) [minimum width=7.5em,minimum height=6.5em,right=7.5em of service] {Поток \\ данных};

	\node(_tmp) [coordinate,above=1.5em of object.east] {};
	\draw[->] (_tmp) -- node[label,above,pos=0.175] {сериализация} (_tmp -| stream.west);
	\node(_tmp) [coordinate,below=1.5em of object.east] {};
	\draw[<-] (_tmp) -- node[label,below,pos=0.175] {десериализация} (_tmp -| stream.west);
\end{tikz*}

			\caption{Сериализация и десериализация могут осуществляться с использованием 
				встроенных или сторонних сервисов}
		\end{figure}
	}

	\frame{
		\frametitle{Достоинства и недостатки сериалиазации}

		\textbf{Достоинства:}
		\begin{itemize}
			\item
			компактность хранения данных (для двоичных форматов);
			\item
			легкость восприятия и отладки (для текстовых форматов);
			\item
			универсальность;
			\item
			независимость от языка программирования (для~текстовых и~двоичных форматов, 
			определенных внешними стандартами).
		\end{itemize}

		\vspace{1ex}
		\textbf{Недостатки:}
		\begin{itemize}
			\item
			линейность формата хранения данных $\Rightarrow$ для~извлечения части информации (напр., отдельного поля объекта) 
			в худшем случае требуется прочесть весь поток данных;
			\item
			проблемы валидации данных;
			\item
			проблемы совместимости для различных сред выполнения.
		\end{itemize}
	}

	\frame{
		\frametitle{Классификация методов сериализации}

		\textbf{Варианты использования:}
		\begin{itemize}
			\item
			передача данных: 
			\begin{itemize}
				\item в распределенных приложениях (напр., в~архитектуре~CORBA); 
				\item при удаленном вызове функций (напр., в веб-сервисах на~основе~SOAP или~REST);
			\end{itemize}

			\item
			хранение данных.
		\end{itemize}

		\vspace{1ex}
		\textbf{Формат данных:}
		\begin{itemize}
			\item
			двоичный — оптимизация объема данных;
			\item
			текстовый — читаемость сериализованных данных человеком.
		\end{itemize}

		\vspace{1ex}
		\textbf{Дополнительные характеристики:}
		\begin{itemize}
			\item
			система типов данных, возможность определения пользовательских типов;
			\item
			наличие / отсутствие пользовательских схем данных и возможности валидации;
			\item
			поддержка ссылок на части сериализованных данных.
		\end{itemize}
	}

	\subsection{Схема данных}

	\frame{
		\frametitle{Схема данных}

		\begin{Definition}
			\textbf{Схема данных} \engterm{data schema} — спецификация порядка, 
			в~котором хранятся сериализованные данные, их~семантики, 
			а~также описание отображения встроенных и~пользовательских структур данных 
			в~сериализованный формат и обратно.
		\end{Definition}

		\vspace{1ex}
		\textbf{Назначение:}
		\begin{itemize}
			\item корректная десериализация;
			\item проверка формата данных, полученных из внешних источников.
		\end{itemize}

		\vspace{1ex}
		\textbf{Спецификация:}
		\begin{itemize}
			\item
			для двоичных форматов: ABI сервиса сериализации и / или ЯП;

			\item
			для текстовых форматов:
			\begin{itemize}
				\item
				спецификация формата данных;
				\item
				спецификация формата данных + заданная программистом схема валидации 
				(напр., XML~Schema или~DTD для~XML).
			\end{itemize}
		\end{itemize}
	}

	\subsection{Сериализация в Java}

	\frame{
		\frametitle{Сериализация в Java}

		\textbf{Принципы сериализации:}
		\begin{itemize}
			\item
			Сериализуемые объекты должны помечаться интерфейсом 
			\extlink{http://docs.oracle.com/javase/8/docs/api/java/io/Serializable.html}{\code{java.io.Serializable}} (не~имеет методов).

			\vspace{1ex}
			\item
			Сериализуются все нестатические поля объекта, не~отмеченные ключевым словом \codekw{transient}.

			\vspace{1ex}
			\item
			Корректно сериализуются ссылки на другие объекты (в~т.\,ч. множественные ссылки на~один объект 
			и~рекурсивные ссылки).

			\vspace{1ex}
			\item
			К сериализуемым классам относятся:
			\begin{itemize}
				\item примитивные типы (\codekw{byte}, \codekw{char}, \codekw{int}, …);
				\item обертки (Byte, Character, Integer, …);
				\item массивы;
				\item встроенные реализации коллекций (ArrayList, HashSet, HashMap, TreeSet, TreeMap), …
			\end{itemize}
		\end{itemize}
	}

	\frame{
		\frametitle{Сериализация в Java}

		\textbf{Принципы сериализации (продолжение):}
		\begin{itemize}
			\item
			Для чтения / записи используются потоки 
			\extlink{http://docs.oracle.com/javase/8/docs/api/java/io/ObjectInputStream.html}{ObjectInputStream} / 
			\extlink{http://docs.oracle.com/javase/8/docs/api/java/io/ObjectOutputStream.html}{ObjectOutputStream}.

			\vspace{1ex}
			\item
			Для изменения поведения при чтении / записи используются специальные функции, 
			декларируемые в классе: 
			\lstinputlisting[language=java]{code-ser-methods.java}

			\vspace{1ex}
			\item
			Специальные функции вызываются в корректном порядке при~наследовании. 
			Сериализуемый класс может~быть потомком несериализуемого класса.
		\end{itemize}
	}

	\frame{
		\frametitle{Сериализация в Java — пример сериализуемого класса}

		\lstinputlisting[language=java]{code-serialization.java}
	}

	\frame{
		\frametitle{Сериализация в Java — пример использования}

		\lstinputlisting[language=java]{code-serialization-i.java}
	}

	\subsection{Сериализация в Python}

	\frame{
		\frametitle{Сериализация в Python}

		\textbf{Принципы сериализации:}
		\begin{itemize}
			\item
			Для сериализации используется модуль \extlink{https://docs.python.org/2/library/pickle.html}{\code{pickle}}
			(входит в~стандартную библиотеку).

			\vspace{0.5ex}
			\item
			Для быстрой сериализации может использоваться модуль \code{cPickle} (аналог pickle, 
			написанный на~C).

			\vspace{0.5ex}
			\item
			Поддерживается сериализация:
			\begin{itemize}
				\item
				примитивных типов (\codekw{bool}, \codekw{int}, \codekw{long}, \codekw{float}, \codekw{complex}, \codekw{None});
				\item
				строк (\codekw{str}, \codekw{unicode}); 
				\item
				коллекций (\codekw{tuple}, \codekw{list}, \codekw{set}, \codekw{dict}, …);
				\item
				функций и классов, определенных на уровне модуля, т.\,е. не являющихся вложенными 
				(сериализуется только имя класса / функции, но не код или переменные);
				\item
				экземпляров таких классов.
			\end{itemize}

			\vspace{0.5ex}
			\item
			По умолчанию сериализуется словарь объекта (\code{obj.\_\_dict\_\_}).

			\vspace{0.5ex}
			\item
			Для изменения поведения при сериализации используются функции \code{\_\_getstate\_\_}, \code{\_\_setstate\_\_}.
		\end{itemize}
	}

	\frame{
		\frametitle{Сериализация в Python — пример сериализуемого класса}

		\lstinputlisting[language=python,firstline=4]{code-serialization.py}
	}

	\frame{
		\frametitle{Сериализация в Python — пример использования}

		\lstinputlisting[language=python,firstline=4,morekeywords={with,as}]{code-serialization-i.py}
	}

	\section{XML и JSON}

	\subsection{XML}

	\frame{
		\frametitle{XML}

		\begin{Definition}
			\textbf{XML} (extensible markup language) — язык разметки документов, созданный международным комитетом~W3C 
			и~ориентированный на воспринимаемость человеком и~программами.
		\end{Definition}

		\vspace{1ex}
		\textbf{Преимущества:}
		\begin{itemize}
			\item
			легок для восприятия и отладки человеком;
			\item
			широко используется в веб-приложениях (HTML — $\approx$~частный вариант XML);
			\item
			большое количество инструментов для работы с XML во многих ЯП;
			\item
			наличие стандартных механизмов валидации (DTD, XML Schema) 
			и~преобразования формата данных (XSLT).
		\end{itemize}

		\vspace{1ex}
		\textbf{Недостатки:}
		\begin{itemize}
			\item
			определенная громоздкость языка;
			\item
			отсутствие встроенной семантики типов и структур данных.
		\end{itemize}
	}

	\frame{
		\frametitle{Структура XML-документов}

		\textbf{Составляющие XML:}
		\begin{itemize}
			\item
			текст;

			\vspace{0.5ex}
			\item
			теги (открывающиеся \code{<foo>}, закрывающиеся \code{</foo>}, пустые \code{<foo/>});

			\vspace{0.5ex}
			\item
			атрибуты тегов (\code{<foo \alert{bar="bazz"}>});

			\vspace{0.5ex}
			\item
			комментарии (\code{<!-- comment -->});

			\vspace{0.5ex}
			\item
			сущности \engterm{entity} $\simeq$ макросы, использующиеся в~тексте для~устранения неоднозначностей. 

			\vspace{0.5ex}
			\textbf{Пример:} \code{\&gt;} заменяет символ \code{>}, \code{\&lt;} — \code{<}.

			\vspace{0.5ex}
			\item
			декларация XML, определяющая версию XML (1.0 или 1.1) и кодировку (UTF-8, UTF-16).

			\vspace{0.5ex}
			\textbf{Пример:} \code{<?xml version="1.1" encoding="UTF-8"?>}.
		\end{itemize}
	}

	\frame{
		\frametitle{Пример XML-документа}

		\lstinputlisting[language=xml,morekeywords={encoding}]{code-sample.xml}
	}

	\frame{
		\frametitle{Чтение / запись XML}

		\begin{itemize}
			\item
			\textbf{На основе событий.} Отдельные события соответствуют появлению текста, открывающих / закрывающих тегов, атрибутов.

			\vspace{1ex}
			\textbf{Использование:} для извлечения малого объема информации из больших документов.

			\vspace{0.5ex}
			\textbf{Пример:} Simple API for XML (\extlink{http://www.saxproject.org/}{SAX}).

			\vspace{1ex}
			\item
			\textbf{На основе потоков} чтения / записи \engterm{pull parsing}. 
			Документ представляется как последовательность символов (тегов, атрибутов, текста, …).

			\vspace{1ex}
			\textbf{Использование:} для извлечения малого объема информации из больших документов.

			\vspace{0.5ex}
			\textbf{Пример:} \extlink{http://www.xml.com/pub/a/2003/09/17/stax.html}{StAX}.
		\end{itemize}
	}

	\frame{
		\frametitle{Чтение / запись XML (продолжение)}

		\begin{itemize}
			\item
			\textbf{Объектный анализатор.} 
			Документ представляется как дерево объектов, соответствующих тегам и другим элементам XML.

			\vspace{1ex}
			\textbf{Использование:} для анализа документа целиком (напр., в веб-приложениях).

			\vspace{0.5ex}
			\textbf{Пример:} Document Object Model (\extlink{http://www.w3.org/DOM/}{DOM}).

			\vspace{1ex}
			\item
			\textbf{Привязка данных.} Элементы XML-документа отображаются в поля объектов.

			\vspace{1ex}
			\textbf{Использование:} для быстрой имплементации сериализации в XML.

			\vspace{0.5ex}
			\textbf{Пример:} Java Architecture for XML Binding (\extlink{https://jaxb.java.net/}{JAXB}).
		\end{itemize}
	}

	\frame{
		\frametitle{Схемы данных в XML}

		\textbf{Схема данных в XML:}
		\begin{itemize}
			\item
			определяет семантику элементов (допустимые атрибуты, дочерние элементы, …);
			\item
			позволяет проводить проверку корректности (валидацию) XML при чтении.
		\end{itemize}

		\vspace{1ex}
		\textbf{Определения схемы данных:}
		\begin{itemize}
			\item
			\textbf{DTD} (document type definition) — на основе регулярных выражений.

			\vspace{1ex}
			\textbf{+:} широкая распространенность; краткость по сравнению со~схемами на~основе~XML.

			\textbf{\textminus{}:} неполная поддержка свойств XML (напр., пространств имен); 
			отсутствие поддержки типов данных.

			\vspace{1ex}
			\item
			\textbf{XSD} (XML schema definition) — на основе XML.

			\vspace{1ex}
			\textbf{+:} поддержка типов данных и сложных ограничений на допустимые элементы и~атрибуты; 
			поддержка всех~возможностей XML~1.1.

			\textbf{\textminus{}:} громоздкость.
		\end{itemize}
	}

	\frame{
		\frametitle{API для работы с XML}

		\begin{itemize}
			\item
			\textbf{XSLT} — декларативный язык программирования на основе XML, 
			позволяющий преобразовывать XML-документы в другие XML-документы.

			\vspace{1ex}
			\textbf{Пример использования:} преобразование XML, извлеченного из~базы~данных, 
			в~XHTML (вариант HTML, совместимый с~XML).

			\vspace{1ex}
			\item
			\textbf{XPath} — язык запросов для выбора элементов XML-документа.

			\vspace{1ex}
			\textbf{Пример использования:} эффективное извлечение информации из~XML-документов.

			\vspace{1ex}
			\item
			\textbf{XQuery} — функциональный язык программирования для произвольных преобразований XML-документов.

			\vspace{1ex}
			\textbf{Пример использования:} преобразование XML, извлеченного из~базы~данных,
			в~HTML или~другой формат (напр., JSON).
		\end{itemize}
	}

	\subsection{JSON}


	\frame{
		\frametitle{JSON}

		\begin{Definition}
			\textbf{JSON} (JavaScript object notation) — формат разметки на~основе языка программирования JavaScript, 
			предназначенный для~передачи данных в~виде пар «атрибут — значение».
		\end{Definition}

		\textbf{Использование:} 
		\begin{itemize}
			\item в веб-приложениях как альтернатива XML;
			\item в документно-ориентированных базах данных (напр., \extlink{http://mongodb.org/}{MongoDB}).
		\end{itemize}

		\vspace{1ex}
		\textbf{Преимущества:}
		\begin{itemize}
			\item легкость восприятия человеком;
			\item меньший объем по сравнению с XML;
			\item наличие встроенной семантики типов данных.
		\end{itemize}

		\vspace{1ex}
		\textbf{Недостатки:} \\
		отсутствие стандартных API для многих задач (напр., валидации).
	}

	\frame{
		\frametitle{Структура JSON-документов}

		\textbf{Типы данных:}
		\begin{itemize}
			\item
			числа (напр., \code{1}, \code{-1.5}, \code{+2.71828E+10});

			\vspace{0.5ex}
			\item
			булев тип (\codekw{true}, \codekw{false});

			\vspace{0.5ex}
			\item
			\codekw{null};

			\vspace{0.5ex}
			\item
			строки (напр., \code{"a"}, \code{"string"});

			\vspace{0.5ex}
			\item
			массивы — упорядоченная коллекция из нуля или более элементов.

			\vspace{0.5ex}
			\textbf{Примеры:} \code{[]}; \code{["foo", 5, \textbf{false}]}.

			\vspace{0.5ex}
			\item
			объекты — неупорядоченный набор строковых ключей (атрибутов) 
			и~соответствующих значений.

			\vspace{0.5ex}
			\textbf{Примеры:} \code{\{\}}; \code{\{"foo": "bar", "bazz": \textbf{false}\}}.
		\end{itemize}
	}

	\frame{
		\frametitle{Пример документа JSON}

		\lstinputlisting[language=javascript]{code-sample.json}
	}

	\frame{
		\frametitle{API для JSON}

		\begin{itemize}
			\item
			\textbf{Чтение / запись:} встроенные или сторонние библиотеки во многих ЯП:
			\begin{itemize}
				\item
				JavaScript: \extlink{https://developer.mozilla.org/en-US/docs/Web/JavaScript/Reference/Global_Objects/JSON}{JSON}
				(также доступен через библиотеки, напр., \extlink{http://jquery.com}{jQuery});
				\item
				Java: \extlink{https://code.google.com/p/google-gson/}{Google GSON};
				\item
				PHP: функции \extlink{http://php.net/manual/en/book.json.php}{json\_encode, json\_decode};
				\item
				Python: модуль \extlink{https://docs.python.org/2/library/json.html}{json}.
			\end{itemize}

			\vspace{1ex}
			\item
			\textbf{Валидация:} \extlink{http://json-schema.org/}{JSON Schema} (на стадии разработки).

			\vspace{1ex}
			\item
			\textbf{Удаленный вызов процедур:} \extlink{http://www.jsonrpc.org/specification}{JSON-RPC}.

			\vspace{1ex}
			\item
			\textbf{Загрузка данных из сторонних источников:} \extlink{http://www.json-p.org/}{JSONP}.

			\vspace{1ex}
			\item
			\textbf{Преобразование в двоичный формат:} \extlink{http://bsonspec.org/}{BSON}.
		\end{itemize}
	}

	\section{ORM}

	\frame{
		\frametitle{Объектно-реляционные отображения}

		\textbf{Реляционные БД:}
		\begin{itemize}
			\item
			сущности — строки таблицы или представления в базе данных (т.\,е.~кортежи разнородных данных);
			\item
			нормализация с целью сократить объем хранимой информации;
			\item
			идентификация сущностей с помощью первичных ключей (primary~key);
			\item
			связи между сущностями с помощью внешних ключей (foreign~key).
		\end{itemize}

		\vspace{1ex}
		\textbf{ООП:}
		\begin{itemize}
			\item
			сущности — объекты (набор разнородных свойств + методы);
			\item
			идентификация сущностей с помощью сравнения;
			\item
			связи между сущностями с помощью ссылок на~другие объекты.
		\end{itemize}
	}

	\frame{
		\frametitle{Объектно-реляционные отображения}

		\begin{Problem}
			Определить двустороннее преобразование от~объектно-ориентированного 
			к~реляционному представлению данных.
		\end{Problem}

		\begin{Definition}
			\textbf{Объектно-реляционное отображение} \engterm{object-relational mapping, ORM} — 
			технология для~взаимодействия с~базами данных с~использованием парадигмы 
			объектно-ориентированного программирования.
		\end{Definition}

		\begin{figure}
			\begin{tikz*}[%
	class/.style={draw,rectangle split,rectangle split parts=2,align=left}
]
	\node(entity) [class,text width=8em] {%
		\hfill\textbf{Entity}\hfill\strut{}
		\nodepart{two}
		$+$ Поле 1 \\
		$+$ Поле 2 \\
		\hspace{2em}$\vdots$ \\
		$+$ Поле $n$
	};

	\node(table) [text width=20em,right=4em of entity,align=center,draw] {%
		\textbf{База данных} \\[.5ex]
		\renewcommand{\arraystretch}{1.5}%
		\begin{tabular}{|c|c|c|c|}
			\hline
			\textbf{Поле 1} & \textbf{Поле 2} & … & \textbf{Поле $n$} \cr
			\hline
			$a_{11}$ & $a_{12}$ & … & $a_{1n}$ \cr
			\hline
			$a_{21}$ & $a_{22}$ & … & $a_{2n}$ \cr
			\hline
			\multicolumn{4}{c}{$\vdots$}
		\end{tabular}
	};

	\draw[<->] (entity) -- (table);
\end{tikz*}
\vspace{-1ex}
			\caption{Связь между объектами и СУБД в рамках ORM}
		\end{figure}
	}

	\frame{
		\frametitle{Преимущества и недостатки ORM}

		\textbf{Преимущества} (по~сравнению с~использованием низкоуровневого доступа к~БД):
		\begin{itemize}
			\item
			высокоуровневые структуры данных (объекты вместо массивов или~ассоциативных таблиц) 
			упрощают программирование;
			\item
			упрощение управления БД, в~частности, миграции.
		\end{itemize}

		\vspace{1ex}
		\textbf{Преимущества} (по сравнению с~сериализацией):
		\begin{itemize}
			\item
			структуризация хранения данных упрощает их~анализ;
			\item
			встроенные в~БД ограничения целостности.
		\end{itemize}

		\vspace{1ex}
		\textbf{Недостатки:}
		\begin{itemize}
			\item
			повышенное потребление памяти;
			\item
			иллюзия простоты: неаккуратное использование~ORM может~привести 
			к~значительному повышению нагрузки на~БД.
		\end{itemize}
	}

	\subsection{Базовые понятия}

	\frame{
		\frametitle{Базовые понятия ORM}

		\begin{itemize}
			\item
			\textbf{Сущность} \engterm{entity} — $\sim$ класс в ООП, отдельное понятие предметной области 
			(соответствует таблице или~представлению~БД). 
			Экземпляры сущности соответствуют строкам таблицы / представления.

			\vspace{1ex}
			\item
			\textbf{Первичный ключ} \engterm{primary key}, \textbf{идентификатор} — поле или~набор полей, 
			позволяющий идентифицировать и~различать экземпляры сущности. 
			Соответствует декларации SQL \code{PRIMARY KEY}.

			\vspace{1ex}
			\item
			\textbf{Отношения между сущностями} \engterm{entity relationship} — связи, 
			определяемые через~декларации SQL \code{FOREIGN KEY}:
			\begin{itemize}
				\item один к одному;
				\item один ко многим;
				\item многие ко многим (через вспомогательную таблицу).
			\end{itemize}
		\end{itemize}
	}

	\frame{
		\frametitle{Базовые понятия ORM (продолжение)}

		\begin{itemize}
			\item
			\textbf{Ленивая загрузка} \engterm{lazy load} / \textbf{немедленная загрузка} \engterm{eager fetching} — 
			различные стратегии загрузки связанных объектов при~извлечении сущности из~БД 
			(при~загрузке основного объекта с~помощью~SQL \code{JOIN} или~при~попытке доступа).

			\vspace{1ex}
			\item
			\textbf{Ограничения} \engterm{constraint} — условия, накладываемые на~значения полей. 
			Соответствуют ограничениям на~типы данных SQL 
			(напр., максимальная длина строки для~поля типа~\code{VARCHAR(n)}). 
			Ограничения проверяются перед~сохранением данных в~БД.

			\vspace{1ex}
			\item
			\textbf{Язык запросов} \engterm{query language} — язык, похожий на~SQL, 
			предназначенный для~выполнения запросов к~БД, возвращающих сущности.
		\end{itemize}
	}

	\subsection{Шаблоны использования}

	\frame{
		\frametitle{Шаблон ActiveRecord}

		\begin{center}
			\begin{tabular}{p{0.2\textwidth}p{0.7\textwidth}}
				\textbf{Описание:} & \raggedright каждый объект обладает методами для~сохранения, 
					обновления и~удаления из~БД. \cr
				\raggedright \textbf{Псевдокод (Java):} & \raggedright 
				\vspace{-2ex}\lstinputlisting[language=java]{code-ar.java} 

				\vspace{-2.5ex}\strut{}\cr
				\textbf{Примеры:} & \raggedright Yii Framework. \cr
				\textbf{Достоинства:} & \raggedright краткость и очевидность кода. \cr
				\textbf{Недостатки:} & \raggedright повышенное число запросов к БД. \cr
			\end{tabular}
		\end{center}
	}

	\frame{
		\frametitle{Шаблон DataMapper}

		\begin{center}
			\begin{tabular}{p{0.2\textwidth}p{0.7\textwidth}}
				\textbf{Описание:} & \raggedright для извлечения и~сохранения объектов в~БД 
					используется вспомогательный класс. \cr
				\raggedright \textbf{Псевдокод (Java):} & \raggedright
				\vspace{-2ex}\lstinputlisting[language=java]{code-data-mapper.java} 

				\vspace{-2.5ex}\strut{}\cr
				\textbf{Примеры:} & \raggedright Doctrine (PHP), Java Persistence API. \cr
				\textbf{Достоинства:} & \raggedright возможность управления агрегацией операций. \cr
			\end{tabular}
		\end{center}
	}

	\frame{
		\frametitle{Пример использования}

		\begin{figure}
			\begin{tikz*}[%
	table/.style={draw,rectangle split,rectangle split parts=2,align=left},
	qty/.style={font=\footnotesize},
	fkey/.style={font=\footnotesize\ttfamily}
]
	\node(reader) [table] {%
		reader
		\nodepart{two}
		id: integer \\
		firstname: varchar(30) \\
		lastname: varchar(30) \\
		location\_id: integer
	};

	\node(checked-out) [table,right=10em of reader] {%
		checkedout
		\nodepart{two}
		reader\_id: integer \\
		book\_id: integer \\
		takendate: date \\
		duedate: date
	};

	\node(book) [table,below=5em of checked-out] {%
		book
		\nodepart{two}
		id: integer \\
		isbn: varchar(20) \\
		title: varchar(100) \\
		language\_id: tinyint \\
		author\_id: integer \\
		publisher\_id: smallint
	};

	\node(location) at (reader.center |- book.center) [table] {%
		location
		\nodepart{two}
		id: integer \\
		country: varchar(20) \\
		city: varchar(20)
	};

	\draw (reader) -- (location)
		node[qty,right,very near start] {1..*}
		node[qty,right,very near end] {1}
		node[fkey,left,midway,align=right] {FOREIGN KEY(location\_id) \\ REFERENCES location(id)};

	\draw (reader) -- (checked-out)
		node[qty,below,very near start] {0..*}
		node[fkey,above,midway,align=center] {FOREIGN KEY(reader\_id) \\ REFERENCES reader(id)};

	\draw (book) -- (checked-out)
		node[qty,right,very near start] {0..*}
		node[fkey,left,midway,align=right] {FOREIGN KEY(book\_id) \\ REFERENCES book(id)};
\end{tikz*}

			\caption{Схема организации данных в~БД. 
				Большинство ORM-систем умеют создавать схему на~основе классов сущностей.}
		\end{figure}
	}

	\frame{
		\frametitle{Пример использования}

		\lstinputlisting[language=java]{code-orm.java}
	}

	\section{Заключение}

	\subsection{Выводы}
	
	\frame{
		\frametitle{Выводы}

		\begin{enumerate}
			\item
			Задача хранения данных (напр., состояния объектов) вне~оперативной памяти 
			может~решаться с~помощью сериализации или~создания объектно-реляционного отображения 
			для~сущностей предметной области.

			\vspace{0.5ex}
			\item
			Сериализация — сохранение данных в~поток (двоичный или~текстовый). 
			Средства сериализации присутствуют в~большинстве языков программирования.

			\vspace{0.5ex}
			\item
			Два популярных стандарта текстовой сериализации — XML и~JSON — широко используются при~передаче данных, 
			напр., в~веб-приложениях.

			\vspace{0.5ex}
			\item
			Объектно-реляционные отображения используются для~установления связи между~БД и~программой, 
			написанной на~объектно-ориентированном~ЯП.
		\end{enumerate}
	}
	
	\subsection{Материалы}
	
	\frame{
		\frametitle{Материалы}
		
		\begin{thebibliography}{9}
			\bibitem[1]{1}
			Walsh, Norman
			\newblock Technical Introduction to XML.
			\newblock {\footnotesize\url{http://www.xml.com/pub/a/98/10/guide0.html}}

			\bibitem[2]{2}
			ECMA
			\newblock Introducing JSON.
			\newblock {\footnotesize\url{http://www.json.org/}}

			\bibitem[3]{3}
			Oracle
			\newblock Introduction to the Java Persistence API.
			\newblock {\footnotesize\url{http://docs.oracle.com/javaee/7/tutorial/persistence-intro001.htm}
			 	\\ (введение в ORM на примере JPA)}

			\bibitem[4]{4}
			Doctrine DevTeam
			\newblock Getting Started with Doctrine.
			\newblock {\footnotesize\url{http://docs.doctrine-project.org/projects/doctrine-orm/en/latest/tutorials/getting-started.html}
				\\ (введение в ORM на примере Doctrine)}
		\end{thebibliography}
	}
	
	\frame{
		\frametitle{}
		
		\begin{center}
			\Huge Спасибо за внимание!
		\end{center}
	}

\end{document}

