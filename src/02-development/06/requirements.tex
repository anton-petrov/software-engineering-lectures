\documentclass{a4beamer}
%% Lectures - common definitions

\usextensions{tikz}
\usetikzlibrary{shapes.multipart,shapes.callouts,shapes.geometric}
\input{fix-callouts.inc} % Fixes absolute positioning of rectangle callouts

\newif\ifbigpages \bigpagesfalse
\ifdim\paperwidth >20cm
	\bigpagestrue
\fi

\tikzset{%
	note/.style={rectangle callout,draw=none,callout pointer width=1em,%
		align=flush left,font=\footnotesize,inner sep=0.5em,%
		fill=blue!15,fill opacity=0.95,text opacity=1.0,callout absolute pointer=#1},
	node distance=2em and 2.75em
}
\ifbigpages
	% Scale all arrow tips by the factor of 2.5
	\let\old@pgf@arrow@call=\pgf@arrow@call
	\def\pgf@arrow@call#1{%
		\@tempdima=\pgflinewidth%
		\pgfsetlinewidth{2.5\pgflinewidth}%
		\old@pgf@arrow@call{#1}%
		\pgfsetlinewidth{\@tempdima}%
	}
	\def\pgfarrowsleftextend#1{\pgfmathsetlength{\pgf@xa}{1.5*#1}}
	\def\pgfarrowsrightextend#1{\pgfmathsetlength{\pgf@xb}{1.5*#1}}
\fi

%% Load listings package
\usepackage{listings}

%% Are we inside a comment?
\newif\iflstcomment \lstcommentfalse

\lstset{%
	tabsize=4,
	showstringspaces=false,
	basicstyle=\linespread{1.25}\ttfamily\small,
	keywordstyle=\bfseries,
	commentstyle=\lstcommentstyle,
	numbers=left,
	numberstyle=\footnotesize\color{gray},
	xleftmargin=2.5em,
	extendedchars=true,
	escapechar=\$,
	escapebegin=\iflstcomment\begingroup\lstcommentstyle\fi,
	escapeend=\iflstcomment\endgroup\fi
}

\def\lstcommentstyle{\color{gray}}

\lst@AddToHook{AfterBeginComment}{\global\lstcommenttrue}
\let\orig@lst@EndComment=\lst@EndComment
\def\lst@EndComment{\global\lstcommentfalse\orig@lst@EndComment}
\lst@AddToHookAtTop{EOL}{%
	\lst@ifLmode\global\lstcommentfalse\fi% XXX Sloppy way to determine comment end
}

%% Python with docstrings treated as comments
\lstdefinelanguage[doc]{python}[]{python}{%
	deletestring=[s]{"""}{"""},%
	morecomment=[s]{"""}{"""}%
}%

%% JavaScript language
\lstdefinelanguage{javascript}%
	{morekeywords={break,case,catch,%
		const,constructor,continue,default,do,else,false,%
		finally,for,function,if,in,instanceof,%
		new,null,prototype,%
		return,switch,this,throw,%
		true,try,typeof,var,while},%
	sensitive,%
	morecomment=[l]//,%
	morecomment=[s]{/*}{*/},%
	morestring=[b]",%
	morestring=[b]',%
}[keywords,comments,strings]%

%% C# language (4.0?)
\lstdefinelanguage{csharp}%
	{morekeywords={abstract,as,%
		base,bool,byte,case,catch,char,%
		checked,class,const,continue,%
		decimal,default,delegate,do,double,%
		else,enum,event,explicit,extern,%
		false,finally,fixed,float,for,foreach,%
		goto,if,implicit,in,int,interface,%
		internal,is,lock,long,%
		namespace,new,null,object,operator,out,%
		override,params,private,protected,public,%
		readonly,ref,return,sbyte,sealed,%
		short,sizeof,stackalloc,static,string,%
		struct,switch,this,throw,true,try,%
		typeof,uint,ulong,unchecked,unsafe,ushort,%
		using,virtual,void,volatile,while%
	},%
	sensitive,%
	morecomment=[l]//,%
	morecomment=[s]{/*}{*/},%
	morestring=[b]",%
	morestring=[b]',%
}[keywords,comments,strings]%

%% Translation for fact environment
\deftranslation[to=russian]{Fact}{Наблюдение}

%% Inline code snippets
\def\code#1{\texttt{#1}}
\def\codekw#1{\code{\textbf{#1}}}

\def\quoteauthor#1{\par\footnotesize\upshape\hfill—~#1}

%% English term
\def\engterm#1{(англ. \textit{#1})}
%% Term with explanation below (to be used in diagrams)
\def\termwithexpl#1#2{#1\strut{}\\\small\color{gray}(\textit{#2})\strut{}}
%% External link
\def\extlink#1#2{\href{#1}{\color[rgb]{0.7,0.7,1.0}\dashbar{#2}}}
%% Internal link
\def\inlink#1#2{\hyperlink{#1}{\color[rgb]{0.7,0.7,1.0}\dashbar{#2}}}
%% Explanation for a list item
\def\itemexpl#1{\begingroup\small\vspace{0.75ex}#1\par\endgroup}




\lecturetitle{Программная инженерия. Лекция №6 — Инженерия требований.}
\title[Инженерия требований]{Инженерия требований к~программному обеспечению}
\author{Алексей Островский}
\institute{\small{Физико-технический учебно-научный центр НАН Украины}\vspace{2ex}}
\date{24 октября 2014 г.}

\begin{document}
	\frame{\titlepage}

	\frame{
		\frametitle{Требования к ПО}

		\begin{Definition}
			\textbf{Требования к ПО} — это:
			\begin{itemize}
				\item свойства системы, необходимые для выполнения предложенных заказчиком функций;
				\item ограничения на функционирование системы.
			\end{itemize}
		\end{Definition}

		\vspace{1ex}
		\textbf{Инженерия требований:}
		\begin{center}
			\begin{tikz*}[%
	iteration/.style={rectangle,align=center}
]
	\node[iteration] (elic) {
		\termwithexpl{определение}{elicitation}
	};
	\node[iteration,right=of elic] (analysis) {
		\termwithexpl{анализ}{analysis}
	};
	\node[iteration,right=of analysis] (spec) {
		\termwithexpl{спецификация}{specification}
	};
	\node[iteration,right=of spec] (valid) {
		\termwithexpl{проверка}{validation}
	};
	\node[iteration,right=of valid] (manage) {
		\termwithexpl{управление}{management}
	};
	
	\draw[->] (elic) to (analysis);
	\draw[->] (analysis) to (spec);
	\draw[->] (spec) to (valid);
	\draw[->] (valid) to (manage);
\end{tikz*}

		\end{center}
	}

	\section[Классификация]{Классификация требований}

	\subsection{Пользовательские и системные требования}

	\frame{
		\frametitle{Классификация требований}

		\begin{itemize}
			\item
			\textbf{Пользовательские требования} \engterm{user requirements} — описание
			на~естественном языке ожидаемой функциональности системы и~присущих~ей ограничений.

			\vspace{0.5ex}
			\textbf{Источник:} предлагаются заказчиком ПО.

			\vspace{0.5ex}
			\textbf{Инструменты:} естественный язык + диаграммы.

			\vspace{1ex}
			\item
			\textbf{Системные требования} \engterm{system requirements} — детальное описание функциональности
			системы и ограничений.

			\vspace{0.5ex}
			\textbf{Источник:} результат совместной работы заказчика и разработчика.

			\vspace{0.5ex}
			\textbf{Инструменты:} формальные языки, шаблоны, спецификации.
		\end{itemize}
	}

	\frame<1>[label=l:sys-req]{
		\frametitle{Классификация требований — пример}

		\textbf{Пример.} Веб-сервис для вычисления чисел Фибоначчи
		\[ F_i = F_{i-1} + F_{i-2}. \]

		\textbf{Пользовательское требование:}
		веб-сервис должен отображать ряд чисел Фибоначчи фрагментами по 100 чисел с помощью HTML-страниц.

		\vspace{1ex}
		\textbf{Системные требования:}
		\begin{itemize}
			\item
			Веб-сервис должен отображать числа Фибоначчи $i,\dots,i+99$ при доступе к~веб-сервису с~помощью URL вида
			\code{http://fib.example.com/fib/}\emph{i}.

			\item
			Каждая сгенерированная страница должна содержать навигацию для доступа к~следующим ста
			и (если применимо) к предыдущим ста числам Фибоначчи.

			\item
			При попытке доступа к сервису с помощью URL \code{http://fib.example.com/fib/}\emph{str},
			где~\emph{str} не~является натуральным числом,
			должна выдаваться страница оговоренного вида с~HTTP-кодом 400.
		\end{itemize}
	}

	\subsection{Функциональные и нефункциональные требования}

	\frame{
		\frametitle{Функциональные и нефункциональные требования}

		\textbf{Функциональные требования} — это:
		\begin{itemize}
			\item определение предоставляемых программным продуктом услуг;
			\item описание реакции на различные входные данные;
			\item описание поведения системы в различных ситуациях;
			\item (необязательно) спецификация запретов.
		\end{itemize}

		\vspace{1ex}
		\textbf{Нефункциональные требования} — ограничения на функции, предоставляемые ПП:
		\begin{itemize}
			\item временные ограничения;
			\item ограничения на процесс разработки;
			\item ограничения, связанные со стандартами разработки ПО.
		\end{itemize}
	}

	\frame{
		\frametitle{Связь между требованиями}

		\begin{figure}
			\begin{tikz*}[%
	every node/.style={rectangle,align=center,minimum height=3em,inner xsep=0.5em,minimum width=15em},
	label/.style={font=\footnotesize\itshape,minimum height=0pt,minimum width=0pt}
]
	\node(nonfunc) {\bfseries Нефункциональное требование};
	\node(func) [right=6em of nonfunc]{\bfseries Функциональное требование};
	\node(confid) [below=of nonfunc] {Защита конфиденциальных \\ данных};
	\node(auth) [below=of func] {Система авторизации};
	\node(mem) [below=of confid] {Ограничение на \\ занимаемую память};
	\node(del) [below=of auth] {Периодическое удаление \\ лишних данных};
	\node(recover) [below=of mem] {Отказоустойчивость};
	\node(copy) [below=of del] {Система резервных \\ копий данных};
	
	\draw[->] (confid) -- node[below,label]{уточнение} (auth);
	\draw[->] (mem) -- node[below,label]{уточнение} (del);
	\draw[->] (recover) -- node[below,label]{уточнение} (copy);
\end{tikz*}

			\caption{Нефункциональные требования могут в процессе уточнения
				порождать новые функциональные требования.}
		\end{figure}
	}

	\frame{
		\frametitle{Нефункциональные требования}

		\begin{tikz*}[%
	every node/.style={rectangle,draw,align=center,minimum height=3em},
	sub/.style={font=\small,minimum height=0pt,minimum width=10em},
	hilight/.style={font=\small\only<#1>{\color{red}}},
]
	\node(nonfunc) [minimum width=30em]{\bfseries Нефункциональные требования};
	
	\node(org) [below=of nonfunc] {%
		\termwithexpl{Организационные требования}{organizational req.}
	};
	\node(env) [sub,below=of org] {среда выполнения \\ \emph{(environmental req.)}};
	\node(oper) [sub,below=0.25em of env,hilight=2] {операционные \\ \emph{(operational req.)}};
	\node(dev) [sub,below=0.25em of oper,hilight=3] {ограничения \\ на разработку \\ \emph{(development req.)}};
	\draw[->] (org) to (env);
	
	\node(prod) [left=1.75em of org] {%
		\termwithexpl{Требования на продукт}{product requirements}
	};
	\node(usability) [sub,below=of prod] {интерфейс \emph{(usability)}};
	\node(efficiency) [sub,below=0.25em of usability] {продуктивность \\ \emph{(efficiency)}};
	\node(dep) [sub,below=0.25em of efficiency] {надежность \\ \emph{(dependability)}};
	\node(sec) [sub,below=0.25em of dep] {безопасность \emph{(security)}};
	\draw[->] (prod) to (usability);
	
	\node(ext) [right=1.75em of org]{%
		\termwithexpl{Внешние требования}{external req.}
	};
	\node(reg) [sub,below=of ext,hilight=4] {регламентные \\ \emph{(regulatory req.)}};
	\node(legal) [sub,below=0.25em of reg,hilight=5] {юридические \\ \emph{(legislative req.)}};
	\node(ethical) [sub,below=0.25em of legal] {этические \\ \emph{(ethical req.)}};
	\draw[->] (ext) to (reg);
	
	\draw[->] (nonfunc) to (org);
	\draw[->] (nonfunc) to (prod);
	\draw[->] (nonfunc) to (ext);
	
	\node<2> [note={(oper.north)},above=1em of oper.north,anchor=south] {
		Определяют, каким образом  \\ будет использоваться система \\ (напр., порядок ее запуска).
	};
	\node<3> [note={(dev.north)},above=1em of dev.north,anchor=south] {
		Напр., язык программирования \\ или среда разработки.
	};
	\node<4> [note={(reg.west)},left=1em of reg.west,anchor=east] {
		Напр., требования центрального банка \\ для банковских систем.
	};
	\node<5> [note={(legal.west)},left=1em of legal.west,anchor=east] {
		Напр., требования безопасности \\ в соответствии с законом \\ про защиту персональных данных.
	};
\end{tikz*}

	}

	\frame{
		\frametitle{(Не)функциональные требования — пример}

		\textbf{Пример.} Веб-сервис для вычисления чисел Фибоначчи.

		\vspace{1ex}
		\textbf{Функциональные требования:}
		\begin{itemize}
			\item требования к отображению информации (\inlink{l:sys-req}{см. выше});
			\item интерфейс администратора;
			\item учет количества посетителей.
		\end{itemize}

		\vspace{1ex}
		\textbf{Нефункциональные требования:}
		\begin{itemize}
			\item \emph{(usability)} использование адаптивного дизайна для ПК, планшетов и смартфонов;
			\item \emph{(производительность)} генерация любой страницы за $\leq 0{,}5$~с;
			\item \emph{(среда выполнения)} Linux, MySQL, Apache HTTP Server;
			\item \emph{(разработка)} использование Python/Django.
		\end{itemize}
	}

	\section[Спецификация]{Спецификация требований}

	\frame{
		\frametitle{Спецификация требований}

		\begin{Definition}
			\textbf{Спецификация требований} — запись требований в виде, обеспечивающем их ясность,
			однозначность, простоту понимания, полноту и непротиворечивость.
		\end{Definition}

		\vspace{0.5ex}
		\begin{center}\linespread{1.0}
			\begin{tabular}{|p{0.25\textwidth}|p{0.6\textwidth}|}
				\hline
				\centering \textbf{Роль} & \centering \textbf{Использование требований} \cr
				\hline
				потребители & спецификация и уточнение требований \\
				\hline
				менеджеры & \raggedright оценка затрат на систему; планирование процесса разработки \cr
				\hline
				разработчики & детализация характеристик системы \\
				\hline
				тестеры & разработка тестов для проверки системы \\
				\hline
				\raggedright отдел сопровождения & \raggedright понимание системы и взаимоотношений между~ее~частями \cr
				\hline
			\end{tabular}
		\end{center}
	}

	\subsection{Документ спецификации}

	\frame{
		\frametitle{Документ спецификации}

		\begin{enumerate}
			\item Предварительные замечания (версия документа, основания для ее создания).
			\item Вступление (общее назначение системы, ее взаимодействие с другим ПО).
			\item Словарь технических терминов.
			\item Описание пользовательских требований (+ нефункциональные системные требования).
			\item Архитектура системы.
			\item Описание системных требований.
			\item Системные модели (взаимодействие между компонентами, со средой выполнения и~т.\,п.).
			\item Эволюция системы (ожидаемые изменения системы).
			\item Приложения.
		\end{enumerate}
	}

	\frame{
		\frametitle{Запись требований}

		\textbf{Способы записи:}
		\begin{itemize}
			\item
			естественный язык

			\itemexpl{(подходит для записи пользовательских требований);}

			\item
			структурированный язык (таблицы или шаблоны)

			\itemexpl{(подходит для спецификации системных требований);}

			\item
			язык описания архитектуры

			\itemexpl{(используется редко, в основном для спецификации интерфейсов);}

			\item
			графическая нотация (напр., UML-диаграммы)

			\itemexpl{(подходит для детализации системных требований);}

			\item
			математическая спецификация (напр., конечные автоматы)

			\itemexpl{(используется для критических требований в области безопасности).}
		\end{itemize}
	}

	\frame{
		\frametitle{Запись требований — пример}

		\textbf{Числа Фибоначчи — математический модуль}

		\begin{center}\linespread{1.0}
			\begin{tabular}{rp{0.67\textwidth}}
				\textbf{Функция:} & вычисление ряда чисел Фибоначчи. \cr
				\textbf{Описание:} & \raggedright вычисляет значение ста последовательных чисел Фибоначчи. \cr
				\textbf{Вход:} & \raggedright $i$ — индекс первого числа Фибоначчи, которое нужно вычислить. \cr
				\textbf{Источник данных:} & HTTP-запрос пользователя. \cr
				\textbf{Выход:} & \raggedright значения чисел Фибоначчи $F_i,F_{i+1}\dots,F_{i+99}$. \cr
				\textbf{Назначение данных:} & \raggedright цикл обработки HTTP-запроса. \cr
				\textbf{Действие:} & \raggedright Числа $F_i$ и~$F_{i+1}$ вычисляются по~\inlink{l:fib}{формуле} быстрого возведения в~степень.
					Оставшиеся числа вычисляются согласно определению чисел Фибоначчи. \cr
				\textbf{Требования:} & \raggedright $i$ должно быть целым неотрицательным числом. \cr
				\textbf{Побочные эффекты:} & нет. \cr
			\end{tabular}
		\end{center}
	}

	\section[Инженерия]{Процесс инженерии}

	\frame{
		\frametitle{Процесс инженерии требований}

		{\footnotesize\begin{tikz*}[%
	x=2em,y=2em,
	xscale=3.0,yscale=2.1,
	every node/.style={rectangle,align=center}
]
	\draw[thick,draw=blue,domain=3.1416:21.991,samples=150,smooth,variable=\t,scale=0.19] plot({\t*sin(\t r)}, {\t*cos(\t r)});
	
	\draw (0,0) to (30:4.3);
	\draw (0,0) to (150:4.3);
	\draw (0,0) to (-90:4.3);
	
	\node at (-165:0.4) {Начало};
	\node at (90:0.7) {Спецификация \\ бизнес-требований};
	\node at (-45:0.9) {Исследование \\ выполнимости};
	\node at (-130:1.3) {Определение \\ польз. требований};
	\node at (90:1.7) {Спецификация \\ польз. требований};
	\node at (-67:2.25) {Прототипирование};
	\node at(-120:2.35) {Определение \\ системных требований};
	\node at (90:2.9) {Спецификация \\ и моделирование \\ системных требований};
	\node at (-75:3.55) {Рецензирование};
	
	\node at (-90:4.0) [anchor=east] {\bfseries Документ спецификации \\ \bfseries системных требований};
	
	\node at (-165:3.9) {\bfseries Определение};
	\node at (90:3.85) {\bfseries Спецификация};
	\node at (-30:4.5) {\bfseries Валидация};
\end{tikz*}
}
	}

	\subsection{Определение и анализ}

	\frame{
		\frametitle{Определение и анализ требований}

		\begin{figure}
			\begin{tikz*}[%
	xscale=2.0,
	every node/.style={rectangle,draw,align=center,minimum height=3em}
]
	\node(discovery) at (90:7.5em) {\alert<2>{Сбор требований}};
	\node(class) at (0:7.5em) {\alert<3>{Классификация} \\ \alert<3>{и организация}};
	\node(prio) at (-90:7.5em) {\alert<4>{Выделение приоритетов} \\ \alert<4>{и согласование}};
	\node(spec) at (180:7.5em) {\alert<5>{Спецификация}};
	
	\draw[->] (discovery.east) -| (class.north);
	\draw[->] (class.south) |- (prio.east);
	\draw[->] (prio.west) -| (spec.south);
	\draw[->] (spec.north) |- (discovery.west);
	
	\node<2> [note={(discovery.south)},below=1em of discovery.south,anchor=north] {
		Взаимодействие с заинтересованными сторонами \\
		(заказчиками, конечными пользователями, \\ обслуживающим персоналом, …) 
		для определения \\ их требований.
		
		\\[0.5ex]
		\textbf{Методы:} интервью, сценарии, прецеденты, \\ варианты применения (use case).
	};
	
	\node<3> [note={(class.west)},left=1em of class.west,anchor=east] {
		Упорядочивание требований в связанные группы \\
		с использованием архитектуры системы.
	};

	\node<4> [note={(prio.north)},above=1em of prio.north,anchor=south] {
		Согласование требований с заинтересованными \\
		сторонами и нахождение компромиссов.
	};
	
	\node<5> [note={(spec.east)},right=1em of spec.east,anchor=west] {
		Описание требований с помощью формальных \\
		или неформальных методов.
	};
\end{tikz*}

			\caption{Процессы определения и анализа требований}
		\end{figure}
	}

	\subsection{Валидация}

	\frame{
		\frametitle{Валидация требований}

		\textbf{Проверки:}
		\begin{itemize}
			\item
			корректность (согласованы ли требования со всеми заинтересованными сторонами?);
			\item
			непротиворечивость (есть ли конфликты между требованиями?);
			\item
			полнота (описывают ли требования все функции системы?);
			\item
			реалистичность (возможно ли реализовать требования?);
			\item
			верифицируемость (существуют ли тесты, проверяющие выполнение требований?).
		\end{itemize}

		\vspace{1ex}
		\textbf{Методы валидации:}
		\begin{itemize}
			\item рецензирование;
			\item прототипирование;
			\item создание тестов.
		\end{itemize}
	}

	\subsection{Управление требованиями}

	\frame{
		\frametitle{Управление требованиями}

		\begin{Definition}
			\textbf{Управление требованиями} — процесс выявления и~контроля изменений в~системных требованиях.
		\end{Definition}

		\vspace{1ex}
		\textbf{Причины изменений:}
		\begin{itemize}
			\item изменение среды выполнения (новое оборудование, новые приоритеты,
			изменение регламентирующих документов или законодательства, …);
			\item различие между пониманием системы заказчиком и конечными пользователями;
			\item изменение баланса между различными группами пользователей.
		\end{itemize}
	}

	\frame{
		\frametitle{Процесс изменения требований}

		\textbf{Традиционная модель ЖЦ:}
		\begin{enumerate}
			\item
			Анализ проблемы и~спецификация изменения. Анализ пересылается заказчику изменения для внесения дополнений или отказа от~изменения.
			\item
			Анализ изменения и~оценка затрат.
			\item
			Имплементация изменений в~общую спецификацию требований, а~также в~архитектуру и~имплементацию системы.
		\end{enumerate}

		\vspace{1ex}
		\textbf{Agile development:}
		\begin{enumerate}
			\item Оценка приоритета изменения.
			\item Модификация плана следующего цикла разработки.
		\end{enumerate}
	}

	\section{Заключение}

	\subsection{Выводы}

	\frame{
		\frametitle{Выводы}

		\begin{enumerate}
			\item
			Требования к~ПО определяют его возможности (функциональные тр.) и~ограничения на~процесс разработки (нефункциональные тр.).

			\vspace{0.5ex}
			\item
			Процесс инженерии требований включает в~себя анализ выполнимости, выработку и~анализ требований, их~спецификацию, проверку,
			а~также управление требованиями.

			\vspace{0.5ex}
			\item
			Существует несколько инструментов спецификации требований, в~частности формальные языки и диаграммы UML.
		\end{enumerate}
	}

	\subsection{Материалы}

	\frame{
		\frametitle{Материалы}

		\begin{thebibliography}{9}
			\bibitem[1]{1}
			Лавріщева К.\,М.
			\newblock Програмна інженерія (підручник).
			\newblock {\footnotesize К., 2008. — 319 с.}

			\bibitem[2]{2}
			Sommerville, Ian
			\newblock Software Engineering.
			\newblock {\footnotesize Pearson, 2011. — 790 p.}
		\end{thebibliography}
	}

	\frame{
		\frametitle{}

		\begin{center}
			\Huge Спасибо за внимание!
		\end{center}
	}

	\subsection{Приложение. Вычисление чисел Фибоначчи}

	\frame<1>[label=l:fib]{
		\frametitle{Приложение. Вычисление чисел Фибоначчи}

		\[
			F_n = F_{n-1} + F_{n-2}, \quad F_0 = 0, F_1 = 1.
		\]

		\vspace{1ex}
		$F_n$, $n \geq 1$ можно вычислить за время $O(\log n)$ с помощью формулы
		\[
			\left( \begin{array}{c} F_n \\ F_{n-1} \end{array} \right) =
				\left( \begin{array}{r@{\hspace{2ex}}l} 1 & 1 \\ 1 & 0 \end{array} \right) ^ {n-1} \cdot
				\left( \begin{array}{c} 1 \\ 0 \end{array} \right),
		\]
		где возведение матрицы в степень выполняется с помощью \extlink{https://ru.wikipedia.org/wiki/\%D0\%90\%D0\%BB\%D0\%B3\%D0\%BE\%D1\%80\%D0\%B8\%D1\%82\%D0\%BC_\%D0\%B1\%D1\%8B\%D1\%81\%D1\%82\%D1\%80\%D0\%BE\%D0\%B3\%D0\%BE_\%D0\%B2\%D0\%BE\%D0\%B7\%D0\%B2\%D0\%B5\%D0\%B4\%D0\%B5\%D0\%BD\%D0\%B8\%D1\%8F_\%D0\%B2_\%D1\%81\%D1\%82\%D0\%B5\%D0\%BF\%D0\%B5\%D0\%BD\%D1\%8C}{быстрого алгоритма}.
		Для~вычислений необходима поддержка целых чисел с произвольной разрядностью.

		\vspace{1ex}
		Другой способ — использование формулы
		\[
			F_n = \left[ \frac{1}{\sqrt{5}} \left( \frac{1 + \sqrt{5}}{2} \right)^n \right]
		\]
		(необходима поддержка \emph{вещественных} чисел с произвольной разрядностью).
	}
\end{document}
