\documentclass{a4beamer}
%% Lectures - common definitions

\usextensions{tikz}
\usetikzlibrary{shapes.multipart,shapes.callouts,shapes.geometric}
\input{fix-callouts.inc} % Fixes absolute positioning of rectangle callouts

\newif\ifbigpages \bigpagesfalse
\ifdim\paperwidth >20cm
	\bigpagestrue
\fi

\tikzset{%
	note/.style={rectangle callout,draw=none,callout pointer width=1em,%
		align=flush left,font=\footnotesize,inner sep=0.5em,%
		fill=blue!15,fill opacity=0.95,text opacity=1.0,callout absolute pointer=#1},
	node distance=2em and 2.75em
}
\ifbigpages
	% Scale all arrow tips by the factor of 2.5
	\let\old@pgf@arrow@call=\pgf@arrow@call
	\def\pgf@arrow@call#1{%
		\@tempdima=\pgflinewidth%
		\pgfsetlinewidth{2.5\pgflinewidth}%
		\old@pgf@arrow@call{#1}%
		\pgfsetlinewidth{\@tempdima}%
	}
	\def\pgfarrowsleftextend#1{\pgfmathsetlength{\pgf@xa}{1.5*#1}}
	\def\pgfarrowsrightextend#1{\pgfmathsetlength{\pgf@xb}{1.5*#1}}
\fi

%% Load listings package
\usepackage{listings}

%% Are we inside a comment?
\newif\iflstcomment \lstcommentfalse

\lstset{%
	tabsize=4,
	showstringspaces=false,
	basicstyle=\linespread{1.25}\ttfamily\small,
	keywordstyle=\bfseries,
	commentstyle=\lstcommentstyle,
	numbers=left,
	numberstyle=\footnotesize\color{gray},
	xleftmargin=2.5em,
	extendedchars=true,
	escapechar=\$,
	escapebegin=\iflstcomment\begingroup\lstcommentstyle\fi,
	escapeend=\iflstcomment\endgroup\fi
}

\def\lstcommentstyle{\color{gray}}

\lst@AddToHook{AfterBeginComment}{\global\lstcommenttrue}
\let\orig@lst@EndComment=\lst@EndComment
\def\lst@EndComment{\global\lstcommentfalse\orig@lst@EndComment}
\lst@AddToHookAtTop{EOL}{%
	\lst@ifLmode\global\lstcommentfalse\fi% XXX Sloppy way to determine comment end
}

%% Python with docstrings treated as comments
\lstdefinelanguage[doc]{python}[]{python}{%
	deletestring=[s]{"""}{"""},%
	morecomment=[s]{"""}{"""}%
}%

%% JavaScript language
\lstdefinelanguage{javascript}%
	{morekeywords={break,case,catch,%
		const,constructor,continue,default,do,else,false,%
		finally,for,function,if,in,instanceof,%
		new,null,prototype,%
		return,switch,this,throw,%
		true,try,typeof,var,while},%
	sensitive,%
	morecomment=[l]//,%
	morecomment=[s]{/*}{*/},%
	morestring=[b]",%
	morestring=[b]',%
}[keywords,comments,strings]%

%% C# language (4.0?)
\lstdefinelanguage{csharp}%
	{morekeywords={abstract,as,%
		base,bool,byte,case,catch,char,%
		checked,class,const,continue,%
		decimal,default,delegate,do,double,%
		else,enum,event,explicit,extern,%
		false,finally,fixed,float,for,foreach,%
		goto,if,implicit,in,int,interface,%
		internal,is,lock,long,%
		namespace,new,null,object,operator,out,%
		override,params,private,protected,public,%
		readonly,ref,return,sbyte,sealed,%
		short,sizeof,stackalloc,static,string,%
		struct,switch,this,throw,true,try,%
		typeof,uint,ulong,unchecked,unsafe,ushort,%
		using,virtual,void,volatile,while%
	},%
	sensitive,%
	morecomment=[l]//,%
	morecomment=[s]{/*}{*/},%
	morestring=[b]",%
	morestring=[b]',%
}[keywords,comments,strings]%

%% Translation for fact environment
\deftranslation[to=russian]{Fact}{Наблюдение}

%% Inline code snippets
\def\code#1{\texttt{#1}}
\def\codekw#1{\code{\textbf{#1}}}

\def\quoteauthor#1{\par\footnotesize\upshape\hfill—~#1}

%% English term
\def\engterm#1{(англ. \textit{#1})}
%% Term with explanation below (to be used in diagrams)
\def\termwithexpl#1#2{#1\strut{}\\\small\color{gray}(\textit{#2})\strut{}}
%% External link
\def\extlink#1#2{\href{#1}{\color[rgb]{0.7,0.7,1.0}\dashbar{#2}}}
%% Internal link
\def\inlink#1#2{\hyperlink{#1}{\color[rgb]{0.7,0.7,1.0}\dashbar{#2}}}
%% Explanation for a list item
\def\itemexpl#1{\begingroup\small\vspace{0.75ex}#1\par\endgroup}



\usetikzlibrary{shapes.misc}

\lecturetitle{Программная инженерия. Лекция №27 — Облачные вычисления}
\title[Веб-сервисы]{Введение в облачные вычисления}
\author{Алексей Островский}
\institute{\small{Физико-технический учебно-научный центр НАН Украины}\vspace{2ex}}
\date{21 мая 2015 г.}

\usepackage{listings}
\lstset{%
	tabsize=4,
	showstringspaces=false,
	basicstyle=\linespread{1.25}\ttfamily\small,
	keywordstyle=\bfseries,
	commentstyle=\color{gray},
	numbers=left,
	numberstyle=\footnotesize\color{gray},
	xleftmargin=2.5em,
	texcl=true,
	extendedchars=true,
	escapechar=\$
}

\def\code#1{\texttt{#1}}
\def\codekw#1{\texttt{\textbf{#1}}}

\begin{document}
	\frame{\titlepage}

	\section[Облако]{Облачные вычисления}

	\frame{
		\frametitle{Облачные вычисления}

		\begin{Definition}[NIST — Национальный институт стандартов и~технологий~США]
			\textbf{Облачные вычисления} \engterm{cloud computing} — модель для~предоставления 
			повсеместного удобного сетевого доступа к~конфигурируемым вычислительным ресурсам, 
			которые могут~быть введены в~использование быстро и~с~минимальными затратами 
			на~управление или~взаимодействие с~провайдером.
		\end{Definition}

		\vspace{1ex}
		\begin{center}
			{\small\begin{tikz*}[%
	every node/.style={align=center,minimum height=2.5em}
]
	\node(cloud) [font=\bfseries] {Облако};
	\node(comp) [below right=2em and 1em of cloud,anchor=north west] {вычислительные \\ мощности};
	\node(data) [above left=2em and 1em of cloud,anchor=south east] {данные};
	\node(app) [right=of cloud] {приложения};
	\node(service) [above right=2em and 1em of cloud,anchor=south west] {сервисы};
	\node(mon) [below left=2em and 1em of cloud,anchor=north east] {мониторинг}; 
	\node(conf) [left=of cloud] {конфигурация};
	\node(wrap) [rectangle,draw,dashed,fit=(conf.west) (app.east) (mon.south west) (comp.south east) %
			(data.north west) (service.north east)] {};

	\node(srv) [font=\bfseries,right=3em of wrap] {Серверы};
	\node(laptop) [font=\bfseries,below left=3em and 3em of wrap.west] {Настольные \\ компьютеры};
	\node(mobile) [font=\bfseries,above left=3em and 3em of wrap.west] {Мобильные \\ устройства};
	\draw[<->] (srv) -- (wrap);
	\draw[<->] (laptop) -- (laptop -| wrap.west);
	\draw[<->] (mobile) -- (mobile -| wrap.west);
\end{tikz*}
}

			\vspace{1ex}
			\figureexpl{Облачная архитектура скрывает детали распределения ресурсов от пользователя}
		\end{center}
	}

	\subsection{Характеристики}

	\frame{
		\frametitle{Характеристики облачных вычислений}

		\begin{itemize}
			\item
			\textbf{Самообслуживание} — возможность заказа потребителями дополнительных ресурсов 
			автоматически с~помощью системы управления (без~взаимодействия со~службой поддержки провайдера).

			\vspace{1ex}
			\item
			\textbf{Широкий сетевой доступ} — возможность доступа к~ресурсам через~стандартные сетевые протоколы 
			для~широкого круга устройств (смартфоны, планшеты, настольные компьютеры).

			\vspace{1ex}
			\item
			\textbf{Пулинг (объединение) ресурсов} — динамическое распределение мощностей оборудования провайдера 
			для~обслуживания актуальных запросов потребителей. 
			Прозрачность распределения ресурсов для~заказчиков (с~возможностью спецификации высокого уровня, 
			напр., страны и~информационного центра, на~котором хостится приложение).
		\end{itemize}
	}

	\frame{
		\frametitle{Характеристики облачных вычислений (продолжение)}

		\begin{itemize}
			\item
			\textbf{Эластичность} — возможность быстрого выделения дополнительных ресурсов 
			(по~заказу или~автоматически) для~обеспечения производительности 
			при~повышении потребления; иллюзия бесконечных ресурсов.

			\vspace{1ex}
			\item
			\textbf{Измерение предоставляемых услуг} — автоматизированный контроль и~оптимизация 
			использования ресурсов; модель оплаты на~основе количества предоставленных услуг 
			(вычислительных мощностей, объема данных, …). 
			Возможность мониторинга и~предоставления отчетности потребителям.

			\vspace{1ex}
			\item
			\textbf{Минимизация затрат} на приобретение, конфигурацию и поддержку оборудования.
		\end{itemize}
	}

	\subsection{Предшествующие технологии}

	\frame{
		\frametitle{Предшествующие технологии}

		\textbf{Интернет-технологии:}
		\begin{itemize}
			\item сервисная архитектура приложений (SOA);
			\item веб-сервисы (SOAP / WSDL и REST);
			\item объединения сервисов \engterm{mashup};
			\item Web 2.0.
		\end{itemize}

		\vspace{1ex}
		\textbf{Распределенные вычисления:}
		\begin{itemize}
			\item
			грид-вычисления \engterm{grid computing} — форма распределенной архитектуры, 
			представленная виртуальным суперкомпьютером, составленным из~слабо связанных 
			и~географически разделенных независимых компьютеров или~кластеров;

			\item
			утилитарные вычисления \engterm{utility computing} — модель предоставления ресурсов 
			по~заказу потребителей.
		\end{itemize}
	}

	\frame{
		\frametitle{Предшествующие технологии (продолжение)}

		\textbf{Виртуализация:} 
		\begin{itemize}
			\item
			виртуализация на уровне операционной системы \engterm{OS-level virtualization}~— 
			создание множества изолированных пространств пользователя (контейнеров) 
			с~поддержкой со~стороны ядра~ОС;

			\item
			поддержка виртуализации оборудованием (напр., многоядерные процессоры) 
			и~ПО (гипервизоры — приложения для~управления~ВМ).
		\end{itemize}

		\vspace{1ex}
		\textbf{Управление:}
		\begin{itemize}
			\item
			автономные вычислительные узлы — компьютеры с~возможностью самостоятельной конфигурации, 
			оптимизации, восстановления после~сбоев и~защиты;

			\item
			инструменты для~мониторинга и~управления распределенными конфигурациями 
			(напр., распределением вычислительных мощностей; предупреждением и~устранением неполадок).
		\end{itemize}
	}

	\subsection{Уровни и развертывание}

	\frame{
		\frametitle{Уровни облачной архитектуры}

		\begin{center}\small
			\begin{tabular}{|p{0.2\textwidth}|p{0.3\textwidth}|p{0.35\textwidth}|}
				\hline
				\centering Уровень & \centering Средства доступа и~управления & \centering Содержимое \cr
				\hline
				\raggedright ПО как~сервис (SaaS) & Веб-браузер & 
					\raggedright\textbf{Облачные приложения:}
					социальные сети, офисные приложения, системы управления содержимым, 
					интеллектуальная обработка данных. \cr
				\hline
				\raggedright Платформа как~сервис (PaaS) & \raggedright Облачная среда разработки & 
					\raggedright\textbf{Облачная платформа:}
					языки программирования, библиотеки, утилиты конфигурирования композиций сервисов, 
					структурированные данные. \cr
				\hline
				\raggedright Инфраструктура как~сервис (IaaS) & 
					\raggedright Система управления виртуальной инфраструктурой & 
					\raggedright\textbf{Облачная инфраструктура:} вычислительные сервера, хранилища данных, 
					организация сетевых соединений (брандмауэры, балансировка нагрузки). \cr
				\hline
			\end{tabular}
		\end{center}
	}

	\frame{
		\frametitle{Уровни облачной архитектуры}

		\begin{Definition}
			\textbf{Инфраструктура как сервис} \engterm{infrastructure as a service, IaaS}~— 
			предоставление потребителям спецификации базовых ресурсов (для~вычислений, 
			хранения и~передачи данных, …) с~возможностью развертывания произвольного~ПО 
			(ОС~и~приложений).
		\end{Definition}

		\vspace{1ex}
		\textbf{Степень контроля:}
		\begin{itemize}
			\item ОС;
			\item системы хранения данных;
			\item развернутые приложения;
			\item (частично) сетевые компоненты, напр., брандмауэры.
		\end{itemize}

		\vspace{1ex}
		\textbf{Пример:} Amazon EC2.
	}

	\frame{
		\frametitle{Уровни облачной архитектуры}

		\begin{Definition}
			\textbf{Платформа как сервис} \engterm{platform as a service, PaaS}~— 
			предоставление возможности развертывания пользовательских и~аналитических приложений 
			на~основе поддерживаемых провайдером~ЯП, библиотек, сервисов и~инструментов.
		\end{Definition}

		\vspace{1ex}
		\textbf{Степень контроля:}
		\begin{itemize}
			\item развернутые приложения;
			\item (частично) конфигурация среды выполнения.
		\end{itemize}

		\vspace{1ex}
		\textbf{Пример:} Google AppEngine.
	}

	\frame{
		\frametitle{Уровни облачной архитектуры}

		\begin{Definition}
			\textbf{ПО как сервис} \engterm{software as a service, SaaS}~— 
			предоставление возможности использовать приложения, 
			работающие в~облачном окружении и~доступные с~помощью~клиентов (напр., веб-браузера) или~сетевого~API.
		\end{Definition}

		\vspace{1ex}
		\textbf{Степень контроля:}
		\begin{itemize}
			\item (возможно) конфигурация приложения;
			\item создаваемые пользователями данные.
		\end{itemize}

		\vspace{1ex}
		\textbf{Пример:} Microsoft Office 365.
	}

	\frame{
		\frametitle{Модели развертывания}

		\begin{itemize}
			\item
			\textbf{Частное облако} \engterm{private cloud, enterprise cloud}~— 
			выделение инфраструктуры для~эксклюзивного использования некоторой организацией.

			\vspace{0.5ex}
			\item
			\textbf{Общественное облако} \engterm{community cloud}~— 
			выделение инфраструктуры для~пользования объединением организаций (напр., из~соображений безопасности).

			\vspace{0.5ex}
			\item
			\textbf{Публичное облако} \engterm{public cloud, Internet cloud}~— 
			выделение облачной инфраструктуры для~открытого использования частными лицами или~организациями.

			\vspace{0.5ex}
			\item
			\textbf{Гибридное облако} \engterm{hybrid cloud} — композиция нескольких видов инфраструктуры, 
			объединенных средствами коммуникации для~обмена данными и~приложениями.
		\end{itemize}
	}

	\subsection{Итоги}

	\frame{
		\frametitle{Достоинства и недостатки облачных вычислений}

		\textbf{Достоинства:}
		\begin{itemize}
			\item
			минимизация затрат на создание, конфигурацию и~поддержку распределенной инфраструктуры;
			\item
			эластичность — возможность быстрой адаптации к~росту нагрузки (в~т.\,ч.~географически неоднородной);
			\item
			доступность сервисов для~широкого круга пользователей.
		\end{itemize}

		\vspace{1ex}
		\textbf{Недостатки:}
		\begin{itemize}
			\item
			угрозы безопасности данных, злонамеренного использования сервисов и~т.\,п.;
			\item
			возможная ограниченность средств разработки, необходимость адаптации 
			к~принципам распределенной / облачной архитектуры;
			\item
			отсутствие общепринятых стандартов разработки.
		\end{itemize}
	}

	\frame{
		\frametitle{Технологии облачных вычислений}

		\textbf{Оборудование:}
		\begin{itemize}
			\item
			аппаратная виртуализация;
			\item
			информационные и вычислительные центры, кластеры;
			\item
			сетевые соединения, Интернет.
		\end{itemize}

		\vspace{1ex}
		\textbf{Программное обеспечение:}
		\begin{itemize}
			\item
			распределенные файловые системы;
			\item
			облачные базы данных и~другие технологии хранения (напр., распределенные системы кэширования);
			\item
			средства распределения нагрузки в узлах сети;
			\item
			инструменты для обработки данных в облачном окружении;
			\item
			веб-API и веб-сервисы.
		\end{itemize}
	}

	\subsection{Примеры}

	\frame{
		\frametitle{Примеры облачных платформ}

		\textbf{Amazon Web Services} (\extlink{http://aws.amazon.com/}{ссылка}):
		\begin{itemize}
			\item
			предоставляет услуги IaaS, PaaS;
			\item
			Elastic Compute Cloud (EC2) — масштабируемые сервера для~вычислений;
			\item
			Elastic MapReduce (EMR) — аналитика;
			\item
			Simple Storage Service (S3) — хранилище данных на~основе веб-сервисов;
			\item
			DynamoDB, SimpleDB — базы данных.
		\end{itemize}

		\vspace{1ex}
		\textbf{Google App Engine} (\extlink{https://cloud.google.com/appengine/}{ссылка}):
		\begin{itemize}
			\item
			предоставляет услуги PaaS;
			\item
			автоматическое масштабирование в зависимости от количества запросов;
			\item
			поддерживаются ЯП Java (+~другие, использующие JVM, напр., Scala), Python, Go и~PHP;
			\item
			ограниченный перечень API: БД \extlink{http://en.wikipedia.org/wiki/BigTable}{BigTable}, 
			HTTP-запросы, обработка изображений, …
		\end{itemize}
	}

	\frame{
		\frametitle{Примеры облачных платформ (продолжение)}

		\textbf{Microsoft Azure} (\extlink{http://azure.microsoft.com/}{ссылка}):
		\begin{itemize}
			\item
			предоставляет услуги PaaS и IaaS;
			\item
			управление виртуальными машинами под управлением Windows Server и Linux;
			\item
			БД SQL Azure (облачная версия MS SQL Server);
			\item
			веб-приложения на основе ASP.NET, PHP, Node.js, Python;
			\item
			аналитика при помощи доступных SDK, в~частности, 
			\extlink{http://azure.microsoft.com/en-in/services/hdinsight/}{Hadoop} 
			и~\extlink{http://azure.microsoft.com/ru-ru/services/machine-learning/}{машинное обучение}.
		\end{itemize}

		\vspace{1ex}
		\textbf{Heroku} (\extlink{http://heroku.com/}{ссылка}):
		\begin{itemize}
			\item
			предоставляет услуги PaaS;
			\item
			веб-интефейс и интерфейс командной строки для большинства операций, 
			поддержка быстрого добавления модулей (Heroku Elements);
			\item
			поддержка ЯП Ruby, Java, JavaScript / Node.js, Scala, Clojure, Python, PHP;
			\item
			БД PostgreSQL (реляционная), MongoDB, Redis (нереляционные).
		\end{itemize}
	}

	\section{Big Data}

	\frame{
		\frametitle{Big Data}

		\begin{Definition}
			\textbf{Большие данные} \engterm{big data} — наборы данных, характеризующиеся большим объемом, 
			высокой скоростью прироста и~слабой структурированностью, 
			для~которых невозможны или~затруднены традиционные методы хранения и~обработки (напр., реляционные~БД).
		\end{Definition}

		\vspace{1ex}
		\textbf{Источники данных:}
		\begin{itemize}
			\item мобильные устройства; 
			\item Web 2.0 (социальные сети, поиск данных, …); 
			\item наука (метеорология, биоинформатика, физика, …); 
			\item коммерческие организации (данные клиентов).
		\end{itemize}
	}

	\subsection{Характеристики}

	\frame{
		\frametitle{Характеристики Big Data}

		\textbf{Основные характеристики} (3V):
		\begin{itemize}
			\item
			объем \engterm{volume} — большой размер данных, влияющий на~выбор средств их~обработки;

			\vspace{0.5ex}
			\item
			разнообразие \engterm{variety} — отсутствие общей для~данных структуры, 
			наличие различных типов данных из~многих источников; 
			неструктурированные (естественный текст) или~полуструктурированные (XML, JSON) данные.

			\vspace{0.5ex}
			\item
			скорость \engterm{velocity} — высокие темпы накопления данных и~требования к~скорости их~обработки.
		\end{itemize}
	}

	\frame{
		\frametitle{Характеристики Big Data (продолжение)}

		\textbf{Дополнительные характеристики:}
		\begin{itemize}
			\item
			изменчивость \engterm{variability} — несогласованность между данными из~различных источников;

			\vspace{0.5ex}
			\item
			(не)достоверность \engterm{veracity} — возможность существенных различий в~качестве исходных данных;

			\vspace{0.5ex}
			\item
			сложность \engterm{complexity} — незаурядные требования к~алгоритмам и~реализациям 
			систем анализа данных для~связывания и~извлечения полезной информации.
		\end{itemize}
	}

	\subsection{Теория}

	\frame{
		\frametitle{Теория распределенных хранилищ}

		\begin{theorem}[CAP-теорема, Э. Брюэр, С. Гильберт, 2002]
			Не существует распределенной компьютерной системы, удовлетворяющей одновременно трем условиям:
			\begin{itemize}
				\item
				\textbf{согласованность данных} \engterm{consistency} — все узлы системы имеют доступ к~одним и~тем~же~данным 
				в~произвольный момент времени;

				\item
				\textbf{доступность} \engterm{availability} — на каждый запрос к~данным будет получен ответ об~успешности 
				его~выполнения;

				\item
				\textbf{масштабируемость} \engterm{partition tolerance} — система продолжает функционировать, 
				несмотря на~возможную потерю сообщений между~узлами или~отказ части системы.
			\end{itemize}
		\end{theorem}

		\begin{corollary}
			При разработке распределенных систем хранения данных выбираются 
			два~из~трех требований (чаще всего — AP или~CP), в~зависимости от~условий использования.
		\end{corollary}
	}

	\frame{
		\frametitle{MapReduce}

		\begin{Definition}
			\textbf{MapReduce} — программная модель для~параллельной обработки больших объемов данных 
			в~распределенных системах, сходная с~применением функций \code{map} и~\code{reduce} 
			в~функциональном программировании.
		\end{Definition}

		\vspace{1ex}
		\textbf{Этапы вычисления:}
		\begin{enumerate}
			\item
			подготовка данных для процедуры \code{Map()} на~узлах системы, устранение дублирующихся данных;
			\item
			выполнение кода \code{Map()}, заданного пользователем;
			\item
			реорганизация данных \engterm{shuffle} для~выполнения функции \code{Reduce()};
			\item
			выполнение кода \code{Reduce()}, заданного пользователем;
			\item
			вывод полученного результата.
		\end{enumerate}
	}

	\frame{
		\frametitle{Пример MapReduce}

		\textbf{Задача.} Определить файл (один из файлов) с~максимальным количеством слов.

		\vspace{0.5ex}
		\textbf{Вход:} набор имен файлов.

		\textbf{Выход:} словарь с одним вхождением (файл $\rightarrow$~количество слов).

		\vspace{0.5ex}
		\textbf{Псевдокод (Python):}
		\begin{overlayarea}{\textwidth}{0.6\textheight}
			\only<1>{\vspace{-1ex}\lstinputlisting[language=python]{code-mapreduce.py}}
			\only<2>{\vspace{-1ex}\lstinputlisting[language=python]{code-mapreduce2.py}}
		\end{overlayarea}
	}

	\frame{
		\frametitle{Характеристики MapReduce}

		\textbf{Особенности:}
		\begin{itemize}
			\item
			Процедура Reduce может выполняться в~несколько этапов по~мере поступления данных 
			на~каждом узле и~при~агрегации данных на~различных узлах.
			\item
			Метод MapReduce эффективен для обработки больших объемов данных ($\sim$~Гб–Тб).
		\end{itemize}

		\vspace{1ex}
		\textbf{Области применения:}
		\begin{itemize}
			\item
			распределенный поиск и индексирование;
			\item
			распределенная сортировка;
			\item
			получение статистики по документам в распределенных хранилищах;
			\item
			математические приложения (напр., 
			\extlink{http://en.wikipedia.org/wiki/Singular_value_decomposition}{сингулярное разложение матриц});
			\item
			машинное обучение (напр., кластеризация документов, машинный перевод, …).
		\end{itemize}
	}

	\subsection{Облачные ФС}

	\frame{
		\frametitle{Облачные файловые системы}

		\begin{Definition}
			\textbf{Облачная файловая система} \engterm{distributed file system for cloud}~— 
			файловая система с~распределенной архитектурой, предоставляющая пользователям 
			одновременный полноценный сетевой доступ к~данным / файлам.
		\end{Definition}

		\vspace{1ex}
		\textbf{Цели:}
		\begin{itemize}
			\item
			оптимизация пакетной обработки данных (напр., с помощью MapReduce);
			\item
			высокая доступность \engterm{high availability} — доступ к~данным при~возможности отказа узлов системы;
			\item
			поддержка сложной топологии системы (географически разделенные узлы и~кластеры);
			\item
			поддержка больших файлов (до~нескольких~Тб) и~большого количества файлов;
			\item
			использование TCP/IP и удаленного вызова процедур для~доступа к~данным.
		\end{itemize}
	}

	\frame{
		\frametitle{Облачные файловые системы}

		\textbf{Характеристики облачных ФС:}
		\begin{itemize}
			\item
			разделение файлов на~блоки ($\sim$~несколько Мб) для~оптимизации доступа;
			\item
			дублирование блоков на нескольких узлах для отказоустойчивости. 
			Часто подбираются географически разделенные узлы для оптимизации скорости доступа.
			\item
			выделенные серверы для хранения метаданных (соответствия блоков файлам, 
			положение файлов в~директориях, …).
		\end{itemize}

		\vspace{1ex}
		\textbf{Примеры облачных ФС:}
		\begin{itemize}
			\item
			\extlink{http://static.googleusercontent.com/media/research.google.com/en//archive/gfs-sosp2003.pdf}{Google File System};
			\item
			Hadoop Distibuted File System (\extlink{http://hadoop.apache.org/}{HDFS});
			\item
			\extlink{http://lustre.org/}{Lustre};
			\item
			IBM General Parallel File System (\extlink{http://www-01.ibm.com/support/knowledgecenter/SSFKCN/gpfs_welcome.html?lang=en}{GPFS}).
		\end{itemize}
	}

	\frame{
		\frametitle{Архитектура облачных ФС}

		\begin{center}
			\begin{tikz*}[%
	every node/.style={rectangle,draw,align=center,minimum height=2.5em},
	edge/.style={draw=none,font=\footnotesize\itshape,minimum height=0pt}
]
	\node(app) {Приложение}; 
	\node(name-node) [right=7em of app] {Сервер имен}; 
	\node(shadow-nn) [below=4em of name-node] {Запасной \\ сервер имен}; 
	\node(block-srv2) [right=7em of name-node] {Сервер блоков 2};
	\node(block-srv1) [above=4em of block-srv2] {Сервер блоков 1};
	\node(block-srv3) [below=4em of block-srv2] {Сервер блоков 3};
	\node(block2) [above=of block-srv1] {Файл 1 \\ блок 2};
	\node(block1) [left=1em of block2] {Файл 1 \\ блок 1};
	\node(block3) [right=1em of block2] {Файл 2 \\ блок 2};

	\draw[<-] (app) -- node[above,edge]{соответствия \\ файлов блокам} (name-node);
	\draw[<-,bend left] (app) to node[above left,edge]{прямой доступ \\ к блокам} (block-srv1);

	\draw[->] (name-node) -- node[left,edge]{дублирование \\ операций} (shadow-nn);

	\draw[->] (name-node) -- node[above left,edge]{контроль} (block-srv1);
	\draw[->] (name-node) -- (block-srv2);
	\draw[->] (name-node) -- (block-srv3);

	\draw[<->,dashed] (block-srv1) -- node[right,edge]{автоматическая \\ репликация} (block-srv2);
	\draw[<->,dashed] (block-srv2) -- (block-srv3);

	\draw (block1) -- (block-srv1);
	\draw (block2) -- (block-srv1);
	\draw (block3) -- (block-srv1);
\end{tikz*}


			\vspace{1ex}
			\figureexpl{Типичная архитектура облачных файловых систем}
		\end{center}
	}

	\subsection{Распределенные БД}

	\frame{
		\frametitle{Распределенные БД}

		\textbf{Недостатки реляционных БД для облачных приложений:}
		\begin{itemize}
			\item
			невозможность линейного горизонтального масштабирования (\emph{scaling out} — линейный рост производительности 
			при~увеличении количества узлов), слабая совместимость с~распределенными системами;
			\item
			отсутствие или недостаточность встроенных механизмов кэширования;
			\item
			фрагментация при хранении больших объемов данных;
			\item
			жесткость схемы данных, необходимость структуризации входящей информации;
			\item
			транзакции для соблюдения согласованности данных, замедляющие работу системы;
			\item
			уменьшающие производительность операции нормализации данных и~объединения таблиц 
			(оператор~SQL \codekw{JOIN}).
		\end{itemize}
	}

	\frame{
		\frametitle{NoSQL}

		\begin{Definition}
			\textbf{NoSQL} (not only SQL) — модель данных для распределенного хранения данных, 
			отличающаяся от~реляционной алгебры традиционных~СУБД.
		\end{Definition}

		\vspace{1ex}
		\textbf{Характеристики:}
		\begin{itemize}
			\item
			отсутствие жесткой схемы данных, проектирование структур данных согласно~заранее заданным 
			шаблонам запросов (а~не~наоборот, как~в~РСУБД);
			\item
			упрощение структуры данных по~сравнению с~реляционными таблицами, отсутствие нормализации;
			\item
			отказ от транзакций в пользу отложенной согласованности \engterm{delayed consistency};
			\item
			встроенная поддержка распределенной архитектуры и (часто) кэширования.
		\end{itemize}
	}

	\frame{
		\frametitle{Структуры данных в NoSQL}

		\textbf{Типы баз данных (от простых к сложным):}
		\begin{itemize}
			\item
			\textbf{Пары «ключ — значение».} 
			Используются для кэширования; зачастую данные хранятся исключительно в оперативной памяти.

			\vspace{0.5ex}
			\textbf{Примеры:} \extlink{http://redis.io/}{Redis}; \extlink{http://www.memcached.org/}{memcached}.

			\vspace{0.5ex}
			\item
			\textbf{На основе столбцов} \engterm{column-oriented}. 
			Используются для~хранения просто структурированных данных 
			при~необходимости быстрого доступа.

			\vspace{0.5ex}
			\textbf{Примеры:} \extlink{http://cassandra.apache.org/}{Apache Cassandra}, \extlink{http://hbase.apache.org/}{Apache HBase}.

			\vspace{0.5ex}
			\item
			\textbf{Графовые.}
			Хранят отношения между сущностями (напр., followers / followed~by в~Twitter).

			\vspace{0.5ex}
			\textbf{Примеры:} \extlink{http://neo4j.com/}{Neo4j}; \extlink{http://orientdb.com/}{OrientDB}.

			\vspace{0.5ex}
			\item
			\textbf{Документно-ориентированные.}
			Используются для~хранения произвольных документов со~схемой, задающейся форматом сериализации (напр., JSON).

			\vspace{0.5ex}
			\textbf{Примеры:} \extlink{http://couchdb.apache.org/}{Apache CouchDB}, \extlink{https://www.mongodb.org/}{MongoDB}.
		\end{itemize}
	}

	\subsection{Обработка данных}

	\frame{
		\frametitle{Обработка данных}

		\textbf{Машинное обучение} — извлечение полезной информации из~данных 
		при~помощи методов оптимизации / мат.~статистики:
		\begin{itemize}
			\item
			регрессия (полиномиальная, \extlink{http://en.wikipedia.org/wiki/Multivariate_adaptive_regression_splines}{MARS});
			\item
			классификация (байесовские методы, решающие деревья, SVM, бустинг, …);
			\item
			структурное распознавание (обработка изображений, текста, …);
			\item
			кластеризация (k-means, гауссовские смеси);
			\item
			сокращение размерности (сингулярное разложение и другие методы).
		\end{itemize}

		\vspace{1ex}
		\textbf{Обратное индексирование данных} — подготовка индекса для~полнотекстового поиска 
		в~большом объеме документов.
	}

	\frame{
		\frametitle{Apache Hadoop}

		\begin{Definition}
			\textbf{Apache Hadoop} — оболочка для~распределенного хранения и~обработки данных, 
			написанная на~ЯП~Java.
		\end{Definition}

		\vspace{1ex}
		\textbf{Модули:}
		\begin{itemize}
			\item
			HDFS — распределенная файловая система для хранения данных;
			\item
			система выполнения заданий MapReduce — JobTracker (центральный модуль управления заданиями), 
			TaskTracker (выполнение процедур \code{Map} и~\code{Reduce});
			\item
			планировщик заданий.
		\end{itemize}
	}

	\frame{
		\frametitle{Подключаемые модули Hadoop}

		\begin{itemize}
			\item
			\textbf{Файловые системы} (доступны через плагины).

			\vspace{0.5ex}
			\item
			\textbf{NoSQL-база данных} \extlink{http://hbase.apache.org/}{HBase} (устанавливается поверх HDFS).

			\vspace{0.5ex}
			\item
			\extlink{https://spark.apache.org/}{Apache Spark} — архитектура для~выполнения \textbf{распределенной обработки данных}, 
			альтернативная MapReduce.

			\vspace{0.5ex}
			\item
			\extlink{http://mahout.apache.org/}{Apache Mahout} — библиотека \textbf{машинного обучения}, 
			написанная на~Java.

			\vspace{0.5ex}
			\item
			\extlink{http://pig.apache.org/}{Apache Pig} — высокоуровневая \textbf{платформа для~создания заданий 
			типа~MapReduce} на~основе процедурного~ЯП Pig~Latin ($\sim$~SQL) и~пользовательских функций 
			на~Python, Java, JavaScript.

			\vspace{0.5ex}
			\item
			\extlink{http://hive.apache.org}{Apache Hive} — \textbf{инфраструктура для~обработки данных} 
			с~использованием языка запросов~HiveQL, который транслируется в~набор заданий для~Hadoop.

			\vspace{0.5ex}
			\item
			\extlink{http://zookeeper.apache.org/}{Apache ZooKeeper} — \textbf{централизованный сервер координации} 
			в~распределенных системах.
		\end{itemize}
	}

	\section{Заключение}

	\subsection{Выводы}
	
	\frame{
		\frametitle{Выводы}

		\begin{enumerate}
			\item
			Облачные вычисления — платформа для выполнения распределенных приложений 
			(как~веб-, так~и~аналитических), основанная на~принципе горизонтальной масштабируемости.

			\item
			Есть три уровня облачной архитектуры: инфраструктура как~сервис (IaaS), платформа как~сервис (PaaS) 
			и~ПО~как~сервис (SaaS).

			\item
			Основой облачной архитектуры является хранение данных (при~помощи распределенных~ФС и~NoSQL-баз данных) 
			и~их обработка (напр., с~помощью инструментов типа~MapReduce).

			\item
			Основные характеристики распределенных хранилищ данных — структура хранимых объектов 
			и~характеристики из~набора доступность, согласованность данных и~масштабируемость. 
			Согласно CAP-теореме, выполнение трех~характеристик одновременно невозможно.
		\end{enumerate}
	}
	
	\subsection{Материалы}
	
	\frame{
		\frametitle{Материалы}
		
		\begin{thebibliography}{9}
			\bibitem[1]{1}
			Sommerville, Ian
			\newblock Software Engineering.
			\newblock {\footnotesize Pearson, 2011. — 790 p.}

			\bibitem[2]{2}
			Voorsluys W., Broberg J., Buyya R.
			\newblock Introduction to Cloud Computing.
			\newblock {\footnotesize\url{http://media.johnwiley.com.au/product_data/excerpt/90/04708879/0470887990-180.pdf}}

			\bibitem[3]{3}
			Cloud Security Alliance
			\newblock Big Data Taxonomy.
			\newblock {\footnotesize\url{https://downloads.cloudsecurityalliance.org/initiatives/bdwg/Big_Data_Taxonomy.pdf}}
		\end{thebibliography}
	}
	
	\frame{
		\frametitle{}
		
		\begin{center}
			\Huge Спасибо за внимание!
		\end{center}
	}

\end{document}

