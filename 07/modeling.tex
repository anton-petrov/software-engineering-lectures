\documentclass{a4beamer}
%% Lectures - common definitions

\usextensions{tikz}
\usetikzlibrary{shapes.multipart,shapes.callouts,shapes.geometric}
\input{fix-callouts.inc} % Fixes absolute positioning of rectangle callouts

\newif\ifbigpages \bigpagesfalse
\ifdim\paperwidth >20cm
	\bigpagestrue
\fi

\tikzset{%
	note/.style={rectangle callout,draw=none,callout pointer width=1em,%
		align=flush left,font=\footnotesize,inner sep=0.5em,%
		fill=blue!15,fill opacity=0.95,text opacity=1.0,callout absolute pointer=#1},
	node distance=2em and 2.75em
}
\ifbigpages
	% Scale all arrow tips by the factor of 2.5
	\let\old@pgf@arrow@call=\pgf@arrow@call
	\def\pgf@arrow@call#1{%
		\@tempdima=\pgflinewidth%
		\pgfsetlinewidth{2.5\pgflinewidth}%
		\old@pgf@arrow@call{#1}%
		\pgfsetlinewidth{\@tempdima}%
	}
	\def\pgfarrowsleftextend#1{\pgfmathsetlength{\pgf@xa}{1.5*#1}}
	\def\pgfarrowsrightextend#1{\pgfmathsetlength{\pgf@xb}{1.5*#1}}
\fi

%% Load listings package
\usepackage{listings}

%% Are we inside a comment?
\newif\iflstcomment \lstcommentfalse

\lstset{%
	tabsize=4,
	showstringspaces=false,
	basicstyle=\linespread{1.25}\ttfamily\small,
	keywordstyle=\bfseries,
	commentstyle=\lstcommentstyle,
	numbers=left,
	numberstyle=\footnotesize\color{gray},
	xleftmargin=2.5em,
	extendedchars=true,
	escapechar=\$,
	escapebegin=\iflstcomment\begingroup\lstcommentstyle\fi,
	escapeend=\iflstcomment\endgroup\fi
}

\def\lstcommentstyle{\color{gray}}

\lst@AddToHook{AfterBeginComment}{\global\lstcommenttrue}
\let\orig@lst@EndComment=\lst@EndComment
\def\lst@EndComment{\global\lstcommentfalse\orig@lst@EndComment}
\lst@AddToHookAtTop{EOL}{%
	\lst@ifLmode\global\lstcommentfalse\fi% XXX Sloppy way to determine comment end
}

%% Python with docstrings treated as comments
\lstdefinelanguage[doc]{python}[]{python}{%
	deletestring=[s]{"""}{"""},%
	morecomment=[s]{"""}{"""}%
}%

%% JavaScript language
\lstdefinelanguage{javascript}%
	{morekeywords={break,case,catch,%
		const,constructor,continue,default,do,else,false,%
		finally,for,function,if,in,instanceof,%
		new,null,prototype,%
		return,switch,this,throw,%
		true,try,typeof,var,while},%
	sensitive,%
	morecomment=[l]//,%
	morecomment=[s]{/*}{*/},%
	morestring=[b]",%
	morestring=[b]',%
}[keywords,comments,strings]%

%% C# language (4.0?)
\lstdefinelanguage{csharp}%
	{morekeywords={abstract,as,%
		base,bool,byte,case,catch,char,%
		checked,class,const,continue,%
		decimal,default,delegate,do,double,%
		else,enum,event,explicit,extern,%
		false,finally,fixed,float,for,foreach,%
		goto,if,implicit,in,int,interface,%
		internal,is,lock,long,%
		namespace,new,null,object,operator,out,%
		override,params,private,protected,public,%
		readonly,ref,return,sbyte,sealed,%
		short,sizeof,stackalloc,static,string,%
		struct,switch,this,throw,true,try,%
		typeof,uint,ulong,unchecked,unsafe,ushort,%
		using,virtual,void,volatile,while%
	},%
	sensitive,%
	morecomment=[l]//,%
	morecomment=[s]{/*}{*/},%
	morestring=[b]",%
	morestring=[b]',%
}[keywords,comments,strings]%

%% Translation for fact environment
\deftranslation[to=russian]{Fact}{Наблюдение}

%% Inline code snippets
\def\code#1{\texttt{#1}}
\def\codekw#1{\code{\textbf{#1}}}

\def\quoteauthor#1{\par\footnotesize\upshape\hfill—~#1}

%% English term
\def\engterm#1{(англ. \textit{#1})}
%% Term with explanation below (to be used in diagrams)
\def\termwithexpl#1#2{#1\strut{}\\\small\color{gray}(\textit{#2})\strut{}}
%% External link
\def\extlink#1#2{\href{#1}{\color[rgb]{0.7,0.7,1.0}\dashbar{#2}}}
%% Internal link
\def\inlink#1#2{\hyperlink{#1}{\color[rgb]{0.7,0.7,1.0}\dashbar{#2}}}
%% Explanation for a list item
\def\itemexpl#1{\begingroup\small\vspace{0.75ex}#1\par\endgroup}



\usetikzlibrary{shapes.misc}
%%
%% Human-like shape for UML diagrams. Expects 'draw' style and no node text.
%% (c) Alexei Ostrovski, 2014
%%
\begingroup
\makeatletter
\pgfdeclareshape{human}{
	\inheritsavedanchors[from=rectangle]
	\inheritanchorborder[from=rectangle]
	% Anchors
	\inheritanchor[from=rectangle]{center}
	\inheritanchor[from=rectangle]{north}
	\inheritanchor[from=rectangle]{east}
	\inheritanchor[from=rectangle]{south}
	\inheritanchor[from=rectangle]{west}
	\inheritanchor[from=rectangle]{north east}
	\inheritanchor[from=rectangle]{north west}
	\inheritanchor[from=rectangle]{south east}
	\inheritanchor[from=rectangle]{south west}
	\backgroundpath{%
		\pgf@process{\northeast}% 
		\pgf@xb=.5\pgf@x%
		\pgf@yb=.5\pgf@y%
		\pgf@process{\southwest}%
		\pgf@xa=.5\pgf@x%
		\pgf@ya=.5\pgf@y%
		\advance\pgf@xa by \pgf@xb%
		\advance\pgf@ya by \pgf@yb%
		%(\pgf@xa,\pgf@ya) = center
		\pgf@process{\northeast}%
		\pgf@xb=\pgf@x%
		\pgf@yb=\pgf@y%
		\pgf@process{\southwest}%
		\advance\pgf@xb by -\pgf@x%
		\advance\pgf@yb by -\pgf@y%
		\pgf@x=\pgfkeysvalueof{/pgf/outer xsep}
		\pgf@y=\pgfkeysvalueof{/pgf/outer ysep}
		\advance\pgf@xb by -2\pgf@x
		\advance\pgf@yb by -2\pgf@y
		%(\pgf@xb,\pgf@yb) = (width,height)
		% Torso
		\pgf@xc=\pgf@xa \pgf@yc=\pgf@ya
		\advance\pgf@yc by 0.2\pgf@xb
		\pgfpathmoveto{\pgfpoint{\pgf@xc}{\pgf@yc}}
		\advance\pgf@yc by -0.4\pgf@xb
		\pgfpathlineto{\pgfpoint{\pgf@xc}{\pgf@yc}}
		\pgfpathclose
		% Legs
		\pgf@xc=\pgf@xa \pgf@yc=\pgf@ya
		\advance\pgf@yc by -0.2\pgf@xb
		\pgfpathmoveto{\pgfpoint{\pgf@xc}{\pgf@yc}}
		\advance\pgf@xc by -0.3\pgf@xb
		\advance\pgf@yc by -0.3\pgf@yb
		\pgfpathlineto{\pgfpoint{\pgf@xc}{\pgf@yc}}
		\pgfpathclose
		\pgf@xc=\pgf@xa \pgf@yc=\pgf@ya
		\advance\pgf@yc by -0.2\pgf@xb
		\pgfpathmoveto{\pgfpoint{\pgf@xc}{\pgf@yc}}
		\advance\pgf@xc by 0.3\pgf@xb
		\advance\pgf@yc by -0.3\pgf@yb
		\pgfpathlineto{\pgfpoint{\pgf@xc}{\pgf@yc}}
		\pgfpathclose
		% Arms
		\pgf@xc=\pgf@xa \pgf@yc=\pgf@ya
		\advance\pgf@xc by -.4\pgf@xb
		\pgfpathmoveto{\pgfpoint{\pgf@xc}{\pgf@yc}}
		\advance\pgf@xc by 0.8\pgf@xb
		\pgfpathlineto{\pgfpoint{\pgf@xc}{\pgf@yc}}
		\pgfpathclose
		% Head
		\pgf@xc=\pgf@xa \pgf@yc=\pgf@ya
		\advance\pgf@yc by 0.3\pgf@yb
		\pgfpathcircle{\pgfpoint{\pgf@xc}{\pgf@yc}}{0.15\pgf@yb}
	}
}
\endgroup



\lecturetitle{Программная инженерия. Лекция №7 — Моделирование программных систем.}
\title[Моделирование ПС]{Моделирование программных систем}
\author{Алексей Островский}
\institute{\small{Физико-технический учебно-научный центр НАН Украины}\vspace{2ex}}
\date{07 ноября 2014 г.}

\begin{document}
	\frame{\titlepage}
	
	\section[Основы]{Основы моделирования}
	
	\subsection{Определение}
	
	\frame{
		\frametitle{Моделирование в разработке ПО}

		\begin{center}\small
			\begin{tikz*}[%
	scale=1.25,
	every node/.style={rectangle,align=center,minimum height=3em},
	every label/.style={,minimum height=0pt,font=\small\itshape}
]
	\node(comp) {Проектирование \\ компонентов};
	\node(constr) [right=of comp] {Конструирование};
	\node(arch) [above=of comp] {Архитектура};
	\node(model) [above=of constr] {\bfseries Моделирование};
	\node(req) [right=of model] {Инженерия \\ требований};
	\node(test) [right=of constr] {Тестирование};
	\node(maint) [right=of test] {Сопровождение};
	
	\draw[<->] (req) to (model);
	\draw[->] (constr) to (test);
	\draw[->] (test) to (maint);
	\draw[->] (model) to (arch);
	\draw[->] (arch) to (comp);
	\draw[->] (comp) to (constr);
	
	\node(design) [draw,dashed,inner ysep=1em,inner xsep=2.5em,
		fit=(model.north east) (arch.north west) (comp.south west) (constr.south east),label=below:Проектирование и программирование] {};
\end{tikz*}

		\end{center}
		
		\begin{Definition}
			\textbf{Системное моделирование} — процесс создания абстрактных моделей программной системы, 
			отображающих различные ее аспекты.
		\end{Definition}
	}
	
	\frame{
		\frametitle{Цели моделирования}

		\begin{center}\linespread{1.0}
			\begin{tabular}{|p{0.22\textwidth}|p{0.65\textwidth}|}
				\hline
				\centering\textbf{Процесс} & \centering\textbf{Цель моделирования} \cr
				\hline
				\raggedright Инженерия требований & \raggedright объяснение предложенных требований заинтересованным сторонам. \cr
				\hline
				Проектирование & \raggedright создание общей структуры системы архитекторами; планирование и документирование общих указаний по~имплементации. \cr
				\hline
				Программирование & \raggedright частичная или полная имплементация системы с помощью генераторов кода. \cr
				\hline
				Сопровождение & \raggedright объяснение структуры системы для команды сопровождения; базис для внесения изменений в систему. \cr
				\hline
			\end{tabular}
		\end{center}
	}
	
	\subsection{Представления системы}
	
	\frame{
		\frametitle{Представления системы}

		\begin{Definition}
			\textbf{Представление} — абстрактная модель системы, выделяющая ее характеристики, 
			соответствующие определенному аспекту ее функционирования.
		\end{Definition}

		\vspace{1ex}
		\textbf{Основные представления:}
		\begin{itemize}
			\item контекстное представление — модель окружения, в котором выполняется система;
			\item взаимодействия — связи системы с окружением, а также элементов в системе;
			\item структурное представление — организация системы и данных для обработки;
			\item поведение — модель реагирования системы в ответ на внешние события.
		\end{itemize}
	}
	
	\subsection{Язык моделирования UML}
	
	\frame{
		\frametitle{Язык моделирования UML}

		\begin{Definition}
			\textbf{Унифицированный язык моделирования} \engterm{unified modeling language, UML} — язык маркировки общего назначения, 
			целью которого является стандартизация графического представления архитектуры и дизайна ПС. 
		\end{Definition}
		
		\vspace{1.5ex}
		\begin{tabular}{r@{\ —\ }l}
			\textbf{1996 г.} & UML 1.0 (Grady Booch, Ivar Jacobson, James Rumbaugh). \cr
			\textbf{2000 г.} & стандарт ISO. \cr
			\textbf{2005 г.} & UML 2.0 (новые виды диаграмм, расширение семантики языка). \cr
		\end{tabular}
		
		\vspace{2ex}
		\textbf{Инструменты UML:} 
		\begin{itemize}
			\item Eclipse Modeling Tools; 
			\item Papyrus; 
			\item Rational Software Architect/Modeler, …
		\end{itemize}
	}

	\frame{
		\frametitle{Основные диаграммы UML}
		
		\begin{itemize}
			\item
			\textbf{Диаграмма деятельности} \engterm{activity diagram}: составляющие деятельности по~обработке данных.
			\item
			\textbf{Диаграмма вариантов использования} \engterm{use case diagram}: взаимодействие между~системой и ее окружением.
			\item
			\textbf{Диаграмма последовательности} \engterm{sequence diagram}: 
			взаимодействие системы с~актерами и компонентов системы друг с другом.
			\item
			\textbf{Диаграмма классов} \engterm{class diagram}: структура классов, используемых в системе, и~отношения между классами.
			\item
			\textbf{Диаграмма состояний} \engterm{state diagram}: реагирование системы на внутренние и~внешние события.
		\end{itemize}
	}
	
	\section[Контекст]{Контекстные модели}
	
	\subsection{Контекст системы}
	
	\frame{
		\frametitle{Контекстные модели}
		
		\textbf{Цель:} 
		\begin{itemize}
			\item разграничение функций системы и ее окружения;
			\item определение компонентов, которые надо имплементировать, и используемых интерфейсов.
		\end{itemize}
		
		\vspace{0.5ex}
		\begin{center}
			\begin{tikz*}[%
	scale=1.25,
	every node/.style={rectangle,draw,align=center,minimum height=3em,minimum width=7.5em}
]
	\node(lib) {«система» \\ \textbf{Библиотека}};
	\node(scan) [left=of lib] {«система» \\ Сканирование книг};
	\node(archive) [right=of lib] {«система» \\ Электронный архив};
	\node(plan) [above=of lib] {«система» \\ Планирование выдачи};
	\node(report) [below=of lib] {«система» \\ Сбор отчетности};
	
	\draw (lib) to (scan);
	\draw (lib) to (archive);
	\draw (lib) to (plan);
	\draw (lib) to (report);
\end{tikz*}

		
			\vspace{1ex}
			\figureexpl{\textbf{Пример:} контекст электронной библиотечной системы.}
		\end{center}
	}
	
	\subsection{Модель процесса}
	
	\frame{
		\frametitle{Модель процесса}
		
		\label{p:activity}
		\begin{center}
			{\small\begin{tikz*}[%
	every node/.style={draw,align=center,font=\small},
	activity/.style={rounded rectangle,minimum height=3em},
	decision/.style={diamond,minimum width=1em,minimum height=1em},
	label/.style={draw=none,font=\footnotesize},
	system/.style={rectangle,minimum height=3em}
]
	\node(start) [circle,fill,minimum width=1em,minimum height=1em] {};
	\node(data) [activity,below=of start] {\alert<4>{Извлечь данные}};
	\node(sink1) [rectangle,fill,minimum width=5em,below=of data] {};
	\node(check-exist) [activity,below=of sink1] {\alert<4>{Проверить} \\ \alert<4>{наличие}};
	\node(check-e) [activity,below left=2em and 5em of sink1] {Проверить \\ эл. версию};
	\node(sys-e) [system,above=of check-e] {\alert<9>{Электронный} \\ \alert<9>{архив}};
	\node(check-access) [activity,below right=2em and 4em of sink1]{Проверить \\ доступ};
	\node(sink2) [rectangle,fill,minimum width=5em,below=of check-exist] {};
	\node(decision) [decision,below=of sink2] {};
	\node(unavailable) [activity,below=of decision] {Уведомить};
	\node(e-version) at (decision.west -| check-e.south) [activity] {Загрузить \\ эл. версию};
	\node(below-un) [coordinate,below=1.5em of unavailable] {};
	
	\node(sink3) [rectangle,fill,minimum height=5em,right=10em of decision] {};
	\node(db) [activity,right=of sink3]{Обновить БД};
	\node(appointment) [activity,above=of db]{Назначить \\ встречу};
	\node(sys-app) [system,above=of appointment]{\alert<9>{Планирование} \\ \alert<9>{выдачи}};
	\node(doc) [activity,below=of db]{\alert<4>{Запросить} \\ \alert<4>{документ}};
	\node(sink4) [rectangle,fill,minimum height=5em,right=of db] {};
	\node(end1) [circle,fill,minimum width=1em,minimum height=1em,right=of sink4] {};
	\node(end) [circle,draw,minimum width=1.35em,minimum height=1.35em] at (end1.center) {};
	
	\draw[->] (start) to (data);
	\draw[->] (data) to (sink1);
	\draw[->] (sink1) to (check-exist);
	\draw[->] (sink1) to (check-e);
	\draw[->] (sink1) to (check-access);
	\draw[<-] (check-e) to (sys-e);
	\draw[<-] (sink2) to (check-exist);
	\draw[<-] (sink2) to (check-e);
	\draw[<-] (sink2) to (check-access);
	\draw[->] (sink2) to (decision);
	\draw[->] (decision) -- node(lbl-no)[label,right] {\alert<6>{[нет]}} (unavailable);
	\draw[->] (decision) -- node(lbl-e)[label,above] {\alert<6>{[эл. версия]}} (e-version);
	\draw[->] (decision) -- node(lbl-paper)[label,above] {\alert<6>{[бум. версия]}} (sink3);
	\draw[->] (sink3) to (db);
	\draw[->] (sink3) to (appointment);
	\draw[->] (sink3) to (doc);
	\draw[<-] (sink4) to (db);
	\draw[<-] (sink4) to (appointment);
	\draw[<-] (sink4) to (doc);
	\draw[->] (sys-app) to (appointment);
	\draw[->] (sink4) to (end);
	\draw (unavailable) to (below-un);
	\draw (e-version.south) |- (below-un);
	\draw[->] (below-un) -| (end);
	\draw[->] (sys-e.west) -- ++(-1em,0) |- (e-version.west);
	
	\node<2> [note={(start.south east)},below right=1em of start] {начальное состояние};
	\node<3> [note={(end.north west)},above left=1em of end] {конечное состояние};
	\node<4> [note={(data.east)},right=1em of data,callout pointer width=0.5em] {действие};
	\node<4> [note={(check-exist.south east)},below right=1em of check-exist] {действие};
	\node<4> [note={(doc.west)},left=1em of doc,callout pointer width=0.5em] {действие};
	\node<5> [note={(decision.south east)},below right=1em of decision] {решение};
	\node<6> [note={(lbl-no.south)},below=1em of lbl-no] {условие};
	\node<6> [note={(lbl-paper.south)},below=1em of lbl-paper] {условие};
	\node<6> [note={(lbl-e.south)},below=1em of lbl-e] {условие};
	\node<7> [note={(sink1.east)},right=1em of sink1] {начало выполнения \\ параллельных действий};
	\node<7> [note={(sink3.south west)},below left=1em of sink3] {начало выполнения \\ параллельных действий};
	\node<8> [note={(sink2.east)},right=1em of sink2] {конец выполнения \\ параллельных действий};
	\node<8> [note={(sink4.south west)},below left=1em of sink4] {конец выполнения \\ параллельных действий};
	\node<9> [note={(sys-e.east)},right=1em of sys-e,callout pointer width=0.5em] {внешняя система};
	\node<9> [note={(sys-app.south)},below=1em of sys-app] {внешняя система};
\end{tikz*}
}
			
			\vspace{1ex}
			\figureexpl{Диаграмма деятельности UML \engterm{activity diagram} для обработки запроса на выдачу документа.}
		\end{center}
	}
	
	\section[Взаимодействие]{Модели взаимодействия}
	
	\frame{
		\frametitle{Модели взаимодействия}
		
		\textbf{Цели:} 
		\begin{itemize}
			\item отладка взаимодействия с пользователями; 
			\item выработка пользовательских требований; 
			\item определение возможных «узких мест» и проблем коммуникации; 
			\item отработка производительности \engterm{performance} и надежности \engterm{dependability}.
		\end{itemize}
		
		\vspace{1ex}
		\textbf{Типы моделей:}
		\begin{itemize}
			\item
			варианты использования — моделирование взаимодействия системы с актерами (пользователями или другими системами);
			\item
			диаграммы последовательностей — моделирование взаимодействия компонентов системы.
		\end{itemize}
	}
	
	\subsection{Варианты использования}
	
	\frame{
		\frametitle{Варианты использования}
		
		\begin{center}
			\begin{overlayarea}{\textwidth}{0.21\textheight}
				\begin{tikz*}[%
	every node/.style={draw,align=center},
	label/.style={draw=none,font=\small\bfseries},
	actor/.style={human,minimum width=2em,minimum height=3em,outer sep=6pt},
	action/.style={ellipse,minimum height=2.5em}
]
	\node(reader) [actor] {};
	\node(lbl-reader) [label,below=0pt of reader] {\alert<2>{Читатель}};
	\node(act) [action,right=5em of reader] {\alert<3>{Выдача книги}};
	\node(lib) [actor,right=5em of act] {};
	\node(lbl-lib) [label,below=0pt of lib] {\alert<2>{Библиотекарь}};
	
	\draw (reader) to (act);
	\draw (act) to (lib);
	
	\node<2> [note={(lbl-reader.east)},right=1em of lbl-reader,callout pointer width=0.5em] {актер};
	\node<2> [note={(lbl-lib.west)},left=1em of lbl-lib,callout pointer width=0.5em] {актер};
	\node<3> [note={(act.south)},below=1em of act] {взаимодействие};
\end{tikz*}

			\end{overlayarea}
			
			\vspace{1ex}
			\figureexpl{Взаимодействие между читателем и библиотекарем на диаграмме вариантов использования \engterm{use case diagram}}
		\end{center}
		
		\vspace{0.9ex}
		\begin{center}\linespread{1.0}
			\begin{tabular}{rp{0.7\textwidth}}
				\textbf{Актеры:} & читатель, библиотекарь \cr
				\textbf{Описание:} & \raggedright Библиотекарь выдает книгу на руки пользователю 
					и вносит соответствующие данные в систему. \cr
				\textbf{Данные:} & \raggedright идентификатор книги, дата выдачи, кому выдана, на какой срок. \cr
				\textbf{Побуждение:} & \raggedright команда системы, полученная от библиотекаря. \cr
				\textbf{Отклик:} & \raggedright подтверждение внесения изменений в систему. \cr
				\textbf{Доп. условия:} & \raggedright библиотекарь должен быть авторизован в системе; 
					читатель должен иметь разрешение на выдачу книги. \cr
			\end{tabular}
		\end{center}
	}
	
	\frame{
		\frametitle{Варианты использования (продолжение)}
		
		\begin{center}
			{\small\begin{tikz*}[%
	every node/.style={draw,align=center},
	actor-label/.style={draw=none,font=\small\bfseries},
	label/.style={draw=none,font=\small},
	actor/.style={human,minimum width=2.5em,minimum height=4em,outer sep=6pt},
	action/.style={ellipse,minimum height=3.5em}
]
	\node(reader) [actor] {};
	\node(lbl-reader) [actor-label,below=0pt of reader] {Читатель};
	\node(reg-reader) [action,left=of reader] {Регистрация \\ читателя};
	\node(query-book) [action,below=of reg-reader] {Запрос книги};
	\node(activity) [action,above=of reg-reader] {Просмотр \\ активности};
	\node(take-book) [action,above right=0em and 5em of reader] {Выдача книги};
	\node(read-book) [action,above=4em of take-book] {Выдача для \\ ознакомления};
	\node(give-book) [action,below right=0em and 5em of reader] {Сдача книги};
	\node(lib) [actor,below right=0em and 5em of take-book] {};
	\node(lbl-lib) [actor-label,below=0pt of lib] {Библиотекарь};
	\node(reg-book) [action,below=3em of lbl-lib] {Регистрация \\ книги};
	\node(upd-book) [action,left=7em of reg-book] {Обновление \\ данных};
	
	\draw (reader) to (reg-reader);
	\draw (reader) to (query-book);
	\draw (reader) to (activity);
	\draw (reader) to (take-book);
	\draw (reader) to (give-book);
	\draw (lib) to (take-book);
	\draw (lib) to (give-book);
	\draw[->,dashed] (read-book) -- 
		node(relation1)[label,right] {«расширяет»} 
		node(comment)[label,left] {\alert<3>{\{при невозможности} \\ \alert<3>{выдачи на абонемент\}}} 
		(take-book);
	\draw (lbl-lib) to (reg-book);
	\draw[->,dashed] (upd-book) -- node(relation2)[label,below] {\alert<2>{«расширяет»}} (reg-book);
	
	\node<2> [note={(relation2.north)},above=1em of relation2] {отношение между взаимодействиями \\ («расширяет» или «использует»)};
	\node<3> [note={(comment.south)},below=1em of comment]{комментарий};
\end{tikz*}
}
			
			\vspace{1ex}
			\figureexpl{Более сложный пример, демонстрирующий связи между взаимодействиями}
		\end{center}
	}
	
	\subsection{Диаграммы последовательностей}
	
	\frame{
		\frametitle{Диаграммы последовательностей}
		
		\label{p:sequence}
		\begin{center}
			\begin{tikz*}[%
	every node/.style={draw,align=center},
	actor-label/.style={draw=none,font=\small\bfseries},
	lifeline/.style={draw=blue,fill=white},
	actor/.style={human,minimum width=1.0em,minimum height=2.25em},
	action/.style={draw=none,font=\footnotesize},
	condition/.style={draw=none,font=\small,fill=white}
]
	\node(lib) [actor] {};
	\node(lbl-lib) [actor-label,above=0pt of lib] {\alert<2>{Библиотекарь}};
	\draw[dashed] (lib.south) ++(0, -1em) -- ++(0, -18em);
	\node(line-lib) [lifeline,below=1em of lib.south,minimum height=16em] {};
	
	\node(bookinfo) [right=10em of lib.south,anchor=south] {B: BookInfo};
	\draw[dashed] (bookinfo.south) ++(0, -1em) -- ++(0, -18em);
	\node(line-bookinfo) [lifeline,below=1.5em of bookinfo.south,anchor=north,minimum height=14.5em] {};
	
	\node(db) [right=10em of bookinfo.south,anchor=south] {\alert<2>{D: BookDB}};
	\draw[dashed] (db.south) ++(0, -1em) -- ++(0, -18em);
	\node(line-db) [lifeline,below=2.5em of db.south,anchor=north,minimum height=13em] {};
	
	\node(auth) [right=10em of db.south,anchor=south] {AS: Authorization};
	\draw[dashed] (auth.south) ++(0, -1em) -- ++(0, -18em);
	\node(line-auth) [lifeline,below=3.5em of auth.south,anchor=north,minimum height=3.5em] {};
	
	\node(alt-box-ne) [coordinate,below=2em of line-auth.south] {};
	\node(alt-box-sw) [coordinate,below=0pt of line-lib.south] {};
	\node(alt-box) [fit=(alt-box-ne) (alt-box-sw),inner xsep=2em,inner ysep=1em] {};
	\node(lbl-alt-box) at (alt-box.north west) [anchor=north west,fill=white] {\alert<6>{Alt}};
	\node(alt-success) [condition,below right=2pt and 2pt of lbl-alt-box.south west,anchor=north west] {\alert<7>{[успешная авторизация]}};
	\draw[dashed](alt-success.south -| alt-box.west) ++ (0, -1em) --
		($ (alt-success.south -| alt-box.east) + (0, -1em) $);
	\node(alt-fail) [condition,below right=1.1em and 2pt of alt-success.south -| alt-box.west,anchor=north west] {\alert<7>{[ошибка авторизации]}};
	
	\draw[<-] (line-bookinfo.north west) ++(0, -0.5em) -- node(act-view-info)[action,above]{\alert<4>{ViewInfo(BID)}}
		($ (line-bookinfo.north west -| line-lib.east) + (0, -0.5em) $);
	\draw[<-] (line-db.north west) ++(0, -0.5em) -- node(act-view)[action,above]{\alert<4>{View(\textbf{out} Info, BID, UID)}}
		($ (line-db.north west -| line-bookinfo.east) + (0, -0.5em) $);
	\draw[<-] (line-auth.north west) ++(0, -0.5em) -- node(act-auth)[action,above]{Auth(UID, \textit{просмотр})}
		($ (line-auth.north west -| line-db.east) + (0, -0.5em) $);
	\draw[->,dashed] (line-auth.south west) ++(0, 1em) -- node(resp-auth)[action,above]{\alert<5>{авторизация}}
		($ (line-auth.south west -| line-db.east) + (0, 1em) $);
	\draw[->,dashed] (alt-box.north -| line-db.west) ++(0, -3em) --node(resp-info)[action,above]{\alert<5>{информация о книге}}
		($ (alt-box.north -| line-bookinfo.east) + (0, -3em) $);
	\draw[->,dashed] (alt-fail.north -| line-db.west) ++(0, -2em) --node(resp-error)[action,above]{Error(\textit{нет доступа})}
		($ (alt-fail.north -| line-bookinfo.east) + (0, -2em) $);
		
	\node<2> [note={(lib.south)},below=1em of lib.south,anchor=north west] {задействованный актер};
	\node<2> [note={(db.south)},below=1em of db.south] {задействованный объект};
	
	\node<3> [lifeline,fill=blue!50,below=1em of lib.south,minimum height=16em] {};
	\node<3> [lifeline,fill=blue!50,below=2.5em of db.south,anchor=north,minimum height=13em] {};
	\node<3> [note={(line-db.south east)},right=1em of line-db.south east,callout pointer width=0.5em] 
		{время существования объекта \\ \engterm{lifeline}};
	\node<3> [note={(line-lib.east)},above right=2em and 1em of line-lib.east] 
		{время существования объекта \\ \engterm{lifeline}};
		
	\node<4> [note={(act-view-info.south)},below=1em of act-view-info.south] {сообщение};
	\node<4> [note={(act-view.south)},below=1em of act-view.south] {сообщение};
	
	\node<5> [note={(resp-auth.south)},below=1em of resp-auth.south] {ответное сообщение};
	\node<5> [note={(resp-info.north)},above=1em of resp-info.north] {ответное сообщение};
	
	\node(alt-border) [coordinate] at ($ (alt-success.south -| alt-box.east) + (-8em, -1em) $) {};
	\node<6> [note={(lbl-alt-box.north)},above=1em of lbl-alt-box.north,anchor=south west] 
		{альтернативное развитие \\ событий};
	\node<6> [note={(alt-border)},below=1em of alt-border] 
		{граница между вариантами};
	
	\node<7> [note={(alt-success.north)},above=1em of alt-success.north,anchor=south west] {условие выполнения};
	\node<7> [note={(alt-fail.south)},below=1em of alt-fail.south,anchor=north west] {условие выполнения};
\end{tikz*}

			
			\vspace{1ex}
			\figureexpl{Диаграмма последовательности для просмотра информации о книге}
		\end{center}
	}
	
	\frame{
		\frametitle{Диаграммы последовательностей (продолжение)}
		
		\textbf{Интерпретация примера:}
		\begin{enumerate}
			\item
			\textbf{Библиотекарь} запрашивает информацию по книге по ее идентификатору \textbf{BID} (напр., ISBN). 
			Для отображения информации создается экземпляр \textbf{B} класса \textbf{BookInfo}, отображающий информацию в виде таблицы.
			
			\item
			\textbf{B} запрашивает базу данных \textbf{D}, предоставляя ей идентификатор пользователя \textbf{UID}.
		
			\item
			База данных проверяет право \textbf{UID} на \textbf{просмотр сведений} о книге с помощью системы авторизации \textbf{AS}.
			
			\item
			Если авторизация выполнена успешно, \textbf{D} возвращает данные о книге. \textbf{B} отображает их.
			
			\item
			Если авторизация не удалась, возвращается \textbf{ошибка}. \textbf{B} отображает сведения об ошибке.
		\end{enumerate}
	}
	
	\section[Структура]{Структурные модели}
	
	\subsection{Диаграммы классов}
	
	\frame{
		\frametitle{Структурные модели}
		
		\textbf{Цели:}
		\begin{itemize}
			\item определение архитектуры системы;
			\item определение структур хранения данных.
		\end{itemize}
		
		\vspace{0.5ex}
		\textbf{Диаграмма классов UML:}
		\vspace{0.5ex}
		\begin{center}
			\begin{tikz*}[%
	class/.style={draw,rectangle split,rectangle split parts=3},
	qty/.style={font=\footnotesize},
	role/.style={font=\small}
]
	\node(reader) [class,text width=7.5em] {
		\hfill\alert<2>{\textbf{Читатель}}\hfill\strut{}
		\nodepart{two}
		\alert<3>{Имя} \\ 
		\alert<3>{\textit{Адрес}} \\ 
		\alert<3>{e-mail} \\ 
		\alert<3>{Телефон}
		\nodepart{three}
		\alert<4>{Задолженность}
	};
	\node(book) [class,text width=7.5em,right=6.5em of reader] {
		\hfill\textbf{Книга}\hfill\strut{}
		\nodepart{two}
		Название \\ 
		\textit{Автор} \\ 
		\textit{Издатель} \\
		Год издания \\
		ISBN \\ 
		Кол-во страниц
		\nodepart{three}
		\alert<4>{Ссылка} \\ 
		\alert<4>{Похожие книги}
	};
	\node(category) [class,text width=7.5em,right=6.5em of book] {
		\hfill\alert<2>{\textbf{Категория}}\hfill\strut{}
		\nodepart{two}
		Название \\ 
		Описание
		\nodepart{three}
		Кол-во книг
	};
	
	\node(_tmp) [coordinate] at ($ (reader.text split east) + (0,-0.5em) $) {};
	\draw (_tmp) -- (_tmp -| book.west)
		node(qty1)[qty,above,very near start]{\alert<6,7>{0..1}} 
		node(role1)[role,below,midway]{\alert<5>{выдача}} 
		node(qty2)[qty,above,very near end]{\alert<6,8>{0..20}};
	\node(_tmp) [coordinate] at ($ (reader.two split east) + (0,0.5em) $) {};
	\draw (_tmp) -- (_tmp -| book.west)
		node(qty3)[qty,above,very near start]{0..*} 
		node(role2)[role,below,midway]{история} 
		node(qty4)[qty,above,very near end]{0..*};
	\draw (book) -- (category)
		node(qty5)[qty,above,very near start]{\alert<6,9>{1..*}} 
		node(role3)[role,below,midway]{} 
		node(qty6)[qty,above,very near end]{\alert<10>{0..*}};
		
	\node<2> [note={(reader.text split)},below=1em of reader.text split,anchor=north west] {название класса};
	\node<2> [note={(category.text split)},below=1em of category.text split,anchor=north east] {название класса};
	
	\node<3> [note={(reader.two east)},right=1em of reader.two east,callout pointer width=0.5em] {поля класса};
	\node<4> [note={(reader.three east)},right=1em of reader.three east,callout pointer width=0.5em] {операции класса};
	\node<4> [note={(book.three east)},right=1em of book.three east,callout pointer width=0.5em] {операции класса};
	
	\node<5> [note={(role1.north)},above=1em of role1.north] {отношение между классами};
	\node<5> [note={(role3.north)},above=1em of role3.north] {отношение между классами};
	
	\node<6> [note={(qty1.south)},below=1em of qty1.south] {кратность};
	\node<6> [note={(qty2.south)},below=1em of qty2.south] {кратность};
	\node<6> [note={(qty5.south)},below=1em of qty5.south] {кратность};
	
	\node<7> [note={(qty1.south)},below=1em of qty1.south] {%
		Каждая книга в любой момент времени \\ может находиться у 0 либо 1 читателя.};
	\node<8> [note={(qty2.south)},below=1em of qty2.south] {%
		Каждый читатель может взять \\ не более 20 книг.};
	\node<9> [note={(qty5.south)},below=1em of qty5.south] {%
		Категория соответствует \\ как минимум одной книге.};
	\node<10> [note={(qty6.south)},below=1em of qty6.south] {%
		У каждой книги может быть \\ произвольное число категорий.};
\end{tikz*}

			
			\vspace{1ex}
			\figureexpl{Диаграмма для нескольких классов предметной области «Библиотека». \textit{Курсив} = другие классы.}
		\end{center}
	}
	
	\subsection{Отношения обобщения}
	
	\frame{
		\frametitle{Отношения обобщения}
		
		\begin{center}
			\begin{tikz*}[%
	class/.style={draw,rectangle split,rectangle split parts=3},
	qty/.style={font=\footnotesize},
	role/.style={font=\small}
]

	\node(reader) [class,text width=8em] {
		\hfill\textbf{Читатель}\hfill\strut{}
		\nodepart{two}
		\nodepart{three}
		Задолженность
	};
	\node(lib) [class,text width=8em,right=of reader.north east,anchor=north west] {
		\hfill\textbf{Библиотекарь}\hfill\strut{}
		\nodepart{two}
		Зарплата \\
		Стаж
		\nodepart{three}
	};
	\node(person) [class,text width=8em,above=4em of $ (reader.north east)!0.5!(lib.north west) $,anchor=south] {
		\hfill\textbf{Личность}\hfill\strut{}
		\nodepart{two}
		Имя \\
		Фамилия \\
		Телефон \\
		e-mail
		\nodepart{three}
	};
	\node(library) [class,text width=8em,right=6em of lib.north east,anchor=north west] {
		\hfill\textbf{Библиотека}\hfill\strut{}
		\nodepart{two}
		Название \\
		Информация
		\nodepart{three}
	};
	\node(address) [class,text width=8em,above=4em of library,anchor=south] {
		\hfill\textbf{Адрес}\hfill\strut{}
		\nodepart{two}
		Город \\
		Улица \\
		Дом
		\nodepart{three}
		Расстояние
	};

	\node(gen) [coordinate,above=2em of $ (reader.north east)!0.5!(lib.north west) $]{};
	\draw (reader) |- (gen);
	\draw (lib) |- (gen);
	\draw[->,>=open triangle 60] (gen) -- (person);
	\draw[->] (person.east) -- (address.west |- person.east)
		node(qt1)[qty,above,very near start]{\alert<3>{0..*}}
		node[role,below,midway]{живет}
		node(qt2)[qty,above,very near end]{\alert<5>{0..1}};
	\draw[->] (lib.east) -- (library.west |- lib.east)
		node[qty,above,very near start]{1..*}
		node[role,below,midway]{работает}
		node[qty,above,very near end]{1};
	\draw[->] (library.north) -- (address.south)
		node(qt3)[qty,right,very near start]{\alert<4>{0..1}}
		node[role,left,midway]{находится}
		node[qty,right,very near end]{1};

	\node<2> [note={(gen)},above right=1em and 2em of gen] {%
		отношение обобщения};
	\node<3> [note={(qt1.south)},below=1em of qt1.south] {%
		Адрес может не соответствовать человеку. \\
		Один адрес может соответствовать \\ нескольким людям.};
	\node<4> [note={(qt3.west)},left=1em of qt3.west] {%
		Адрес может не соответствовать \\ библиотеке.};

	\node<5> [note={(qt2.south)},below=1em of qt2.south] {%
		Человек может не предоставить \\ адрес проживания.};
\end{tikz*}

			
			\vspace{1ex}
			\figureexpl{Диаграмма классов, демонстрирующая отношения обобщения \engterm{generalization}.}
		\end{center}
	}
	
	\subsection{Отношения агрегации и композиции}
	
	\frame{
		\frametitle{Отношения агрегации и композиции}
		
		\textbf{Агрегация («часть/целое», слабая связь)}
		\begin{center}
			\begin{tikz*}[%
				class/.style={draw,rectangle,align=center,minimum height=3em,minimum width=7.5em,font=\bfseries},
				qty/.style={font=\footnotesize},
				role/.style={font=\small}
			]
				\node(library) [class]{Библиотека};
				\node(book) [class,right=6em of library] {Книга};
				
				\draw[-open diamond] (book) -- (library)
					node[qty,very near start,above] {1..*}
					node[role,midway,below] {включает}
					node[qty,very near end,above] {1};
			\end{tikz*}
			
			\vspace{1ex}
			\figureexpl{Каждая книга принадлежит одной из библиотек. Эта библиотека может измениться.}
		\end{center}
		
		\vspace{2ex}
		\textbf{Композиция («часть/целое», сильная связь)}
		\begin{center}
			\begin{tikz*}[%
				class/.style={draw,rectangle,align=center,minimum height=3em,minimum width=7.5em,font=\bfseries},
				qty/.style={font=\footnotesize},
				role/.style={font=\small}
			]
				\node(library) [class]{Журнал};
				\node(book) [class,right=6em of library] {Статья};
				
				\draw[-diamond] (book) -- (library)
					node[qty,very near start,above] {1..*}
					node[role,midway,below] {включает}
					node[qty,very near end,above] {1};
			\end{tikz*}
			
			\vspace{1ex}
			\figureexpl{Журнал состоит из нескольких статей; для каждой статьи содержащий ее журнал фиксирован.}
		\end{center}
	}
	
	\section[Поведение]{Модели поведения}
	
	\frame{
		\frametitle{Модели поведения}
		
		\textbf{Цель:} определение реакции системы на внешние и внутренние входные сигналы.
		
		\vspace{2ex}
		\textbf{Виды сигналов:}
		\begin{itemize}
			\item данные (\emph{data-driven modeling})
			
			\textbf{Область применения:} системы обработки данных (напр., транзакций).
			
			\textbf{Диаграммы UML:} диаграмма  \inlink{p:activity}{деятельности}, диаграмма \inlink{p:sequence}{последовательности}.
			
			\vspace{1ex}
			\item события (\emph{event-driven modeling})
			
			\textbf{Область применения:} системы реального времени (напр., микроконтроллеры).
			
			\textbf{Диаграммы UML:} диаграмма состояний.
		\end{itemize}
	}
	
	\subsection{Диаграммы состояний}
	
	\frame{
		\frametitle{Диаграммы состояний}
		
		\begin{center}
			{\small\begin{tikz*}[%
	state/.style={rounded rectangle,draw,align=center,minimum width=10em,minimum height=4em,font=\footnotesize},
	label/.style={font=\footnotesize}
]
	\def\statecap#1{%
		{\small\textbf{#1}\strut{}}%
	}
	\node(active) [state] {%
		\alert<3>{\statecap{Выполнение}} \\[.75ex]
		\textbf{do/} обработка событий
	};
	\node(inactive) [state,below=5em of active] {%
		\statecap{Приостановка} \\[.75ex]
		\textbf{entry/} сохр. состояния
	};
	\node(restart) [state,right=7.5em of inactive] {%
		\statecap{Перезапуск} \\[.75ex]
		\alert<5>{\textbf{do/} onRestart()}
	};
	\node(kill) [state,left=7.5em of inactive] {%
		\alert<3>{\statecap{Уничтожение}} \\ \alert<3>{\statecap{процесса}}
	};
	\node(init) [state] at (active.west -| kill.north) {%
		\statecap{Инициализация} \\[.75ex]
		\textbf{entry/} восст. \\ состояния
	};
	\node(start) [circle,minimum width=1em,fill,above=3em of init] {};
	\node(end1) [circle,minimum width=1em,fill,below=3em of kill] {};
	\node(end) [circle,minimum width=1.35em,draw] at (end1.center) {};
	
	\draw[->] (start) -- (init);
	\draw[->] (kill) -- (end);
	\draw[->] (active) -- node[label,right,align=left]{активация \\ другого приложения} (inactive);
	\draw[->] (inactive) -- node(return)[label,above,align=center]{\alert<4>{возврат} \\ \alert<4>{к приложению}} (restart);
	\draw[->] (inactive) -- node[label,above,align=center]{нехватка \\ памяти} (kill);
	\draw[->] (restart.north) |- (active.east);
	\draw[->] (init) -- node(trans)[coordinate]{} (active);
	\draw[->] (active) -- node[label,right]{\quad выход} (kill);
	
	\node<2> [note={(start.east)},below right=1em of start] {начальное состояние};
	\node<2> [note={(end.east)},above right=1em of end] {конечное состояние};
	\node<3> [note={(kill.south east)},below right=1em of kill.south east] {состояние};
	\node<3> [note={(kill.south east)},below right=1em of kill.south east] {состояние};
	\node<3> [note={(active.north east)},above right=1em of active.north east] {состояние};
	\node<4> [note={(trans)},above=1em of trans] {переход};
	\node<4> [note={(return.south)},below=1em of return.south] {условный переход};
	\node<5> [note={(restart.north)},above=1em of restart.north,anchor=south east,text width=17.5em] {%
		условие выполнения действий:
		\begin{itemize}
			\item \textbf{entry} — при входе в~состояние;
			\item \textbf{exit} — при выходе из~состояния;
			\item \textbf{do} — во время нахождения в~состоянии.
		\end{itemize}
	};
\end{tikz*}
}
			
			\vspace{1ex}
			\figureexpl{Упрощенная диаграмма состояний для жизненного цикла программы (\emph{activity}) в Android}
		\end{center}
	}
	
	\section{Заключение}

	\subsection{Выводы}

	\frame{
		\frametitle{Выводы}
		
		\begin{enumerate}
			\item
			Модели нужны для разработки отдельных аспектов программных систем: 
			контекста выполнения, взаимодействий, структуры и поведения системы.
			
			\item
			Моделирование важно для детализации требований к программному обеспечению, 
			а~также для проектирования общей архитектуры системы и отдельных элементов.
			
			\item
			Одним из стандартов моделирования являются графические модели на основе языка UML. 
			В прикладном моделировании используются 5 основных типов диаграмм UML: 
			диаграммы деятельности, последовательности, вариантов использования, классов и~состояний.
		\end{enumerate}
	}

	\subsection{Материалы}
	
	\frame{
		\frametitle{Материалы}
		
		\begin{thebibliography}{9}
			\bibitem[2]{2}
			Sommerville, Ian
			\newblock Software Engineering.
			\newblock {\footnotesize Pearson, 2011. — 790 p.}
		
			\bibitem[1]{1}
			Лавріщева К.\,М. 
			\newblock Програмна інженерія (підручник). 
			\newblock {\footnotesize К., 2008. — 319 с.}
		\end{thebibliography}
	}
	
	\frame{
		\frametitle{}
		
		\begin{center}
			\Huge Спасибо за внимание!
		\end{center}
	}
\end{document}
