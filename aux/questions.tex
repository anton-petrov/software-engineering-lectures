\documentclass[a4paper,10pt]{article}

\usepackage{xtensions}
\hypersetup{hidelinks}

\textwidth=170mm
\oddsidemargin=0pt
\hoffset=-5.4mm % Левое и правое поле 2 см
\topmargin=-35pt
\headheight=11pt
\headsep=24pt
\textheight=257mm
\voffset=-5.4mm
\pagestyle{plain}
\def\baselinestretch{1.4}

\title{Список вопросов для экзамена по~программной~инженерии}
\author{Алексей Островский}
\date{\url{http://softandware.org.ua}}

\newcounter{globalenum}
\setcounter{globalenum}{0}

\newenvironment{genumerate}{%
	\begin{enumerate}%
	\afterlabel.%
	\setcounter{enumi}{\theglobalenum}%
}{%
	\setcounter{globalenum}{\theenumi}%
	\end{enumerate}%
}

\begin{document}

\maketitle

\section{Осенний семестр}

\subsubsection*{Введение в программную инженерию}

\begin{genumerate}
	\item
	Цель программной инженерии. Индустриальный подход к производству 
	программного обеспечения.

	\item
	Дисциплины программной инженерии: научная, инженерная, производственная, 
	управленческая и экономическая.

	\item
	Обзор основных областей программной инженерии согласно стандарту SWEBOK: 
	инженерия требований, проектирование, конструирование, тестирование 
	и~сопровождение~ПО.

	\item
	Обзор вспомогательных областей программной инженерии согласно стандарту SWEBOK: 
	конфигурация, управление инженерией, управление процессами разработки, 
	инструменты и~методы~ПИ, инженерия качества.

	\item
	Классификация моделей жизненного цикла ПО. Каскадная, инкрементная 
	и~эволюционная модели.

	\item
	Принципы гибкой методологии разработки ПО (agile development).
\end{genumerate}

\subsubsection*{Проектирование ПО}

\begin{genumerate}
	\item
	Инженерия требований к ПО. Функциональные и нефункциональные требования.

	\item
	Моделирование и представления программных систем.

	\item
	Унифицированный язык моделирования (UML) и его роль в разработке ПО.

	\item
	Архитектура ПО. Примеры архитектур: MVC, многослойная архитектура, клиент — сервер.

	\item
	Объектно\=ориентированное проектирование. Классификация шаблонов проектирования. Примеры шаблонов.
\end{genumerate}

\subsubsection*{Прикладные методы программирования}

\begin{genumerate}
	\item
	Декларативное программирование. Функциональные и логические языки программирования.

	\item
	Императивное программирование. Аналогии и различия в процедурном, структурном и модульном программировании.

	\item
	Основы объектно\=ориентированного программирования. Классификация объектно\=ориентированных ЯП.

	\item
	Программирование с компонентами. Концепция повторного использования.

	\item
	Основы аспектно-ориентированного программирования. Принцип разделения ответственности (separation of concerns).

	\item
	Сервисное программирование. Архитектура веб-сервисов.

	\item
	Синтаксис и семантика языков программирования. Виды семантики: статическая, динамическая и формальная.

	\item
	Виды метапрограммирования. Понятие рефлексии.

	\item
	Порождающее программирование. Языки описания предметных областей (DSL). Примеры DSL.
\end{genumerate}

\section{Весенний семестр}

\subsubsection*{Сопровождение ПО}

\begin{genumerate}
	\item
	Эволюция ПО. Сопровождение как частный случай эволюции. 
	Типы сопровождения: исправление дефектов, адаптация, совершенствование.

	\item
	Реинженерия и рефакторинг программ.

	\item
	Документирование ПО. Формы документации. 
	Автоматические генераторы документации: javadoc, doxygen.
\end{genumerate}

\subsubsection*{Интерфейсы и взаимодействие}

\begin{genumerate}
	\item
	Понятие интерфейса в программной инженерии. Двоичные и программные интерфейсы.

	\item
	Связывание компонентов приложения с помощью виртуальных машин. 
	Примеры виртуальных машин: Java Virtual Machine, Common Language Runtime.

	\item
	Интерфейсы внешних функций. Java Native Interface.

	\item
	Теория типов данных. Типобезопасность и безопасность памяти. 
	Классификация систем типов данных в языках программирования: 
	статическая и динамическая типизация; номинальная, структурная и утиная типизация.

	\item
	Приведение типов данных и полиморфизм. 
	Виды полиморфизма: специальный, параметрический и полиморфизм подтипов. 
	Отличия между наследованием и полиморфизмом.

	\item
	Интерфейсы в языках программирования. Интерфейс как контракт. 
	Принцип подстановки Барбары Лисков.

	\item
	Ковариантность и контравариантность параметрических конструкций в языках программирования (Java, C\#).

	\item
	Фундаментальные типы данных согласно стандарту ISO~11404. 
	Типы данных, независимые от ЯП.

	\item
	Понятие интероперабельности компонентов программных систем. 
	Низкоуровневая и высокоуровневая интероперабельность.

	\item
	Архитектура посредников доступа к объектам (object request broker). 
	Язык спецификации интерфейсов (IDL).

	\item
	Очереди сообщений. Архитектуры очереди: point to point и publish / subscribe.
\end{genumerate}

\subsubsection*{Вспомогательные области разработки ПО}

\begin{genumerate}
	\item
	Централизованные и распределенные системы управления версиями. Git.

	\item
	Методы автоматизации сборки ПО. Утилиты make и ant.

	\item
	Управление выпусками ПО. Выпуски в контексте эволюции программных продуктов.

	\item
	Основные принципы непрерывной интеграции (continuous integration).

	\item
	Управление качеством программных систем. Модели качества.

	\item
	Измерение программного обеспечения. Метрики ПО и их связь с характеристиками качества.

	\item
	Управление программным проектом. Планирование разработки.

	\item
	Управление рисками при разработке ПО.
\end{genumerate}

\subsubsection*{Прикладные аспекты программной инженерии}

\begin{genumerate}
	\item
	Постоянное хранение данных (data persistence). Сериализация данных. 
	Обзор стандартов XML и JSON.

	\item
	Объектно-реляционные отображения (ORM). Шаблоны проектирования, связанные с ORM: 
	ActiveRecord, DataMapper.

	\item
	Сервисная архитектура приложений. Веб-сервисы на основе SOAP / WSDL.

	\item
	Понятие REST (передача репрезентативного состояния) для сервисов. 
	Отличительные особенности веб-сервисов на основе REST.

	\item
	Понятие BigData. Базы данных NoSQL.

	\item
	Вспомогательные технологии для облачных вычислений. Классификация облачных архитектур.
\end{genumerate}

\begin{thebibliography}{99}
	\bibitem{1}
    Лаврищева Е.\,М., Петрухин В.\,А.
	Методы и средства программного обеспечения.~— М.:~Мин. образования РФ.~— 2007.~— 415~с.

	\bibitem{2}
    Лавріщева К.\,М.
	Програмна інженерія.~— К.:~Академперіодика.~— 2008.~— 319~с.

	\begin{otherlanguage}{english}
	\bibitem{3}
    Sommerville I. 
	Software engineering, 9th~ed.~— Boston, Massachusetts: Addison-Wesley.~— 2011.~— 790~p.

	\bibitem{4}
    Pfleeger S.\,L., Atlee J.\,M.
	Software engineering: theory and practice.~— Upper Saddle River, New Jersey: Prentice Hall.~— 2010.~— 756~p.

	\bibitem{5}
    Guide to the software engineering body of knowledge, version 3.0~/ 
	ed.~by~Bourque~P., Fairley R.\,E.~— IEEE Computer Society.~— 2014.~— URL: \url{http://swebok.org/}.

	\bibitem{6}
    McConnell S.
	Code complete.~— Upper Saddle River, New Jersey: Microsoft Press.~— 2009.~— 960~p.

	\bibitem{7}
    Pressman R.\,S.
	Software engineering: a practitioner’s approach.~— Basingstoke: Palgrave Macmillan.~— 2005.~— 880 p.

	\bibitem{8}
    Object-oriented analysis and design with applications~/ 
	Booch~G., Maksimchuck~R.\,A., Engle~M.\,W. et~al.~— 
	Upper Saddle River, New Jersey: Pearson Education.~— 2007.~— 720~p.

	\bibitem{9}
    Jacobson I., Ng P.-W.
	Aspect-oriented software development with use cases.~— Boston, Massachusetts: Addison-Wesley.~— 2005.~— 418~p.

	\bibitem{10}
    Bell M.
	Service-oriented modeling.~— Hoboken, New Jersey: John Wiley \& Sons.~— 2008.~— 368~p.

	\bibitem{11}
    Hansen M.\,R., Rischel H.
	Functional programming using F\#.~— Cambridge: Cambridge University Press.~— 2013.~— 361~p.

	\bibitem{12}
    Fowler M., Beck K., Brant J., Opdyke W., Roberts D.
	Refactoring: improving the design of existing code.~— Boston, Massachusetts: Addison-Wesley.~— 2012.~— 455~p.

	\bibitem{13}
    Hohpe G., Woolf B.
	Enterprise Integration Patterns.~— Boston, Massachusetts: Addison-Wesley.~— 2012.~— 735~p.

	\bibitem{14}
    Fowler M.
	Patterns of enterprise application architecture.~— Boston, Massachusetts: Addison-Wesley.~— 2012.~— 557~p.

	\bibitem{15}
    Shore J., Warden S.
	The art of agile development.~— Sebastopol, California: O’Reilly Media.~— 2008.~— 409~p.

	\bibitem{16}
    Wiegers K., Beatty J.
	Software requirements. — Upper Saddle River, New Jersey: Pearson Education.~— 2013.~— 672~p.

	\bibitem{17}
    Myers G.\,J., Sandler C., Badgett T.
	The art of software testing.~— Hoboken, New Jersey: John Wiley \& Sons.~— 2011.~— 256~p.

	\bibitem{18}
    Tian J.
	Software quality engineering. — Hoboken, New Jersey: John Wiley \& Sons.~— 2005.~— 440~p.

	\bibitem{19}
    Buyya R., Broberg J., Goscinski A.\,M.
	Cloud computing: principles and paradigms.~— Hoboken, New Jersey: John Wiley \& Sons.~— 2010.~— 664~p.

	\bibitem{20}
    Coulouris G.\,F., Dollimore J., Kindberg T.
	Distributed systems: concepts and design.~— Upper Saddle River, New Jersey: Pearson Education.~— 2011.~— 927~p.
	\end{otherlanguage}
\end{thebibliography}

\end{document}
