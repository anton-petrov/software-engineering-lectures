\documentclass{a4beamer}
%% Lectures - common definitions

\usextensions{tikz}
\usetikzlibrary{shapes.multipart,shapes.callouts,shapes.geometric}
\input{fix-callouts.inc} % Fixes absolute positioning of rectangle callouts

\newif\ifbigpages \bigpagesfalse
\ifdim\paperwidth >20cm
	\bigpagestrue
\fi

\tikzset{%
	note/.style={rectangle callout,draw=none,callout pointer width=1em,%
		align=flush left,font=\footnotesize,inner sep=0.5em,%
		fill=blue!15,fill opacity=0.95,text opacity=1.0,callout absolute pointer=#1},
	node distance=2em and 2.75em
}
\ifbigpages
	% Scale all arrow tips by the factor of 2.5
	\let\old@pgf@arrow@call=\pgf@arrow@call
	\def\pgf@arrow@call#1{%
		\@tempdima=\pgflinewidth%
		\pgfsetlinewidth{2.5\pgflinewidth}%
		\old@pgf@arrow@call{#1}%
		\pgfsetlinewidth{\@tempdima}%
	}
	\def\pgfarrowsleftextend#1{\pgfmathsetlength{\pgf@xa}{1.5*#1}}
	\def\pgfarrowsrightextend#1{\pgfmathsetlength{\pgf@xb}{1.5*#1}}
\fi

%% Load listings package
\usepackage{listings}

%% Are we inside a comment?
\newif\iflstcomment \lstcommentfalse

\lstset{%
	tabsize=4,
	showstringspaces=false,
	basicstyle=\linespread{1.25}\ttfamily\small,
	keywordstyle=\bfseries,
	commentstyle=\lstcommentstyle,
	numbers=left,
	numberstyle=\footnotesize\color{gray},
	xleftmargin=2.5em,
	extendedchars=true,
	escapechar=\$,
	escapebegin=\iflstcomment\begingroup\lstcommentstyle\fi,
	escapeend=\iflstcomment\endgroup\fi
}

\def\lstcommentstyle{\color{gray}}

\lst@AddToHook{AfterBeginComment}{\global\lstcommenttrue}
\let\orig@lst@EndComment=\lst@EndComment
\def\lst@EndComment{\global\lstcommentfalse\orig@lst@EndComment}
\lst@AddToHookAtTop{EOL}{%
	\lst@ifLmode\global\lstcommentfalse\fi% XXX Sloppy way to determine comment end
}

%% Python with docstrings treated as comments
\lstdefinelanguage[doc]{python}[]{python}{%
	deletestring=[s]{"""}{"""},%
	morecomment=[s]{"""}{"""}%
}%

%% JavaScript language
\lstdefinelanguage{javascript}%
	{morekeywords={break,case,catch,%
		const,constructor,continue,default,do,else,false,%
		finally,for,function,if,in,instanceof,%
		new,null,prototype,%
		return,switch,this,throw,%
		true,try,typeof,var,while},%
	sensitive,%
	morecomment=[l]//,%
	morecomment=[s]{/*}{*/},%
	morestring=[b]",%
	morestring=[b]',%
}[keywords,comments,strings]%

%% C# language (4.0?)
\lstdefinelanguage{csharp}%
	{morekeywords={abstract,as,%
		base,bool,byte,case,catch,char,%
		checked,class,const,continue,%
		decimal,default,delegate,do,double,%
		else,enum,event,explicit,extern,%
		false,finally,fixed,float,for,foreach,%
		goto,if,implicit,in,int,interface,%
		internal,is,lock,long,%
		namespace,new,null,object,operator,out,%
		override,params,private,protected,public,%
		readonly,ref,return,sbyte,sealed,%
		short,sizeof,stackalloc,static,string,%
		struct,switch,this,throw,true,try,%
		typeof,uint,ulong,unchecked,unsafe,ushort,%
		using,virtual,void,volatile,while%
	},%
	sensitive,%
	morecomment=[l]//,%
	morecomment=[s]{/*}{*/},%
	morestring=[b]",%
	morestring=[b]',%
}[keywords,comments,strings]%

%% Translation for fact environment
\deftranslation[to=russian]{Fact}{Наблюдение}

%% Inline code snippets
\def\code#1{\texttt{#1}}
\def\codekw#1{\code{\textbf{#1}}}

\def\quoteauthor#1{\par\footnotesize\upshape\hfill—~#1}

%% English term
\def\engterm#1{(англ. \textit{#1})}
%% Term with explanation below (to be used in diagrams)
\def\termwithexpl#1#2{#1\strut{}\\\small\color{gray}(\textit{#2})\strut{}}
%% External link
\def\extlink#1#2{\href{#1}{\color[rgb]{0.7,0.7,1.0}\dashbar{#2}}}
%% Internal link
\def\inlink#1#2{\hyperlink{#1}{\color[rgb]{0.7,0.7,1.0}\dashbar{#2}}}
%% Explanation for a list item
\def\itemexpl#1{\begingroup\small\vspace{0.75ex}#1\par\endgroup}



\usetikzlibrary{shapes.misc}


\lecturetitle{Программная инженерия. Лекция №24 — Управление программным проектом}
\title[Управление]{Управление программным проектом. Планирование и риски}
\author{Алексей Островский}
\institute{\small{Физико-технический учебно-научный центр НАН Украины}\vspace{2ex}}
\date{7 мая 2015 г.}

\begin{document}
	\frame{\titlepage}

	\section{Управление}

	\frame{
		\frametitle{Управление программным проектом}

		\begin{Definition}
			\textbf{Управление программным проектом} \engterm{software project management} — деятельность, 
			направленная на~производство качественного программного обеспечения 
			при~ограниченных бюджетных и~временных ресурсах.
		\end{Definition}

		\vspace{1ex}
		\textbf{Цели управления:}
		\begin{itemize}
			\item поставка ПО в заданный срок;
			\item соблюдение расходов в рамках бюджета;
			\item поставка ПО, удовлетворяющего запросы заказчика;
			\item организация продуктивной команды разработки.
		\end{itemize}

		\vspace{1ex}
		\textbf{Проблемы управления производством ПО:}
		\begin{itemize}
			\item неосязаемость продуктов производства;
			\item уникальность программных проектов;
			\item специфичность процессов производства ПО.
		\end{itemize}
	}

	\subsection{Составляющие}

	\frame{
		\frametitle{Составляющие управления}

		\begin{itemize}
			\item
			\textbf{Планирование проекта:}
			\begin{itemize}
				\item планирование, оценка и составление планов по разработке ПО; 
				\item назначение ответственных за выполнение заданий; 
				\item контроль выполнения заданий.
			\end{itemize}

			\vspace{0.5ex}
			\item
			\textbf{Отчетность:} составление отчетов о ходе выполнения проекта для заказчика и~организации-разработчика.

			\vspace{0.5ex}
			\item
			\textbf{Управление рисками:} 
			\begin{itemize}
				\item оценка рисков, влияющих на проект; 
				\item мониторинг рисков; 
				\item устранение проблем, связанных с рисками.
			\end{itemize}

			\vspace{0.5ex}
			\item
			\textbf{Управление персоналом:}
			\begin{itemize}
				\item подбор персонала для проекта;
				\item организация коммуникации между разработчиками.
			\end{itemize}

			\vspace{0.5ex}
			\item
			Создание \textbf{проектных планов для заключения контракта} с заказчиком.
		\end{itemize}
	}

	\section{Риски}

	\frame{
		\frametitle{Риски}

		\begin{Definition}
			\textbf{Риск} — нежелательное событие при выполнении проекта, которое может иметь 
			непредвиденные отрицательные последствия.
		\end{Definition}

		\vspace{1ex}
		\textbf{Категории рисков:}
		\begin{itemize}
			\item
			проектные риски — риски, связанные с расписанием проекта или ресурсами 
			(напр., потеря опытного разработчика);

			\item
			риски, связанные с продуктом — риски, влияющие на~качество или~производительность~ПО 
			(напр., неудовлетворительное поведение приобретенного компонента);

			\item
			бизнес-риски — риски, влияющие на~организацию-разработчика 
			(напр., выход на~рынок конкурирующего продукта).
		\end{itemize}
	}

	\frame{
		\frametitle{Примеры рисков}

		\begin{center}
			\begin{tabular}{|p{0.3\textwidth}|p{0.6\textwidth}|}
				\hline
				\textbf{Категория} & \textbf{Риск} \cr
				\hline
				Проектные & смена персонала; \cr
				& изменение структуры управления; \cr  
				& отсутствие необходимого оборудования. \cr
				\hline
				Проектные & изменение требований; \cr 
				и производственные & задержки с определением спецификации; \cr 
				& недооценка размера и сложности системы. \cr
				\hline
				Производственные & недостаточная эффективность CASE-инструментов. \cr
				\hline
				Бизнес & устаревание базовых технологий; \cr
				& появление конкурирующих продуктов. \cr
				\hline
			\end{tabular}
		\end{center}
	}

	\subsection{Управление рисками}

	\frame{
		\frametitle{Управление рисками}

		\textbf{Этапы управления рисками:}
		\begin{enumerate}
			\item
			Идентификация рисков.

			\vspace{0.5ex}
			\textbf{Результат:} перечень потенциальных рисков.

			\vspace{0.5ex}
			\item
			Анализ рисков.

			\vspace{0.5ex}
			\textbf{Результат:} установка приоритетов для рисков.

			\vspace{0.5ex}
			\item
			Планирование рисков.

			\vspace{0.5ex}
			\textbf{Результат:} планы по избеганию и минимизации рисков.

			\vspace{0.5ex}
			\item
			Мониторинг рисков.

			\vspace{0.5ex}
			\textbf{Результат:} оценка актуальных рисков на~данный момент работы с~проектом.
		\end{enumerate}
	}

	\subsection{Идентификация рисков}

	\frame{
		\frametitle{Идентификация рисков}

		\begin{itemize}
			\item
			\textbf{Технологические риски} (связанные с~оборудованием или~ПО, используемым в~разработке).

			\vspace{0.5ex}
			\textbf{Примеры:}
			\begin{itemize}
				\item недостаточная скорость обработки транзакций СУБД;
				\item дефекты в компонентах повторного использования.
			\end{itemize}

			\vspace{0.5ex}
			\item
			\textbf{Кадровые риски.}

			\vspace{0.5ex}
			\textbf{Примеры:}
			\begin{itemize}
				\item
				невозможность набора компетентного персонала;
				\item
				отсутствие должных навыков у членов команды разработки;
				\item
				недоступность ключевого персонала (напр., по~причине болезни) 
				в~ключевые периоды.
			\end{itemize}

			\vspace{0.5ex}
			\item
			\textbf{Организационные риски} (связанные с организацией разработки).

			\vspace{0.5ex}
			\textbf{Примеры:}
			\begin{itemize}
				\item
				реструктуризация управления проектом во время разработки;
				\item
				сокращение финансирования проекта.
			\end{itemize}
		\end{itemize}
	}

	\frame{
		\frametitle{Идентификация рисков (продолжение)}

		\begin{itemize}
			\item
			\textbf{Инструментальные риски} (вызванные инструментами разработки).

			\vspace{0.5ex}
			\textbf{Примеры:}
			\begin{itemize}
				\item
				неэффективность кода, сгенерированного автоматически;
				\item
				невозможность интеграции различных инструментов разработки.
			\end{itemize}

			\vspace{0.5ex}
			\item
			\textbf{Риски, связанные с требованиями.}

			\vspace{0.5ex}
			\textbf{Примеры:}
			\begin{itemize}
				\item
				необходимость кардинального изменения архитектуры, связанная с~изменениями требований.
			\end{itemize}

			\vspace{0.5ex}
			\item
			\textbf{Оценочные риски} (связанные с неправильной оценкой ресурсов).

			\vspace{0.5ex}
			\textbf{Примеры:}
			\begin{itemize}
				\item
				недооценка времени, требуемого на разработку;
				\item
				переоценка скорости исправления дефектов системы;
				\item
				недооценка размера разрабатываемого ПО.
			\end{itemize}
		\end{itemize}
	}

	\subsection{Анализ рисков}

	\frame{
		\frametitle{Анализ рисков}

		\textbf{Вероятность рисковой ситуации:} 
		\begin{itemize}
			\item очень низкая ($< 10$~\%); 
			\item низкая ($10$–$25$~\%); 
			\item средняя ($25$–$50$~\%); 
			\item высокая ($50$–$75$~\%); 
			\item очень высокая ($> 75$~\%).
		\end{itemize}

		\vspace{1ex}
		\textbf{Последствия рисковой ситуации:}
		\begin{itemize}
			\item катастрофические (угроза существованию проекта);
			\item серьезные (большие задержки);
			\item терпимые (задержки в пределах плана); 
			\item незначительные.
		\end{itemize}
	}

	\subsection{Планирование рисков}

	\frame{
		\frametitle{Планирование рисков}

		\textbf{Стратегии реагирования на риски:}
		\begin{itemize}
			\item
			\textbf{Избегание рисков} — минимизация возникновения рисковой ситуации.

			\vspace{0.5ex}
			\textbf{Риск:} дефективные компоненты.
			 
			\textbf{Стратегия:} использование проверенных компонентов 
			(напр., использованных в~предыдущих проектах).

			\vspace{0.5ex}
			\item
			\textbf{Минимизация последствий.}

			\vspace{0.5ex}
			\textbf{Риск:} недоступность персонала в ключевой период разработки.

			\textbf{Стратегия:} реорганизация разработки для~усиления взаимодействия 
			между~разработчиками и~повышения понимания сути работы коллег.

			\vspace{0.5ex}
			\item
			\textbf{План действий} в чрезвычайной ситуации.

			\vspace{0.5ex}
			\textbf{Риск:} сокращение финансирования проекта.

			\textbf{Стратегия:} подготовка документа для~начальства с~описанием важности проекта 
			для~организации и~нецелесообразности сокращения бюджета.
		\end{itemize}
	}

	\subsection{Мониторинг рисков}

	\frame{
		\frametitle{Мониторинг рисков}

		\begin{Definition}
			\textbf{Мониторинг рисков} — проверка базовых предположений о~рисках, 
			их~вероятности и~влияния на~процесс разработки.
		\end{Definition}

		\vspace{1ex}
		\begin{center}
			\begin{tabular}{|p{0.25\textwidth}|p{0.65\textwidth}|}
				\hline
				\textbf{Типы рисков} & \textbf{Факторы} \cr
				\hline
				Технологические & \raggedright задержки в доставке ПО или оборудования; проблемы работы с ними. \cr
				\hline
				Кадровые & \raggedright плохие отношения в команде разработки; высокая текучесть кадров. \cr
				\hline
				Организационные & \raggedright бездействие начальства. \cr
				\hline
				Инструментальные & \raggedright недостаточное использование инструментов; 
					запрос на~более мощные рабочие компьютеры. \cr
				\hline
				Требования & \raggedright большое число запросов на изменение требований. \cr
				\hline
				Оценочные & \raggedright отставания в расписании; низкая скорость устранения дефектов. \cr
				\hline
			\end{tabular}
		\end{center}
	}

	\section{Планирование}

	\frame{
		\frametitle{Планирование проекта}

		\textbf{Задачи:}
		\begin{itemize}
			\item разбиение процесса разработки на составляющие;
			\item назначение ответственных за выполнение заданий;
			\item оценка затрат на выполнение проекта;
			\item управление рисками;
			\item оценка текущей степени выполнения проекта.
		\end{itemize}

		\vspace{1ex}
		\textbf{Этапы планирования:}
		\begin{enumerate}
			\item
			предварительный план (составляется до~подписания контракта с~заказчиком);
			\item
			начальный план (составляется в~начале разработки);
			\item
			периодические планы (уточняются по~ходу~выполнения проекта).
		\end{enumerate}
	}

	\frame{
		\frametitle{Разработка через планирование}

		\begin{Definition}
			\textbf{Разработка через планирование} \engterm{plan-driven development}, \textbf{классическая модель разработки} — 
			подход к~разработке~ПО, основанный на~детальном планировании процессов производства.
		\end{Definition}

		\vspace{0.5ex}
		\textbf{Достоинства:}
		\begin{itemize}
			\item
			раннее выделение организационных проблем (доступность персонала, взаимодействие с~другими проектами);
			\item
			ранняя оценка затрат на проект;
			\item
			упрощение взаимодействия с~заказчиком и~третьими сторонами (напр., для~сертификации).
		\end{itemize}

		\textbf{Недостатки:}
		\begin{itemize}
			\item
			громоздкость системы документирования проекта;
			\item
			высокие затраты на внесение изменений по~ходу проекта;
			\item
			низкая степень адаптации к изменениям в~среде выполнения.
		\end{itemize}
	}

	\subsection{Проектный план}

	\frame{
		\frametitle{Проектный план}

		\textbf{Составляющие плана:}
		\begin{enumerate}
			\item
			\textbf{Введение} — решаемые задачи; ограничения (бюджетные, временные, …), 
			влияющие на~управление проектом.

			\vspace{0.5ex}
			\item
			\textbf{Организация проекта} — организация команды разработки, роли разработчиков.

			\vspace{0.5ex}
			\item
			\textbf{Анализ рисков} — риски, вероятность их~появления, стратегии реагирования.

			\vspace{0.5ex}
			\item
			\textbf{Ресурсные требования} на~оборудование и~ПО, оценка стоимости приобретения.

			\vspace{0.5ex}
			\item
			\textbf{Разбиение работы} на~составляющие процессы, выделение рубежей \engterm{milestone} 
			и~артефактов производства \engterm{deliverable}.

			\vspace{0.5ex}
			\item
			\textbf{Расписание проекта} — зависимости между~процессами, оценка времени выполнения, 
			распределение кадров.

			\vspace{0.5ex}
			\item
			Механизмы \textbf{мониторинга} и \textbf{отчетности}.
		\end{enumerate}
	}

	\frame{
		\frametitle{Дополнительные планы}

		\begin{itemize}
			\item
			\textbf{План качества} — описание деятельности по~достижению, измерению и~контролю качества 
			процессов и~артефактов производства.

			\vspace{0.5ex}
			\item
			\textbf{План валидации} — описание подхода к~проверке программной системы, 
			ресурсов и~расписания валидации.

			\vspace{0.5ex}
			\item
			\textbf{План управления конфигурацией} — описание методов управления конфигурацией проекта 
			и~используемых утилит (систем управления версиями, утилит построения, …).

			\vspace{0.5ex}
			\item
			\textbf{План сопровождения} — прогноз требований, затрат и~стоимости сопровождения проекта.

			\vspace{0.5ex}
			\item
			\textbf{План квалификации кадров} — описание повышения квалификации и~опыта разработчиков 
			по~ходу выполнения проекта.
		\end{itemize}
	}

	\frame{
		\frametitle{Процесс планирования}

		{\small\begin{tikz*}[%
	every node/.style={draw,align=center,font=\small},
	activity/.style={rounded rectangle,minimum height=3em},
	decision/.style={diamond,minimum width=1em,minimum height=1em},
	label/.style={draw=none,font=\footnotesize},
	sink/.style={rectangle,fill,inner sep=0.1em}
]
	\node(start) [circle,fill,minimum width=1em,minimum height=1em] {};
	\node(sink1) [sink,minimum height=12.5em,right=1.5em of start] {};
	\node(risk) [activity,right=1em of sink1] {Идентификация \\ рисков};
	\node(constr) [activity,above=of risk] {Выделение \\ ограничений};
	\node(milestone) [activity,below=of risk] {Определение \\ рубежей \\ и артефактов};
	\node(sink2) [sink,minimum height=12.5em,right=1em of risk] {};
	\node(schedule) [activity,right=1.5em of sink2] {Составление \\ расписания};
	\node(sink3) [sink,minimum height=8em,right=1.5em of schedule] {};
	\node(work) [activity,above right=1em and 5em of sink3.center,anchor=south] {Выполнение \\ работы}; 
	\node(monitor) [activity,below right=1em and 5em of sink3.center,anchor=north] {Мониторинг \\ прогресса};
	\node(dec-problems) [decision,below=3em of monitor] {};
	\node(dec-complete) [decision,right=7.5em of dec-problems] {};
	\node(end1) [circle,fill,minimum width=1em,minimum height=1em,below=4em of dec-complete] {};
	\node(end) [circle,draw,minimum width=1.35em,minimum height=1.35em] at (end1.center) {};
	\node(sink4) [sink,minimum width=13.5em,below=6em of dec-problems] {};
	\node(strategy) [activity,below left=4em and 1em of sink4.south,anchor=east] {Противодействие \\ рискам};
	\node(replan) [activity,below right=4em and 1em of sink4.south,anchor=west] {Перепланирование};

	\draw[->] (start) -- (sink1);
	\draw[->] (sink1.east |- constr.west) -- (constr.west);
	\draw[->] (sink1.east |- risk.west) -- (risk.west);
	\draw[->] (sink1.east |- milestone.west) -- (milestone.west);
	\draw[<-] (sink2.west |- constr.east) -- (constr.east);
	\draw[<-] (sink2.west |- risk.east) -- (risk.east);
	\draw[<-] (sink2.west |- milestone.east) -- (milestone.east);
	\draw[->] (sink2) -- (schedule);
	\draw[->] (schedule) -- (sink3);
	\draw[->] (sink3.east |- work.west) -- (work.west);
	\draw[->] (sink3.east |- monitor.west) -- (monitor.west);
	\draw[->] (monitor) -- (dec-problems);
	\draw[->] (dec-problems) -- node[label,above]{[нет проблем]} (dec-complete);
	\draw[->] (dec-problems) -| node[label,pos=0.25,above]{[незначительные \\ проблемы]} (schedule);
	\draw[->] (dec-complete) |- node[label,pos=0.25,right]{[есть работа]} (work);
	\draw[->] (dec-complete) -- node[label,right] {[проект \\ завершен]} (end);
	\draw[->] (dec-problems) -- node[label,left]{[серьезные проблемы]} (sink4);
	\draw[->] (sink4.south -| strategy.north) -- (strategy.north);
	\draw[->] (sink4.south -| replan.north) -- (replan.north);
\end{tikz*}
}
	}

	\subsection{Расписание}

	\frame{
		\frametitle{Составление расписания проекта}

		\begin{Definition}
			\textbf{Процесс} \engterm{activity} — автономная часть разработки, которая характеризуется:
			\begin{itemize}
				\item длительностью (1–8 недель);
				\item оценкой объема работ (в человеко-днях);
				\item граничным сроком завершения;
				\item условием завершения (напр., удачное выполнение всех тестов).
			\end{itemize}
		\end{Definition}

		\vspace{1ex}
		\textbf{Этапы составления расписания:}
		\begin{enumerate}
			\item
			выделение процессов на основе требований и предварительной архитектуры;
			\item
			определение взаимодействия и зависимостей между процессами;
			\item
			оценивание ресурсов для выполнения процессов;
			\item
			распределение членов команды разработки по процессам;
			\item
			создание расписания и его визуальных представлений (графиков, диаграмм, …).
		\end{enumerate}
	}

	\frame{
		\frametitle{Представление расписания}

		\begin{figure}
			\begin{tikz*}[%
	milestone/.style={diamond,fill,minimum width=0.5em,minimum height=0.5em,inner sep=0pt},
	task/.style={rectangle,minimum height=0.75em,fill}
]
	\node(week) {Неделя};
	\node(w0) [right=1em of week] {0};
	\node(w1) [right=3em of w0.center,anchor=center] {1};
	\node(w2) [right=3em of w1.center,anchor=center] {2};
	\node(w3) [right=3em of w2.center,anchor=center] {3};
	\node(w4) [right=3em of w3.center,anchor=center] {4};
	\node(w5) [right=3em of w4.center,anchor=center] {5};
	\node(w6) [right=3em of w5.center,anchor=center] {6};
	\node(w7) [right=3em of w6.center,anchor=center] {7};
	\node(w8) [right=3em of w7.center,anchor=center] {8};
	\node(w9) [right=3em of w8.center,anchor=center] {9};

	\node(start) [milestone,below=1em of w0.south] {};
	\node [right=0em of start] {Начало};
	\node(t1) [task,minimum width=6em,below=1em of start,anchor=west] {};
	\node [left=0em of t1] {T1};
	\node(t2) [task,minimum width=9em,below=1.5em of t1.west,anchor=west] {};
	\node [left=0em of t2] {T2};
	\node(m1) [milestone,below=2.5em of t1.east] {};
	\node [right=0em of m1] {M1 / T1};
	\node(t3) [task,minimum width=9em,below=1em of m1,anchor=west] {};
	\node [left=0em of t3] {T3};
	\node(t4) [task,minimum width=6em,below=4em of t2.west,anchor=west] {};
	\node [left=0em of t4] {T4};
	\node(m2) [milestone,below right=1em and 3em of t4.east] {};
	\node [right=0em of m2] {M2 / T2 \& T4};
	\node(t5) [task,minimum width=6em,below=1em of m2,anchor=west] {};
	\node [left=0em of t5] {T5};
	\node(m3) [milestone,below=1em of t5.west] {};
	\node [right=0em of m3] {M3 / T1 \& T2};
	\node(t6) [task,minimum width=3em,below=1em of m3,anchor=west] {};
	\node [left=0em of t6] {T6};
	\node(t7) [task,minimum width=12em,below=6.5em of t4.east,anchor=west] {};
	\node [left=0em of t7] {T7};
	\node(m4) [milestone,below=2.5em of t6.east] {};
	\node [right=0em of m4] {M4 / T6};
	\node(t8) [task,minimum width=9em,below=1em of m4,anchor=west] {};
	\node [left=0em of t8] {T8};
	\node(m5) [milestone,below=4em of t7.east] {};
	\node [right=0em of m5] {M5 / T3 \& T7};
	\node(t9) [task,minimum width=9em,below=1em of m5,anchor=west] {};
	\node [left=0em of t9] {T9};

	\node<2> [note={(t2.south)},below=1em of t2.south] {задания};
	\node<2> [note={(t7.south)},below=1em of t7.south] {задания};
	\node<3> [note={(m1.south)},below=1em of m1.south] {рубежи и задания, \\ от которых они зависят};
	\node<3> [note={(m4.south)},below=1em of m4.south] {рубежи и задания, \\ от которых они зависят};
\end{tikz*}

			\caption{Представление расписания в виде столбчатой диаграммы}
		\end{figure}
	}

	\subsection{Гибкое планирование}

	\frame{
		\frametitle{Гибкое планирование}

		\begin{Definition}
			\textbf{Гибкое планирование} \engterm{agile planning} — составление 
			инкрементальных итеративных планов по~производству программного продукта 
			по~ходу разработки.
		\end{Definition}

		\vspace{0.5ex}
		\textbf{Базовое предположение:} требования заказчика уточняются и~изменяются в~ходе разработки~ПО.

		\vspace{2ex}
		\textbf{Этапы планирования:}
		\begin{enumerate}
			\item
			Планирование выпусков — определение характеристик следующего выпуска программной системы.

			\vspace{0.5ex}
			\textbf{Длительность:} несколько месяцев.

			\item
			Планирование итераций — определение следующей итерации разработки.

			\vspace{0.5ex}
			\textbf{Длительность:} 2–4 недели.
		\end{enumerate}
	}

	\frame{
		\frametitle{Планирование в XP}

		\textbf{Этапы планирования в экстремальном программировании (XP):}
		\begin{enumerate}
			\item
			\textbf{выделение сценариев использования} \engterm{user story}, 
			по~возможности покрывающих всю~функциональность системы;

			\item
			\textbf{оценка затрат} на имплементацию функций, соответствующих различным сценариям;

			\item
			\textbf{оценка скорости реализации} (из~предыдущего опыта или~путем имплементации 
			нескольких тестовых сценариев);

			\item
			\textbf{планирование выпуска:} подбор и конкретизация сценариев для~следующего выпуска системы 
			(совместно с~заказчиком);

			\item
			\textbf{планирование итерации:} подбор сценариев для~итерации с~учетом скорости реализации, 
			разбиение сценариев на~короткие задания (4–16 часов);

			\item
			\textbf{оценка прогресса} посредине итерации для удаления «отстающих» сценариев из~итерации.
		\end{enumerate}
	}

	\subsection{Оценка затрат}

	\frame{
		\frametitle{Оценка затрат}

		\textbf{Проблемы оценки затрат} на процессы разработки:
		\begin{itemize}
			\item
			большое количество неизвестных параметров на~ранних этапах 
			(новые технологии, навыки команды разработки);

			\item
			невозможность проверки корректности оценки.
		\end{itemize}

		\vspace{1ex}
		\textbf{Методы оценки:}
		\begin{itemize}
			\item
			\textbf{Экспертные методы} (на основе опыта управления предыдущими проектами) — 
			оценка затрат на~производство отдельных артефактов и~суммирование получившихся оценок.

			\vspace{0.5ex}
			\textbf{Проблемы:} невозможность оценки при~использовании новых технологий.

			\vspace{0.5ex}
			\item
			\textbf{Алгоритмические методы} — оценка затрат на~основе метрик проекта и~процессов 
			(напр., размер проекта).
		\end{itemize}
	}

	\frame{
		\frametitle{Алгоритмическое моделирование затрат}


		\textbf{Общая формула оценки затрат:}
		\vspace{-1ex}
		\[
			Effort = A \times Size^B \times M.
		\]

		\begin{itemize}
			\item
			$A$ — константа, зависящая от~типа разрабатываемого~ПО и~организации работы;
			\item
			$Size$ — оценка размера~ПО или~его~сложности;
			\item
			$B$ — показатель, зависящий от~сложности проекта ($1 \leq B \leq 1{,}5$);
			\item
			$M$ — множитель, зависящий от~характеристик проекта (напр., требований к~безотказности~ПО 
			или~навыков команды разработки).
		\end{itemize}

		\vspace{1ex}
		\textbf{Проблемы:}
		\begin{itemize}
			\item сложности с оценкой размера $Size$ на ранних этапах;
			\item субъективность факторов, влияющих на $B$ и $M$.
		\end{itemize}
	}

	\frame{
		\frametitle{Пример: модель COCOMO II}

		\begin{Definition}
			\textbf{COCOMO II} (constructive cost model) — модель для оценки затрат на~производство~ПО, 
			разработанная Б.\,Бёмом (Barry Boehm).
		\end{Definition}

		\vspace{1ex}
		\textbf{Подмодели:}
		\begin{itemize}
			\item
			\textbf{Модель композиции} \engterm{application composition model} — оценка затрат 
			на~построение системы из~КПИ, с~помощью сценариев и~программирования~БД.

			\item
			\textbf{Модель раннего проектирования} \engterm{early design model} — 
			предварительная оценка затрат на~разработку на~основе описания интерфейсов системы.

			\item
			\textbf{Модель повторного использования} \engterm{reuse model} — 
			оценка затрат на~интеграцию~КПИ и~/~или автоматически сгенерированного кода.

			\item
			\textbf{Пост-архитектурная модель} \engterm{post-architecture model} — 
			оценка затрат на~основе характеристик проекта.
		\end{itemize}
	}

	\section{Заключение}

	\subsection{Выводы}
	
	\frame{
		\frametitle{Выводы}

		\begin{enumerate}
			\item
			Управление рисками и планирование проекта — две~важных составляющих управления разработкой~ПО, 
			наряду с~управлением качеством и~управлением конфигурацией.

			\vspace{0.5ex}
			\item
			Управление рисками включает оценку вероятности и~последствий рисковых ситуаций, 
			разработку стратегий борьбы с~ними, а~также мониторинг рисков 
			по~ходу выполнения проекта.

			\vspace{0.5ex}
			\item
			Планирование проекта позволяет разбить разработку на~мелкие процессы, 
			определить расписание их~выполнения и~ответственных разработчиков.

			\vspace{0.5ex}
			\item
			Оценка затрат — одна из~ключевых задач планирования разработки~ПО. 
			Для~ее~решения существуют алгоритмические методы, в~т.\,ч.~COCOMO (constructive cost model).
		\end{enumerate}
	}
	
	\subsection{Материалы}
	
	\frame{
		\frametitle{Материалы}
		
		\begin{thebibliography}{9}
			\bibitem[1]{1}
			Sommerville, Ian
			\newblock Software Engineering.
			\newblock {\footnotesize Pearson, 2011. — 790 p.}

			\bibitem[2]{2}
			Лавріщева К.\,М. 
			\newblock Програмна інженерія (підручник). 
			\newblock {\footnotesize К., 2008. — 319 с.}
		\end{thebibliography}
	}
	
	\frame{
		\frametitle{}
		
		\begin{center}
			\Huge Спасибо за внимание!
		\end{center}
	}

\end{document}

